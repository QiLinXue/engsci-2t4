\documentclass{article}
\usepackage{qilin}
\tikzstyle{process} = [rectangle, rounded corners, minimum width=1.5cm, minimum height=0.5cm,align=center, draw=black, fill=gray!30, auto]
\title{APM426: General Relativity}
\author{QiLin Xue}  
\date{Fall 2022}
\usepackage{mathrsfs}
\usetikzlibrary{arrows}
\usepackage{stmaryrd}
\usepackage{accents}
\newcommand{\ubar}[1]{\underaccent{\overline}{#1}}
\usepackage{pgfplots}
\numberwithin{equation}{section}
\usetikzlibrary{quantikz}
\usepackage[american]{circuitikz}
\newcommand{\equals}{=}

\begin{document}

\maketitle
\tableofcontents
\newpage
\section{Manifolds and Tensor Fields}
\subsection{Review}
\textit{Note:} The manifold section should serve as a review from MAT367, so we will be quickly going over it.
\begin{definition}
    An $n$-dimensional, $C^\infty$, real manifold $M$ is a topological space together with a collection of subsets $\{O_\alpha\}$ satisfying the following properties:
    \begin{enumerate}
        \item $\{O_\alpha\}$ cover $M.$
        \item For each $\alpha$ there is a homeomorphism $\psi_\alpha: O_\alpha \to U_\alpha,$ where $U_\alpha$ is an open subset of $\mathbb{R}^n$.
        \item If any two sets $O_\alpha$ and $O_\beta$ intersect, then $\psi_\beta \circ \psi_\alpha^{-1}$ is smooth.
    \end{enumerate}
    Note that there are a few extra conditions (Hausdorff and paracompact), but they generally aren't important.
\end{definition}
Let $\mathcal{F}$ denote the collection of $C^\infty$ functions from $M$ to $\mathbb{R}.$
\begin{definition}
    Tangent vectors are maps $v:\mathcal{F}\to \mathbb{R}$ which satisfy:
    \begin{enumerate}
        \item Linearity: $v(af + g) = av(f) + g$
        \item Leibniz's rule: $v(fg) = v(f)g + f v(g)$ 
    \end{enumerate}
\end{definition}
The commutator (Lie Bracket) of two tangent vectors $[v,w] = v\circ w - w\circ v$ is also a tangent vector.
\subsection{Tensors}
Now, we can introduce the notion of tensors.
\begin{definition}
    A $(k,\ell)$ tensor over a vector space $V$ is a multilinear map
    \begin{equation*}
        T: \underbrace{V^* \times \cdots \times V^*}_{k} \times \underbrace{V\times \cdots \times V}_{\ell} \to \mathbb{R}
    \end{equation*}
\end{definition}
Some examples:
\begin{itemize}
    \item A $(0,1)$-tensor is a dual vector
    \item A $(1,0)$ tensor is an element of $V^{**}.$
\end{itemize}
An interesting example is a $(1,1)$-tensor, which is a map $V^*\times V\to \mathbb{R}.$ However, we can fix $v\in V$ so $V(\cdot, v)$ is in $V^{**}.$ But since $V^{**}$ is canonically isomorphic to $V,$ we have a linear map from $V$ to $V.$ Similarly, we can also view $T$ as a map from $V^*\to V^*.$

Let $\mathcal{T}(k,\ell)$ be the space of all $(k,\ell)$-tensors. There are two important operations on tensors:
\begin{enumerate}
    \item \emf{Contraction:} This is a map $C:\mathcal{T}(k,\ell) \to \mathcal{T}(k-1,\ell-1),$ defined by
    \begin{equation*}
        CT = \sum_{\sigma=1}^{n} T(\dots,v^{\sigma^*},\dots; \dots , v_{\sigma},\dots).
    \end{equation*}
    \item \emf{Outer product:} Given a $(k,\ell)$-tensor and a $(k',\ell')$-tensor, the outer product is defined by 
    \begin{equation*}
        (T \otimes T')(v^{1^*},\dots,v^{(k+k')^*};v_1,\dots,v_{\ell+\ell'}) = T(v^{1^*},\dots,v^{k^*};v_1,\dots,v_{\ell'})T'(v^{(k+1)^*},\dots,v^{(k+k')^*};v_{\ell'+1},\dots,v_{\ell+\ell'}).
    \end{equation*}
\end{enumerate}
One way to construct tensors is to take the outer product of smaller tensors, i.e. vectors and dual vectors. If this is possible, then the tensor is \emf{simple.} Let $\{v_\mu\}$ be the basis for $V$ and $\{v^{\nu^*}\}$ its dual basis. Then,
\begin{equation*}
    \{v_{\mu_1}\otimes \cdots \otimes v_{\mu_k} \otimes v^{\nu_1^*} \otimes \cdots \otimes v^{\nu_\ell^*}\}
\end{equation*}
forms a basis for $\mathcal{T}(k,\ell).$ Then every $(k,\ell)$ tensor can be written as a linear combination,
\begin{equation*}
    T = \sum_{\mu_1,\dots,\nu_\ell = 1}^n T^{\mu_1\cdots\mu_k}{}_{\nu_1\cdots\nu_\ell} v_{\mu_1} \otimes \cdots \otimes v^{\nu_\ell^*},
\end{equation*}
where the coefficients $T^{\mu_1\cdots\mu_k}{}_{\nu_1\cdots\nu_\ell}$ are known as the \emf{components} of $T.$ Note that it is often more convenient to work with just the components. For example, when applying the contraction and outer product:
\begin{itemize}
    \item Contraction: We have
    \begin{equation*}
        (CT)^{\mu_1\cdots \mu_{k-1}}{}_{\nu_1\cdots \nu_{\ell-1}} = \sum_{\sigma=1}^n T^{\mu_1\cdots \sigma \cdots \mu_{k-1}}{}_{\nu_1\cdots \sigma \cdots \nu_{\ell-1}}.
    \end{equation*}
    \item Outer product: We have
    \begin{equation*}
        (T\otimes T')^{\mu_1\cdots \mu_{k+k'}}{}_{\nu_1 \cdots \nu_{\ell+\ell'}} = T^{\mu_1\cdots \mu_k}{}_{\nu_1\cdots \nu_\ell}T'^{\mu_{k+1}\cdots \mu_{k+k'}}{}_{\nu_{\ell+1}\cdots \nu_{\ell+\ell'}}.
    \end{equation*}
\end{itemize}
\begin{theorem}
    The \emf{tensor transformation law} says that 
    \begin{equation*}
        T'^{\mu_1'\cdots \mu_k'}{}_{\nu_1'\cdots \nu_\ell'} = \sum_{\mu_1,\cdots \nu_\ell =1}^{n} T^{\mu_1 \cdots \mu_k}{}_{\nu_1 \cdots \nu_\ell}\frac{\partial x'^{\mu_1'}}{\partial x^{\mu_1}}\cdots\frac{\partial x^{\nu_\ell}}{\partial x'^{\nu_\ell'}}.
    \end{equation*}
\end{theorem}
\begin{definition}
    A metric is a $(0,2)$-tensor that is also:
    \begin{itemize}
        \item Symmetric: $g_{\mu\nu} = g_{\nu\mu}$
        \item Nondegenerate: $g_{\mu\nu} = 0$ only if $\mu=\nu=0.$
    \end{itemize}
\end{definition}
We can write
\begin{equation*}
    g = \sum_{\mu,\nu} g_{\nu\mu} \dd{x}^\mu \otimes \dd{x}^\nu
\end{equation*}
\subsection{Abstract Index Notation}
\section{Curvature}
How do we compare vectors in a curved space? We can't simply add or subtract since they live in different tangent spaces. To do so, we need to introduce a derivative operator.
\begin{definition}
    A derivative operator $\nabla$ on a manifold $M$ is a map $\nabla: \mathcal{T}(k,\ell) \to \mathcal{T}(k,\ell+1),$ that satisfies the five properties:
    \begin{enumerate}
        \item Linearity: For $A,B\in \mathcal{T}(k,\ell)$ and $\alpha,\beta\in \mathbb{R},$
        \begin{equation*}
            \nabla_c(\alpha A^{a_1\cdots a_k}{}_{b_1\cdots b_\ell}+ \beta B^{a_1\cdots a_k}{}_{b_1\cdots b_\ell}) = \alpha \nabla_c(A^{a_1\cdots a_k}{}_{b_1\cdots b_\ell}) + \beta \nabla_c(B^{a_1\cdots a_k}{}_{b_1\cdots b_\ell}). 
        \end{equation*}
        \item Leibnitz Rule:
        \begin{equation*}
            \nabla_e\left[A^{a_1\cdots a_k}{}_{b_1\cdots b_\ell}B^{c_1\cdots c_{k'}}{}_{d_1\cdots d_{\ell'}}\right] = \left[\nabla_e A^{a_1\cdots a_k}{}_{b_1\cdots b_\ell}\right]B^{c_1\cdots c_{k'}}{}_{d_1\cdots d_{\ell'}} + A^{a_1\cdots a_k}{}_{b_1\cdots b_\ell}\left[\nabla_e B^{c_1\cdots c_{k'}}{}_{d_1\cdots d_{\ell'}}\right].
        \end{equation*}
        \item Commutativity with contraction:
        \begin{equation*}
            \nabla_d(A^{a_1\cdots c\cdots a_k}{}_{b_1\cdots c\cdots b_\ell}) = (\nabla_d A)^{a_1\cdots c\cdots a_k}{}_{b_1\cdots c\cdots b_\ell}
        \end{equation*}
        \item Consistent with tangent vectors. For all $f\in \mathcal{F}$ and $t^a\in V_p,$ we have:
        \begin{equation*}
            t(f) = t^a\nabla_a f
        \end{equation*}
        \item Torsion free:
        \begin{equation*}
            \nabla_a\nabla_b f = \nabla_b\nabla_a f
        \end{equation*}
    \end{enumerate}
\end{definition}
We need to show a few important facts about this derivative operator:
\begin{itemize}
    \item $\nabla$ exists: To do so, pick local coordinates $\left\{\frac{\partial}{\partial x^\mu}\right\}$ and $\{\dd{x}^\mu\}.$ Then trhe ordinary derivative $\partial_a$ defined by:
    \begin{equation*}
        \partial_a: T^{\mu_1\cdots \mu_k}{}_{\nu_1\cdots \nu_\ell} \mapsto \frac{\partial}{\partial x^\sigma}T^{\mu_1\cdots\mu_k}{}_{\nu_1\cdots\nu_\ell}
    \end{equation*}
    \item The derivative operator is almost unique. Given two operators $\tilde{\nabla}_a$ and $\nabla_a,$ their difference is characterized by the tensor field $C^c_{ab},$ which is sometimes denoted as the \emf{Christoffel symbol} $\Gamma^{b}_{ac}$ when $\tilde{\nabla}_a$ is the ordinary derivative oeprator. That is,
    \begin{equation*}
        \nabla_a t^b = \partial_a t^b + \Gamma^b_{ac}t^c
    \end{equation*}
\end{itemize}

    \begin{definition}
        A vector $v^a$ given at each point on the curve is said to be \emf{parallelly transported} as one moves along the curve if the equation
        \begin{equation*}
            t^a\nabla_av^b = 0
        \end{equation*}
        is satisfied along the curve. In general, a tensor of arbitrary rank is parallelly transported if 
        \begin{equation*}
            t^a\nabla_a T^{b_1\cdots b_k}{}_{c_1\cdots c_\ell} = 0.
        \end{equation*}
    \end{definition}
    Consider a vector and choose a coordinate system. Then the above simplifies to:
    \begin{equation*}
        t^a\partial_a v^b + t^a\Gamma^b_{ac}v^c = 0 \iff \frac{dv^\nu}{dt} + \sum_{\mu,\lambda} t^{\mu}\Gamma^{\nu}_{\mu\lambda}v^{\lambda} = 0.
    \end{equation*}
    A vector at a point $p$ on the curve uniquely defines a parallel transported vector everywhere else on the curve. The mathematical structure arising from such a curve dependent identification of the tangent spaces of different points is called a \emf{connection}.
    \begin{theorem}
        Let $g_{ab}$ be a metric. Then there exists a unique derivative operator $\nabla_a$ satisfying $\nabla_ag_{bc}=0.$
    \end{theorem}
    A direct corollary is that a metric $g_{ab}$ naturally determines a derivative operator $\nabla_a$. In particular, we have:
    \begin{equation*}
        \Gamma^c_{ab} = \frac{1}{2}g^{cd}\left(\partial_ag_{bd} + \partial_bg_{ad} - \partial_dg_{ab}\right),
    \end{equation*}
    and the coordinate basis components are
    \begin{equation*}
        \Gamma^{\rho}_{\mu\nu} = \frac{1}{2}\sum_\sigma g^{\rho\sigma}\left(\frac{\partial g_{v\sigma}}{\partial x^\mu} + \frac{\partial g_{\mu\sigma}}{\partial x^\nu} - \frac{\partial g_{\mu\nu}}{\partial x^\sigma}\right).
    \end{equation*}

\subsection*{Motivation to Curvature}
Suppose we are on a riemannian manifold, i.e. the metric is positive definite. Consider a curve $\tau \in C^1([0,1] \to M^n),$ then
\begin{equation*}
    L(\tau)^2 = \int g(\tau'(t), \tau'(t)) \dd{t},
\end{equation*} 
and define the distance between $x_0$ and $x_1$ as
\begin{equation*}
    d(x_0,x_1)^2 = \text{inf}\left\{L(\tau)^2,\right\}
\end{equation*}
where $\tau(0)=x_0$ and $\tau(1)=x_1.$ If $\tau$ attains this infinum, then $L(\tau + \epsilon \sigma) > L(\tau)$ for all $\epsilon > 0$ if $\sigma(0)=\sigma(1)=0.$ Then:
\begin{equation*}
    0 = \frac{d}{d\epsilon}L(\tau+\epsilon\sigma) \iff \dot{\tau} \text{ is parallel transported along }\tau,
\end{equation*}  
where the connection is the Levi-Civita derivative,
\begin{equation}
    \dot{\tau}^a(t)\nabla_a \dot{\tau}^b,
\end{equation}
which is known as the \emf{geodesic equation}. In coordinates, recall that $\nabla_AV^b = \partial_aV^b + \Gamma^b_{ac}V^c,$ so the geodesic equation becomes
\begin{equation}
    \frac{d\tau^\alpha}{dt}\partial_\alpha \frac{d\tau^B}{dt} + \frac{d\tau^\alpha}{dt}\Gamma^{b}_{\alpha\gamma}\frac{d\tau^\gamma}{dt} = 0,
\end{equation}
which is sometimes written as 
\begin{equation}
    \frac{d^2\tau^B}{dt^2} + \Gamma^\beta_{\alpha\gamma}(\tau(t)) \frac{d\tau^\alpha}{dt}\frac{d\tau^\gamma}{dt} = 0.
\end{equation}
If $\tau^\beta(0) = x^\beta$ and $\frac{d\tau^\beta}{dt}(0) = V^\beta,$ then the solution (locally) is
\begin{equation*}
    \tau(t) = \text{exp}_x(tV),
\end{equation*}
where the exponential map is $\text{exp}_x:T_xM \to M, 0 \to x,$ defined by:
\begin{equation*}
    (x,V) \mapsto (x,\text{exp}_xV).
\end{equation*}
We can think of the exponential function as $\exp_{x_0}tv_0$ tells us to go a distance $t(v_0)_g$ in direction $\vec{v}_0.$ The exponential map is smooth, and locally and smoothly defined. If we identify $(T_{x_0}M,g_{x_0})\approx (\mathbb{R}^n,\delta_{ab}),$ i.e. with the euclidean space and metric, then we can call it the \emf{Riemannian normal coordinates at $x_0$,} which is a local coordinate chart at $x_0.$
Now consider two curves $\tau(t) = \exp_x(tW)$ and $\sigma(s) = \exp_xsV$ with $\sigma(0) = \tau(0) = x.$ Therefore,
\begin{equation*}
    d^2(\sigma(s),\tau(0)) = s^2,\quad\quad\quad\quad d^2(\sigma(0),\tau(t)) = t^2,
\end{equation*}
which is true per the geodesic equation. If we taylor expand $d^2(\tau(s),\tau(t))$ around $(s,t)=(0,0),$ then the cross terms are zero, so 
\begin{align*}
    d^2(\sigma(s),\tau(t)) &= |sV-tW|^2 + O(|(s,t
    )|^3) \\ 
    &= |sV-tW|^2 - \frac{s^2t^2}{3}R(v,w,v,w) + O(|(s,t)|^5).
\end{align*}
Here, $R$ is the Riemannian curvature tensor. Therefore, curvature is just a way to describe higher order terms when computing the distance.
\subsection*{Formal Definition of Curvature}
Now we extend to a more usual definition of curvature, which also extends to vectors that cannot be parallelly transported. Consider $f\in \mathcal{F}$ and $\omega \in \mathcal{T}(0,1)$ such that $\nabla_{[a}\nabla_{b]}f=0.$ Then,
\begin{align*}
    \nabla_a\nabla_b(f\omega_c) &= \nabla_a(\omega_c\nabla_b f + f\nabla_b \omega_c).
\end{align*}
The commutator relationship is then:
\begin{equation*}
    \nabla_{[a}\nabla_{b]}(f\omega_c) = f\nabla_{[a}\nabla_{b]}\omega_c.
\end{equation*}
The fact that $f$ pulls through implies that this function can only depend on the value of $\omega_c$ at $p$ and not at any nearby points. The most general linear thing that satisfies this is some arbitrary tensor 
\begin{equation*}
    (\nabla_a\nabla_b - \nabla_b\nabla_a)\omega_c = R_{abc}^d\omega_d,
\end{equation*}
where $R_{abc}^d$ is the Riemann curvature tensor. We can write something similar for $(\nabla_a\nabla_b - \nabla_b\nabla_a)V^c.$ Here, $V^c \in \mathcal{T}(1,0)$ and $\omega_c \in \mathcal{T}(0,1).$ Therefore, we can contract the two together to get,
\begin{align*}
    0 &= (\nabla_a\nabla_b - \nabla_b\nabla_a)(V^c\omega_c) \\ 
    &= V^c R_{abc}^d\omega_d + \omega_d \nabla_{[a}\nabla_{b]}V^d.
\end{align*}
This gives us 
\begin{equation*}
    \nabla_{[a}\nabla_{b]}V^d = -R_{abc}^dV^c.
\end{equation*}
By induction, we can show that if $T\in \mathcal{T}(k,\ell),$ then
\begin{equation*}
    \nabla_{[a}\nabla_{d]} T^{b_1\cdots b_k}{}_{c_1\cdots c_\ell} = -\sum_{i=1}^k R_{ade}{}^{b_i}T^{b_1\cdots e\cdots b_k}{}_{c_1\cdots c_\ell} + \sum_{j=1}^k R_{adc_j}{}^{e}T^{b_1\cdots b_k}{}_{c_1\cdots e\cdots c_\ell}.
\end{equation*}
The standard intuition behind this formulation is the failure to conserve a vector when taken under parallel transport around a closed curve.

First recall that if $U^a,B^v \in \mathcal{T}(1,0)$ and we want a vector field $W$ such that $[U,V]^c = W^c,$ then 
\begin{align*}
    W(f) &=  U(V(f)) - V(U(f)) \\ 
    &= U^b \nabla_b(V^a\nabla_a f) - V^a\nabla_a(U^b\nabla_b f) \\ 
    &= (U^a\nabla_a V^c - V^a \nabla_a U^c) \nabla_c f
\end{align*}
Consider a surface $S.$ Let's attempt to parallel transport $V^a$ around a parallelogram $(0,0) \to (\Delta s, 0) \to (\Delta s, \Delta t) \to (0,\delta t).$ Let us fix $\omega_b \in \mathcal{T}(0,1)$ and let us see how $V^a\omega_a$ changes.

Define $S^a = \left(\frac{\partial}{\partial s}\right)^a$ and $T^b = \left(\frac{\partial}{\partial t}\right)^b$ be coordinate tangent vectors to $S.$ Note that $[S,T]=0.$ Then,
\begin{align*}
    \delta_1 &= (\Delta s) \frac{d}{ds}(v^a\omega_a)\bigg|_{(\Delta s/2, 0)} + O((\Delta s)^3) \\ 
    &= (\Delta s) S^b\nabla_b (v^a\omega_a)\bigg|_{(\Delta s/2, 0)}  \\ 
    &= (\Delta s)S^b v^a \Delta b \omega_a\bigg|_{(\Delta s/2, 0)} 
\end{align*}
Similarly,
\begin{equation*}
    \delta_3 = -(\Delta s)S^b v^a\nabla_b\omega_a\bigg|_{(\Delta s/2, \Delta t)}.
\end{equation*}
However, the $v^a$ in this last expression is the vector we get after being transported halfway across the parallelogram. But first, note that
\begin{equation*}
    \delta_1+\delta_2+\delta_3+\delta_4 = O((\Delta s)^2 + (\Delta t)^2),
\end{equation*}
so since the difference between the vectors at different points vary by second order, we can effectively ignore them. Therefore, 
\begin{equation*}
    \delta_1+\delta_2+\delta_3+\delta_4 = v_\text{new} - v_0 = -S^bT^cR_{bca}^dv_0%a.
\end{equation*}
The Riemann tensor has some symmetries,
\begin{enumerate}
    \item $R_{abc}{}^d=-R_{bac}{}^d$
    \item $R_{[abc]}{}^d=0.$
    \item If $\nabla_a g_{bc} = 0,$ then $R_{abcd}=-R_{abdc}.$
    \item $\nabla_{[e}R_{ab]c}{}^d = 0$ is the \emf{Bianchi} identity.
\end{enumerate}
We can prove these,
\begin{enumerate}
    \item $R_{abc}{}^d\omega_d = \nabla_{[a}\nabla_{b]}\omega_c$
    \item Consider some form $T_{[abc]}.$ Then
    \begin{equation*}
        \nabla_{[a}T_{bcd]}
    \end{equation*}
    is a form. But taking the derivative again 
    \begin{equation*}
        \nabla_{[a}\nabla_bT_{cde]} = 0.
    \end{equation*}
    To see this, we can rewrite 
    \begin{equation*}
        \nabla_{[a} T_{bcd]} = \partial_a T_{bcd} + \Gamma_{[ab]}^{e}T_{ecd} + \cdots.
    \end{equation*}
    Since the $\Gamma$ are symmetric, then the $\Gamma$ terms disappear, and we get $\partial_a T_{bcd}.$ Therefore,
    \begin{equation*}
        \nabla_{[a}\nabla_bT_{cde]} = \partial_{[a}\partial_bT_{cde]} = 0.
    \end{equation*}
    \item 
\end{enumerate}
\subsection{Geodesics}
Let $\Sigma^{n+1} \subseteq M^n$ be an $(n-1)$-dimensional submanifold of $M$, i.e. a hypersurface. Then,
\begin{equation*}
    \mathbb{R}^{n-1} \approx T_p\Sigma \subseteq T_pM \approx \mathbb{R}^n.
\end{equation*} 
There are three cases for $g$
\begin{equation*}
    g\bigg|_{T_p\Sigma \times T_p\Sigma} > 0 \iff \Sigma\text{ is spacelike at } p,
\end{equation*}
which is alwys true for Riemannian metrics. For Lorentizn metrics,
\begin{equation*}
    \det g|_{(T_p\Sigma)^2} = 0 \iff \Sigma \text{ is null at } p
\end{equation*}
\begin{equation*}
    g|_{(T_p\Sigma)^2} < 0 \iff \Sigma \text{ is timelike at } p.
\end{equation*}
We claim that there exists a nonzero normal vector $N\in T_pM$ such that $g(N,X) = 0$ for all $X\in T_p\Sigma.$ Note that $N$ won't be null, i.e. $N\notin T_p\Sigma$ (except possibly in case 2). Therefore WLOG,
\begin{equation*}
    g(N,N) = \pm 1
\end{equation*}
except of course in case 2.

\textbf{Gaussian Normal Coordinates near $\Sigma$}
Let $\underline{\overline{X}} = (x^1,\dots,x^{n-1})$ be any coordinates on a neighbourhood of $p$ in $\Sigma.$ Then,
\begin{equation*}
    (x^1,\dots,x^{n-1},t) \in \mathbb{R} \mapsto \exp_{\underline{\overline{X}}= (x^1,\dots,x^{n-1})}tN \in M
\end{equation*}
gives coordinates in a neighborhood of $\Sigma.$ Notice that 
\begin{equation*}
    N^aN_a = \pm 1
\end{equation*}
and for all $X^b\in T_q\Sigma_0,$ we have
\begin{equation*}
    g(N,X) = N_aX^a = 0.
\end{equation*}
\textbf{Claim:} The geodesic $t\in \mathbb{R} \to \exp_{\underline{\overline{X}}}tN$ remains orthogonal to $\Sigma_t$ for all small $t.$
\begin{proof}
    The tangent vectors 
    \begin{equation*}
        X_i^a = \left(\frac{\partial}{\partial x^i}\right)^a
    \end{equation*}
    for $i\in \{1,\dots,n-1\}$ form a coordinate basis for $T_p\Sigma_t,$ and $N^b = \left(\frac{\partial}{\partial t}\right)^b$ denotes the tangent to the geodesic. Recall,
    \begin{equation*}
        \left[N, X_{(i)}\right]^a = 0.
    \end{equation*}
    This is equivalent to 
    \begin{equation*}
        T^b\nabla_bX^a = X^b\nabla_bT^a,
    \end{equation*}
    for $X\in \{X_1,\dots,x_{n-1}\}.$ We then claim that
    \begin{equation*}
        N^aX_a = 0,
    \end{equation*}
    for all $|t| \ll 1.$ Its derivative along the geodesic is 
    \begin{align*}
        N^b\nabla_b(N^aX_a) &= \cancel{(N^b\nabla_bN^a)}X_a + N^aN^b\nabla_b X_a \\ 
        &= N_aX^b\nabla_bN^a \\ 
        &= \frac{1}{2}X^b\nabla_B(N_aN^a) \\ 
        &= \frac{1}{2}X^b\nabla_B(\pm 1) \\ 
        &= 0.
    \end{align*}
\end{proof}
\subsection{Geodesic Deviation Equation and Jacobi Fields}
Let $t\in I \subseteq \mathbb{R} \to \sigma(t) \in M$ be a geodesic. Now consider a surface
\begin{equation*}
    (s,t) \in B_1(0) \subseteq \mathbb{R}^2 \to \sigma_{s}(t) = \sigma(t;s) \in M
\end{equation*}
formed by geodesics $t\in I_s \subseteq \mathbb{R} \to \sigma_s(t),$ i.e. 
\begin{equation*}
    \dot{\sigma}_s^a \nabla_a \dot{\sigma}_s = 0.
\end{equation*}
If we have a geodesic, we can re-parametrize it. That is, $\sigma(bt+c)$ is also a geodesic. If $\sigma$ depends on $s$, then $\sigma_s(b(s)t+c(s))$ is also a geodesic. We have some freedom to choose $b(s)$ and $c(s).$ Let us choose $b(s)$ such that 
\begin{equation*}
    \dot{\sigma}^a\dot{\sigma}_a = \pm 1.
\end{equation*}
We can also choose $c'(0)$ so that 
\begin{equation*}
    \dot{\sigma}^aX_a\bigg|_{\sigma_0(0)} = 0.
\end{equation*}
Finally, we can choose $X^a = \left(\frac{\partial}{\partial s}\right)^a$ and $c(s)$ such that 
\begin{equation*}
    \dot{\sigma}^aX_a\bigg|_{\sigma_s(0)} = 0.
\end{equation*}
As before, $\dot{\sigma}^b\nabla_b(\dot{\sigma}^aX_a) = 0.$ Therefore, $X^a$ remains orthogonal to $\dot{\sigma}$ along $\sigma_0(t)$ for all $t,$ which is a result we've seen before.

We can linearize the geodesic equation around $\sigma_0(t)$ to get a linear 2nd order equation for $X^a.$ That is, we can compute the velocity
\begin{align*}
    v^a &= \dot{\sigma}^b\nabla_bX^a \\ 
    a^a &= \dot{\sigma}^c\nabla_c v^a
\end{align*}
of a nearby geodesic relative to $\sigma_0(t).$ We can rewrite,
\begin{equation*}
    a^a = \dot{\sigma}^c\nabla_c\left(\dot{\sigma}^b\nabla_bX^a\right),
\end{equation*}
which includes two derivatives. Recall that the order of derivatives matter and if we want to switch the order, we need to include curvature. Note that since $[\dot{\sigma},X] = 0,$ we can rewrite,
\begin{align*}
    a^a &= \dot{\sigma}^c\nabla_c\left(X^b\nabla_b\dot{\sigma}^a\right) \\ 
    &= \dot{\sigma}^c(\nabla_cX^b)\nabla_b\dot{\sigma}^b + X^b\dot{\sigma}^c\nabla_c\nabla_b \dot{\sigma}^a \\ 
    &= (X^c\nabla_c\dot{\sigma}^b)\nabla_b\dot{\sigma}^a + X^b\dot{\sigma}^c\nabla_b\nabla_c\dot{\sigma}^a - R_{cbd}^a\dot{\sigma}^dX^b\dot{\sigma}^c \\ 
    &= -R_{cbd}^a\dot{\sigma}^dX^b\dot{\sigma}^c.
\end{align*}
Note that $a$ is linear in $X,$ so this is the linear second order equation we wanted. In coordinates, this becomes,
\begin{equation*}
    \frac{d^2}{dt^2}X^\alpha + R_{\gamma\beta\delta}^{\alpha}\dot{\sigma}^\gamma \dot{\sigma}^{\delta}X^\beta = 0.
\end{equation*}
The initial conditions are $X^\alpha(0) = X_0^\alpha$ and $\dot{X}^\alpha(0) = V_0^\alpha.$ There are $n$ choices for both, so $2n$ degrees of freedom. Note that $2$ of them correspond to the affine reparametrization of $\sigma_0(t).$ 

Solutions $X^\alpha(t)$ along $\sigma_c(t)$ are called \emf{Jacobi Fields.}
\subsection{Computing the Riemann Tensor}
How do we compute $R_{abc}^d$? There are different methods to do so, but we begin with the coordinate method. Given a tensor field $\omega_d \in \mathcal{T}(0,1),$ then 
\begin{align*}
    \frac{1}{2}R_{abc}^d \omega_d &= \frac{1}{2}(\nabla_a\nabla_b - \nabla_b\nabla_a)\omega_c \\ 
    &= \nabla_{[a}\nabla_{b]}\omega_c.
\end{align*}
In local coordinates,
\begin{equation*}
    \nabla_a\omega_b = \partial_a\omega_b - \Gamma^c{}_{ab}\omega_c.,
\end{equation*}
and we can write it in a more useful form,
\begin{equation*}
    D_b\omega_c = \partial_b\omega_c - \Gamma^d{}_{bc}\omega_d.
\end{equation*}
We have,
\begin{align*}
    & \nabla_a\nabla_b \omega_c = \partial_a(\partial_b\omega_c - \Gamma^d{}_{bc}\omega_d) - \Gamma^e{}_{ab}\left(\partial_e\omega_c - \Gamma^d{}_{ec}\omega_d\right) - \Gamma^e{}_{ac}\left(\partial_b\omega_\ell - \Gamma^d_{be}\omega_d\right)\\
    \implies & \nabla_{[a}\nabla_{b]}\omega_c = -\partial_{[a}\Gamma^d{}_{b]c}\omega_d + \Gamma^e{}_{c[a}\Gamma^d{}_{b]e}\omega_d.
\end{align*}
Therefore,
\begin{equation*}
    R_{\alpha\beta\gamma}{}^{\delta} = -\frac{\partial}{\partial x^\alpha} \Gamma^{\delta}{}_{\beta\gamma} + \frac{\partial}{\partial x^\beta}\Gamma^{\delta}{}_{\alpha\gamma} + \Gamma^{\epsilon}{}_{\gamma\alpha}\Gamma^{\delta}{}_{\epsilon} - \Gamma^{\epsilon}{}_{\gamma\beta}\Gamma^{\delta}{}_{\alpha\epsilon}.
\end{equation*}
There are different types of curvature tensors. We have the \emf{Ricci curvature,}
\begin{equation*}
    R_{ac} = R_{abc}^b
\end{equation*}
and the scalar curvature $R=R_a^a = R_{ac}g^{ca}.$
\subsection{Twice Contracted Identity}
Starting from the Bianchi identity, we can expand it:
\begin{equation*}
    0 = 2\nabla_a R_{bcd}{}^{e} + 2\nabla_b R_{cad}^{e} + 2\nabla_c R_{abd}^e.
\end{equation*}
Contracting on $c,e$ gives 
\begin{equation*}
    \nabla_a R_{bd} - \nabla_b R_{ad} + \nabla_e R_{abd}^e,
\end{equation*}
and contracting it by $ad$ (i.e. multiplying by $g^{ad}$) gives 
\begin{align*}
    0 &= \nabla^a R_{ba} - \nabla_b R_a{}^a + \nabla^e R_{be} \\ 
    &= 2\left(\nabla^a R_{ba} - \frac{1}{2}\nabla_b R\right) \\ 
    &= 2\nabla^a \left(R_{ab} - \frac{1}{2}Rg_{ab}\right),
\end{align*}
where $G_{ab} \equiv R_{ab}-\frac{1}{2}Rg_{ab}$ is often known as the \emf{Einstein tensor}, and the twice contracted identity tells us that it is divergence free, which gives us a conservation law.

Recall that we can write,
\begin{align*}
    0 &= \nabla_a v^a \\ 
    &= \frac{\partial v^a}{\partial x^a} + \Gamma^a{}_{ab}v^b.
\end{align*}
We can write out an explicit formula for the contracted Christoffel symbols:
\begin{align*}
    \Gamma^a{}_{ab} &= \frac{1}{2}g^{ad} \left(\frac{\partial g_{db}}{\partial x^a} + \frac{\partial g_{ad}}{\partial x^b} - \frac{\partial g_{ab}}{\partial x^d}\right) \\ 
    &= \frac{1}{2}\left(\frac{\partial g_{bd}}{\partial x^d} + g^{ad} \frac{\partial g_{ad}}{\partial x^b} - \frac{\partial g_{ab}}{\partial x^a}\right) \\ 
    &= \frac{1}{2}g^{ac}\frac{\partial}{\partial x^b}g_{ac}.
\end{align*}
Alternatively,
\begin{align*}
    \Gamma^{a}{}_{ab} &= \frac{1}{2}\frac{\partial}{\partial x^b}\log |g| \\ 
    &= \frac{1}{2}\frac{\partial}{\partial x^b} \sum_{i=1}^{n} \log |\lambda_i| \\ 
    &= \frac{1}{2} \frac{1}{\lambda_i}\frac{\partial \lambda^i}{\partial x^b} \\ 
    &= \frac{1}{2} g^{ac} \frac{\partial}{\partial x^b}g_{ac}
\end{align*}
where $|g| = |\det g_{ij}|.$
\subsection{Differential Forms}
Let $\nabla_a$ be a connection on $M.$ We can define the derivative operator
\begin{equation*}
    d: \Lambda^p(M) \to \Lambda^{p+1}(M)
\end{equation*}
by 
\begin{equation*}
    \omega_{a_1\dots a_p} \mapsto \nabla_{[b}\omega_{a_1\dots a_p]}
\end{equation*}
which we can expand to 
\begin{equation*}
    \nabla_{[b}\omega_{a_1\dots a_p]} = \frac{\partial}{\partial x^i} \omega_{a_1 \dots a_p} + \sum_{i=1}^p C^c_{[ba_i}\omega_{a_1\dots |c|\dots a_p}.
\end{equation*}
But this last term is zero, so $d$ only depends on the differential topology of the manifold, and not its differential geometry. It is independent of the connection choice! Therefore, we often denote the derivative of $\omega$ as $d\omega.$

Note that $\omega \in \Lambda^p(M)$ is \emf{closed} when $\dd{\omega}=0$ and $\omega$ is \emf{exact} when it can be written as $\omega = \dd{\eta}.$

The manifold $M$ is \emf{orientable} if and only if there exists an $\epsilon \in \Lambda^n(M)$ that is continuous non-vanishing. If $\alpha \in \Lambda^n(M)$ and $M$ is oriented (by $\epsilon$) then 
\begin{equation*}
    \int_M \alpha
\end{equation*}
is defined locally in charts and globally by partitions of unity, and is independent of coordinates on the chart.

However, things become spicy once we bring a metric into play. A pseudo-Riemannian metric $g$ on $M^n$ selects a preferred volume form $\epsilon$ (up to a sign (orientable)). This is to ensure 
\begin{equation*}
    \epsilon^{a_1\dots a_n}\epsilon_{a_1\dots a_n} = (-1)^s n!
\end{equation*}
Many nice things follow from this. In right-handed coordinates on $U \subseteq M,$ we have 
\begin{equation*}
    \epsilon \to \sqrt{|g|} \dd{x}^1 \wedge \cdots \dd{x}^n,
\end{equation*}
where $|g| = |\det g_{\alpha\beta}|$ in some coordinates. Let:
\begin{equation*}
    g_{\mu\nu} = g\left(\frac{\partial}{\partial x^\mu}, \frac{\partial}{\partial x^\nu}\right),
\end{equation*}
and consider the change of basis 
\begin{equation*}
    \frac{\partial}{\partial x^{\bar{\mu}}} = \frac{\partial x^\mu}{\partial x^{\bar{\mu}}} \frac{\partial}{\partial x^\mu} = \Lambda^{\mu}_{\bar{\mu}} \frac{\partial}{\partial x^\mu}.
\end{equation*}
Then,
\begin{align*}
    g_{\bar{\mu}\bar{\nu}} &= \frac{\partial x^{\mu}}{\partial x^{\bar{\mu}}}\frac{\partial x^{\nu}}{\partial x^{\bar{\nu}}}g\left(\frac{\partial}{\partial x^\mu}, \frac{\partial}{\partial x^\nu}\right) \\ 
    &= \Lambda^{\mu}_{\bar{\mu}} \Lambda^{\nu}_{\bar{\nu}} g_{\mu\nu},
\end{align*}
where the $\Lambda$ are Jacobians. Then,
\begin{equation*}
    |\bar{g}|^{1/2} = |\det \Lambda | |g|^{1/2},
\end{equation*}
is how the determinant of the metric changes wrt coordinates.


allows for the reduction to the Standard Stoke's Theorem via integration by parts.
\begin{lemma}
    If the derivative $\nabla_a$ and volume form $\epsilon$ are compatible with $g_{ba}$ then $U$ is a subdomain $U\subset\subset M^n$ implies
    \begin{equation*}
        \int_{U} f(\nabla_a V^a)\epsilon = \int_{\partial U} (fV^a n_a) \dd{A} - \int_U (\nabla_a f)V_a^a \epsilon,
    \end{equation*}
    where $n_a$ is an outer normal form\footnote{For example if $U=\{x\in M, \phi(x) < 0\}$ with $|d\phi| = 1$, then we can identify $n=\dd{\phi}$}. If $f\in C^1(U)$ and $V \in \mathcal{T}(1,0)$ is $c^1.$
\end{lemma}
\begin{proof}
    Assume a coordinate chart $\psi$ covers $U$. Then, we can integrate,
\begin{align*}
    -\int_{U} f(\nabla_a V^a)\epsilon 
    &=  -\int_{\psi(U)} f\left(\frac{\partial V^\alpha}{\partial x^\alpha} + \Gamma^{\alpha}_{\alpha\beta}V^\beta\right)\sqrt{g} \dd{x}^1\wedge \cdots \wedge \dd{x}^n\\ 
    &= -\int_{\psi(U)} f\left(\frac{\partial V^\alpha}{\partial x^\alpha}\sqrt{g} + \frac{V^\beta}{|g|^{1/2}} \frac{\partial |g|^{1/2}}{\partial x^\beta}\right) \dd{x}^1\wedge \cdots \wedge \dd{x}^n \\ 
    &= -\int_{\psi(U)} f\frac{\partial V^\alpha}{\partial x^\alpha}|g|^{1/2} + fV^\beta \frac{\partial |g|^{1/2}}{\partial x^\beta} \dd{x}^1\wedge \cdots \wedge \dd{x}^n, \\ 
    &= +\int_{\psi(U)} V^\alpha \frac{\partial}{\partial x^\alpha}(f\sqrt{g}) + \frac{\partial(fV^\alpha)}{\partial x^\alpha}\sqrt{g} \dd^n{x} - 2\int_{\partial \psi(U)} fV^\alpha \sqrt{g} n_\alpha \dd^{n-1}{x} \\ 
    &= \int_{\psi(U)} \frac{\partial}{\partial x^\alpha}(V^\alpha f \sqrt{g}) + (\partial_\alpha f)V^\alpha \sqrt{g} \dd^n{x} - 2\int_{\partial \psi(U)} fV^\alpha \sqrt{g} n_\alpha \dd^{n-1}x.
\end{align*}
Which gives 
\begin{align*}
    -\int_{\partial \psi(U)} (f\sqrt{g} + V^\alpha n_\alpha) \dd^{n-1}x + \int_{\psi(U)} (\partial_\alpha f)V^{\alpha}\sqrt{g} \dd{x}^n,
\end{align*}
which is the same thing as what we want in our lemma.

Note that the reason there is an extra $\sqrt{g}$ term is because of the \emf{minkowski capacity},
\begin{equation*}
    \text{Area}(\Sigma) = \lim_{\epsilon \to 0} \frac{vol(U_{\epsilon}-U_0)}{\epsilon},
\end{equation*}
and the volume will have a $\sqrt{g}$ factor, so we need to add this to the area when we work with coordinates.
\end{proof}
\section{Stress Energy Tensor}
In Euclidean geometry $\mathbb{R}^3,$ we have $h_{\mu \nu} = \text{diag}(1,1,1),$ which implies that $\Gamma^{\mu}_{\alpha\beta} = 0,$ so the covariant derivative agrees $\nabla_\mu = \partial_\mu,$ geodesics are straight lines, and parallel transport is curve independent. In $\mathbb{R}^4$ in special relativity, we have 
\begin{equation*}
    \eta_{\mu\nu} = \text{diag}(-1,1,1,1),
\end{equation*}
and the Christoffel symbols vanish, etc. Although there is no inertial frame, there is still a future direction and orientation. Consider the $(M^n,g^{\mu\nu})$ Lorentzian manifold:
\begin{itemize}
    \item At each point, we have a lightcone, and we can decide the future direction in a consistent way (orientable).
    \item Massive particles, absent other forces, follow timelike geodesics, parametrized such that 
    \begin{equation*}
        g(\sigma'(s), \sigma'(s)) = -1
    \end{equation*}
    \item The 4-velocity is 
    \begin{equation*}
        U^a = \frac{d\sigma^a}{ds}
    \end{equation*}
    and the 4-momentum is 
    \begin{equation*}
        P^a = mU^a.
    \end{equation*}
    \item The energy of our particle, measured by a non-comoving observer at the same point in spacetime $M,$ is 
    \begin{equation*}
        \tilde{U}^b p_b = g_{ab}(\tilde{U}^b, p^a),
    \end{equation*}
    where $\tilde{U}^b$ is the 4-velocity of observer.
    \item $\tilde{U}$ and $U$ makes an angle of $\theta$ with each other. Note that 
    \begin{equation*}
        \tanh\psi = v = \tan\theta,
    \end{equation*}
    so we can write the energy as 
    \begin{equation*}
        \tilde{U}^b p_b = m\cosh\psi = \frac{m}{\sqrt{1-|v|^2}},
    \end{equation*}
    where $|v|$ is the relative 3-velocity.
    \item If $g_{ab}\tilde{X}^a\tilde{U}^b=0,$ then the momentum in the $\tilde{X}^a$ direction is $g_{ab}\tilde{X}^ap^b = \frac{mv}{\sqrt{1-|v|^2}}.$
\end{itemize}
Something else that is affected by boosts is number density. Consider $N$ particles at rest and draw some box with volume $r^3$ in the $x,y,z$ direction. The density is 
\begin{equation*}
    n = \frac{N}{r^3}.
\end{equation*}
In a moving reference frame, the Lorentz boost, the particles are now moving at some relative velocity $(v,0,0)$ in the $x$-direction. The density now becomes 
\begin{equation*}
    \bar{n} = \frac{n}{\gamma r^3},
\end{equation*}
due to length contraction, so density transforms like the first component of a 4-vector. We can treat this as a 4-vector, where the other 3 directions are the fluxes. The \emf{number flux vector} is
\begin{align*}
    N^a &\xrightarrow[]{\text{comoving}} (n,0,0,0) \\ 
    &\xrightarrow[]{\text{x-boosted}} (n/\sqrt{1-|v|^2},\frac{nv}{\sqrt{1-|v|^2}},0,0) \\ 
    &\xrightarrow[]{\text{boosted}} \left(n/\sqrt{1-|v|^2},\frac{nv^1}{\sqrt{1-|v|^2}},\frac{nv^2}{\sqrt{!-|v|^2}},\frac{nv^3}{\sqrt{1-|v|^2}}\right).
\end{align*}
Note that the number of particles is observer independent, i.e. 
\begin{equation*}
    n^2 = g_{ab}(N^a,N^b),
\end{equation*}
where $n$ is the scalar rest density and is the first component of $N^a$ in comoving coordinates.

This motivates the question of what the energy density of these particles might look like. The energy density is the energy per particle times the density of the particles, i.e.
\begin{align*}
    &\xrightarrow[]{\text{comoving}} mn \\ 
    &\xrightarrow[]{\text{x-boosted}} \frac{mn}{1-|v|^2}.
\end{align*}
This transforms like a $(2,0)$ tensor $T^{ab} = P^a \otimes N^b$. This motivates us to write,
\begin{align*}
    T^{ab} &\xrightarrow[]{\text{comoving}} \begin{pmatrix}
        mn &0 & 0 & 0 \\ 
        0 & 0 & 0 & 0 \\
        0 & 0 & 0 & 0 \\
        0 & 0 & 0 & 0
    \end{pmatrix} \\ 
    &\xrightarrow[]{\text{x-boosted}} \begin{pmatrix}
        \frac{mn}{1-v^2} & \frac{mvn}{1-v^2} & 0 & 0 \\
        \frac{mvn}{1-v^2} & \frac{mnv^2}{1-v^2} & 0 & 0 \\ 
        0 & 0 & 0 & 0 \\
        0 & 0 & 0 & 0
    \end{pmatrix}
\end{align*}
which is the energy momentum tensor for \emf{dust.} More generally, $T^{ab}$ is the \emf{stress-energy tensor} of matter, with the following properties 
\begin{itemize}
    \item The energy-density observed by observer having velocity $v$ is 
    \begin{equation*}
        T_{ab}v^av^b
    \end{equation*}
    \item If $g_{ab}(X^a,V^b)=0,$ Then the energy flux of matter in $X^b$ direction is 
    \begin{equation*}
        T_{ab}v^aX^b
    \end{equation*}
    and 
    \begin{equation*}
        T_{ab}X^av^b
    \end{equation*}
    is the $X^a$ momentum density, and 
    \begin{equation*}
        T_{ab}X^aY^b
    \end{equation*}
    is the  $X^a$ momentum flux in the $Y^b$ direction, where $g_{ab}(X^a,Y^b)=g_{ab}(V^a,Y^b)=0.$
\end{itemize}
\begin{lemma}
    The stress-energy tensor is symmetric, i.e. $T^{ab} = T^{ba}.$
\end{lemma}
\begin{proof}
    Note that $T^{a0}$ is the $X^a$ momentum density and $T^{0a}$ is the energy flux in the $x^a$ direction. These are the same quantity, so 
    \begin{equation*}
        T^{a0} = T^{0a}.
    \end{equation*}
    To show that $T^{ab}=T^{ba}$ for $a,b \neq 0,$ we can use a rotation argument. Consider small volume cuboid elements. The torque in one direction is created by contributions from four sides. For example, the $z$ torque is given by 
    \begin{equation*}
        \left(T^{xy}\bigg|_{y=-r}r^2 + T^{xy}\bigg|_{y=r}r^2\right)  - \left(T^{yx}\bigg|_{x=-r}r^2 + T^{yx}\bigg|_{x=r}r^2\right).
    \end{equation*}
    The $zz$ momentum of inertia scales like $r^5.$ The angular acceleration is a ratio, and to prevent diverging for small $r$, we evaluate at $y=0+r^3 \approx 0$ instead, and we get 
    \begin{equation*}
        T^{xy} = T^{yx}
    \end{equation*}
    in order to prevent infinite acceleration.
\end{proof}
\subsection{Perfect Fluids}
\emf{Perfect fluids} have no viscosity (no heat conduction), so there is no shear force. Therefore, $T^{\alpha\beta}=0$ for all $\alpha\neq \beta \in \{1,2,3\}$ spacelike indices. Since there is no heat conduction, heat (energy density) can only be transported in the direction of the fluid, i.e. 
\begin{equation*}
    T^{\alpha 0} =0 \forall \alpha \in \{1,2,3\}.
\end{equation*}
In a special relativity frame, we have
\begin{align*}
    T^{ab} &\xrightarrow[]{SR} \begin{pmatrix}
        T^{00} & 0 & 0 & 0 \\ 
        0 & T^{11} & 0 & 0 \\
        0 & 0 & T^{22} & 0 \\
        0 & 0 & 0 & T^{33}
    \end{pmatrix} \\ 
    &\xrightarrow[]{} \begin{pmatrix}
        \rho(x,y,z,t) & 0 & 0 & 0 \\ 
        0 & P(x,y,z,t) & 0 & 0 \\
        0 & 0 & P & 0 \\
        0 & 0 & 0 & P
    \end{pmatrix}.
\end{align*}
The stress energy tensor is also divergence free, which is the statement that energy and momentum is conserved locally. Note that in special relativity,
\begin{itemize}
    \item $T_{00}$ is the energy density
    \item $T_{0i}$ is the energy flux in the $i$ direction
    \item $T_{i0}$ is the $i$-momentum density
    \item $T_{ij}$ is the flux of momentum in the $j$ direction.
\end{itemize}
Consider an imaginary cube. The flux through the 6 faces of the cube is given by 
\begin{align*}
    \frac{\partial}{\partial t}\bigg|_{(0,0,0)} (T^{00}L^3) &= -\left(T^{01}\bigg|_{(L,0,0)} - T^{01}\bigg|_{(0,0,0)}\right)\frac{L^2}{L} - \left(T^{02}\bigg|_{(0,L,0)} - T^{02}\bigg|_{(0,0,0)}\right)\frac{L^2}{L} - \left(T^{03}\bigg|_{(0,0,L)} - T^{03}\bigg|_{(0,0,0)}\right)\frac{L^2}{L},
\end{align*}
which gives, after taking the limit $L\to 0,$
\begin{equation}
    \frac{\partial T^{00}}{\partial t} + \frac{\partial T^{01}}{\partial x} + \frac{\partial T^{02}}{\partial y} + \frac{\partial T^{03}}{\partial z} = 0,
\end{equation}
which is the equation for conservation of energy. Similarly, conservation of momentum can be written as 
\begin{equation}
    \frac{\partial T^{j0}}{\partial t} + \frac{\partial T^{j1}}{\partial x} + \frac{\partial T^{j2}}{\partial y} + \frac{\partial T^{j3}}{\partial z} = 0.
\end{equation}
More generally, we can write that 
\begin{equation*}
    \nabla_a T^{ab} = 0.
\end{equation*}
Note that we can write 
\begin{align*}
    T^{ab} &= (\rho + P)U^aU^b + Pg^{ab} \\ 
    T_{ab} &= (\rho+P)U_aU_b + Pg_{ab}.
\end{align*}
Taking the covariant derivative, we have 
\begin{align*}
    0 = \nabla_a T^{ab} &= U^b\nabla_a(\rho U^a) + \rho U^a \nabla_a U^b + P\left(U^b \nabla_a U^a + U^a\nabla_a U^b\right) + (g^{ab} + U^aU^b)\nabla_aP \\ 
    &= U^b\left(\nabla_a(\rho U^a) + P\nabla_a U^a\right) + (\rho+P)U^a \nabla_a U^b + (g^{ab}+U^aU^b)\nabla_a P.
\end{align*}
Contracting with $U^b$ (note: $U^bU_b = -1$) in order to get the motion in a particular direction, 
\begin{align*}
    0 = -1\left[\nabla_a(\rho U^a) + P\nabla_a U^a\right] + \cancel{\frac{1}{2}(\rho+P)U^a\nabla_a(U^bU_b)} + \cancel{(U^a-U^a)\nabla_a P},
\end{align*}
which gives us 
\begin{align*}
    0 &= \nabla_a(\rho U^a) + P\nabla_a U^a = U^a\nabla_a\rho + (\rho+P)\nabla_aU^a\\ 
    0 &= (\rho+P)U^a \nabla_a U^b + (g^{ab}+U^aU^b)\nabla_a P.
\end{align*}
How do we interpret this? In SR, $g_{ab}=\eta_{ab}$ and $\nabla_a = \partial_a,$ and $\rho \gg P$ for $|v| \ll 1.$ Then,
\begin{align*}
    0 &= \frac{\partial \rho}{\partial t} + (\vec{v}\cdot \vec{\nabla})\rho + \rho\left(\vec{\nabla} \cdot \vec{v}\right) = \frac{\partial \rho}{\partial t} + \vec{\nabla} \cdot \left(\rho \vec{v}\right),
\end{align*}
which gives us the \emf{continuity equation.} We can also recover Netwon's second law, for $b=1,2,3.$
\begin{equation*}
    \rho\left(\frac{\partial U^b}{\partial t} + (\vec{v}\cdot \vec{\nabla})U^b\right) = -\partial_b P.
\end{equation*}
Together, they form the 3D compressible Euler equations for fluids. We have five unknowns here, $\rho, U^1,U^2,U^3,P,$ and four equations. We can get a fifth equation by getting an equation of state (i.e. ideal gas law). 
\subsection{Klein-Gordon Wave Equation}
For $\phi: M\to \mathbb{R},$ the Klein-Gordon wave equation is 
\begin{equation*}
    \nabla^a\nabla_a \phi - m^2\phi = 0,
\end{equation*}
which can be written as 
\begin{equation*}
    \left(-\frac{\partial^2}{\partial t^2} + \Delta\right)\phi - m^2\phi = 0.
\end{equation*}
Sometimes, we write 
\begin{equation*}
    -\square = -\frac{\partial^2}{\partial t^2} + \Delta.
\end{equation*}
This comes from conservation of energy/momentum frmo the following stress energy tensor.
\begin{align*}
    T^{ab} = \nabla^a\phi \nabla^\phi - \frac{1}{2}g^{ab}\left(\nabla^c\phi \nabla_c\phi + m^2\phi^2\right)
\end{align*}
Contracting it with $\nabla_a$ gives 
\begin{align*}
    0 = \nabla_a T^{ab} &= (\nabla_a\nabla^a\phi)\nabla^b\phi + \nabla_a\phi(\nabla_a\nabla^a\phi) - \frac{1}{2}g^{ab}\left(\nabla_a\nabla^c \phi \nabla_c \phi + \nabla^c \phi \nabla_a\nabla_c \phi + 2m^2\phi \nabla_a\phi\right) \\ 
    0 &= (\nabla_a\nabla^a \phi - m^2\phi)\nabla^b\phi + \nabla^a\phi (\nabla_a\nabla^b\phi) - \frac{1}{2}\left(\nabla^b\nabla_a\phi\nabla^a\phi + \nabla^c\phi\nabla^b\nabla_c\phi\right).
\end{align*}
By the torsion free condition, we get the desired wave equation.
\subsection{Maxwell's Equations}
We can write the \emf{Faraday tensor} as 
\begin{equation*}
    F_{ab} = F_{[ab]} =
    \begin{pmatrix}
        0 & E_{1} & E_{2} & E_{3} \\
        -E_{1} & 0 & B_{3} & -B_{2} \\
        -E_{2} & -B_{3} & 0 & B_{1} \\
        -E_{3} & B_{2} & -B_{1} & 0
    \end{pmatrix}
\end{equation*}
Maxwell's equations are very simple in this notation. Namely,
\begin{align*}
    \nabla^a F_{ab} &= -4\pi J_b \\ 
    \nabla_{[a}F_{bc]} &= 0,
\end{align*}
where $J^b$ is the charge/current density four-vector. First, we can check that $J^b$ is divergence free. Note that 
\begin{align*}
    \nabla^b J_b &= -\frac{1}{4\pi} \nabla^b\nabla^a F_{ab} \\ 
    &-\frac{1}{4\pi}\partial_a\partial_b F^{ab} \\ 
    &= 0,
\end{align*}
which tells us that current is the flux of the charge density. The second relationship tells us that locally, we can write 
\begin{equation*}
    F_{ab} = \nabla_a A_b - \nabla_b A_a,
\end{equation*}
where $A$ is a four-vector, known as the \emf{vector potential.} Plugging this into the first relationship gives 
\begin{equation*}
    \nabla^a\nabla_a A_b - \nabla_b\nabla_a A^a - R_{abc}^a A^c = -4\pi J_b.
\end{equation*}
Notice that we can write $A_a = A_a' + \nabla_a \chi$ (choosing the Lorentz gauge). Then we can show, with some work, that 
\begin{equation*}
    \nabla_a A^a = \nabla_a A'^{a} + \nabla_a\nabla^a \chi.
\end{equation*}
We can choose $\nabla_a\nabla^a \chi = \nabla_a A^a$ to make $\nabla_a A'^a = 0.$ Therefore, our conservation law gives 
\begin{equation*}
    \nabla^a\nabla_a A_b - R_{bc} A^c = - 4\pi J_b.
\end{equation*}
How will charges move in these fields? The answer is that the acceleration is given by 
\begin{equation*}
    u^a \nabla_a u^b = \frac{q}{m} F^b_c u^c,
\end{equation*}
and corresponds to Newton's second law.
\subsection{Lorentz Gauge}
In special relativity, the Lorentz gauge seeks solutions 
\begin{equation*}
    A_b = C_b e^{iS(t,x,y,z)},
\end{equation*}
where $C^b$ is a constant, i.e. parallel transport. Plugging this into the wave equation in the absence of charges, i.e. $\nabla^a\nabla_a A_b=0.$ Then,
\begin{align*}
    \partial_a A_b &= A_b i\partial_a S \\ 
    \partial^a\partial_a A_b &= A_b\left(-\partial^a S\partial_a S + i\partial^a\partial_a S\right).
\end{align*}
For this to vanish, both the real and imaginary components need to vanish, i.e. 
\begin{equation*}
    \partial^aS \partial_a S = 0 = \partial^a \partial_a S.
\end{equation*}
Similarly, $\nabla_b A^b=0$ gives
\begin{equation*}
    C^b\nabla_b S = 0.
\end{equation*}
Let $k_a = \nabla_a S.$ Then $k_ak^a=0$ is a null vector orthogonal to $C^b.$ Then the surfaces 
\begin{equation*}
    \Sigma_\lambda := \{x:S(x)=\lambda\}
\end{equation*}
are null surfaces. In fact $k^a$ is both normal and tangent to $\Sigma$ and its integral curves. Foliate $\Sigma_\lambda.$

We have 
\begin{align*}
    0 = \partial_b(\partial_a S\partial^a S) &= 2\partial_b\partial_a S\partial^a S \\ 
    &= 2\left(\partial b k_a\right)k^a,
\end{align*}
which is the geodesic equation with tangent $k^a.$
\subsection{Relating Geometry to Physics}
For a manifold $(M^n,g_{ij}),$ there is a tensor (Einstein tensor)
\begin{equation*}
    G_{ab} = R_{ab} - \frac{1}{2}g_{ab}R,
\end{equation*}
that satisfies $\nabla^a G_{ab} = 0.$ But from physics, we know there is another tensor that satisfies 
\begin{equation*}
    \nabla^a T_{ab} = 0.
\end{equation*}
For Maxwell, this tensor is 
\begin{equation*}
    T_{ab} = \frac{1}{4\pi} \left(F_{ac}F^{c}_{a} - g_{ab} F_{cd}F^{cd}\right),
\end{equation*}
where $\nabla^a T_{ab}=0$ if $J_b=0.$ Note that more generally, we need to take the stress energy of charges + currents to get the more general statement.

The brilliant idea of Einstein was that maybe these are the same things. The Einstein Field equations says geometry = physics, and says 
\begin{equation*}
    G_{ab} = \alpha T_{ab},
\end{equation*}
for some constant $\alpha.$ We can determine $\alpha$ be consistencies with Newtonian gravity, where tidal forces at separation $\vec{X}$ are given by 
\begin{equation*}
    F = -\left(\vec{X} \cdot \vec{\nabla}\right) \nabla\phi,
\end{equation*}
where $\nabla\phi = 4\pi \rho.$ In general relativity, the geodesic deviation gives 
\begin{equation*}
    (X^a\nabla_a)(X^b\nabla_b U^d) = -R_{abc}{}^dU^aU^cX^b.
\end{equation*}
We need to somehow match these two ideas. We can identify 
\begin{equation*}
    R_{abc}^dU^aU^c
\end{equation*}
(which has trace $R_{ac}$ with 
\begin{equation*}
    \partial_b\partial^d \phi,
\end{equation*}
which has trace $4\pi \rho.$ This suggests that 
\begin{equation*}
    R_{ac}U^aU^c = 4\pi T_{ac},
\end{equation*}
but the right side is divergence free but the left side is not. We can fix this by contracting with $U^aU^b$ to get 
\begin{equation*}
    4\pi \rho \sim R_{ab}U^aU^b,
\end{equation*}
which eventually gives us $\alpha = 8\pi.$ Therefore, Einstein's equation is 
\begin{equation*}
    G_{ab} = 8\pi T_{ab}.
\end{equation*}
\end{document}
