\begin{sol}
    \begin{enumerate}
        \item There are three criteria:
        \begin{itemize}
            \item {Reflexive:} We have $x\sim x =0 \in \mathbb{Z}$.
            \item {Symmetric:} If $x \sim y \implies x - y = a$ where $a\in \mathbb{Z}$, then $y-x = -a \in \mathbb{Z} \implies y \sim x$.
            \item {Transitive:} If $x \sim y \implies x-y = a \in \mathbb{Z}$ and $y \sim z \implies y-z = b \in \mathbb{Z}$, then: $x-z = (x-y)+(y-z) = a+b \in \mathbb{Z} \implies x\sim z$. 
        \end{itemize}
        \item For each element $x \in \mathbb{R}$, the equivalence class would be $\{x+k | k \in \mathbb{Z}\}$.
        \item The unique representation of each equivalence class of $x$ is $x-\lfloor x \rfloor$. Since $\lfloor x \rfloor \in \mathbb{Z}$, then this is part of the equivalence class.
        \item Let $a,b \in \mathbb{Z}$ such that $x_1-x_2 = a$ and $y_1-y_2=b$. Adding the two equations, we get:
        \begin{equation}
            x_1-x_2+y_1-y_2=a+b \implies (x_1+y_1) - (x_2+y_2) = a+b
        \end{equation} 
        Since $a+b \in \mathbb{Z}$, $x_1+y_1 \sim x_2+y_2$.
        \item All elements in $\mathbb{R}/\mathbb{Z}$ can be represented as $\lfloor x \rfloor$ where $x\in \mathbb{R}$. Since addition is commutative, the group must be abelian. 
    \end{enumerate}
\end{sol}