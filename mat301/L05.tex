\section{Permutation Groups}
\begin{itemize}
    \item Let $X$ be a set. $A$ is a symmetry of $X$ as a set is just a bijection $\sigma: X\to X$ because there is no structure that $\sigma$ should preserve.
    \item We call bijections $\sigma: X \to X$ permutations of $X$.
    \begin{definition}
        The \textbf{symmetric group} on $X$ is the group of all permutations of $X$ with group operation given by composition. It is denoted by $S_x$.
    \end{definition}
    \begin{example}
        Let $X=\{a,b,c\}$, where $a,b,c$ distinct. The map $\sigma: X\to X$ defined by $\sigma(a)=b$, $\sigma(b)=a$, $\sigma(c)=c$ is a permutation of $X$, so $\sigma \in S_x$.

        Similarly, the map $\tau: X\to X$ defined by $\tau(a)=c$, $\tau(b)=a$, $\tau(c)=b$ is a permutation of $X$, so $\tau \in S_X$ also.
    \end{example}
    \begin{proposition}
        For every finite set $X$, $|S_x| = |X|!$.
    \end{proposition}
    \item To prove this proposition rigorously, we can prove this via induction on $n\in \mathbb{Z}_{\ge 0}$ with $|X|=|Y|=n$, the set $\{\sigma:X\to Y|\sigma\text{ is a bijection}\}$ has cardinality $n!$. Then apply that in the case $X=Y$.
    \begin{definition}
        A subgroup of $S_x$ is called a permutation group on $X$.
    \end{definition}
    \item We are most interest in the case when $0<|X|<\infty$.
    \item By choosing a linear ordering $x_1,\dots,x_n$ of the elements of $X$, then we can regard $X$ as the set $\{1,\dots,n\}$.
    \item We may as well, and we will, assume that $X=\{1,\dots, n\}$.
    \item We denote $S_{\{1,\dots,n\}}$ by $S_n$ and we call it the symmetric group on $n$ letters.
    \item The identity of $S_n$ is something denoted by $id$, $1$, $e$, or $\epsilon$. 
    \item If $\sigma \in S_n$, we write:
    \begin{equation}
        \sigma = \begin{pmatrix}
            1 & 2 & \dots & n \\ 
            \sigma(1) & \sigma(2) & \dots & \sigma(n)
        \end{pmatrix}
    \end{equation}
    \begin{example}
        Let $\sigma = \begin{pmatrix}
            1&2&3&4 \\ 
            2&1&4&3
        \end{pmatrix}$ and $\tau = \begin{pmatrix}
            1&2&3&4\\ 2& 3& 4&1
        \end{pmatrix}$. Then:
        \begin{equation}
            \sigma\tau = \begin{pmatrix}
                1&2&3&4\\ 2&1&4&3
            \end{pmatrix}\begin{pmatrix}
                1&2&3&4\\ 2&3&4&1
            \end{pmatrix} = \begin{pmatrix}
                1&2&3&4 \\ 1&4&3&2
            \end{pmatrix}
            and
            \begin{pmatrix}
                1&2&3&4 \\ 
                4&1&2&3
            \end{pmatrix}
        \end{equation}
    \end{example}
    \item For $n\ge 3$, $S_n$ is non-abelian.
    \begin{proof}
        Let $\sigma = \begin{pmatrix}
            1&2&3&\cdots&n\\ 2&1&3&\cdots&n
        \end{pmatrix}$ and $\tau = \begin{pmatrix}
            1&2&3&\cdots & n \\ 
            1&3&2&\cdots & n
        \end{pmatrix}$. Then:
        \begin{equation}
            \sigma\tau = \begin{pmatrix}
                1&2&3&\cdots&n \\ 
                2&3&1 \cdots &n
            \end{pmatrix}
        \end{equation}
        but
        \begin{equation}
            \tau\sigma = \begin{pmatrix}
                1&2&3&\cdots & n \\ 
                3&1&2 & \cdots & n
            \end{pmatrix}
        \end{equation}
        so $\sigma\tau \neq \tau \sigma$.
    \end{proof}
    \item We will now introduce the notion of a cycle
    \begin{definition}
        Let $r\in \mathbb{Z}$, $r\ge 2$. An \textbf{r-cycle} in $S_n$ is a permutation $\gamma \in S_n$ with the following property: There exist $r$ distinct elements $c_1,\dots,c_r \in \{1,\dots,n\}$ such that:
        \begin{enumerate}[label=(\alph*)]
            \item $\gamma(c_i)=c_i+1$ for $1\le i \le r-1$, and $\gamma(c_r)=c_1$.
            \item $\gamma(k)=k$ for all $k \in \{1,\dots,n\} \setminus \{c_1,\dots,c_r\}$.
        \end{enumerate}
        In this case, we write the r-cycle $\gamma$ as:
        \begin{equation}
            \gamma=\begin{pmatrix}
                c_1&c_2&\dots & c_r
            \end{pmatrix}
        \end{equation}
        That is, $\gamma$ is an r-cycle if it moves precisely $r$ elements of $\{1,\dots,n\}$ in a cyclic pattern (and leaves every other element fixed).
    \end{definition}
    \begin{example}
        Let $\gamma = \begin{pmatrix}
            1&2&3&4&5&6&7&8&9 \\ 
            1&5&3&2&6&9&7&4&8
        \end{pmatrix} \in S_9$. We claim that $\gamma$ is a 6-cycle.
        
        Note that $\gamma$ fixes $1,3,7$. We then need to show that the remaining elements are mapped by $\gamma$ in a cyclic pattern:
        \begin{equation}
            2\mapsto 5\mapsto 6\mapsto 9 \mapsto 8 \mapsto 4\mapsto 2
        \end{equation}
        Therefore, $\gamma=\begin{pmatrix}
            2&5&6&9&8&4
        \end{pmatrix}$. Note that this is also equivalent to:
        \begin{equation}
            \gamma=\begin{pmatrix}
                5&6&9&8&4&2
            \end{pmatrix}.
        \end{equation}
    \end{example}
        \begin{proposition}
            Let $r \ge 2$ and let $\gamma = \begin{pmatrix}
                c_1&c_2&\cdots & c_r
            \end{pmatrix}$ be an r-cycle in $S_n$.
            \begin{enumerate}
                \item For all $2\le i \le r$ we have:
                \begin{equation}
                    \gamma = \begin{pmatrix}
                        c_i & c_{i+1} & \dots & c_r & c_1 & c_2 & \dots & c_{i-1}
                    \end{pmatrix}
                \end{equation}
                \item The inverse $\gamma^{-1}$ is given by:
                \begin{equation}
                    \gamma^{-1} = \begin{pmatrix}
                        c_r & c_{r-1} & \dots c_1
                    \end{pmatrix}
                \end{equation}
            \end{enumerate}
        \end{proposition}
        \begin{proof} We prove both parts of the above proposition.
            \begin{enumerate}
                \item Exercise left to reader.
                \item Let $\delta = \begin{pmatrix}
                    c_r & c_{r-1} & \dots c_1
                \end{pmatrix}$. To show that $\delta = \gamma^{-1}$, it suffices to show that $\delta \gamma = \id$. (since $S_n$ is a group). To do so, we must prove that $\forall i \in \{1,\dots,n\}$, we have $\delta\gamma(i)=i$.

                By definition of cycles, we have:
                \begin{equation}
                    \gamma(k) = \begin{cases}
                        k & k\notin \{c_1,\dots,c_r\} \\ 
                        c_{i+1} & k=c_i,1\le i \le r-1 \\ 
                        c_1 & k=c_r
                    \end{cases}
                \end{equation}
                and:
                \begin{equation}
                    \delta(k) = \begin{cases}
                        k & k\notin \{c_1,\dots,c_r\} \\ 
                        c_{i-1} & k=c_i,2\le i \le r \\ 
                        c_r & k=c_1
                    \end{cases}
                \end{equation}
                We can then check for $k\notin \{c_1,\dots,c_r\}$, we have:
                \begin{equation}
                    \delta \gamma(k) = \gamma(k)=k
                \end{equation}
                For $k=c_i$, $1\le i\le r-1$, we have:
                \begin{equation}
                    \delta \gamma (k) = \delta \gamma(c_i) = \delta(c_{i+1}) = c_i = k
                \end{equation}
                For $k=c_r$, we have:
                \begin{equation}
                    \delta \gamma(k) = \delta(c_1) = c_r = k.
                \end{equation}
            \end{enumerate}
        \end{proof}
        \item Let us investigate the product of two cycles.
        \begin{example}
            Let $\gamma=\begin{pmatrix}
                1&3&2&4
            \end{pmatrix}$ and $\delta = \begin{pmatrix}
                2&6&3
            \end{pmatrix}$ where $\gamma,\delta \in S_8$. Then:
            \begin{align}
                \delta \gamma &= \begin{pmatrix}
                    1&3&2&4
                \end{pmatrix}\begin{bmatrix}
                    2&6&3
                \end{bmatrix} \\ 
                &= \begin{pmatrix}
                    1&2&3&4&5&6&7&8 \\ 
                    3&6&4&1&5&2&7&8
                \end{pmatrix}
            \end{align}
            Notice that:
            \begin{equation}
                1\mapsto 3\mapsto 4 \mapsto 1
            \end{equation}
            However, the other elements are not fixed since $6\mapsto 2$. Therefore, $\gamma\delta$ is not a cycle.
        \end{example}
\end{itemize}