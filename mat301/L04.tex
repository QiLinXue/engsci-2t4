\section{Lecture Four}
\begin{itemize}
    \item We begin with the \textbf{Finite Subgroup Test}
    \begin{theorem}
        Let $G$ be a group and let $H$ be a finite nonempty subset of $G$. If $H$ is closed under the group operation of $G$, then $H \le G$.
    \end{theorem}
    \begin{proof}
        By the $2$-step subgroup test, it suffices to prove that $H$ is closed under taking inverses. Let $a \in H$:
        \begin{itemize}
            \item If $a=e$, then $a^{-1}=e \in H$.
            \item If $a\neq e$, consider the set:
            \begin{equation}
                \{a^n | n \ge 1\} = \{a,a^2,a^3,\dots\}
            \end{equation}
            Since $H$ is closed under the group operation and $a \in H$, we have $\{a^n | n \ge 1 \} \subseteq H$ by a short induction argument. Since $H$ is finite, so is $\{a^n | m \ge 1\}$. Therefore, $\exists m,n \ge 1, m \neq n$ such that:
            \begin{equation}
                a^m = a^n
            \end{equation}
            WLOG, we may assume that $m > n$, so $m-n>0$. We have:
            \begin{equation}
                a^{m-n}=e
            \end{equation}
            Since $a\neq e$, $m-n\neq 1$. Therefore, $m-n \ge 2$, so $m-n-1 \ge 1$. Thus:
            \begin{equation}
                a^{m-n-1} \in \{a^k| k \ge 1\} \subseteq H
            \end{equation}
            and
            \begin{equation}
                a^{m-n-1}a = a^{m-n}=e
            \end{equation}
            so:
            \begin{equation}
                a^{m-n-1} = a^{-1}
            \end{equation}
        \end{itemize}
    \end{proof}
    \item We will look at a special class of subgroups: \textbf{subgroups generated by one element.}
    \begin{definition}
        Let $G$ be a group and let $a \in G$. Define:
        \begin{equation}
            \langle a \rangle = \{a^n | n\in \mathbb{Z}\}
        \end{equation}
        We call $\langle a \rangle$ the subgroup of $G$ generated by $a$.
    \end{definition}
    \item We propose that $\langle a \rangle \le G$.
    \begin{proof}
        Since $e=a^0 \in \langle a \rangle$, we have $\langle a \rangle \neq \varnothing$.
        
        If $g,h \in \langle a \rangle$, then $g=a^m$ and $h=a^n$ for some $m,n \in \mathbb{Z}$ and:
        \begin{equation}
            gh^{-1} = a^m(a^n)^{-1} = a^ma^{-n}=a^{m-n} \in \langle a \rangle
        \end{equation}
    \end{proof}
    \begin{example}
        Let $G = (\mathbb{Z}/14\mathbb{Z})^\times = \{1,3,5,9,11,13\}$. We have:
        \begin{equation}
            a=3,\, a^2=9,\, a^3=27=13=-1=-1,\, a^4=-3=11,\,a^5=-9=5,\,a^6=15=1
        \end{equation}
        Similarly:
        \begin{equation}
            a^0=1,\, a^{-1}=5,\, a^{-2}=11,\, a^{-3}=13,\, a^{-4}=9,\, a^{-5}=3,\, a^{-6}=1
        \end{equation}
        Therefore:
        \begin{equation}
            \langle a \rangle = \{1,3,5,9,11,13\} = (\mathbb{Z}/14\mathbb{Z})^x
        \end{equation}
        Therefore, $(\mathbb{Z}/14\mathbb{Z})^\times$ is cyclic.
        \textbf{Remarks:} If $a^n=e$, then for all $k\in \mathbb{Z}$, we have:
        \begin{equation}
            a^{-k}=a^{n-k}
        \end{equation}
        so we can easily figure out negative exponents.
    \end{example}
    \begin{example}
        Let $G=\mathbb{Z}/12\mathbb{Z}$ and $a=2$. We have:
        \begin{equation}
            -a=10,\,0a=0,\,2a=4,\,3a=6,\,4a=8,\,5a=10,\,6a=12=0,\,7a=2
        \end{equation}
        so:
        \begin{equation}
            \langle a \rangle = \{0,2,4,6,8,10\}.
        \end{equation}
    \end{example}
    \begin{example}
        Let $G=\mathbb{R}$ and $a=2\pi$. Here,
        \begin{equation}
            \langle a\rangle = \{n2\pi | n\in \mathbb{Z} \} = 2\pi \mathbb{Z}
        \end{equation}
    \end{example}
    \begin{definition}
        Let $G$ be a group and $a\in G$. If there exists $n\in \mathbb{Z}_{>0}$ such that $a^n=e$, then we say that $a$ has \textbf{finite order} and the \textbf{order of a} is defined to be the smallest $n\in \mathbb{Z}_{>0}$ such that $a^n=e$.
        \vspace{2mm}

        If there does not exist $n\in \mathbb{Z}_{>0}$ such that $a^n=e$, then we say that $a$ has infinite order.
        \vspace{2mm}

        The order of $a$ is denoted by $o(a)$ or $|a|$. If $a$ has infinite order, we write $o(a)=\infty$.
    \end{definition}
    \item Note that:
    \begin{itemize}
        \item $o(a)=1 \iff a=e$
        \item If $o(a)=\infty$, then $a^n=e \iff n=0$.
    \end{itemize}
    \item Let $G$ be a group and $a\in G$.
    \begin{enumerate}
        \item If $o(a)=\infty$, then $\forall i,j \in \mathbb{Z}$ we have:
        \begin{align}
            a^{i-j} = e &\iff i-j=0 \\ 
            &\iff i=j
        \end{align}
        \item If $o(a) = n <\infty$, then $\forall_{i,j}\in \mathbb{Z}$ we have:
        \begin{align}
            a^i = a^j &\iff n|i-j \\ 
            &\iff i\equiv j \pmod{n}
        \end{align}
        In particular, $a^i=e (=a^0) \iff n | i$.
    \end{enumerate}
    \begin{proof}
        Let $i,j\in \mathbb{Z}$. Note $a^i=a^j \implies a^{i-j}=e$.
        \begin{enumerate}
            \item Suppose $o(a)=\infty$. Then $a^{i-j}=e$ iff $i-j=0 \iff i=j$.
            \item Suppose $o(a)=n<\infty$. We must show that $a^{i-j}=e \iff n|i-j$.
            
            (Backwards): If $n|i-j$, then $\exists k \in \mathbb{Z}$ such that $i-j=kn$ so $a^{i-j} = a^{kn}=(a^n)^k = e^k=e$.

            (Forwards) Now suppose $a^{i-j}=e$. By the division algorithm, $\exists!$ $q$ and $0 \le r < n$ such that:
            \begin{equation}
                i-j = qn+r
            \end{equation}
            We have:
            \begin{equation}
                e = a^{i-j}=a^{qn+r}=a^{qn}a^{r} = (a^n)^q a^r = e^qa^r = a^r
            \end{equation}
            Since $n$ is the smallest positive integer with $a^n=e$ and $0 \le r < n$ and satisfies $a^r=e$, we must have $r=0$.

            Therefore, $i-j = qn$ so $n |i-j$.
        \end{enumerate}
    \end{proof}
    \begin{corollary}
        Let $G$ be a group and $a\in G$.
        
        \begin{enumerate}
            \item If $o(a)=\infty$, then $\dots,a^{-2},a^{-1},e,a,a^2,\dots$ are distinct (and $\langle a \rangle = \{a^n|n\in \mathbb{Z}\}$)
            \item If $o(a) = n < \infty$, then $e,a,\dots,a^{n-1}$ are distinct and $\langle a \rangle = \{e,a,\dots,a^{n-1}\}$.
        \end{enumerate}
    \end{corollary}
    \begin{corollary}
        Let $G$ be a group and $a\in G$. Then $o(a) = |\langle a \rangle |$ where $|\langle a \rangle | = \infty$ when $\langle a\rangle $ is infinite.
    \end{corollary}
    \begin{corollary}
        Let $G$ be a group and $a,b\in G$. If $ab=ba$ and $o(a),o(b) < \infty$, then
        \begin{equation}
            o(ab) | o(a)o(b)
        \end{equation}
    \end{corollary}
    \begin{proof}
        Suppose $ab=ba$ and $o(a),o(b)<\infty$. Since:
        \begin{align}
            (ab)^{o(a)o(b)} &= a^{o(a)o(b)}b^{o(a)o(b)} \\ 
            &= (a^{o(a)})^{o(b)}(b^{o(b)})^{o(a)} \\ 
            &= e^{o(b)}e^{o(a)} \\ 
            &= e
        \end{align}
        Therefore, $o(ab) | o(a)o(b)$.  
    \end{proof}
    \item \textbf{Remarks about notation:}
    \begin{itemize}
        \item $\mathbb{Z}/n\mathbb{Z}$ is sometimes denoted by $\mathbb{Z}_n$ or $\mathbb{Z}/(n)$.
        \item $(\mathbb{Z}/n\mathbb{Z})^\times = \{[a]\in \mathbb{Z}/n\mathbb{Z} | [b] \in \mathbb{Z}/n\mathbb{Z} \text{ with } [a][b] = 1\} = \{[a] | \gcd(n,a) = 1\}.$
    \end{itemize}
    \begin{theorem}
        Let $G$ be a group and $a\in G$ with $o(a)=n<\infty$. For any $k\in \mathbb{Z}$, we have:
        \begin{equation}
            o(a^k) = \frac{o(a)}{\gcd(o(a),k)} = \frac{n}{\gcd(n,k)}
        \end{equation}
    \end{theorem}
    \begin{proof}
        By definition, $o(a^k)$ is the smallest $m\in \mathbb{Z}_{>0}$ such that
        \begin{align}
        (a^k)^m = e &\iff a^{mk} = e \\ 
        &\iff n | mk
        \end{align}
        Since $mk$ is a multiple of $k$, we have $n|mk \iff mk \text{ is common multiple of } n \text{ and } k$.

        If there exists $m \in \mathbb{Z}_{>0}$ such that $mk = \lcm(n,k)$, then $m=o(a^k)$. Recall that:
        \begin{equation}
            \frac{nk}{\gcd(n,k)} = \lcm(n,k)
        \end{equation}
        Since $\gcd(n,k) | n$, then $\frac{n}{\gcd(n,k)} \in \mathbb{Z}_{>0}$ with
        \begin{equation}
            \left(\frac{n}{\gcd(n,k)}\right)k = \lcm(n,k)
        \end{equation}
        Therefore:
        \begin{equation}
            o(a^k) = \frac{n}{\gcd(n,k)}
        \end{equation}
    \end{proof}
    \begin{corollary}
        In a finite group $G$, the order of every element divides the order of the group:
        \begin{equation}
            \forall x \in G,\quad o(x) \Big| |G|
        \end{equation}
    \end{corollary}
    \begin{example}
        $\mathbb{Z}=\langle 1\rangle$ is an infinite cyclic group. Meanwhile, $\mathbb{Z}/n\mathbb{Z} = \langle 1 \rangle$ is a finite cyclic group.
    \end{example}
    \item Next we will study subgroups of cyclic groups. Choose a generator $a \in G$ and $G=\langle a \rangle$.
    \item For each $k \in \mathbb{Z}$, $a^k \in \langle a\rangle$. Therefore $\langle a^k \rangle \subseteq \langle a\rangle$.
    \begin{proposition}
        Let $G$ be a group and let $a\in G$. If $H \le G$ and $a\in H$, then $\langle a \rangle \subseteq H$.
    \end{proposition}
    \item One natural question is: \textit{Do we get every subgroup in this way?} If $k,\ell \in \mathbb{Z}$, when is $\langle a^k\rangle = \langle a^\ell \rangle$?
    \begin{theorem}
        \textbf{Classification of subgroups of cyclic groups}: Let $G=\langle a\rangle$ be a cyclic group:
        \begin{enumerate}
            \item If $|G|=\infty$ ($\iff o(a) = \infty$) then every subgroup of $G$ is of the from $\langle a^m \rangle$ for a unique $m \in \mathbb{Z}_{\ge 0}$.
            
            \textit{Remarks:} $\langle a^m \rangle = \langle a^{-m}\rangle $.
            \item If $|G|=n < \infty$ ($\iff o(a) = n < \infty$) then every subgroup of $G$ is of the form $\langle a^m\rangle$ for a unique $m \in \mathbb{Z}_{>0}$ with $m|n$.
            
            Said differently, the order of every subgroup of $G$ divides $n$ and for each $d\in \mathbb{Z}_{>0}$ with $d|n$ there is a unique subgroup of $G$ of order $d$, namely $\langle a^{n/d} \rangle$.
        \end{enumerate}
    \end{theorem}
    \begin{proof}
        Let $H \le G = \langle a\rangle$ with $H \neq \{e\}$. Then $\exists k \in \mathbb{Z} \setminus \{0\}$ such that $a^k$, $a^{-k} \in H$. Therefore, $a^{|k|} \in H$ so $\exists k' \in \mathbb{Z}_{>0}$ such that $a^{k'} \in H$. Let $m$ be the smallest positive integer such that $a^m \in H$ (which exists by the well-ordering principle).
        \vspace{2mm}

        We will prove that $H= \langle a^m \rangle$. Since $a^m \in H$, we have $\langle a^m\rangle \subseteq H$. To prove $H \subseteq \langle a^m \rangle$, it suffices to prove:
        \begin{itemize}
            \item If $a^k \in H$ where $k\in \mathbb{Z}$, then $m|k$.
            
            Let $k\in \mathbb{Z}$ and assume $a^k \in H$. By the division algorithm, $\exists! q,r \in \mathbb{Z}$ such that $0\le r < m$ and:
            \begin{equation}
                k=qm + r
            \end{equation}
            Then:
            \begin{equation}
                a^k=a^{qm+r}=(a^m)^qa^r \implies a^r = (a^m)^{-q}a^k
            \end{equation}
            Since $(a^m)^{-q},a^k \in H$.

            Since $\langle a^m\rangle \subseteq H$, $(a^m)^{-q} \in H$. We assumed $a^k \in H$. Therefore, $a^r \in H$.

            Since $m$ is the smallest positive integer with $a^m \in H$ and $a^r \in H$ and $0 \le r < m$, we have $r=0$. Therefore $k=qm$ so $m|k$.
        \end{itemize}
        If $|G| = n < \infty$, then $o(a) = n$, so $a^n=e\in H$. Therefore by the above point, $m|n$. Now we look at the two cases:
        \begin{enumerate}
            \item Suppose $|G| = \infty$. We prove that every nontrivial subgroup of $G$ is of the form $\langle a^m \rangle$ for some $m \in \mathbb{Z}_{>0}$. Since $\{e\}= \langle a^0\rangle$, we have that every subgroup of $G$ is of the form $\langle a^m \rangle$ for some $m \in \mathbb{Z}_{\ge 0}$.
            
            To prove that $m$ is unique, suppose $H \le G$ and $H = \langle a^m\rangle = \langle a^{m'}\rangle$ for some $m,m' \in \mathbb{Z}_{\ge 0}$.

            Since $a^m \in \langle a^m \rangle = \langle a^{m'} \rangle$, $a^{m} \in \langle a^{m'} \rangle$, so $a^m=a^{m'k}$ for some $k \in \mathbb{Z}$. Since $o(a)=\infty$, we must have $m=m'k$ so $m'|m$. Similarly, $m|m'$. Thus, $m=m'$.
            \item Suppose $|G| = n < \infty$. Then $o(a) = n$, so $a^n=e$ and therefore $\{e\} = \langle a^n \rangle$. We proved above that every nontrivial subgroup of $G$ is of the form $\langle a^m \rangle$ for some $m\in \mathbb{Z}_{>0}$ with $m|n$.
            \item 
            Therefore, every subgroup of $G$ is of the form $\langle a^m\rangle $ for some $m \in \mathbb{Z}_{>0}$ with $m|n$.
            
            To prove that $m$ is unique, suppose $H \le G$ with $H = \langle a^m \rangle = \langle a^{m'}\rangle$ where $m,m'' \in \mathbb{Z}_{>0}$ with $m,m' | n$. Then:
            \begin{equation}
                o(a^m) = |\langle a^m \rangle | = |\langle a^{m'} \rangle | = o(a%{m;})
            \end{equation}
            Since $o(a^k) = \frac{n}{\gcd(n,k)}$ for all $k\in \mathbb{Z}$, we got:
            \begin{equation}
                \frac{n}{\gcd(n,m)} = \frac{n}{\gcd(n,m')}
            \end{equation}
            which implies $\gcd(n,m) = \gcd(n,m')$. Since $m,m'|n$ we have $\gcd(n,m)=m$ and $\gcd(n,m')=m'$ so $m=m'$.
        \end{enumerate}
    \end{proof}
    \begin{corollary}
        Criterion for $\langle a^i \rangle = \langle a^j \rangle$ and $o(a^i) = o(a^j)$.
        \vspace{2mm}

        Let $G=\langle a\rangle$ be a cyclic group and let $i,j \in \mathbb{Z}$.
        \begin{enumerate}
            \item If $|G| = \infty$, then $\langle a^i \rangle = \langle a^j \rangle$ if and only if $j= \pm k$.
            \item If $|G|=n<\infty$, then the following are equivalent:
            \begin{itemize}
                \item $\langle a^i \rangle = \langle a^k \rangle$
                \item $o(a^i) = o(a^j)$
                \item $\gcd(n,i) = \gcd(n,j)$
            \end{itemize}
        \end{enumerate}
    \end{corollary}
    \begin{corollary}
        (The generators of a cyclic group) Let $G=\langle a\rangle$ be a cyclic group. The generators of $G$ are:
        \begin{equation}
            \begin{cases}
                \{a,a^{-1}\} & |G|=\infty \\ 
                \{a^k | \gcd(n,k)= 1\} & |G|=n<\infty
            \end{cases}
        \end{equation}
        This corollary follows from the first corollary.
    \end{corollary}
    \item If $G=\langle a\rangle$ is cyclic of order $n<\infty$, it follows that there are exactly $\phi(n)$ generators where $\phi(n)$ is Euler's Toitent function.
\end{itemize}