\documentclass{article}
\usepackage{gallian}
\title{MAT301: Extra Topics}
\author{QiLin Xue}
% \lhead{MAT301}
% \rhead{QiLin Xue}
\usetikzlibrary{decorations.markings}
\usepackage{eso-pic}

\usepackage{xcolor}
\usepackage{pagecolor}

\usepackage{calc}%
\usepackage{titlesec}%
\titleformat{\section} {\color{blue1}\Large\bfseries}{\makebox[-4.5em][l]{}}{1em}{}
\titleformat{\subsection} {\color{blue1}\large\bfseries}{\makebox[-5em][l]{}}{1em}{\texorpdfstring{$\blacksquare$}{}\hspace{1mm}}
\titleformat{\subsubsection} {\color{blue1}\normalfont\bfseries}{\makebox[-5em][l]{}}{1em}{\texorpdfstring{$\blacksquare$}{}\hspace{1mm}}
\renewcommand{\qedsymbol}{$\textcolor{blue1}{\blacksquare}$}
\setcounter{tocdepth}{2}
\renewenvironment{proof}{{\bfseries\color{blue1} Proof:}}{\qed}

\begin{document}
\maketitle
\AddToShipoutPictureBG{\AtPageLowerLeft{%
        \color{blue2}\rule{.04\paperwidth}{\paperheight}}}

\tableofcontents
\newpage
\section{Free Abelian Group}
The notion of a free abelian group is important. While during class this was introduced as a tool to solve another problem (classification of finitely generated abelian groups), I will introduce it first to prevent the flow from being broken up in the next section.

Before we define free abelian groups, we need to make a connection between abelian groups and linear algebra.
\subsection{Relationship between Abelian Groups and Vector Spaces}
In this section, we make the connection between free abelian groups and vector spaces. The fact that the group operator of abelian groups is typically written as addition $+$ is suggestive of this relationship.

Specifically, just like how vectors can span a vector space, a linear combination of elements can span a group. For formality, we introduce the following definitions:
\begin{definition}{Set of $\mathbb{Z}$ linear combinations of elements of $S$}
    Let $(A,+)$ be an abelian group. Note that if $S \subseteq A$, then
    \begin{equation*}
        \langle S \rangle = \left\{\sum_{i=1}^{m}k_ia_i:m\in \mathbb{Z}_{\ge 0},\,a_i \in S, k_i \in \mathbb{Z}\right\}
    \end{equation*}
    where the right hand side can be denoted as $\text{span}_{\mathbb{Z}}(S)$, which is the set of all $\mathbb{Z}$ linear combinations of elements of $S$.
\end{definition}
Since empty sets are trivial, we have
\begin{equation*}
    \text{span}_{\mathbb{Z}}(\emptyset) = \{0\}
\end{equation*}
\begin{definition}{Linear Independence, Span, Basis}
    Let $S \subseteq A$.
    \begin{enumerate}
        \item $S$ is linearly independent (over $\mathbb{Z}$) if for any $m\in \mathbb{Z}_{>0}$, $a_1,\dots,a_m \in S$, and $k_1,\dots,k_m \in \mathbb{Z}$,
              \begin{equation*}
                  \sum_{i=1}^{m} k_ia_i = 0 \implies a_1=\cdots=a_m=0,
              \end{equation*}
              or equivalently if every element of $A$ can be written as a $\mathbb{Z}$-linear combination of elements of $S$ in at most one way.
        \item $S$ spans $A$ (over $\mathbb{Z}$) if $A=\spanz(S)$, or equivalently every element of $A$ can be written as a $\mathbb{Z}$-linear combination of elements of $A$ in at least one way.
        \item $S$ is a basis (or $\mathbb{Z}$-basis) of $A$ if $S$ is linearly independent and spans $A$, or equivalently if every element of $A$ can be written as a $\mathbb{Z}$-linear combination of elements of $S$ in exactly one way.
    \end{enumerate}
\end{definition}
\begin{example}
    $e_1,\dots,e_m$ is a basis of $\mathbb{Z}^m$.
\end{example}
\subsection{Definition of Free Abelian Group}
\begin{definition}{Free Abelian Group}
    A free abelian group is an abelian group that has a basis.
    \vspace{2mm}

    A free abelian group of finite rank is an abelian group that has a finite basis.
\end{definition}
\begin{example}
    $\mathbb{Z}^m$ is a free abelian group of finite rank for all $m\in \mathbb{Z}_{\ge 0}$. Note that $\mathbb{Z}^0=\{0\}$.
\end{example}
\begin{example}
    If $\{v_1,\dots,v_m\} \subseteq \mathbb{R}^n$ is linearly independent over $\mathbb{R}$ and $A$ is the subgroup of $\mathbb{R}^n$ generated by $\{v_1,\dots,v_n\}$, i.e. $A=\langle v_1,\dots,v_n\rangle=\spanz(\{v_1,\dots,v_n\})$, then $\{v_1,\dots,v_n\}$ is a $\mathbb{Z}$-basis of $A$, so $A$ is a free abelian group of finite rank.
\end{example}
\begin{example}
    If $\{G_i\}_{i\in I}$ is a set of abelian groups, then
    $$\Oplus{i\in I} G_i := \left\{(g_i)_{i\in I} \in \prod_{i\in I}G_i\,\bigg\vert\,g_i = e_{G_i} \text{ for all but finitely many } i \in I\right\}$$
    is a subgroup of the direct product $\prod_{i\in I} G_i$. Note that we can call $\Oplus{i\in I} G_i$ the direct sum of $\{G_i\}_{i \in I}$.

    If $I$ is infinite and $G_i=\mathbb{Z}$ for all $i\in I$, then $\Oplus{i\in I}G_i = \Oplus{i\in I}\mathbb{Z}$ is a free abelian group that is not of finite rank (it has an infinite basis but no finite basis).
\end{example}
Note that free abelian groups are like vector spaces over $\mathbb{Z}$.
\subsection{Properties of Free Abelian Groups}
\subsubsection{Proposition 1.1: Relationship to $\mathbb{Z}^m$}
\begin{idea}
    A group is a free abelian group of finite rank if and only if it is isomorphic to $\mathbb{Z}^m$ for some $m\in \mathbb{Z}_{\ge 0}$.
\end{idea}
\begin{proof}
    Let $A$ be a group. If $m\in \mathbb{Z}_{\ge 0}$ and $\phi: \mathbb{Z}^m\rightarrow A$ is an isomorphism, then $\{\phi(e_1),\cdots,\phi(e_m)\}$ is a basis of $A$.

    Conversely, if $A$ is a free abelian group of finite rank and $\{a_1,\dots,a_n\}$ is a basis of $A$, then
    \begin{align*}
        \phi: \mathbb{Z}^m & \rightarrow A                    \\
        (k_1,\dots,k_m)    & \mapsto k_1a_1 + \cdots + k_ma_m
    \end{align*}
    is an isomorphism.
\end{proof}
\subsubsection{Proposition 1.2: Cardinality of Rank}
\begin{idea}
    Let $A$ be a free abelian group of finite rank. then all bases of $A$ have the same cardinality.
\end{idea}
\begin{proof}
    Let $\{a_1,\dots,a_m\}$ be a basis of $A$.

    Let $\{a'_j\}_{j\in J}$ be a basis of $A$. For each $i=1,\dots,m$, there exists a finite subset $S_i \subseteq \{a_j'\}_{j\in J}$ such that $a_i$ is a linear combination of the elements of $S_i$. Let $S=S_1 \cup \cdots \cup S_m$.

    Then $a_1,\dots,a_m$ are all linear combinations of the elements of $S$. Since $\{a_1,\dots,a_m\}$ spans $A$, it follows that every element of $A$ is a linear combination of the elements of $S$, i.e. $S$ spans $A$. Let $a_{j_1},\dots,a_{j_n}$ be the elements of $S$.

    Suppose for a contradiction that $\{a_j'\}_{j\in J}$ is infinite. Then there exists $j\in J$ such that $a_j' \neq a_j,\dots, a'_{j_n}$. There exist $k_1,\dots,j_n\in \mathbb{Z}$ such that
    \begin{equation}
        a_j' = \sum_{i=1}^{n}k_ia_{j_i}',
    \end{equation}
    but this contradicts linearly independence of $\{a_j'\}_{j\in J}$.

    Therefore $\{a_j'\}_{j\in J}$ is finite. Let $a_1',\dots,a_n'$ be the elements of $\{a_j'\}_{j\in J}$. Since $\{a_1,\dots,a_m\}$ spans $A$, for all $i=1,\dots,n$ we can write
    \begin{equation}
        a_i' = \sum_{j=1}^m M_{ij}a_j
    \end{equation}
    for some $M_{i1},\dots,M_{im}\in \mathbb{Z}$.

    Similarly, since $\{a_1',\dots,a_n'\}$ spans $A$, for all $i=1,\dots,m$ we can write
    \begin{equation}
        a_i = \sum_{j=1}^n N_{ij}a_j'
    \end{equation}
    for some $N_{i1},\dots,N_{in}\in \mathbb{Z}$.

    For all $i=1,\dots,n$, we have
    \begin{align*}
        a_i' & = \sum_{j=1}^m M_{ij}a_j                                  \\
             & = \sum_{j=1}^m M_{ij} \sum_{k=1}^n N_{jk}a_k'             \\
             & = \sum_{j=1}^m\sum_{k=1}^n M_{ij}N_{jk}a_k'               \\
             & = \sum_{k=1}^n \sum_{j=1}^m M_{ij}N_{jk}a_k'              \\
             & = \sum_{k=1}^n\left( \sum_{j=1}^m M_{ij}N_{jk}\right)a_k'
    \end{align*}
    Since $\{a_1',\dots,a_n'\}$ is linearly independent, it follows that
    \begin{equation}
        \left( \sum_{j=1}^m M_{ij}N_{jk}\right) = \delta_{ik}
    \end{equation}
    for all $i,k=1,\dots,n$.

    Similarly, for all $i,k=1,\dots, m$ we have
    \begin{equation}
        \sum_{j=1}^n N_{ij}M_{jk} = \delta_{ik}.
    \end{equation}
    Let $M=[M_{ij}] \in \text{Mat}_{n\times m}(\mathbb{Z})$ and $N=[N_{ij}]\in \text{Mat}_{m\times n}(\mathbb{Z})$.

    Then $MN=I_n$ and $NM=I_m$. Therefore, the $\mathbb{Q}$-linear transformations
    \begin{align*}
        \mathbb{Q}^m & \rightarrow \mathbb{Q}^n \\
        x            & \mapsto Mx
    \end{align*}
    and
    \begin{align*}
        \mathbb{Q}^n & \rightarrow \mathbb{Q}^m \\
        x            & \mapsto Nx
    \end{align*}
    are inverses of each other. Therefore $\mathbb{Q}^m$ and $\mathbb{Q}^n$ are isomorphic vector spaces over $\mathbb{Q}$, so $m=n$.
\end{proof}
\subsection{Definition of Rank}
Therefore, we can remove the assumption that $A$ is of finite rank from the proposition. This also introduces the notion of a rank:
\begin{definition}{Rank of a Free Abelian Group}
    Let $A$ be a free abelian group. The rank of $A$ is the cardinality of some (and hence any) basis of $A$ and is denoted by $\rk(A)$.
\end{definition}
\subsubsection{Proposition 1.3: Finite groups have finite ranks}
\begin{idea}
    Let $A$ be a finitely generated group. Then $A$ is of finite rank if and only if $A$ is a finite group.
\end{idea}
\begin{proof}
    The ``only if'' direction is immediate. Suppose $A$ is a finite group and let $\{g_1,\dots,g_m\}$ be a generating set of $A$. Let $\{a_i\}_{i \in I}$ be a basis of $A$. There is a finite subset $\{a_{i_1},\dots,a_{i_n}\}$ of $\{a_i\}_{i\in I}$ such that
    \begin{equation*}
        \{g_1,\dots,g_m\} \subseteq \spanz\left(\left\{a_{i_1},\dots,a_{i_n}\right\}\right).
    \end{equation*}
    Then
    \begin{equation*}
        A = \spanz(\{g_1,\dots,g_m\}) \subseteq \spanz\left(\left\{a_{i_1},\dots,a_{i_n}\right\}\right) \subseteq A,
    \end{equation*}
    so $\spanz\left(\left\{a_{i_1},\dots,a_{i_n}\right\}\right) = A$. Since $\{a_{i_1},\dots,a_{i_n}\} \subseteq \{a_i\}_{i \in I}$ and $\{a_i\}_{i\in I}$ us linearly independent, it follows that $\{a_{i_1},\dots,a_{i_n}\}$ is linearly independent. Therefore, $\{a_{i_1},\dots,a_{i_n}\}$ is a basis of $A$, so $A$ is of finite rank.
\end{proof}
\vspace{2mm}

\textit{Warning!} Here are a few misconceptions. Take $\mathbb{Z}$ for example. THen:
\begin{itemize}
    \item $\{2,3\}$ is a minimal spanning subset of $\mathbb{Z}$, but it is not a basis as it is linearly dependent.
    \item $\{2,3\}$ spans $\mathbb{Z}$, but does not contain a basis of $\mathbb{Z}$.
    \item $\{2\}$ is a maximal linearly independent subset of $\mathbb{Z}$, but it is not a basis because its span is $2\mathbb{Z} \subsetneq
              \mathbb{Z}$.
    \item $\{2\}$ is linearly independent, but it is not contained in a basis of $\mathbb{Z}$.
\end{itemize}
\subsection{Homomorphisms of Free Abelian Groups}
\subsubsection{Proposition 1.4: Homomorphisms and Bases}
\begin{idea}
    Let $A$ be a free abelian group and let $\{a_i\}_{i\in I}$ be a basis of $A$.
    \vspace{2mm}

    Let $B$ be an abelian group and let $\{b_i\}_{i\in I}$ be a family of elements of $B$.
    \vspace{2mm}

    Then there exists a unique homomorphism $\phi:A\rightarrow B$ such that $\phi(a_i)=b_i$ for all $i\in I$. It is surjective if and only if $\{b_i\}_{i\in I}$ spans $B$, it is injective if and only if $\{b_i\}_{i\in I}$ is linearly independent, and it is an isomorphism iff $\{b_i\}_{i\in I}$ is a basis of $B$.
\end{idea}
Let $A$ be a free abelian group of finite rank $n$. For any basis $\alpha=\{a_1,\dots,a_n\}$ of $A$ there exists a unique isomorphism:
\begin{equation*}
    \theta_\alpha: A \rightarrow \mathbb{Z}^n
\end{equation*}
such that $\theta(a_i)=e_i$ for all $i=1,\dots,n$. Note that:
\begin{itemize}
    \item $\theta_\alpha^{-1}(k_1,\dots,k_n)=\sum_{i=1}^n k_ia_i$
    \item For all $a\in A$, let us write $[a]_\alpha = \theta_\alpha(a) \in \mathbb{Z}^n$.
\end{itemize}
\subsubsection{Proposition 1.5: Matrices and Homomorphisms}
\begin{idea}
    Let $A,B$ be free abelian groups of finite ranks $n$ and $m$, respectively. Let $\alpha=\{a_1,\dots,a_n\}$ be a basis of $A$ and $\beta=\{b_1,\dots,b_m\}$ be a basis of $B$. For all homomorphisms $\phi: A\rightarrow B$ there exists a unique matrix
    \begin{equation*}
        [T]^\alpha_\beta \in \text{Mat}_{m\times n}(\mathbb{Z})
    \end{equation*}
    such that for all $a\in A$ we have
    \begin{equation*}
        [Ta]_\beta = [T]^\alpha_\beta [a]_\alpha.
    \end{equation*}
    Let $C$ be a free abelian group of finite rank $p$ and let $\gamma = \{c_1,\dots,c_p\}$ be a basis of $C$. If $T: A\rightarrow B$ and $S: B\rightarrow C$ are homomorphisms, then
    \begin{equation*}
        [S \circ T]^\alpha_\gamma = [S]^\beta_\gamma [T]^\alpha_\beta
    \end{equation*}
\end{idea}
\begin{proof}
    Let $T:A\rightarrow B$ be a homomorphism. Define
    \begin{equation*}
        [T]^\alpha_\beta = [[Ta_1]_\beta \cdots [Ta_n]_\beta].
    \end{equation*}
    The rest is straightforward.
\end{proof}

In words, this says that we can take any element $a \in A$ to $b \in B$ using the homomorphism $T$ and write it as a vector using the basis of $B$. This is equivalent to first writing $a$ as a vector using the basis of $A$, and multiplying it by some matrix $[T]^\alpha_\beta$ that transforms the vector from an $\alpha$ basis to a $\beta$ basis. Essentially, this is a change of basis matrix.
\subsubsection{Corollary}
\begin{idea}
    A homomorphism $T:A\rightarrow B$ is an isomorphism if and only if there exists $N\in \text{Mat}_{n\times m}(\mathbb{Z})$ such that
    \begin{equation*}
        [T]^\alpha_\beta n = I_m \text{ and } N[T]^\alpha_\beta = I_n
    \end{equation*}
    in which case $m=n$.
\end{idea}
\subsection{Matrices}
As previously seen, we can make connections between free abelian groups and matrices. Here are a few more important theorems which will help later:
\subsubsection{Invertible Element of \texorpdfstring{$\text{Mat}_{n\times n}(\mathbb{Z})$}{Mat(nxn)(Z)}}
\begin{definition}{Invertible Element of $\text{Mat}_{n\times n}(\mathbb{Z})$}
    Let $n$ be a positive integer and $M \in \text{Mat}_{n\times n}(\mathbb{Z})$. We say that $M$ is an invertible element of $\text{Mat}_{n\times n}(\mathbb{Z})$ if there exists $N\in \text{Mat}_{n\times n}(\mathbb{Z})$ such that
    \begin{equation*}
        MN = I_n = NM,
    \end{equation*}
    in which case $N$ is unique, denoted by $M^{-1}$, and called the \textit{inverse of $M$}. We denote the subset of invertible elements of $\text{Mat}_{n\times n}(\mathbb{Z})$ by $\text{GL}_n(\mathbb{Z})$.
\end{definition}
Note that $M \in \text{Mat}_{n\times n}(\mathbb{Z})$ is invertible if and only if it is invertible in $\text{Mat}_{n\times n}(\mathbb{Q})$ and $M^{-1} \in \text{Mat}_{n\times n}(\mathbb{Z})$.

\subsubsection{Classification of $\text{GL}_n(\mathbb{Z})$}
\begin{idea}
    \begin{equation*}
        \text{GL}_n(\mathbb{Z}) = \{M \in \text{Mat}_{n\times n}(\mathbb{Z}) \vert \det(M) \in \{\pm 1\}
    \end{equation*}
\end{idea}
\begin{proof}
    If $M\in \text{GL}_n(\mathbb{Z})$, then
    \begin{equation*}
        \det(M)\det(M^{-1})=\det(I_n) = 1
    \end{equation*}
    Since $\det(M),\det(M^{-1}) \in \mathbb{Z},$ it follows that $\det(M)=\det(M^{-1})=\pm 1$.

    If $M\in \text{Mat}_{n\times n}(\mathbb{Z})$ and $\det(M) = \pm 1$, then the usual formula for $M^{-1}\in \text{Mat}_{n\times n}(\mathbb{Q})$ shows that $M^{-1}\in \text{Mat}_{n\times n}(\mathbb{Z})$. Thus, $M \in \text{GL}_n(\mathbb{Z}).$
\end{proof}
\subsubsection{Proposition 1.6: Ranks of Subgroups}
\begin{idea}
    For each free abelian group $A$ of finite rank, every subgroup $B$ of $A$ is a free abelian group and
    \begin{equation*}
        \rk{B} \le \rk{A}
    \end{equation*}
\end{idea}
\textit{Remarks:} One can drop the assumption that $A$ is of finite rank.

\begin{proof}
    We will proceed by induction on $m=\rk(A)$.

    If $m \ge 0$ and assume that for each abelian group of rank $m$, every subgroup of it is a free abelian group of rank at most $m$.

    Let $A$ be a free abelian group of rank $m+1$ and let $B \le A$. We can choose a basis $\alpha=\{a_1,\dots,a_{m+1}\}$ of $A$ and define
    \begin{equation*}
        A' = \spanz(\{a_1,\dots,a_m\}) \le A
    \end{equation*}
    Then $A'$ is a free abelian group of rank $m$.
\end{proof}

\subsection{Theorem: Smith Normal Form}
\begin{idea}
    Let $M \in \text{Mat}_{m\times n}(\mathbb{Z})$. There exist a sequence of integral elementary row and column operations that transform $M$ to a matrix of the form
    \begin{equation*}
        \left[
            \begin{array}{c|c}
                \begin{array}{ccc}
                    d_1 &        & 0   \\
                        & \ddots &     \\
                    0   &        & d_r
                \end{array} & 0 \\
                \hline
                0                          & 0
            \end{array}
            \right],
    \end{equation*}
    where $d_1|\cdots|d_n$ are positive integers. Moreover, $r=\rk(M)$ and for all $i=1,\dots,r$, $d_i=d_i(M)/d_{i-1}(M)$. In particular, $r,d_1,\dots,r_r$ are unique.
\end{idea}
\begin{definition}{$i^\text{th}$ determinant divisor of $M$}
    For $i=1,\dots,\min\{m,n\}$, define
    \begin{equation*}
        d_i(M) := \gcd\{\text{determinants of $i\times i$ minors of $M$.}\}
    \end{equation*}
    and define $d_0(M)=1$. The number $d_i(M)$ is called the $i^\text{th}$ determinant divisor of $M$.
\end{definition}
Note that if $i < \rk(M)$, then $d_i(M) > 0$.
\subsubsection{Integral Elementary Row Operations}
There are three main operations:
\begin{itemize}
    \item To interchange row $i$ and row $j$, this is equivalent to multiplying on the left by $P_{i,j}$.
    \item To multiply row $i$ by $-1$, we multiply on the left by $D_i$.
    \item To replace row $i$ with row $i$ plus $k$ times row $j$, we multiply on the left by $E_{ij}(k)$.
\end{itemize}
Note that if we were to act on the columns instead, the elementary matrices should be multiplied on the right.
\begin{idea}
    The integral elementary matrices $P_{ij},D_i,E_{ij}(k)$ generate the group $\text{GL}_n(\mathbb{Z})$.
\end{idea}
\subsubsection{Smith Normal Form Algorithm}
If $M=0$, we are done. Assume $M \neq 0$.
\begin{enumerate}
    \item Let $\delta(M) = \min\{|M_{ij}| : M_{ij} \neq 0\}$. Choose $M_{ij} \neq 0$ such that $|M_{ij}|=\delta(M)$.
    \item If $M_{ij}$ does not divide an entry in its row, say $M_{i\ell}$, and $M_{i\ell}=gM_{ij}+r$ where $q,r \in \mathbb{Z}$ and $0<r<|M_{ij}|$, then replace $\text{col}_{\ell}$ with $\text{col}_{\ell}-q\text{col}_{j}$:
    \item This results in a matrix $M'$ with $M'_{i\ell}=r$ and $\delta(M') \le r < |M_{ij}| = \delta(M)$. Let $M$ denote $M'$ now. Go to the previous step.
    \item If $M_{ij}$ does not divide an entry in its column, we do the same thing analogous to the previous step.
    \item If $M_{ij}$ divides every entry in its row and column, we can clear the other entries in row $i$ and column $j$ using $M_{ij}$ (i.e. all the other entries are $0$). Let $M$ denote the resulting matrix.
    \item If $M_{ij}$ divides every entry in $M$, skip this step. Otherwise, choose $M_{k\ell}$ such that $M_{ij} \nmid M_{k\ell}.$ Then replace row $i$ with $\text{row}_i+\text{row}_j$. Let $M$ denote the new matrix. Go to the first step.
    \item $M_{ij}$ divides every entry of $M$. Swap row 1 and row $i$ and swap column 1 and column $j$. If $M_{ij} < 0$, multiply row by $-1$.

          We look at the resulting matrix $M'$ in the bottom right corner inside the larger matrix. Let $M$ denote $M'$.
    \item Repeat steps 1 to 6 until $M'$ in the previous step is the empty matrix.
\end{enumerate}


\newpage
\section{Finitely Generated Abelian Groups}
\subsection{Definition of Finitely Generated Abelian Groups}
\begin{definition}{Finitely Generated Groups}
    A group $G$ is finitely generated (FG) if there exists $g_1,\dots,g_n \in G$ such that
    \begin{align*}
        G & = \langle g_1,\dots, g_n\rangle                  \\
          & := \bigcap_{H \le G,\quad g_1,\dots,g_n \in H} H
    \end{align*}
    or equivalently, every element of $G$ can be written as the product of integer powers of $g_i$.
\end{definition}
\begin{example}
    $D_n$ is finitely generated since $D_n = \langle r,s \rangle$
\end{example}
\begin{example}
    Every cyclic group is finitely generated, so $\mathbb{Z}, \mathbb{Z}/n\mathbb{Z}$ are finitely generated for all $n\in \mathbb{Z}_{\ge 0}$
\end{example}
\begin{example}
    If $G_1,\dots,G_n$ are finitely generated groups, then so is $G_1 \times \cdots \times G_n$.

    Consequently, $\mathbb{Z}^r \times \mathbb{Z}/d_1\mathbb{Z} \times \cdots \times \mathbb{Z}/d_n\mathbb{Z}$ is a finitely generated abelian group for all $r,n \in \mathbb{Z}_{\ge 0}$ and $d_1,\dots,d_n \in \mathbb{Z}_{>0}$.
\end{example}
\subsection{Classification of Finitely Generated Abelian Groups}
\begin{idea}
    Let $A$ be a finitely generated abelian group. Then it can be classified in either of the two (perfectly equivalent) ways:
    \begin{enumerate}
        \item There exists $r,n\in \mathbb{Z}_{\ge 0}$ and positive integers $d_1|\cdots |d_n$ such that
              \begin{equation}
                  A \cong \mathbb{Z}^r \times \mathbb{Z}/d_1\mathbb{Z} \times \cdots \times \mathbb{Z}/d_n\mathbb{Z}.
              \end{equation}
              Moreover, $r,n,d_1,\dots,d_n$ are unique and the $d_i$ are called the \textbf{invariant factors of $A$}
        \item There exist $r,s \in \mathbb{Z}_{\ge 0}$, prime numbers $p_1 < \cdots <p_s$, positive integers $n_1,\cdots, n_s$, and positive integers
              \begin{align*}
                  e_{1,1} \ge & \cdots \ge e_{1,n_1} \\
                  e_{2,1} \ge & \cdots \ge e_{2,n_2} \\
                              & \,\,\,\,\vdots       \\
                  e_{s,1} \ge & \cdots \ge e_{s,n_s}
              \end{align*}
              such that
              \begin{align*}
                  A & \cong \mathbb{Z}^r \times \mathbb{Z}/p_1^{e_{1,1}}\mathbb{Z} \times \cdots \times \mathbb{Z}/p_1^{e_{1,n_1}}\mathbb{Z} \times \cdots \times \mathbb{Z}/p_s^{e_{s,1}}\mathbb{Z} \times \cdots \times \mathbb{Z}/p_s^{e_{s,n_s}}\mathbb{Z} \\
                    & \cong \mathbb{Z}^r \times \prod_{j=1}^{n_1}\mathbb{Z}/p_1^{e_{i,j}}\mathbb{Z} \times \cdots \times \prod_{j=1}^{n_s}\mathbb{Z}/p_s^{e_{s,j}}\mathbb{Z}.
              \end{align*}
              Moreover, $r$ and $s$, and the $p_i,n_i$ and $e_{i,j}$ are all unique.
              \vspace{2mm}

              More concisely, there exist $r,t\in \mathbb{Z}_{>0}$ and prime powers $q_1\le \cdots \le q_t$ such that
              \begin{equation}
                  A \cong \mathbb{Z}^r \times \mathbb{Z}/q_1\mathbb{Z} \times \cdots \times \mathbb{Z}/q_t\mathbb{Z}
              \end{equation}
              and $r,t,q_i$ are unique. Note the $q_i$ are known as the \textbf{elementary divisors of $A$.}
    \end{enumerate}
\end{idea}
Note that $A$ is finite if and only if $r=0$. Therefore, the theorem specializes to the classification of finite abelian groups. To prove this, we will do the following:
\begin{enumerate}
    \item Show that an isomorphism as in (1) exists by reducing the problem to a result in linear algebra.
    \item We will construct an isomorphism as in (2) from an isomorphism as in (1), and vice versa.
\end{enumerate}
\subsubsection{Lemma: Homomorphism from $\mathbb{Z}^m \rightarrow A$}
\begin{idea}
    Let $(A,+)$ be an abelian group. Then $A$ is finitely generated if and only if there exists $m\in \mathbb{Z}_{>0}$ and a surjective homomorphism $\phi:\mathbb{Z}^m \rightarrow A$.
\end{idea}
\begin{proof}
    Suppose that there exist $m\in \mathbb{Z}_{>0}$ and a surjective homomorphism $\phi: \mathbb{Z}^m \rightarrow A$.

    Let $a \in A$. THen, there exists $(x_1,\dots,x_m)\in \mathbb{Z}^m$ such that $a = \phi(x_1,\dots,x_m)$. For $i=1,\dots,m$, let
    \begin{equation}
        e_i = (0,\dots,0,1,0,\dots,0)\in \mathbb{Z}^m
    \end{equation}
    where the $i^\text{th}$ entry is a $1$. Then $(x_1,\dots,x_m) = \sum_{i=1}^m x_ie_i$. Therefore
    \begin{equation}
        a = \phi\left(\sum_{i=1}^mx_ie_i\right) = \sum_{i=1}^m x_i \phi(e_i).
    \end{equation}
    Therefore $A = \langle \phi(e_1),\dots,\phi(e_n)\rangle$, so $A$ is finitely generated.

    To show the converse, suppose $A$ is finitely generated and let $a_1,\dots,a_m\in A$ such that $A = \langle a_1,\dots, a_m\rangle$. Define
    \begin{equation}
        \phi: \mathbb{Z}^m \rightarrow A
    \end{equation}
    by $\phi(x_1,\dots,x_m)=x_1a_1+\cdots + x_ma_m$. Then $\phi$ is a surjective homomorphism.
\end{proof}
\subsection{First Reduction}
Let $A$ be a finitely generated abelian group and let $\phi: \mathbb{Z}^m \rightarrow A$ be a surjective homomorphism. Let $B = \ker\phi \le \mathbb{Z}^m$. By the $1^\text{st}$ isomorphism theorem, we have $A \cong \mathbb{Z}^m/B$. If:
\begin{align*}
    B & = d_1\mathbb{Z}\times \cdots \times d_n\mathbb{Z} \times \{0\} \times \cdots \times \{0\}                                                                  \\
      & \le \underbrace{\mathbb{Z} \times \cdots \times \mathbb{Z}}_{n} \times \underbrace{\mathbb{Z} \times \mathbb{Z} \times \mathbb{Z}}_{r:=m-n} = \mathbb{Z}^m
\end{align*}
then
\begin{align*}
    A & \cong \mathbb{Z}^m/B                                                                                                                                                                                       \\
      & \cong \frac{{\mathbb{Z} \times \cdots \times \mathbb{Z}} \times {\mathbb{Z} \times \mathbb{Z} \times \mathbb{Z}}}{d_1\mathbb{Z}\times \cdots \times d_n\mathbb{Z} \times \{0\} \times \cdots \times \{0\}} \\
      & \cong \mathbb{Z}/d_1\mathbb{Z} \times \cdots \times \mathbb{Z}/d_n\mathbb{Z} \times \mathbb{Z}/\{0\} \times \mathbb{Z}/\{0\}                                                                               \\
      & \cong \mathbb{Z}/d_1\mathbb{Z} \times \cdots \times \mathbb{Z}/d_n\mathbb{Z} \times \mathbb{Z} \times \cdots \times \mathbb{Z}                                                                             \\
      & = \mathbb{Z}d_1\mathbb{Z} \times \cdots \times \mathbb{Z}/d_n\mathbb{Z} \times \mathbb{Z}^r                                                                                                                \\
      & \cong \mathbb{Z}^r \times \mathbb{Z}/d_1\mathbb{Z} \times \cdots \times \mathbb{Z}/d_n \mathbb{Z}
\end{align*}
where the third line is a result in the exercise sheet. However, $B$ might not have this form.

\subsubsection{Lemma}
\begin{idea}
    If $\phi: G_1\rightarrow G_2$ is an isomorphism, $N_1 \trianglelefteq G_1$, $N_2 \trianglelefteq G_2$, and $\phi(N_1)=N_2$, then
    \begin{align*}
        G_1/N_1 \rightarrow G_2/N_2 \\
        g_1N_1 \mapsto \phi(g_1)N_2
    \end{align*}
    is a well defined isomorphism.
\end{idea}
We can actually make this result more general:

\begin{idea}
    Let $\phi: G_1\rightarrow G_2$ be a homomorphism and $N_2 \trianglelefteq G_2$. Then $\phi^{-1}(N_2) \trianglelefteq G_1$ and the map
    \begin{align*}
        G_1/\phi^{-1}(N_1) \rightarrow G_2/N_2 \\
        g_1\phi^{-1}(N_1) \mapsto \phi(g_1)N_2
    \end{align*}
    is a well defined injective homomorphism with $\im{\phi}=\phi(G_1)N_2/N_2$.
\end{idea}
To prove that an inversion factor decomposition exists, it suffices to construct an isomorphism $\mathbb{Z}^m \rightarrow \mathbb{Z}^m$ that maps $B$ to $d_1\mathbb{Z} \times \cdots \times d_n\mathbb{Z} \times \{0\} \times \cdots \times \{0\},$ where $d_1,\dots,d_n \in \mathbb{Z}_{>0}$ such that $d_1|\cdots|d_n$. Indeed, if this is the case then from the first reduction section, we would have shown that
\begin{equation}
    A \cong \mathbb{Z}^r \times \mathbb{Z}/d_1\mathbb{Z} \times \mathbb{Z}/d_n\mathbb{Z}
\end{equation}
To construct an isomorphism $\mathbb{Z}^m \rightarrow \mathbb{Z}^m$ that maps $B$ to the prescribed subgroup of $\mathbb{Z}^m$, we need a better understanding of group isomorphisms to $\mathbb{Z}^m$, especially their subgroups and isomorphisms between them. This leads us to the study of free abelian groups, which we have done in the first section.

\subsection{Second Reduction}
Let $A$ be a finitely generated abelian group, $\phi: \mathbb{Z}^m \rightarrow A$ is a surjective homomorphism, and $B=\ker\phi \le \mathbb{Z}^m$. Recall that it suffices to construct an isomorphism $\mathbb{Z}^m \rightarrow \mathbb{Z}^m$ that maps $B$ to $$d_1\mathbb{Z} \times \cdots \times d_n\mathbb{Z} \times \{0\} \times \cdots \times \{0\} \le \mathbb{Z}^m$$
for some positive integers $d_1|\cdots | d_n$.

Since $B \le \mathbb{Z}^m$, we now know that $B$ is a free abelian group of rank $n \le m$. Let $r=m-n$. It then suffices to prove the following theorem (too lazy to write proof, can be found in Lec 20):
\begin{idea}
    Let $C$ be a free abelian group of finite rank $m$ and let $B \le C$. Then $B$ is a free abelian group of rank at most $m$. Let $n=\rk(B) \le m$.
    \vspace{2mm}

    Then, there exists bases $\beta =\{b_1,\dots,b_n\}$ of $B$ and $\gamma=\{c_1,\dots,c_m\}$ of $C$ and positive integers $d_1|\cdots | d_n$ such that
    \begin{equation*}
        b_i = d_i c_i
    \end{equation*}
    for all $i=1,\dots, n$. Moreover, $d_1,\dots,d_n$ are unique.
\end{idea}
Indeed, suppose that this theorem holds and apply it to $B=\ker\phi \le \mathbb{Z}^n = C$. The isomorphism
\begin{equation*}
    \mathbb{Z}^m = C \xrightarrow{[\cdot]_\gamma} \mathbb{Z}^m
\end{equation*}
maps $b_i = d_ic_i$ to $d_ie_i$ for all $i=1,\dots, n$. Therefore, the isomorphism maps $B$ to $d_1\mathbb{Z} \times \cdots \times d_n\mathbb{Z}\times \{0\} \times \cdots \times \{0\}$.

We will prove a more general theorem.
\begin{idea}
    Let $B$ and $C$ be free abelian groups of finite ranks $n$ and $m$, respectively. Let $\Psi: B\rightarrow C$ be a homomorphism.
    \vspace{2mm}

    Then there exists bases $\beta=\{b_1,\dots,b_n\}$ of $B$ and $\gamma=\{c_1,\dots,c_m\}$ of $C$, there exists a positive integer $r \le m,n$ and there exists positive integers $d_1|\cdots | d_r$ such that
    \begin{equation*}
        \Psi(b_i) = \begin{cases}
            d_ic_i & 1\le i \le r \\
            0      & r < i \le n
        \end{cases}
    \end{equation*}
    or equivalently
    \begin{equation*}
        [\Psi]^\beta_\gamma =
        \left[
            \begin{array}{c|c}
                \begin{array}{ccc}
                    d_1 &        & 0   \\
                        & \ddots &     \\
                    0   &        & d_r
                \end{array} & 0 \\
                \hline
                0                          & 0
            \end{array}
            \right]
    \end{equation*}
    Moreover, $r,d_1,\dots,d_r$ are unique.
\end{idea}
Let $\beta_0,\gamma_0$ be bases of $B,C$, respectively. The theorem is equivalent to the assertion that there exists matrices $P\in \text{GL}_m(\mathbb{Z})$, $Q\in \text{GL}_n(\mathbb{Z})$ such that
\begin{equation*}
    P[\Psi]_{\gamma_0}^{\beta_0}Q =     \left[
        \begin{array}{c|c}
            \begin{array}{ccc}
                d_1 &        & 0   \\
                    & \ddots &     \\
                0   &        & d_r
            \end{array} & 0 \\
            \hline
            0                          & 0
        \end{array}
        \right]
\end{equation*}
for some positive integers $d_1|\cdots |d_r$, and $r,d_1,\dots,d_r$ are unique. It turns out that slightly more is true.
\subsection{Uniqueness}
Now that we have shown it is possible to create an invariant factor decomposition, let us show this is unique. To do this, we make use of torsion subgruops.
\subsubsection{Torsion Subgroup}
\begin{definition}{Torsion subgroup of $A$}
    Let $A$ be an abelian group. For each $n\in \mathbb{Z}_{>0}$, we define the $n$-torsion subgroup of $A$ to be
    \begin{equation*}
        A[n] := \{a\in A: na = 0\}
    \end{equation*}
    we define the $n$-power torsion subgroup of $A$ to be
    \begin{equation*}
        A[n^\infty] := \left\{a\in A: n^k a = 0 \text{ for some } k \in \mathbb{Z}_{\ge 0}\right\}
    \end{equation*}
    and we define the torsion subgroup of $A$ to be
    \begin{equation*}
        \text{Tor}(A) := \left\{a\in A: ma=0 \text{ for some } m\in \mathbb{Z}_{>0}\right\} = \bigcup_{n\in \mathbb{Z}_{>0}}A[n].
    \end{equation*}
\end{definition}
\subsubsection{Proposition 2.1 Isomorphisms of Torsion Groups}
\begin{idea}
    Let $A$ be a finitely generated group. If $A \cong \mathbb{Z}^r \times T$, where $r \in \mathbb{Z}_{\ge 0}$ and $T$ is a finite abelian group, then $T \cong \text{Tor}(A)$ and $\mathbb{Z}^r \cong A/\text{Tor}(A)$.
    \vspace{2mm}

    Consequently, $T$ is unique up to isomorphism and $r$ is unique.
\end{idea}
\begin{proof}
    First, note that if $\phi: B\rightarrow C$ is an isomorphism between abelian groups, then $\phi(\text{Tor}B)=\text{Tor}C$, so $\phi$ restricts to an isomorphism $\phi: \Tor B \rightarrow \Tor C$.

    Let $\phi: A\rightarrow \mathbb{Z}^r \times T$ be an isomorphism as in the proposition statement.

    Since $\Tor(\mathbb{Z}^r \times T) = \{(0,\dots,0)\} \times T \cong T$, we have $\Tor(A) \cong \{0\} \times T \cong T$.

    Also since $\phi(\Tor A)= \{0\} \times T$, the map
    \begin{align*}
        A/\Tor A   & \rightarrow \mathbb{Z}^r \times T/(\{0\} \times T) \\
        a + \Tor A & \mapsto \phi(a) + (\{0\} \times T)
    \end{align*}
    is a well defined isomorphism. Therefore,
    \begin{align*}
        A/\Tor A & \cong \mathbb{Z}^r \times T/(\{0\} \times T) \\
                 & \cong \mathbb{Z}^r/\{0\} \times T/T          \\
                 & \cong \mathbb{Z}^r \times \{0\}              \\
                 & \cong \mathbb{Z}^r
    \end{align*}
\end{proof}
\subsubsection{Proposition 2.2: Isomorphisms of $n$-power Torsion Groups}
\begin{idea}
    Let $A$ be a finitely generated abelian group. If $A \cong \mathbb{Z}^r \times P \times B$, where $r \in \mathbb{Z}_{\ge 0}$, $P$ is a finite abelian group with $|P|=p^k$ for some prime $p$, and $B$ is a finite abelian group with $p\nmid |B|$, then
    \begin{equation*}
        A[p^\infty] \cong P
    \end{equation*}
\end{idea}
\subsubsection{Proposition 2.3: Uniqueness}
To prove uniqueness of the invariant factor and elementary divisor decompositions, it remains for us to prove that $d_1,\dots,d_n$ and $e_{i,1},\dots,e_{i,n_i}$ are unique for each $i=1,\dots, s$.
\begin{idea}
    Let $A$ be a finite abelian group. If $A \cong \mathbb{Z}/a_1\mathbb{Z} \times \cdots \times \mathbb{Z}/a_n \mathbb{Z}$ for positive integers $a_1|\cdots | a_n,$ then $a_1,\dots,a_n$ are uniquely determined by $A$.
\end{idea}
To prove this, we make use of the following lemma:
\begin{idea}
    If $a\in \mathbb{Z}$, $b\in \mathbb{Z}_{>0}$, then
    \begin{equation*}
        a(\mathbb{Z}/b\mathbb{Z}) = \gcd(a,b)\mathbb{Z}/b\mathbb{Z}
    \end{equation*}
\end{idea}
\begin{proof}
    We have
    \begin{align*}
        a(\mathbb{Z}/b\mathbb{Z}) & = \{a(n+b\mathbb{Z}): n\in \mathbb{Z} \}                                                                       \\
                                  & = \{an+b\mathbb{Z} : n \in \mathbb{Z}\}
                                  & = \{x+b\mathbb{Z} : x\in a\mathbb{Z}\}                                                                         \\
                                  & = \text{ image of $a\mathbb{Z}$ under the quotient homomorphism }\mathbb{Z} \rightarrow \mathbb{Z}/b\mathbb{Z} \\
                                  & = (a\mathbb{Z}+b\mathbb{Z})/b\mathbb{Z}                                                                        \\
                                  & = \gcd(a,b)\mathbb{Z}/b\mathbb{Z}
    \end{align*}
    where the last line follows from Bezout's Lemma.
\end{proof}

and another lemma:
\begin{idea}
    Let $p$ be a prime and let $b\in \mathbb{Z}_{>0}$, for all $e\in \mathbb{Z}_{>0}$, we have
    \begin{equation*}
        \frac{p^{e-1}(\mathbb{Z}/b\mathbb{Z})}{p^e(\mathbb{Z}/b\mathbb{Z})} \cong \begin{cases}
            \{0\}                  & p^e \nmid b \\
            \mathbb{Z}/p\mathbb{Z} & p^e | b
        \end{cases}
    \end{equation*}
\end{idea}
% Let $m,n \in \mathbb{Z}_{>0}$ with $m|n$. For any gruop $(G,\cdot)$, we have $G^n = \{g^n: g\in G\} = \{\left(g^{n/m}\right)^m : g\in G\} \subseteq G^m$. If $(A,+)$ is an abelian group written additively, then the above is written as $mA \subseteq nA$.
\begin{proof}
    Let $e\in \mathbb{Z}_{>0}$. By Lemma 1, we have
    \begin{equation*}
        p^{e-1}(\mathbb{Z}/b\mathbb{Z}) = \gcd(p^{e-1},b)\mathbb{Z}/b\mathbb{Z}
    \end{equation*}
    and
    \begin{equation*}
        p^e(\mathbb{Z}/b\mathbb{Z}) = \gcd(p^e,b)\mathbb{Z}/b\mathbb{Z}
    \end{equation*}
    Suppose $p^e \nmid b$. Then $\gcd(p^e,b)=\gcd(p^{e-1},b)$. Therefore $p^{e-1}(\mathbb{Z}/b\mathbb{Z})=p^e(\mathbb{Z}/b\mathbb{Z})$, and
    \begin{equation*}
        \frac{p^{e-1}(\mathbb{Z}/b\mathbb{Z})}{p^e(\mathbb{Z}/b\mathbb{Z})} \cong \{0\}
    \end{equation*}
    Suppose $p^e \mid b$. Then $\gcd(p^e,b)=p^e$ and $\gcd(p^{e-1},b)=p^{e-1}$. Therefore:
    \begin{align*}
        \frac{p^{e-1}(\mathbb{Z}/b\mathbb{Z})}{p^e(\mathbb{Z}/b\mathbb{Z})} & = \frac{p^{e-1}\mathbb{Z}/b\mathbb{Z}}{p^e\mathbb{Z}/b\mathbb{Z}} \\
                                                                            & \cong p^{e-1}\mathbb{Z}/p^e\mathbb{Z}
    \end{align*}
    where we have used the third isomorphism theorem. The map
    \begin{align*}
        \mathbb{Z}/p\mathbb{Z} & \rightarrow p^{e-1}\mathbb{Z}/p^e\mathbb{Z} \\
        n+p\mathbb{Z} \mapsto p^{e-1}n+p^e\mathbb{Z}
    \end{align*}
    is a well defined isomorphism (consider $\mathbb{Z} \rightarrow p^{e-1}\mathbb{Z}/p^e\mathbb{Z}$ and $n \mapsto p^{e-1}n+p^e\mathbb{Z}$ and apply the first isomorphism theorem).

    Therefore,
    \begin{equation*}
        p^{e-1}\mathbb{Z}/p^e\mathbb{Z}  \cong \mathbb{Z}/p\mathbb{Z}
    \end{equation*}
    and we are done.
\end{proof}

Let $\phi: A \rightarrow \mathbb{Z}/a_1\mathbb{Z} \times \cdots \times \mathbb{Z}/a_n\mathbb{Z} =: B$ be an isomorphism, where $a_1|\cdots | a_n$ are positive integers.

Let $p$ be a prime and $e\in \mathbb{Z}_{>0}$. Then $\phi(p^{e-1}A)=p^{e-1}B$ and $\phi(p^eA)=p^eB$. Therefore, the map
\begin{align*}
    p^{e-1}A/p^eA & \rightarrow p^{e-1}B/p^eB \\
    x+p^eA        & \mapsto \phi(x) + p^eB
\end{align*}
is an isomorphism. We have
\begin{align*}
    \frac{p^{e-1}A}{p^eA} & \cong \frac{p^{e-1}B}{p^eB}                                                                                                                                                       \\
                          & = \frac{p^{e-1}(\mathbb{Z}/a_1\mathbb{Z} \times \cdots \times \mathbb{Z}/a_n\mathbb{Z})}{p^e(\mathbb{Z}/a_1\mathbb{Z} \times \cdots \times \mathbb{Z}/a_n \mathbb{Z})}            \\
                          & = \frac{p^{e-1}(\mathbb{Z}/a_1\mathbb{Z})\times \cdots \times p^{e-1}(\mathbb{Z}/a_n\mathbb{Z})}{p^e(\mathbb{Z}/a_1\mathbb{Z})\times \cdots \times p^e(\mathbb{Z}/a_n\mathbb{Z})} \\
                          & \cong \frac{p^{e-1}(\mathbb{Z}/a_1\mathbb{Z})}{p^e(\mathbb{Z}/a_1\mathbb{Z})} \times \cdots \times \frac{p^{e-1}(\mathbb{Z}/a_n\mathbb{Z})}{p^e(\mathbb{Z}/a_n\mathbb{Z})}        \\
                          & \cong (\mathbb{Z}/p\mathbb{Z})^r
\end{align*}
where $r$ is the number of $i\in \{1,\dots,n\}$ such that $p^e | a_i$.

Let $i\in \{1,\dots,n\}$.

If $p^e | a_i$, then $p^e | a_i, a_{i+1},\dots, a_n$, so there are at least $n-(i-1)$ elements $j\in \{1,\dots,n\}$ such that $p^e | a_j$ and therefore

$$r \ge n - (i-1).$$

If $p^e \nmid a_i$, then $p^e \nmid a_1,\dots, a_i$, so $r \le n-i$.

Therefore $p^e| a_i$ if and only if $p^{e-1}A/p^eA \cong (\mathbb{Z}/p\mathbb{Z})^r$ for some $r \ge n-(i-1)$ if and only if $|p^{e-1}A/p^eA| = p^r$ for some $r \ge n-(i-1)$.

Thus the prine powers that divide each $a_i$ are determined by $A$ and therefore the $a_i$ are determined by $A$.
\subsection{Elementary Divisors to Inversion Factors}
Now we wish to show we can create an isomorphism between the two ways to decompose an abelian group, and thus show that they are equivalent.

Suppose that $A \cong \mathbb{Z}^r \times \prod_{i=1}^s \prod_{j=1}^{n_i} \mathbb{Z}/(p_i^{e_{i,j}}\mathbb{Z})$. Recall that $p_1<\cdots < p_S$ and $e_{i,1}\ge \cdots \ge e_{i,n_i}$ for all $i=1,\dots, s$.

Let $n=\underset{1\le i \le s}{\max}(n_i)$. For each $i\in \{1,\dots, s\},$ if $n_i < n$, define $e_{ij}=0$ for all $n_i < j \le n$. Then
\begin{align*}
    A & \cong \mathbb{Z}^r \times \prod_{i=1}^s \prod_{j=1}^n \mathbb{Z}/(p_i^{e_{i,j}}\mathbb{Z})  \\
      & \cong \mathbb{Z}^r \times \prod_{j=1}^n \prod_{i=1}^s \mathbb{Z}/(p_i^{e_{i,j}}\mathbb{Z}).
\end{align*}
Now,
\begin{equation*}
    \prod_{i=1}^s \mathbb{Z}/(p_i^{e_{i,j}}\mathbb{Z}) \cong \mathbb{Z}/(p_1^{e_{1,j}}\cdots p_n^{e_{n,j}}\mathbb{Z})
\end{equation*}
for all $j=1,\dots, n$.

Define $d_i = p_1^{e_{1,n-i+1}\cdots p_n^{e_{n,n-i+1}}}$ for $i=1,\dots, n$. Then $d_1|\cdots | d_n$ and
\begin{equation*}
    A \cong \mathbb{Z}^r \times \mathbb{Z}/d_1\mathbb{Z} \times \cdots \times \mathbb{Z}/d_n\mathbb{Z}
\end{equation*}

\section{Group Actions}
\subsection{Permutation Representations}
Let $G$ be a gruop and let $X$ be a set.
\begin{definition}{Permutation Representation of $G$ on $X$}
    A permutation of $G$ on $X$ is a homomorphism $\phi:G\rightarrow S_X$.
\end{definition}
\begin{example}
    If $G$ is a permutation group on $X$, i.e. $G \le S_x$, then the inclusion $i:G\rightarrow S_x$ and $g\mapsto g$ is a permutation representation.
\end{example}
If $\phi:G\rightarrow S_x$ is a permutation representation, we define
\begin{align*}
    \alpha = \alpha_\phi: G \times X & \rightarrow X \\
    (g,x) \mapsto \phi(g)(x)
\end{align*}
\subsection{Properties of Permutation Representations}
\begin{enumerate}
    \item For all $x\in X$, we have
          \begin{equation*}
              \alpha(e,x)=\phi(e)(x)=\id_x(x)=x
          \end{equation*}
    \item For all $g_1,g_2\in G$ and $x\in X$, we have
          \begin{align*}
              \alpha(g_1g_2,x) & = \phi(g_1g_2)(x)              \\
                               & = (\phi(g_1)\circ\phi(g_2))(x) \\
                               & = \phi(g_1)(\phi(g_2)(x))      \\
                               & = \phi(g_1)(\alpha(g_2,x))     \\
                               & =\alpha(g_1,\alpha(g_2,x))
          \end{align*}
\end{enumerate}
If we write $g \cdot x$ instead of $\alpha(g,x)$ for all $g\in G, x\in X$, then the above becomes:
\begin{enumerate}
    \item For all $x\in X$, $e\cdot x = x$.
    \item For all $g_1,g_2\in G$ and $x\in X$ we have $(g_1g_2)\cdot x = g_1 \cdot (g_2 \cdot x)$
\end{enumerate}
\subsection{Definition of Group Action}
\begin{definition}{Group Action}
    A group action of $G$ on $X$ is a map $\alpha: G\times X \rightarrow X$ such that
    \begin{enumerate}
        \item For all $x\in X$, $\alpha(e,x)=x$.
        \item For all $g_1,g_2 \in G$ and $x\in X$,
              \begin{equation*}
                  \alpha(g_1g_2,x)=\alpha(g_1,\alpha(g_2,x))
              \end{equation*}
    \end{enumerate}
\end{definition}
Note that we often use other notation to denote group actions, such as $g\cdot x$, $g*x$, ${}^gx$, or $gx$ instead of $\alpha(g,x)$.
\subsubsection{Proposition 3.1: Permutation Representations are Group Actions}
\begin{idea}
    If $\phi:G\rightarrow S_x$ is a permutation representation, then
    \begin{align*}
        \alpha_\phi:G\times X & \rightarrow X \\
        (g,x) \mapsto \phi(g)(x)
    \end{align*}
    is a group action.
\end{idea}
\subsubsection{Proposition 3.2: Permutation Representation from Group Actions}
\begin{idea}
    If $\alpha:G\times X \rightarrow X$ is a group action, then for all $g\in G$ the map
    \begin{align*}
        \phi_\alpha(g) = \alpha(g,\cdot): X & \rightarrow X       \\
        x                                   & \mapsto \alpha(g,x)
    \end{align*}
    is an element of $S_x$ and the map $\phi_a:G\rightarrow S_x$ is a permutation representation.
\end{idea}
\subsection{Theorem 3.1: Relating Permutation Representations and Group Actions}
\begin{idea}
    The maps
    \begin{align*}
        \{\text{permutation representations of $G$ on $X$}\} \leftrightarrow \{\text{group actions of $G$ on $X$}\} \\
        \phi \mapsto \alpha_\phi                                                                                    \\
        \phi_\alpha \mapsfrom \alpha
    \end{align*}
    are inverses of each other.
\end{idea}
Thus, we can switch back and forth between permutation representations and group actions of $G$ on $X$.
\subsection{Right Group Action}
What we have called a group action is more precisely called a left group action.

We can similarly define right group actions.
\begin{definition}{Right Group Actions}
    A right group action of $G$ on $X$ is a map $\alpha: X \times G \rightarrow X$, $(x,g)\mapsto x\cdot g$ such that
    \begin{enumerate}
        \item For all $x\in X$, $x\cdot e=x$
        \item For all $g_1,g_2\in G$ and $x\in X$, $x\cdot (g_1g_2) = (x\cdot g_1)\cdot g_2$.
    \end{enumerate}
\end{definition}
\subsubsection{Proposition 3.3:}
\begin{idea}
    If $\alpha: G\times X\rightarrow X$ is a left group action, then $\alpha_R: X\times G \rightarrow X$ defined by
    \begin{equation*}
        \alpha_R(x,g)=\alpha(g^{-1},x)
    \end{equation*}
    for all $x\in X$, $g\in G$, is a right group action.
    \vspace{2mm}

    Similarly, if $\alpha: X\times G\rightarrow X$ is a right group action, then the map $\alpha_L : G\times X\rightarrow X$ defined by
    \begin{equation*}
        \alpha_L(g,x)=\alpha(x,g^{-1})
    \end{equation*}
    for all $g\in G$, $x\in X$, is a left group action.
    \vspace{2mm}

    The maps
    \begin{align*}
        \{\text{left actions of $G$ on $X$}\} \leftrightarrow \{\text{right actions of $G$ on $X$}\} \\
        \alpha \mapsto \alpha_R                                                                      \\
        \alpha_L \mapsfrom \alpha
    \end{align*}
    are inverses of each other.
\end{idea}
\subsection{Terminology}
Let $\alpha: G \times X\rightarrow X$ be a group action and $(g,x)\mapsto g\cdot x$ and let $\phi: G\rightarrow S_x$ be the corresponding permutation representation.
\begin{definition}{Faithfulness}
    $\phi$ and $\alpha$ are faithful if $\phi$ is injective, or equivalently for all $g\in G \setminus \{e\}$, there exists $x\in X$ such that $g\cdot x \neq x$.
\end{definition}
\begin{definition}{Fixed Points and Stabilisers}
    For all $g\in G$, $\fix(g)=X^g:=\{x\in X: g\cdot x=x\}$ is the set of fixed points of $g$.
    \vspace{2mm}

    For all $x\in X$, $\stab(x)=G_x:=\{g\in G: g\cdot x=x\}$ is the stabiliser of $x$ (or isotropy group of $x$).
\end{definition}
\begin{idea}
    For all $x\in X$, $\stab(x) \le G$.
\end{idea}
\begin{definition}{Freeness}
    $\alpha$ is free if for all $g\in G \setminus \{e\}$, we have $\fix(g) = \phi$.
\end{definition}
\begin{idea}
    If $\alpha$ is free, it implies that $\alpha$ is faithful.
\end{idea}
\begin{definition}{Invariants (noun)}
    The set of invariants of $G$ is:
    \begin{align*}
        X^G : & = \{x\in X: g\cdot x=x \text{ for all }g\in G\} \\
              & = \bigcap_{g\in G}\fix(g)
    \end{align*}
\end{definition}
\begin{definition}{Invariant/stable (adj) and Fixed}
    Let $S \subseteq X$.
    \begin{itemize}
        \item $S$ is invariant/stable under $g\in G$ if $g\cdot S = S$ where $g\cdot S := \{g\cdot s: s\in S\}$.
        \item $S$ is invariant/stable under $G$ if $g\cdot S = S$ for all $g\in G$.
        \item $S$ is fixed by $g\in G$ if $S\subseteq \fix(g)$.
        \item $S$ is fixed by $G$ if $S\subseteq X^G$.
    \end{itemize}
\end{definition}
\begin{definition}{Orbits}
    For $x\in X$, $G\cdot x = \orb(x):=\{g\cdot x:g\in G\}$ is the orbit of $x$ under $\alpha$.
    \vspace{2mm}

    The set of orbits of $\alpha$ is
    \begin{equation*}
        G\bs X = \{G\cdot x:x\in X\}
    \end{equation*}
\end{definition}
\begin{definition}{Transitivity}
    $\alpha$ is transitive if one of the following equivalent conditions hold:
    \begin{enumerate}
        \item For all $x,y\in X$, there exists $g\in G$ such that $g\cdot x=g$.
        \item (If $X \neq \phi$) there exists $x\in X$ such that $X=G\cdot x$.
    \end{enumerate}
    $\alpha$ is simply transitive if $\alpha$ is transitive and free, or equivalently for all $x,y\in X$, there exists a unique $g\in G$ such that $g\cdot x = y$.
\end{definition}
\begin{example}
    If $G\le S_x$, then $g\cdot x:= g(x)$ is an action of $G$ on $X$.
\end{example}
\begin{example}
    If $X$ is a geometric object and $G=\sym(X)$, then $G \subseteq S_x$, so we have an action of $G$ on $X$.
    \vspace{2mm}

    For example, if $X=P_n=\{(\cos(2\pi k/n),\sin(2\pi k/n)):k=0,\dots,n-1\}$, $G=D_n=\sym(P_n)$, then $G$ acts on $X$.
    \vspace{2mm}

    If $F$ is a field, $X=F^n$, $G=\text{GL}_n(F)$.
\end{example}
\begin{example}
    If $F$ is a field,
    \begin{align*}
        X & = P^{n-1}(F) := \{\ell : \ell \text{ is a line in $F^n$ through $0$}, \\
        G & = \text{GL}_n(F),
    \end{align*}
    then
    \begin{equation*}
        g\cdot \ell := g\ell = \{gv:v\in \ell\}
    \end{equation*}
    defines an action of $\text{GL}_n(F)$ on $X$.
\end{example}
\begin{example}
    $G=\text{SL}_2(\mathbb{R})$, $X=\{z\in \mathbb{C}|\Im(z)>0\}$, then
    \begin{equation*}
        \begin{pmatrix}
            a & b \\c&d
        \end{pmatrix} \cdot z := \frac{az+b}{cd+d}
    \end{equation*}
    defines an action of $\text{SL}_2(\mathbb{R})$ on $X$.
\end{example}
\begin{example}
    $G=S_n$, $X=\mathbb{C}[x_1,\dots,x_n]$, then
    \begin{equation*}
        \sigma\cdot p(x_1,\dots,x_n):=p(x_{\sigma(1)},\dots,x_{\sigma(n)})
    \end{equation*}
    defines an action of $S_n$ on $\mathbb{C}[x_1,\dots,x_n]$.
    \begin{equation*}
        \mathbb{C}[x_1,\dots,x_n]^{S_n} =\{\text{invariant polynomials in $n$-variables}\}
    \end{equation*}
\end{example}
\begin{example}
    $X=G$, $g\cdot x:= gx$ defines a left action of $G$ on itself (The left regular action of $G$ on itself).

    The corresponding permutation representation $\phi:G\rightarrow S_G$ is the left regular permutation representation of $G$.
\end{example}
\begin{idea}
    Cauchy's Theorem tells us that the action in the previous example is faithful. The action is simply transitive.
\end{idea}
\begin{example}
    $X=G$, then $g\cdot x=gxg^{-1}$ defines a left action of $G$ on itself.
    \begin{itemize}
        \item The permutation representation: $G\rightarrow \text{Int}(G) \le \text{Aut}(G) \subseteq S_G$
        \item For every $x\in X=G$, $\stab(x)=C_G(x)$.
        \item For $g\in G$, $\fix(g)=C_G(g)$.
        \item $X^G=Z(G)$.
        \item $H\le G$, then $H$ is invertible/stable under $G$ if and only if $H \trianglelefteq G$
        \item $H \le G$, then $H$ is fixed by $G$ if and only if $H \subseteq Z(G)$.
        \item For $x\in X=G$, $G\cdot x = \{gxg^{-1}:g\in G\}$, which is the conjugacy class of $x$.
    \end{itemize}
\end{example}
\begin{example}
    Let $H\subseteq G$. THen $H\times G\rightarrow G$, $(h,g)\mapsto hg$ is an action of $H$ on $G$.
    \begin{itemize}
        \item The action is free.
        \item For $g\in G$, its orbit is $H\cdot g=Hg$.
        \item The set of orbits $H \bs G$ is the set of right cosets of $H$ in $G$ (also denoted by $H\bs G$),
    \end{itemize}
\end{example}
\begin{example}
    $H\subseteq G$, $X=G/h$. For all $g\in G$ and $g_o H\in X = G/H$, $g\cdot g_o H := gg_oH$ defines an action of $G$ on $G/H$
    \begin{itemize}
        \item If $\phi:G\rightarrow S_X$ is the corresponding permutation representation then $\ker\phi \le H$.
    \end{itemize}
\end{example}
\begin{example}
    Let $X=\{H:H\le G\}$, $g\cdot H=gHg^{-1}$.
    \begin{itemize}
        \item $\stab(H)=N_G(H)$
        \item $X^G=\{H:H\triangleright G\}$
        \item $G\cdot H =\text{Orb}(H) = \{gHg^{-1}:g\in G\}$
    \end{itemize}
\end{example}
\subsection{Fundamental Results}
Let $\alpha:G\times X\rightarrow X$, $g\mapsto g\cdot x$ be a group action.
\subsubsection{Orbit Stabiliser Theorem}
\begin{idea}
    For all $x\in X$, the map
    \begin{align*}
        G/\stab(x) & \rightarrow G\cdot x \\
        g\stab(x) \mapsto g\cdot x
    \end{align*}
    is a well=defined bijection. Thus,
    \begin{equation*}
        |G\cdot x| = [G:\stab(x)],
    \end{equation*}
    and if $|G|<\infty$, then
    \begin{equation*}
        |G\cdot x| = |G|/|\stab(x)|
    \end{equation*}
\end{idea}
\begin{proof}
    Let $g,g'\in G$. THen
    \begin{align*}
        g'\cdot x = g\cdot x & \iff (g^{-1}g')\cdot x = x \\
        \iff g^{-1}g' \in \stab(x)                        \\
        \iff g'\stab(x) = g\stab(x).
    \end{align*}
    Thus, the map is well defined and injective. It is clearly surjective.
\end{proof}
\subsubsection{Proposition 3.4}
\begin{idea}
    For all $x,y\in X$, we have
    \begin{equation*}
        G\cdot x \cap G\cdot y \neq \phi \implies G\cdot x = G\cdot y
    \end{equation*}
\end{idea}
\subsubsection{Corollary}
\begin{idea}
    Let $X=\coprod_{g\cdot x \in G \bs X} G\cdot x$. Therefore:
    \begin{equation*}
        |X|=\sum_{G\cdot x\in G\bs X}|G\cdot x| = \sum_{G\cdot x\in G\bs X}[G:\stab(x)]
    \end{equation*}
\end{idea}
\subsubsection{Proposition 3.5}
\begin{idea}
    For all $x\in X$ and $g\in G$
    \begin{equation*}
        \stab(g\cdot x)=g\stab(x)g^{-1}
    \end{equation*}
\end{idea}
\begin{proof}
    For $g'\in G$, $g'\cdot (g\cdot x)=g\cdot x$ if and only if $(g^{-1}g'g)\cdot x = x$ if and only if $g^{-1}g'g\in \stab(x)$ which is true if and only if $g'\in g\stab(x)g^{-1}$.
\end{proof}
\subsubsection{Burnside's Lemma}
Also known as Cauchy-Frobenius Lemma/Not Burnside's Lemma/ORbit-Counting Theorem:
\begin{idea}
    Let $|G||G\bs X|=\sum_{g\in G}|\fix(G)|$. If $|G|<\infty$, we have
    \begin{equation*}
        |G\bs X| = \frac{1}{|G|}\sum_{g\in G}|\fix(g)|
    \end{equation*}
    i.e. the number of orbits is the average number of fixed points of an element of $G$.
\end{idea}
\begin{proof}
    We have
    \begin{equation*}
        \coprod_{g\in G}\{g\}\times \fix(g) = \{(g,x)\in G\times X:g\cdot x=x\} = \coprod_{x\in X}\stab(x) \times \{x\}.
    \end{equation*}
    Therefore:
    \begin{equation*}
        \sum_{g\in G}|\fix(g)|=\left|\{(g,x)\in G\times X:g\cdot x=x\}\right|=\sum_{x\in X}|\stab(x)|.
    \end{equation*}
    Since $X=\coprod_{G\cdot y \in G\bs X} G\cdot y$, we have
    \begin{equation*}
        \sum_{x\in X}|\stab(x)| = \sum_{G\cdot y\in G\bs X}\sum_{x\in G\cdot y}|\stab(x)|.
    \end{equation*}
    For all $y\in X$ and $x\in G\cdot y$, $x=g\cdot y$ for some $g\in G$ and $\stab(x)=g\stab(y)g^{-1}$, so
    \begin{equation*}
        |\stab(x)|=|\stab(y)|.
    \end{equation*}
    Thus,
    \begin{align*}
        \sum_{x\in X} & = \sum_{G\cdot y\in G\bs X}\sum_{x\in G\cdot y}|\stab(x)| \\
                      & = \sum_{G\cdot y\in G\bs X}|G\cdot y| |\stab(y)|          \\
                      & = \sum_{G\cdot y\in G\bs X}[G:\stab(y)]|\stab(y)|         \\
                      & = \sum_{G\cdot y\in G\bs X} |G|                           \\
                      & = |G||G\bs X|
    \end{align*}
    where the antepenultimate and penultimate equalities follow from the Orbit-Stabilizer Theorem and Lagrange's Theorem, respectively.
\end{proof}
\subsection{Class Equation Revisited}
Let $G$ be a group, $X$ is a set, so $G\times X\rightarrow X$ is a group action. Recall that:
\begin{equation*}
    X = \coprod_{G\cdot x \in G \bs X}G\cdot x
\end{equation*}
so
\begin{equation*}
    |X|=\sum_{G\cdot x \in G\bs X}|G\cdot x| = \sum_{G\cdot x \in G\bs X}[G:\stab(x)]
\end{equation*}
However, there is another way to re-write this:
\subsubsection{Lemma}
\begin{idea}
    For all $x\in X$,
    \begin{align*}
        |G\cdot x| = 1 & \iff \stab(x) = G     \\
                       & \iff x \in X^G        \\
                       & \iff G\cdot x = \{x\}
    \end{align*}
\end{idea}
\begin{proof}
    By the orbit-stabilizer theorem,
    \begin{equation*}
        |G\cdot x| = [G:\stab(x)]
    \end{equation*}
    It follows that the $|G\cdot x|=1$ if and only if $\stab(x) = G$.

    We also know that $\stab(x)=G\iff g\cdot x=x$ for all $g\in G$. This by definition is equivalent to $x\in X^G$.

    Lastly, if $x\in X^G$, that happens if and only if $g\cdot x=x$ for all $g\in G$ which happens if and only if $G\cdot x:=\{g\cdot x: g\in G\}=\{x\}$
\end{proof}

Therefore,
\begin{align*}
    |X| & = \sum_{G\cdot x\in G \bs X \text{(s.t. $|G\cdot x|=1$)}} |G\cdot x| +\sum_{G\cdot x\in G \bs X \text{(s.t. $|G\cdot x|\neq1$)}} |G\cdot x| \\
        & = \sum_{\{x\}\text{ s.t. $x\in X^G$}} 1 + \sum_{G\cdot x\in G \bs X \text{(s.t. $|G\cdot x|\neq1$)}} [G:\stab(x)]                           \\
        & = |X^G| + \sum_{G\cdot x\in G \bs X \text{(s.t. $|G\cdot x|\neq1$)}} [G:\stab(x)]
\end{align*}
We also call this the class equation of this action. It is a bit more useful in this form.
\subsection{The Class Equation of a Group}
Let $G$ be a group, $X=G$, and $G\times X\rightarrow X$ and $(g,x) \mapsto gxg^{-1}$ is the action of $G$ on itself by conjugation.

Then for all $x\in X=G$, $$G\cdot x = \{gxg^{-1}:g\in G\}=: \text{cl}(x)$$
and
\begin{equation*}
    G\bs X = \{\text{cl}(x):x\in G\} =: \text{cl}(G).
\end{equation*}
Note that
\begin{itemize}
    \item $X^G = Z(G)$
    \item For all $x\in X=G$, $\stab(x)=C_G(x)$.
\end{itemize}
The class equation for this action is
\begin{equation*}
    |G| = |Z(G)| + \sum_{\text{cl}(x)\in\text{cl}(G) \text{ s.t. } |\text{cl}(x)| \neq 1} [G:C_G(x)]
\end{equation*}
and is known as the class equation of $G$.
% \section{Sylow Theorems}
\subsection{p-groups}
Let $p$ be a prime.
\begin{definition}{p-group}
    A finite group $G$ is a p-group if $|G|=p^n$ for some $n\in \mathbb{Z}_{\ge 0}$.
\end{definition}
\subsubsection{Fundamental Theorem for $p$-groups (Invariants of a $p$-group)}
\begin{idea}
    Let $G$ be a $p$-group and let $X$ be a finite set, and let $\cdot$ be an action of $G$ on $X$: $G\times X \rightarrow X$. Then
    \begin{equation*}
        |X^G| \equiv |X| \pmod{p}.
    \end{equation*}
    Consequently, if $p \nmid |X|,$ then $X^G \neq \emptyset$.
\end{idea}
\begin{proof}
    If $G=\{e\},$ then $X^G=X,$ and the result follows.

    Assume $G\neq \{e\}$. The class equation is
    \begin{equation*}
        |X| = |X^G| + \sum_{G\cdot x \in G\bs x, |G\cdot x|\neq 1} [G:\stab(x)].
    \end{equation*}
    It suffices to show that the second term on the right is divisible by $p$. To do that, note that if $x\in X$ and $|G\cdot x| \neq 1$, then
    \begin{equation*}
        p \mid [G:\stab(x)]
    \end{equation*}
    by Lagrange's Theorem, since $|G\cdot x| \neq 1$.
\end{proof}
\subsubsection{Theorem 3.2}
\begin{idea}
    If $G$ is a non-trivial $p$-group, then $Z(G) \neq \{e\}$.
\end{idea}
\begin{proof}
    Applying the preceding lemma to the action of $G$ on itself by conjugation gives
    \begin{equation*}
        |Z(G)| \equiv |G| \pmod{p}
    \end{equation*}
    Since $G$ is a non-trivial $p$ group, $p \mid |G|$. Therefore $|Z(G)|\equiv 0 \pmod{p}$, i.e. $p \mid |Z(G)|$. Since $e\in Z(G)$, $|Z(G)|>0$. Therefore, $|Z(G)|\ge p > 1$, so $Z(G) \neq \{e\}$.
\end{proof}
\subsubsection{Corollary}
\begin{idea}
    Every group of order $p^2$ is abelian.
\end{idea}
\begin{proof}
    Let $G$ be a group of order $p^2$. Then it has a non-trivial center, we have $|Z(G)|=p\text{ or }p^2$ by Lagrange's Theorem. In tha ltter case, $Z(G)=G$ and we're done.

    Suppose $|Z(g)|=p$. Then $G/Z(G)$ is a group of order $p$, so it is cyclic, and thus $G$ is abelian by an earlier result.
\end{proof}
\newpage
\section{Sylow's Theorems}
\subsection{Sylow p-subgroups}
Let $G$ be a finite group.
\begin{definition}{Sylow $p$-subgroup of $G$}
    Let $p$ be a prime. Let $n$ be the unique positive non-negative integer such that $p^n \bigg| |G|$ and $p^{n+1}\nmid |G|$, i.e. $p^n$ is the largest power of $p$ that divides $|G|$.
    \vspace{2mm}

    A subgroup of $G$ of order $p^n$ is called a Sylow $p$-subgroup of $G$.
    \vspace{2mm}

    The set of Sylow $p$-subgroups of $G$ is denoted by $\text{Sylp}(G)$. And define
    \begin{equation*}
        n_p = |\text{Sylp}(G)|.
    \end{equation*}
\end{definition}
Note that if $p \nmid |G|$, then $\text{Sylp}(G) = \{\{e\}\}$. If $p \mid |G|$, then Sylow $p$-subgroups are non-trivial if they exist.
\subsection{Sylow's Theorems, 1872}
\begin{idea}
    Let $p$ be a finite group and let $p$ be a prime.
    \begin{enumerate}
        \item For each $k\in \mathbb{Z}_{\ge 0}$ such that $p^k \mid |G|$, there exists a subgroup of $G$ of order $p^k$. In particular, there exists a sylow $p$-subgroup of $G$.
              \vspace{2mm}

              Note that this is a partial converse to Lagrange's Theorem.
        \item If $H$ is a $p$-group of $G$ and if $P\in\text{Sylp}(G)$, then there exists $g\in G$ such that $H \le gPg^{-1}$. In particular, every $p$-subgroup of $G$ is contained in a sylow $p$-subgroup.

              Furthermore, if $H$ is a sylow $p$-subgroup of $G$, then it follows that $H=gPg^{-1}$, i.e. any two sylow $p$-subgroups are conjugate to each other.
        \item For any $p\in\text{Sylp}(G)$, we have
              \begin{equation*}
                  n_p := |\text{Sylp}(G)| = [G:N_G(P)] \bigg| |G|.
              \end{equation*}
              Also, $n_p \equiv 1 \pmod{p}.$
    \end{enumerate}
\end{idea}

\end{document}