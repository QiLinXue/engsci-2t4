\documentclass{article}
\usepackage{gallian}
\title{MAT301: Extra Topics}
\author{QiLin Xue}
% \lhead{MAT301}
% \rhead{QiLin Xue}
\usetikzlibrary{decorations.markings}
\usepackage{eso-pic}

\usepackage{xcolor}
\usepackage{pagecolor}

\usepackage{calc}%
\usepackage{titlesec}%
\titleformat{\section} {\color{blue1}\large\bfseries}{\makebox[-5em][l]{}}{1em}{}
\titleformat{\subsection} {\color{blue1}\normalfont\bfseries}{\makebox[-5em][l]{}}{1em}{\texorpdfstring{$\blacksquare$}{}\hspace{1mm}}
\titleformat{\subsubsection} {\color{blue1}\normalfont\bfseries}{\makebox[-5em][l]{}}{1em}{\texorpdfstring{$\blacksquare$}{}\hspace{1mm}}
\renewcommand{\qedsymbol}{$\textcolor{blue1}{\blacksquare}$}
\setcounter{tocdepth}{2}
\renewenvironment{proof}{{\bfseries\color{blue1} Proof:}}{\qed}

\begin{document}
\maketitle
\AddToShipoutPictureBG{\AtPageLowerLeft{%
        \color{blue2}\rule{.04\paperwidth}{\paperheight}}}

\tableofcontents
\section{Free Abelian Groups}
\begin{definition}{Set of $\mathbb{Z}$ linear combinations of elements of $S$}
    Let $(A,+)$ be an abelian group. Note that if $S \subseteq A$, then
    \begin{equation*}
        \langle S \rangle = \left\{\sum_{i=1}^{m}k_ia_i:m\in \mathbb{Z}_{\ge 0},\,a_i \in S, k_i \in \mathbb{Z}\right\}
    \end{equation*}
    where the right hand side can be denoted as $\text{span}_{\mathbb{Z}}(S)$, which is the set of all $\mathbb{Z}$ linear combinations of elements of $S$.
\end{definition}
Since empty sets are trivial, we have
\begin{equation}
    \text{span}_{\mathbb{Z}}(\emptyset) = \{0\}
\end{equation}
\begin{definition}{Linear Independence, Span, Basis}
    Let $S \subseteq A$.
    \begin{enumerate}
        \item $S$ is linearly independent (over $\mathbb{Z}$) if for any $m\in \mathbb{Z}_{>0}$, $a_1,\dots,a_m \in S$, and $k_1,\dots,k_m \in \mathbb{Z}$,
              \begin{equation*}
                  \sum_{i=1}^{m} k_ia_i = 0 \implies a_1=\cdots=a_m=0,
              \end{equation*}
              or equivalently if every element of $A$ can be written as a $\mathbb{Z}$-linear combination of elements of $S$ in at most one way.
        \item $S$ spans $A$ (over $\mathbb{Z}$) if $A=\spanz(S)$, or equivalently every element of $A$ can be written as a $\mathbb{Z}$-linear combination of elements of $A$ in at least one way.
        \item $S$ is a basis (or $\mathbb{Z}$-basis) of $A$ if $S$ is linearly independent and spans $A$, or equivalently if every element of $A$ can be written as a $\mathbb{Z}$-linear combination of elements of $S$ in exactly one way.
    \end{enumerate}
\end{definition}
\begin{example}
    $e_1,\dots,e_m$ is a basis of $\mathbb{Z}^m$.
\end{example}
\begin{definition}{Free Abelian Group}
    A free abelian group is an abelian group that has a basis.
    \vspace{2mm}

    A free abelian group of finite rank is an abelian group that has a finite basis.
\end{definition}
\begin{example}
    $\mathbb{Z}^m$ is a free abelian group of finite rank for all $m\in \mathbb{Z}_{\ge 0}$. Note that $\mathbb{Z}^0=\{0\}$.
\end{example}
\begin{example}
    If $\{v_1,\dots,v_m\} \subseteq \mathbb{R}^n$ is linearly independent over $\mathbb{R}$ and $A$ is the subgroup of $\mathbb{R}^n$ generated by $\{v_1,\dots,v_n\}$, i.e. $A=\langle v_1,\dots,v_n=\spanz(\{v_1,\dots,v_n\})$, then $\{v_1,\dots,v_n\}$ is a $\mathbb{Z}$-basis of $A$, so $A$ is a free abelian group of finite rank.
\end{example}
\begin{example}
    If $\{G_i\}_{i\in I}$ is a set of groups, then
    $$\prod_{i\in I}' G_i := \left\{(g_i)_{i\in I} \in \prod_{i\in I}G_i\,\bigg\vert\,g_i = e_{G_i} \text{ for all but finitely many } i \in I\right\}$$
    is a subgroup of the direct product $\prod_{i\in I} G_i$. If the $G_i$ are abelian, then we denote $\prod_{i\in I}' G_i$ by $\Oplus{i\in I} G_i$, and call it the direct sum of $\{G_i\}_{i \in I}$.

    If $I$ is infinite and $G_i=\mathbb{Z}$ for all $i\in I$, then $\Oplus{i\in I}G_i = \Oplus{i\in I}\mathbb{Z}$ is a free abelian group that is not of finite rank (it has an infinite basis but no finite basis).
\end{example}
Note that free abelian groups are like vector spaces over $\mathbb{Z}$.
\subsubsection{Corollary}
\begin{idea}
    For all $m,n \in \mathbb{Z}_{>0}$, we have $\mathbb{Z}^m \cong \mathbb{Z}^n$ if and only if $m=n$.
\end{idea}
\subsection{Proposition 1}
\begin{idea}
    Let $A$ be a finitely generated group. Then $A$ is of finite rank if and only if $A$ is a finite group.
\end{idea}
\begin{proof}
    The ``only if'' direction is immediate. Suppose $A$ is a finite group and let $\{g_1,\dots,g_m\}$ be a generating set of $A$. Let $\{a_i\}_{i \in I}$ be a basis of $A$. There is a finite subset $\{a_{i_1},\dots,a_{i_n}\}$ of $\{a_i\}_{i\in I}$ such that
    \begin{equation}
        \{g_1,\dots,g_m\} \subseteq \spanz\left(\left\{a_{i_1},\dots,a_{i_n}\right\}\right).
    \end{equation}
    Then
    \begin{equation}
        A = \spanz(\{g_1,\dots,g_m\}) \subseteq \spanz\left(\left\{a_{i_1},\dots,a_{i_n}\right\}\right) \subseteq A,
    \end{equation}
    so $\spanz\left(\left\{a_{i_1},\dots,a_{i_n}\right\}\right) = A$. Since $\{a_{i_1},\dots,a_{i_n}\} \subseteq \{a_i\}_{i \in I}$ and $\{a_i\}_{i\in I}$ us linearly independent, it follows that $\{a_{i_1},\dots,a_{i_n}\}$ is linearly independent. Therefore, $\{a_{i_1},\dots,a_{i_n}\}$ is a basis of $A$, so $A$ is of finite rank.
\end{proof}
\vspace{2mm}

\textit{Warning!} Here are a few misconceptions. Take $\mathbb{Z}$ for example. THen:
\begin{itemize}
    \item $\{2,3\}$ is a minimal spanning subset of $\mathbb{Z}$, but it is not a basis as it is linearly dependent.
    \item $\{2,3\}$ spans $\mathbb{Z}$, but does not contain a basis of $\mathbb{Z}$.
    \item $\{2\}$ is a maximal linearly independent subset of $\mathbb{Z}$, but it is not a basis because its span is $2\mathbb{Z} \subsetneq
              \mathbb{Z}$.
    \item $\{2\}$ is linearly independent, but it is not contained in a basis of $\mathbb{Z}$.
\end{itemize}
\subsection{Proposition 2: Homomorphisms and Bases}
\begin{idea}
    Let $A$ be a free abelian group and let $\{a_i\}_{i\in I}$ be a basis of $A$.

    Let $B$ be an abelian group and let $\{b_i\}_{i\in I}$ be a family of elements of $B$.

    Then there exists a unique homomorphism $\phi:A\rightarrow B$ such that $\phi(a_i)=b_i$ for all $i\in I$. IT is surjective if and only if $\{b_i\}_{i\in I}$ spans $B$, it is injective if and only if $\{b_i\}_{i\in I}$ is linearly independent, and it is an isomorphism iff $\{b_i\}_{i\in I}$ is a basis of $B$.
\end{idea}
Let $A$ be a free abelian group of finite rank $n$. For any basis $\alpha=\{a_1,\dots,a_n\}$ of $A$ there exists a unique isomorphism:
\begin{equation}
    \theta_\alpha: A \rightarrow \mathbb{Z}^n
\end{equation}
such that $\theta(a_i)=e_i$ for all $i=1,\dots,n$. Note that:
\begin{itemize}
    \item $\theta_\alpha^{-1}(k_1,\dots,k_n)=\sum_{i=1}^n k_ia_i$
    \item For all $a\in A$, let us write $[a]_\alpha = \theta_\alpha(a) \in \mathbb{Z}^n$.
\end{itemize}
\subsection{Proposition 3}
\begin{idea}
    Let $A,B$ be free abelian groups of finite ranks $n$ and $m$, respectively. Let $\alpha=\{a_1,\dots,a_n\}$ be a basis of $A$ and $\beta=\{b_1,\dots,b_m\}$ be a basis of $B$. For all homomorphisms $\phi: A\rightarrow B$ there exists a unique matrix
    \begin{equation}
        [T]^\alpha_\beta \in \text{Mat}_{m\times n}(\mathbb{Z})
    \end{equation}
    such that for all $a\in A$ we have
    \begin{equation}
        [Ta]_\beta = [T]^\alpha_\beta [a]_\alpha.
    \end{equation}
    Let $C$ be a free abelian group of finite rank $p$ and let $\gamma = \{c_1,\dots,c_p\}$ be a basis of $C$. If $T: A\rightarrow B$ and $S: B\rightarrow C$ are homomorphisms, then
    \begin{equation}
        [S \circ T]^\alpha_\gamma = [S]^\beta_\gamma [T]^\alpha_\beta
    \end{equation}
\end{idea}
\begin{proof}
    Let $T:A\rightarrow B$ be a homomorphism. Define
    \begin{equation}
        [T]^\alpha_\beta = [[Ta_1]_\beta \cdots [Ta_n]_\beta].
    \end{equation}
    The rest is straightforward.
\end{proof}
\subsubsection{Corollary}
\begin{idea}
    A homomorphism $T:A\rightarrow B$ is an isomorphism if and only if there exists $N\in \text{Mat}_{n\times m}(\mathbb{Z})$ such that
    \begin{equation}
        [T]^\alpha_\beta n = I_m \text{ and } N[T]^\alpha_\beta = I_n
    \end{equation}
    in which case $m=n$.
\end{idea}
\subsection{Invertible Element of \texorpdfstring{$\text{Mat}_{n\times n}(\mathbb{Z})$}{Mat(nxn)(Z)}}
\begin{definition}{Invertible Element of $\text{Mat}_{n\times n}(\mathbb{Z})$}
    Let $n$ be a positive integer and $M \in \text{Mat}_{n\times n}(\mathbb{Z})$. We say that $M$ is an invertible element of $\text{Mat}_{n\times n}(\mathbb{Z})$ if there exists $N\in \text{Mat}_{n\times n}(\mathbb{Z})$ such that
    \begin{equation}
        MN = I_n = NM,
    \end{equation}
    in which case $N$ is unique, denoted by $M^{-1}$, and called the \textit{inverse of $M$}. We denote the subset of invertible elements of $\text{Mat}_{n\times n}(\mathbb{Z})$ by $\text{GL}_n(\mathbb{Z})$.
\end{definition}
Note that $M \in \text{Mat}_{n\times n}(\mathbb{Z})$ is invertible if and only if it is invertible in $\text{Mat}_{n\times n}(\mathbb{Q})$ and $M^{-1} \in \text{Mat}_{n\times n}(\mathbb{Z})$.
\subsection{Proposition 4}
\begin{idea}
    \begin{equation}
        \text{GL}_n(\mathbb{Z}) = \{M \in \text{Mat}_{n\times n}(\mathbb{Z}) \vert \det(M) \in \{\pm 1\}
    \end{equation}
\end{idea}
\begin{proof}
    If $M\in \text{GL}_n(\mathbb{Z})$, then
    \begin{equation}
        \det(M)\det(M^{-1})=\det(I_n) = 1
    \end{equation}
    Since $\det(M),\det(M^{-1}) \in \mathbb{Z},$ it follows that $\det(M)=\det(M^{-1})=\pm 1$.

    If $M\in \text{Mat}_{n\times n}(\mathbb{Z})$ and $\det(M) = \pm 1$, then the usual formula for $M^{-1}\in \text{Mat}_{n\times n}(\mathbb{Q})$ shows that $M^{-1}\in \text{Mat}_{n\times n}(\mathbb{Z})$. Thus, $M \in \text{GL}_n(\mathbb{Z}).$
\end{proof}
\subsection{Proposition 5}
\begin{idea}
    For each free abelian group $A$ of finite rank, every subgroup $B$ of $A$ is a free abelian group and
    \begin{equation}
        \rk{B} \le \rk{A}
    \end{equation}
\end{idea}
\textit{Remarks:} One can drop the assumption that $A$ is of finite rank.

\begin{proof}
    We will proceed by induction on $m=\rk(A)$.

    If $m \ge 0$ and assume that for each abelian group of rank $m$, every subgroup of it is a free abelian group of rank at most $m$.

    Let $A$ be a free abelian group of rank $m+1$ and let $B \le A$. We can choose a basis $\alpha=\{a_1,\dots,a_{m+1}\}$ of $A$ and define
    \begin{equation}
        A' = \spanz(\{a_1,\dots,a_m\}) \le A
    \end{equation}
    Then $A'$ is a free abelian group of rank $m$.
\end{proof}
\subsection{Second Reduction Theorem}
Let $A$ be a finitely generated abelian group, $\phi: \mathbb{Z}^m \rightarrow A$ is a surjective homomorphism, and $B=\ker\phi \le \mathbb{Z}^m$. Recall that it suffices to construct an isomorphism $\mathbb{Z}^m \rightarrow \mathbb{Z}^m$ that maps $B$ to $$d_1\mathbb{Z} \times \cdots \times d_n\mathbb{Z} \times \{0\} \times \cdots \times \{0\} \le \mathbb{Z}^m$$
for some positive integers $d_1|\cdots | d_n$.

Since $B \le \mathbb{Z}^m$, we now know that $B$ is a free abelian group of rank $n \le m$. Let $r=m-n$. It then suffices to prove the following theorem (too lazy to write proof, can be found in Lec 20):
\begin{idea}
    Let $C$ be a free abelian group of finite rank $m$ and let $B \le C$. Then $B$ is a free abelian group of rank at most $m$. Let $n=\rk(B) \le m$.
    \vspace{2mm}

    Then, there exists bases $\beta =\{b_1,\dots,b_n\}$ of $B$ and $\gamma=\{c_1,\dots,c_m\}$ of $C$ and positive integers $d_1|\cdots | d_n$ such that
    \begin{equation}
        b_i = d_i c_i
    \end{equation}
    for all $i=1,\dots, n$. Moreover, $d_1,\dots,d_n$ are unique.
\end{idea}
Indeed, suppose that this theorem holds and apply it to $B=\ker\phi \le \mathbb{Z}^n = C$. The isomorphism
\begin{equation}
    \mathbb{Z}^m = C \xrightarrow{[\cdot]_\gamma} \mathbb{Z}^m
\end{equation}
maps $b_i = d_ic_i$ to $d_ie_i$ for all $i=1,\dots, n$. Therefore, the isomorphism maps $B$ to $d_1\mathbb{Z} \times \cdots \times d_n\mathbb{Z}\times \{0\} \times \cdots \times \{0\}$.

We will prove a more general theorem.
\begin{idea}
Let $B$ and $C$ be free abelian groups of finite ranks $n$ and $m$, respectively. Let $\Psi: B\rightarrow C$ be a homomorphism.
\vspace{2mm}

Then there exists bases $\beta=\{b_1,\dots,b_n\}$ of $B$ and $\gamma=\{c_1,\dots,c_m\}$ of $C$, there exists a positive integer $r \le m,n$ and there exists positive integers $d_1|\cdots | d_r$ such that
\begin{equation}
    \Psi(b_i) = \begin{cases}
        d_ic_i\quad 1\le i \le r \\
        0 & r < i \le n
    \end{cases}
\end{equation}
or equivalently
\begin{equation}
    [\Psi]^\beta_\gamma =
    \left[
        \begin{array}{c|c}
            \begin{array}{ccc}
                d_1 &        & 0   \\
                    & \ddots &     \\
                0   &        & d_r
            \end{array} & 0 \\
            \hline
            0                          & 0
        \end{array}
        \right]
\end{equation}
Moreover, $r,d_1,\dots,d_r$ are unique.
\end{idea}
Let $\beta_0,\gamma_0$ be bases of $B,C$, respectively. The theorem is equivalent to the assertion that there exists matrices $P\in \text{GL}_m(\mathbb{Z})$, $Q\in \text{GL}_n(\mathbb{Z})$ such that
\begin{equation}
    P[\Psi]_{\gamma_0}^{\beta_0}Q =     \left[
        \begin{array}{c|c}
            \begin{array}{ccc}
                d_1 &        & 0   \\
                    & \ddots &     \\
                0   &        & d_r
            \end{array} & 0 \\
            \hline
            0                          & 0
        \end{array}
        \right]
\end{equation}
for some positive integers $d_1|\cdots |d_r$, and $r,d_1,\dots,d_r$ are unique. It turns out that slightly more is true.
\subsection{Theorem: Smith Normal Form}
\begin{idea}
    Let $M \in \text{Mat}_{m\times n}(\mathbb{Z})$. There exist a sequence of integral elementary row and column operations that transform $M$ to a matrix of the form
    \begin{equation}
        \left[
        \begin{array}{c|c}
            \begin{array}{ccc}
                d_1 &        & 0   \\
                    & \ddots &     \\
                0   &        & d_r
            \end{array} & 0 \\
            \hline
            0                          & 0
        \end{array}
        \right],
    \end{equation}
    where $d_1|\cdots|d_n$ are positive integers. Moreover, $r=\rk(M)$ and for all $i=1,\dots,r$, $d_i=d_i(M)/d_{i-1}(M)$. In particular, $r,d_1,\dots,r_r$ are unique.
\end{idea}
\begin{definition}{$i^\text{th}$ determinant divisor of $M$}
    For $i=1,\dots,\min\{m,n\}$, define
    \begin{equation}
        d_i(M) := \gcd\{\text{determinants of $i\times i$ minors of $M$.}\}
    \end{equation}
    and define $d_0(M)=1$. The number $d_i(M)$ is called the $i^\text{th}$ determinant divisor of $M$.
\end{definition}
Note that if $i < \rk(M)$, then $d_i(M) > 0$.
\subsection{Integral Elementary Row Operations}
There are three main operations:
\begin{itemize}
    \item To interchange row $i$ and row $j$, this is equivalent to multiplying on the left by $P_{i,j}$.
    \item To multiply row $i$ by $-1$, we multiply on the left by $D_i$.
    \item To replace row $i$ with row $i$ plus $k$ times row $j$, we multiply on the left by $E_{ij}(k)$.
\end{itemize}
Note that if we were to act on the columns instead, the elementary matrices should be multiplied on the right.
\begin{idea}
    The integral elementary matrices $P_{ij},D_i,E_{ij}(k)$ generate the group $\text{GL}_n(\mathbb{Z})$.
\end{idea}
\subsection{Smith Normal Form Algorithm}
If $M=0$, we are done. Assume $M \neq 0$.
\begin{enumerate}
    \item Let $\delta(M) = \min\{|M_{ij}| : M_{ij} \neq 0\}$. Choose $M_{ij} \neq 0$ such that $|M_{ij}|=\delta(M)$.
    \item If $M_{ij}$ does not divide an entry in its row, say $M_{i\ell}$, and $M_{i\ell}=gM_{ij}+r$ where $q,r \in \mathbb{Z}$ and $0<r<|M_{ij}|$, then replace $\text{col}_{\ell}$ with $\text{col}_{\ell}-q\text{col}_{j}$:
    \item This results in a matrix $M'$ with $M'_{i\ell}=r$ and $\delta(M') \le r < |M_{ij}| = \delta(M)$. Let $M$ denote $M'$ now. Go to the previous step.
    \item If $M_{ij}$ does not divide an entry in its column, we do the same thing analogous to the previous step.
    \item If $M_{ij}$ divides every entry in its row and column, we can clear the other entries in row $i$ and column $j$ using $M_{ij}$ (i.e. all the other entries are $0$). Let $M$ denote the resulting matrix.
    \item If $M_{ij}$ divides every entry in $M$, skip this step. Otherwise, choose $M_{k\ell}$ such that $M_{ij} \nmid M_{k\ell}.$ Then replace row $i$ with $\text{row}_i+\text{row}_j$. Let $M$ denote the new matrix. Go to the first step.
    \item $M_{ij}$ divides every entry of $M$. Swap row 1 and row $i$ and swap column 1 and column $j$. If $M_{ij} < 0$, multiply row by $-1$.
    
    We look at the resulting matrix $M'$ in the bottom right corner inside the larger matrix. Let $M$ denote $M'$.
    \item Repeat steps 1 to 6 until $M'$ in the previous step is the empty matrix.
\end{enumerate}
\subsection{Torsion Subgroup}
\begin{definition}{Torsion subgroup of $A$}
    Let $A$ be an abelian group. For each $n\in \mathbb{Z}_{>0}$, we define the $n$-torsion subgroup of $A$ to be
    \begin{equation}
        A[n] := \{a\in A: na = 0\}
    \end{equation}
    we define the $n$-power torsion subgroup of $A$ to be
    \begin{equation}
        A[n^\infty] := \left\{a\in A: n^k a = 0 \text{ for some } k \in \mathbb{Z}_{\ge 0}\right\}
    \end{equation}
    and we define the torsion subgroup of $A$ to be 
    \begin{equation}
        \text{Tor}(A) := \left\{a\in A: ma=0 \text{ for some } m\in \mathbb{Z}_{>0}\right\} = \bigcup_{n\in \mathbb{Z}_{>0}}A[n].
    \end{equation}
\end{definition}
\subsection{Proposition 6}
\begin{idea}
    Let $A$ be a finitely generated group. If $A \cong \mathbb{Z}^r \times T$, where $r \in \mathbb{Z}_{\ge 0}$ and $T$ is a finite abelian group, then $T \cong \text{Tor}(A)$ and $\mathbb{Z}^r \cong A/\text{Tor}(A)$.
    \vspace{2mm}

    Consequently, $T$ is unique up to isomorphism and $r$ is unique.
\end{idea}
\begin{proof}
    First, note that if $\phi: B\rightarrow C$ is an isomorphism between abelian groups, then $\phi(\text{Tor}B)=\text{Tor}C$, so $\phi$ restricts to an isomorphism $\phi: \Tor B \rightarrow \Tor C$.

    Let $\phi: A\rightarrow \mathbb{Z}^r \times T$ be an isomorphism as in the proposition statement. 

    Since $\Tor(\mathbb{Z}^r \times T) = \{(0,\dots,0)\} \times T \cong T$, we have $\Tor(A) \cong \{0\} \times T \cong T$.

    Also since $\phi(\Tor A)= \{0\} \times T$, the map
    \begin{align*}
        A/\Tor A &\rightarrow \mathbb{Z}^r \times T/(\{0\} \times T) \\ 
        a + \Tor A &\mapsto \phi(a) + (\{0\} \times T)
    \end{align*}
    is a well defined isomorphism. Therefore,
    \begin{align*}
        A/\Tor A & \cong \mathbb{Z}^r \times T/(\{0\} \times T) \\ 
        &\cong \mathbb{Z}^r/\{0\} \times T/T \\ 
        &\cong \mathbb{Z}^r \times \{0\} \\ 
        &\cong \mathbb{Z}^r
    \end{align*}
\end{proof}
\subsection{Proposition 7}
\begin{idea}
    Let $A$ be a finitely generated abelian group. If $A \cong \mathbb{Z}^r \times P \times B$, where $r \in \mathbb{Z}_{\ge 0}$, $P$ is a finite abelian group with $|P|=p^k$ for some prime $p$, and $B$ is a finite abelian group with $p\nmid |B|$, then
    \begin{equation}
        A[p^\infty] \cong P
    \end{equation}
\end{idea}
\end{document}