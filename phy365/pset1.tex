\documentclass{article}
\usepackage{qilin}
\tikzstyle{process} = [rectangle, rounded corners, minimum width=1.5cm, minimum height=0.5cm,align=center, draw=black, fill=gray!30, auto]
\title{PHY365: Quantum Information \\ Problem Set 1}
\author{QiLin Xue}
\date{Winter 2022}
\usepackage{mathrsfs}
\usetikzlibrary{arrows}
\usepackage{stmaryrd}
\usepackage{accents}
\newcommand{\ubar}[1]{\underaccent{\bar}{#1}}
\usepackage{pgfplots}
\numberwithin{equation}{section}

\begin{document}

\maketitle
\begin{enumerate}
    \item \begin{enumerate}
        \item We can compute $\det(\hat{X}-I\lambda) = \det\begin{bmatrix}
            -\lambda & 1 \\ 
            1 & -\lambda
        \end{bmatrix} = \lambda^2-1.$ This is zero if and only if $\lambda = \pm 1$, which are the eigenvalues. The corresponding eigenvectors are $\begin{bmatrix}
            1 \\ 1
        \end{bmatrix}$ and $\begin{bmatrix}
            1 \\ -1
        \end{bmatrix}$ for $\lambda=1,-1$ respectively. The diagonal representation is then
        \begin{equation}
            \hat{X} = \begin{bmatrix}
                0 & 1\\ 
                1 & 0
            \end{bmatrix}
            =
            \begin{bmatrix}
                1 & 1 \\ 
                1 & -1 
            \end{bmatrix}
            \begin{bmatrix}
                1 & 0 \\ 
                0 & -1
            \end{bmatrix}
            \begin{bmatrix}
                1/2 & 1/2 \\ 
                1/2 & -1/2
            \end{bmatrix}
        \end{equation}
        \item Via the same process, we have
        \begin{equation}
            \hat{Y} = \begin{bmatrix}
                0 & -i \\ 
                i & 0
            \end{bmatrix} = \begin{bmatrix}
                i & -i \\ 
                i & 0
            \end{bmatrix} \begin{bmatrix}
                i & -i \\ 
                1 & 1
            \end{bmatrix} \begin{bmatrix}
                -i/2 & 1/2 \\ 
                i/2 & 1/2
            \end{bmatrix}
        \end{equation}
        \item and 
        \begin{equation}
            \hat{Z} = \begin{bmatrix}
                1 & 0 \\ 
                0 & -1
                \end{bmatrix} = \begin{bmatrix}
                    1 & 0 \\
                    0 & -1
                \end{bmatrix}\begin{bmatrix}
                    0 & 1 \\ 
                    1 & 0
                \end{bmatrix}\begin{bmatrix}
                    0 & 1 \\ 
                    1 & 0
                \end{bmatrix}
        \end{equation}
    \end{enumerate}
    \emf{Remarks:} The three Pauli matrices form a group, known as the Pauli group $G_1 = \langle \hat{X},\hat{Y},\hat{Z} \rangle.$
    \item These can be shown by straight matrix multiplication. However, we can use the special trick that
    \begin{equation}
        \hat{X}\hat{Y}\hat{Z} = iI
    \end{equation}
    and $\hat{X},\hat{Y},\hat{Z}$ are all order $2$.  The first and third relations then follow immediately. The second relation also follows immediately from their anti-commutative property. Thus:
    \begin{align}
        \hat{X}\hat{Z}\hat{X} &= i\hat{X}\hat{Y} \\ 
        &= i^2\hat{Z}.
    \end{align} 
    \item We can compute 
    \begin{align}
        \hat{U}^\dagger &= \cos\alpha \hat{I} + \sin\alpha\left(\sin\theta\cos\phi\hat{X}^\dagger + \sin\theta\sin\phi\hat{Y}^\dagger + \cos\theta\hat{Z}^\dagger\right) \\ 
        &= \cos\alpha \hat{I} + \sin\alpha\left(\sin\theta\cos\phi\hat{X} - \sin\theta\sin\phi\hat{Y} + \cos\theta\hat{Z}\right)
    \end{align}
    \item Let us compute $\hat{U},\hat{U}^\dagger$:
    \begin{align}
        \hat{U} &= \begin{bmatrix}
            \cos\alpha + i\sin\alpha\cos\theta & \sin\alpha\sin\theta\sin\phi + i\sin\alpha\sin\theta\cos\phi \\
             - \sin\alpha\sin\theta\sin\phi + i\sin\alpha\sin\theta\cos\phi & \cos\alpha - i\sin\alpha\cos\theta
        \end{bmatrix} \\ 
        \hat{U}^\dagger &= \begin{bmatrix}
            \cos\alpha - i\sin\alpha\cos\theta & - \sin\alpha\sin\theta\sin\phi - i\sin\alpha\sin\theta\cos\phi \\
            \sin\alpha\sin\theta\sin\phi - i\sin\alpha\sin\theta\cos\phi & \cos\alpha + i\sin\alpha\cos\theta
        \end{bmatrix}
    \end{align}
    and their product by looking at each element
    \begin{align*}
        a_{11} &= (\cos^2\alpha+\sin^2\alpha\cos^2\theta) + (\sin^2\alpha\sin^2\theta\sin^2\phi+\sin^2\alpha\sin^2\theta\cos^2\phi) \\ 
        &= \cos^2\alpha+\sin^2\alpha\cos^2\theta + \sin^2\alpha\sin^2\theta \\ 
        &= \cos^2\alpha + \sin^2\alpha \\ 
        &= 1 \\ 
        a_{12} &= (\cos\alpha + i\sin\alpha\cos\theta)(- \sin\alpha\sin\theta\sin\phi - i\sin\alpha\sin\theta\cos\phi) \\ 
        &\quad + (\sin\alpha\sin\theta\sin\phi + i\sin\alpha\sin\theta\cos\phi)(\cos\alpha + i\sin\alpha\cos\theta) \\ 
        &= 0
    \end{align*}
    Due to similar patterns, we can indeed verify that $a_{21}=0$ and $a_{22}=0$. Therefore, $\hat{U}\hat{U}^\dagger = \hat{I}$ implies that $\hat{U}$ is unitary.
    % using properties of the complex transpose. Then the product:
    % \begin{align*}
    %     \hat{U}\hat{U}^\dagger &= \left(\cos\alpha \hat{I} + \sin\alpha\left(\sin\theta\cos\phi\hat{X} - \sin\theta\sin\phi\hat{Y} + \cos\theta\hat{Z}\right)\right) \\ &\quad\cdot\left(\cos\alpha \hat{I} + i\sin\alpha\left(\sin\theta\cos\phi\hat{X} + \sin\theta\sin\phi\hat{Y} + \cos\theta\hat{Z}\right)\right) \\ 
    %     &\equiv \cos^2\alpha \hat{I} + i\cos\alpha\sin\alpha\sin\theta\cos\phi\hat{X} + i\cos\alpha\sin\alpha\sin\theta\sin\phi\hat{Y} + i\cos\alpha\sin\alpha\cos\theta \hat{Z} \\ 
    %     &\quad + \sin\alpha\cos\alpha\sin\theta\cos\phi \hat{X} + i\sin^2\alpha\sin^2\theta\cos^2\phi\hat{X}^2 \\ 
    %     &\quad\quad + i\sin^2\alpha\sin^2\theta\cos\phi\sin\phi \hat{X}\hat{Y} + i\sin^2\alpha\sin\theta\cos\theta\cos\phi \hat{X}\hat{Z} \\ 
    %     &\quad -\sin\alpha\cos\alpha\sin\theta\sin\phi\hat{Y} - i\sin^2\alpha\sin^2\theta\sin\phi\cos\phi\hat{Y}\hat{X} \\ 
    %     &\quad\quad - i\sin^2\alpha\sin^2\theta\sin^2\phi\hat{Y}^2  - i\sin^2\alpha\sin\theta\cos\theta\sin\phi\hat{Y}\hat{Z} \\
    %     &\quad +\sin\alpha\cos\cos\alpha\cos\theta\hat{Z} + i\sin^2\alpha\cos\theta\sin\theta\cos\phi\hat{Z}\hat{X} \\ 
    %     &\quad\quad + i\sin^2\alpha \cos\theta \sin\theta\sin\phi \hat{Z}\hat{Y} + i\sin^2\alpha\cos^2\theta\hat{Z}^2. \\ 
    %     &=\cos^2\alpha \hat{I} + i\cos\alpha\sin\alpha\sin\theta\cos\phi\hat{X} + i\cos\alpha\sin\alpha\sin\theta\sin\phi\hat{Y} + i\cos\alpha\sin\alpha\cos\theta \hat{Z} \\ 
    %     &\quad + \sin\alpha\cos\alpha\sin\theta\cos\phi \hat{X} + i\sin^2\alpha\sin^2\theta\cos^2\phi\hat{I} \\ 
    %     &\quad\quad - \sin^2\alpha\sin^2\theta\cos\phi\sin\phi\hat{Z} + \sin^2\alpha\sin\theta\cos\theta\cos\phi \hat{Y} \\ 
    %     &\quad -\sin\alpha\cos\alpha\sin\theta\sin\phi\hat{Y} - \sin^2\alpha\sin^2\theta\sin\phi\cos\phi\hat{Z} \\ 
    %     &\quad\quad - i\sin^2\alpha\sin^2\theta\sin^2\phi\hat{I}  + \sin^2\alpha\sin\theta\cos\theta\sin\phi\hat{X} \\
    %     &\quad +\sin\alpha\cos\cos\alpha\cos\theta\hat{Z} - \sin^2\alpha\cos\theta\sin\theta\cos\phi\hat{Y} \\ 
    %     &\quad\quad + \sin^2\alpha \cos\theta \sin\theta\sin\phi \hat{X} + i\sin^2\alpha\cos^2\theta\hat{I}. \\ 
    %     % $
    %     &=\left(\cos^2\alpha  + i\sin^2\alpha\sin^2\theta\cos^2\phi- i\sin^2\alpha\sin^2\theta\sin^2\phi + i\sin^2\alpha\cos^2\theta\right)\hat{I} \\ 
    %     &\quad + \left(i\cos\alpha\sin\alpha\sin\theta\cos\phi+\sin\alpha\cos\alpha\sin\theta\cos\phi+\sin^2\alpha\sin\theta\cos\theta\sin\phi+\sin^2\alpha \cos\theta \sin\theta\sin\phi\right)\hat{X} \\ 
    %     &\quad + \left(i\cos\alpha\sin\alpha\sin\theta\sin\phi+\sin^2\alpha\sin\theta\cos\theta\cos\phi-\sin\alpha\cos\alpha\sin\theta\sin\phi-\sin^2\alpha\cos\theta\sin\theta\cos\phi\right)\hat{Y} \\ 
    %     &\quad + \left(i\cos\alpha\sin\alpha\cos\theta-\sin^2\alpha\sin^2\theta\cos\phi\sin\phi- \sin^2\alpha\sin^2\theta\sin\phi\cos\phi+\sin\alpha\cos\cos\alpha\cos\theta\right)\hat{Z} \\ 
    %     % $
    %     &=\left(\cos^2\alpha  + i\sin^2\alpha\sin^2\theta\cos^2\phi- i\sin^2\alpha\sin^2\theta\sin^2\phi + i\sin^2\alpha\cos^2\theta\right)\hat{I} \\ 
    %     &\quad + \sin\theta\left(i\cos\alpha\sin\alpha\cos\phi+\sin\alpha\cos\alpha\cos\phi+\sin^2\alpha\cos\theta\sin\phi+\sin^2\alpha \cos\theta \sin\phi\right)\hat{X} \\ 
    %     &\quad + \sin\theta\sin\alpha\cos\alpha\sin\phi\left(i-1\right)\hat{Y} \\ 
    %     &\quad + \left(i\cos\alpha\sin\alpha\cos\theta-2\sin^2\alpha\sin^2\theta\cos\phi\sin\phi-\sin\alpha\cos\cos\alpha\cos\theta\right)\hat{Z} \\ 
    % \end{align*}
\end{enumerate}
\end{document}