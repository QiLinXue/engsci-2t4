\documentclass{article}
\usepackage{qilin}
\tikzstyle{process} = [rectangle, rounded corners, minimum width=1.5cm, minimum height=0.5cm,align=center, draw=black, fill=gray!30, auto]
\title{PHY365: Quantum Information \\ Midterm \#1 Review}
\author{QiLin Xue}
\date{Winter 2022}
\usepackage{mathrsfs}
\usetikzlibrary{arrows}
\usepackage{stmaryrd}
\usepackage{accents}
\newcommand{\ubar}[1]{\underaccent{\bar}{#1}}
\usepackage{pgfplots}
\numberwithin{equation}{section}
\usetikzlibrary{quantikz}

\begin{document}

\maketitle
\tableofcontents
\newpage
\section{Quantum Gates}
All quantum gates are unitary matrices, i.e. $U^\dagger U = UU^\dagger = I,$ and in general can be represented as
\begin{equation}
    \hat{U} = \begin{pmatrix}
        a & b \\ 
        -b^* & a^*
    \end{pmatrix}.
\end{equation}
Some common gates:
\begin{itemize}
    \item The Pauli gates $\hat{X},\hat{Y},\hat{Z}$ are given by
    \begin{align*}
        \hat{X} &= \begin{pmatrix}
            0 & 1 \\ 
            1 & 0
        \end{pmatrix} \\ 
        \hat{Y} &= \begin{pmatrix}
            0 & -i \\ 
            i & 0
        \end{pmatrix} \\
        \hat{Z} &= \begin{pmatrix}
            1 & 0 \\ 
            0 & -1
        \end{pmatrix}
    \end{align*}
    \item And the Hadamard gate 
    \begin{equation}
        \hat{H} = \frac{1}{\sqrt{2}} \begin{pmatrix}
            1 & 1 \\ 
            1 & -1
        \end{pmatrix}.
    \end{equation}
    sends $\ket{0} \mapsto \ket{+}$ and $\ket{1} \mapsto \ket{-}$.
\end{itemize}
\section{Measurements}
Quantum measurements are described by a collection $\{M_m\}$ of \emf{measurement operators.} These are operators acting on the state space of the system being measured. The index $m$ refers to the measurement outcomes that may occur in the experiment. If the state of the quantum system is $\ket{\psi}$ immediately before the measurement, then the probability that result $m$ occurs is given by 
\begin{equation}
    p(m) = \bra{\psi} M_m^\dagger M_m\ket{\psi}.
\end{equation}
and the state of the system after the measurement is given by
\begin{equation}
    \frac{M_m\ket{\psi}}{\sqrt{p(m)}}.
\end{equation}
The measurement operators satisfy the \emf{completeness equation} (fancy talk for probabilities summing to one):
\begin{equation}
    \sum_m M_m^\dagger M_m = I.
\end{equation}
To measure in a basis $\ket{m}$, where $\ket{m}$ forms an orthonormal basis, simply means to perform the projective measurement with projectors $P_m = \ket{m}\bra{m}$. Note in lecture, we used $\Pi$ as the symbol (but I believe $P$ is more commonly used).
\section{Quantum Circuits}
A control gate is in the form of:
\begin{center}
    \begin{quantikz}
        \lstick{$\ket{a}$} & \ghost{X} & \qw & \ctrl{1} & \qw & \qw & \rstick{$\ket{a}$} \\
        \lstick{$\ket{b}$} & \ghost{X} & \qw & \gate{\hat{U}} & \qw &  \qw & \rstick{$(\hat{I}\otimes \hat{U})(\ket{a}\otimes \ket{b})$}
    \end{quantikz}
\end{center}
where the dot represents the \emf{control.} In a sense, we apply $\hat{U}$ if and only if the control is $1$.
\section{Schmidt Decomposition Theorem}
Any two-qubit pure state can be written as
\begin{equation}
    \ket\Psi = \hat{U}_A \otimes \hat{U}_B \left(\lambda_0 \ket{00} + \lambda_1{11}\right),
\end{equation}
where $\lambda_0,\lambda_1$ are real, positive constants known as $\emf{singular values}$ and they satisfy $\lambda_0^2 + \lambda_1^2 = 1$. The operators $\hat{U}_A,\hat{U}_B$ are unitaries applied separately to each qubit.

The unitary operators $\hat{U}_A,\hat{U}_B$ are given by unitary matrices that satisfy the relationships  
\begin{equation}
    \hat{\chi}\hat{\chi}^\dagger = \hat{U}_A\Lambda^2\hat{U}_A^\dagger,\quad\quad \hat{\chi}^\dagger\hat{\chi} = \hat{U}_B\Lambda^2\hat{U}_B^\dagger.
\end{equation}
where $\chi$ is a matrix where entries are the coefficients of $\ket{\Psi}$ and $\Lambda^2 = \text{diag}(\lambda_0^2,\lambda_1^2)$, where $\lambda_i$ are solutions to the quadratic equation:
\begin{equation}
    \lambda^4 - \lambda^2 + (C/2)^2 = 0,
\end{equation}
where $C$ is the concurrence.
\section{Entangled States}
The \emf{concurrence} of $\alpha \ket{00} + \beta\ket{01} + \gamma\ket{10} + \delta\ket{11}$ is given by
\begin{equation}
    C = 2|\alpha\delta - \beta\gamma|
\end{equation}
where it is maximally entangled if $C = 1$ and separable if $C = 0$.

The \emf{fundamental theorem of entanglement} says that for a $C = 1$ two-qubit system,
\begin{equation}
    (\hat{I} \otimes \hat{U})\ket{\beta} = -(\hat{U}^\dagger \otimes \hat{I} \ket{\beta})
\end{equation}
and 
\begin{equation}
    (\hat{U} \otimes \hat{U})\ket{\beta} = -\ket{\beta}
\end{equation}
The \emf{Bell States} are four maximally entangled two qubit states that form a basis for the four-dimensional Hilbert space for two qubits. They are given by:
\begin{align*}
    \ket{\beta_0} &= \frac{1}{\sqrt 2}\left(\ket{00}+\ket{11}\right) = \ket{\Phi}\\ 
    \ket{\beta_1} &= \frac{i}{\sqrt 2}\left(\ket{01}+\ket{10}\right) = i\ket{\Psi_+}\\
    \ket{\beta_2} &= \frac{-1}{\sqrt 2}\left(\ket{01}-\ket{10}\right) = -i\ket{\Psi_-}\\
    \ket{\beta_3} &= \frac{i}{\sqrt 2}\left(\ket{00}-\ket{11}\right) = -\ket{\Phi_-}
\end{align*}
\end{document}