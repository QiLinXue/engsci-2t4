\documentclass{article}
\usepackage{qilin}
\tikzstyle{process} = [rectangle, rounded corners, minimum width=1.5cm, minimum height=0.5cm,align=center, draw=black, fill=gray!30, auto]
\title{PHY365: Quantum Information}
\author{QiLin Xue}
\date{Fall 2021}
\usepackage{mathrsfs}
\usetikzlibrary{arrows}
\usepackage{stmaryrd}
\usepackage{accents}
\newcommand{\ubar}[1]{\underaccent{\bar}{#1}}
\usepackage{pgfplots}
\numberwithin{equation}{section}
\usetikzlibrary{quantikz}

\begin{document}

\maketitle
\tableofcontents
\newpage
\section{Overview of Quantum Computing}
\subsection{Quantum Coins}
Consider a quantum coin that can be in a superposition of heads and tails. We can write its state as a vector:
\begin{equation}
    |\Psi\rangle = \alpha|H\rangle + \beta|T\rangle  
\end{equation}
which lives in the \emf{Hilbert Space.} Inner products of these vectors can be written as 
\begin{equation}
    \langle \Psi_1 | \Psi_2 \rangle.
\end{equation}
\emf{Born's Rule} tells us we can compute the probability of tails to be $|\beta|^2$ and the probability of heads is $|\alpha|^2$. When there are two quantum coins, there can be four combinations of heads and tails, written as:
\begin{equation}
    |\Psi\rangle = \alpha|HH\rangle + \beta|HT\rangle + \gamma TH\rangle + \delta |TT\rangle.
\end{equation}
In quantum mechanics, we can construct the following state:
\begin{equation}
    |\Psi\rangle = \frac{1}{\sqrt{2}}|HH\rangle + \frac{1}{\sqrt{2}}|TT\rangle,
\end{equation}
which represents \emf{entanglement.} If we measure the first coin, we can instantly know the outcome of the second coin, even if they are lightyears apart.
\subsection{Building a Better Computer}
How might we use quantum coins to help us build a ``better'' computer? Before we begin to understand and answer this question, let us understand some key concepts.

First, we can measure \emf{information} as the number of bits (binary digits) that are needed to specify a message. Each bit in a computer requires a physical system that has two possible configurations.
\begin{itemize}
    \item In semiconductor circuits, we use voltage.
    \item Magnetization is sometimes also used (i.e. in hard drives).
    \item Pits in optical storage.
    \item Paper tape with holes in it
\end{itemize}
Now let's extend the idea to quantum bits, i.e. \emf{qubits}. Let us use $|0\rangle$ and $|1\rangle$ to represent the two possible states of a quantum coin, and we can write a qubit as 
\begin{equation}
    |\Psi_1\rangle = \alpha|0\rangle + \beta|1\rangle,
\end{equation}
which isn't necessarily interesting.  If we have two qubits, we can write the state as
\begin{equation}
    |\Psi_2\rangle = \alpha|00\rangle + \beta|01\rangle + \gamma|10\rangle + \delta|11\rangle,
\end{equation}
where the following notation are equivalent:
\begin{equation}
    |00\rangle = |0\rangle|0\rangle = |0\rangle \otimes |0\rangle
\end{equation}
where $\otimes$ is the \emf{tensor product} of two vectors. To make it easier to write, we can also write it as:
\begin{equation}
    | \Psi_2 \rangle = \alpha |0_2\rangle + \beta|1_2\rangle + \gamma |2_2\rangle + \delta|3_2\rangle.
\end{equation}
For three qubits, we have
\begin{equation}
    |\Psi_3\rangle = \alpha |000\rangle + \beta|001\rangle + \gamma|010\rangle + \delta|011\rangle + \epsilon|100\rangle + \zeta|101\rangle + \eta|110\rangle + \theta|111\rangle.
\end{equation}
Therefore, $N$ qubits will have $2^N$ possible states. This suggests that quantum memory can get big, fast.
\subsubsection{Quantum Parallelism}
However, this is not the only difference. Each qubit operation, i.e. $|0\rangle \longleftrightarrow |1\rangle$ affect \textit{all} the probability amplitudes. This also suggests that quantum computers can be extremely efficient.

However, when we make measurements, $N$ qubits only leads to $N$ bits of information. Therefore, even though it is very efficient and quick, there is only a small amount of output.
\begin{example}
    Consider $f:\mathbb{Z}^+\rightarrow \mathbb{R}$ a periodic function that maps $x\in [0,2^L-1]$ (i.e. takes in an $L$ bit integer). There is some $X$ such that $f(x+X)=f(x)$ and we wish to find $X$.
    \vspace{2mm}

    In a classical computer, we would evaluate $f(x)$ for multiple values of $x$. In general, we would expect around $2^{L-1}$ calls in the routine.
    \vspace{2mm}

    However, in a quantum computer, we need $L$ qubits to store values of $x$ (i.e. in the. argument register) and $L$ qubits to store the result of $f(x)$ in the function register. Through a series of bit flips, we can create the state 
    \begin{equation}
        |x\rangle|0\cdots 0\rangle
    \end{equation}
    where the first braket is the input and the second braket is the function register. Then suppose we have a \emf{quantum operation} $\hat{U}_f$ defined such that
    \begin{equation}
        \hat{U}_f|x\rangle |0\rangle = |x\rangle |f(x)\rangle.
    \end{equation}
    But if we prepare the initial state of the register not in $x$, but in a superposition (achieved via a \emf{Hadamard gate}), then we can write:
    \begin{equation}
        \hat{U}_f\frac{1}{N}\left(\sum_{x=0}^{2^k-1} |x\rangle\right)|0\rangle = \frac{1}{N} \underbrace{\sum_{x=0}^{2^k-1} |x\rangle |f(x)\rangle}_{\text{massively entangled state}}.
    \end{equation}
    The difference is that all values of $f(x)$ are generated by a single call on $\hat{U}_f$. If we now apply something called the \emf{Quantum Fourier Transform}
    \begin{equation}
        \hat{U}_{QFT}\sum_x |x\rangle |f(x)\rangle = \frac{1}{N} \sum_x |x\rangle |\tilde{f}(x)\rangle,
    \end{equation}
    where $\tilde{f}$ is the \emf{fourier transform,} which you will get a discrete graph of vertical bars separated a distance by $\frac{n}{X}.$ If we do this a few times, we can extract what $X$ is.
\end{example}
Quantum computers allow us in principle to evaluate periods very efficient. This is a very important problem in \emf{number theory} since period finding helps a great deal in factoring.

Consider coprime $n,a$ and define 
\begin{equation}
    f(x) = a^x \text{ mod } n.
\end{equation} 
This is a periodic function with period $r$. If we can figure out what $r$ is, then
\begin{equation}
    \gcd(a^{r/2} \pm 1, n)
\end{equation}
is a factor of $n$. This is known as \emf{Shor's Algorithm.}
\subsection{Quantum Mechanics of Quantum Computers}
Suppose there are three qubits. Recall that there are $2^3=8$ possible configurations. These form a basis for a $8$-dimensional vector space. These basis states are known as a \emf{computational basis}.

For a single basis $|\Psi \rangle = \alpha |0\rangle + \beta 1\rangle$, where $\alpha,\beta$ are complex probability amplitudes, then we have
\begin{equation}
    |\alpha|^2+|\beta|^2 = 1 \iff \begin{pmatrix}
        \alpha^*, \beta^*
    \end{pmatrix}\begin{pmatrix}
        \alpha \\ \beta 
    \end{pmatrix} = 1.
\end{equation}
Now suppose we apply a transformation (i.e. operators and gates): 
\begin{align*}
    |\Psi \rangle &\mapsto |\Psi'\rangle \\ 
    \alpha \mapsto \alpha' \\
    \beta \mapsto \beta'.
\end{align*}
We can assume linearity (which has been experimentally validated), and therefore
\begin{align*}
    \alpha' = u_{00}\alpha + u_{01}\beta \\ 
    \beta' = u_{10}\alpha + u_{11}\beta
\end{align*}
which can be written as a matrix
\begin{equation}
    \begin{pmatrix}
        \alpha' \\ \beta'
    \end{pmatrix} = \begin{pmatrix}
        u_{00} & u_{01} \\ u_{10} & u_{11}\end{pmatrix} \begin{pmatrix}
        \alpha \\ \beta
    \end{pmatrix} \iff |\Psi'\rangle = \hat{U} |\Psi\rangle.
\end{equation}
And the complex conjugates are
\begin{equation}
    (\alpha'^*, \beta'^*) = (\alpha^*,\beta^*)\begin{pmatrix}
        u_{00}^* & u_{10}^* \\ u_{01}^* & u_{11}^*
    \end{pmatrix} \iff \langle \Psi'| = \langle \Psi | \hat{U}^\dagger.
\end{equation}
Here are some properties of the complex conjugate:
\begin{itemize}
    \item $(\hat{A}\hat{B})^\dagger = \hat{B}^\dagger\hat{A}^\dagger$
    \item $\langle \psi'|\psi'\rangle = \langle \psi | \hat{U}^\dagger\hat{U} | \Psi\rangle = 1 \iff \hat{U} \text{ is unitary,}$ which is true for all valid quantum operations on a closed system.
\end{itemize}
Let's look at some example gates:
\begin{itemize}
    \item Bit-flip gate:
    \begin{equation}
        \hat{X} = \begin{pmatrix}
            0 & 1 \\ 
            1 & 0.
        \end{pmatrix}
    \end{equation}
    along with the rest of the Pauli matrices:
    \begin{align}
        \hat{Y} &= \begin{pmatrix}
            0 & -i \\ 
            i & 0.
        \end{pmatrix} \\ 
        \hat{Z} &= \begin{pmatrix}
            1 & 0 \\ 
            0 & -1.
        \end{pmatrix} \\
        \hat{I} &= \begin{pmatrix}
            1 & 0 \\ 
            0 & 1.
        \end{pmatrix}.
    \end{align}
    \item Phase-flip gate: $\hat{Z}.$ Note that the overall \emf{phase}, or ``global'' phase is irrelevant, since the norm of the probabilities stay the same.
\end{itemize}
\section{Unitary Operators}
\subsection{SU(2)}
An arbitrary $2\times 2$ unitary is a matrix $\begin{pmatrix}
    a & b \\ c & d
\end{pmatrix}$ such that $|ad-bc|^2=1$. In general, $ad-bc = e^{i\phi} \neq 1.$ However in quantum computing, we don't typically care about the \emf{phase} of our qubits, so without loss of generality, we can assume that $ad-bc=1$. These are known as \emf{special unitary matrices with dimension $2$, or $SU(2)$.} We can therefore write it as 
\begin{equation*}
    \hat{U} = \begin{pmatrix}
        a & b \\ 
        -b^* & a^*
    \end{pmatrix}.
\end{equation*}
Any unitary matrix can be written as a linear combination of $\hat{I}, \hat{X}, \hat{Y},\hat{Z}..$ Particularly,
\begin{equation}
    \hat{U} = \begin{pmatrix}
        a_1 + ia_2 & b_1+ib_2 \\ 
        -b_1 + ib_2 & a_1-ia_2
    \end{pmatrix} = a_1\hat{I} + ib_2\hat{X} + ib_1\hat{Y} + ia_2\hat{Z}.
\end{equation}
Note that
\begin{align}
    1 &= a_1^2 + a_2^2 + b_1^2 + b_2^2 \\
    a_1 &= \cos\theta \\ 
    \{b_2,b_1,a_2\} &= \sin\theta\{n_x,n_y,n_z\}.
\end{align}
We can thus express $\hat{U} = \cos\theta \hat{I} + i\sin\theta \bm{n} \cdot \bm{\sigma}$
\subsection{Basis Change}
We can introduce new bases use unitaries. Namely, $\hat{U}\ket 0 = \ket u, \hat{U}\ket 1 = \ket {u_\perp}$ are new basis vectors. These two will still be orthogonal.
\subsection{Time Evolution}
Suppose we have an evolving unitary
\begin{equation}
    \ket{\Psi(t)} = \hat{U}(t)\ket{\Psi(0)}.
\end{equation}
Taking the partial time derivative, and substituting in the above identity for $\ket{\Psi(0)}$, we have:
\begin{align*}
    \frac{\partial}{\partial t}\ket{\Psi(t)} &= \frac{\partial \hat{U}(t)}{\partial t} \ket{\Psi(0)} \\ 
    &= \left\{\frac{\partial \hat{U}(t)}{\partial t}\hat{U}^\dagger(t)\right\}\ket{\Psi(t)}.
\end{align*}
We can apply the product rule and the identity $(AB)^\dagger = B^\dagger A^\dagger$ to obtain
\begin{align*}
    \hat{U}\hat{U}^\dagger &= I \\ 
    \frac{\partial \hat{U}}{\partial t} \hat{U}^\dagger + \hat{U} \frac{\partial \hat{U}^\dagger}{\partial t} &= 0 \\ 
    \frac{\partial \hat{U}}{\partial t} \hat{U}^\dagger &= -\left(\frac{\partial \hat{U}}{\partial t} \hat{U}^\dagger\right)^\dagger,
\end{align*}
which is an \emf{anti-hermitian operator.} We can relate it to a hermitian operator $\hat{H}$.
\begin{equation}
    \frac{\partial \hat{U}}{\partial t} \hat{U}^\dagger = \frac{\hat{H}}{i\hbar},
\end{equation}
where $\hat{H}$ is the \emf{Hamiltonian}. Altogether, we end up with \emf{Schrodinger's Equation:}
\begin{equation}
    i\hbar \frac{\partial}{\partial t}\ket{\Psi(t)} = \hat{H}\ket{\Psi(t)}.
\end{equation}
Usually we choose $\{\ket 0, \ket 1\}$ as the eigenstates of the Hamiltonian.
\subsection{Measurements and Non-Unitary Operations}
\begin{quotation}
    If the particle is in a state $\ket\Psi$, measure of the variable $\hat{\Omega}$ will yield one of the eigenvalues of $\Omega$ with probability $P(\omega) = |\bra{\omega}\ket{\Psi}|^2$. The state of the system will change from $\ket{\Psi}$ to $\ket\omega$ as a result. - Shankar
\end{quotation}
For a qubit with the measurement operator $\hat{\Omega} = \begin{pmatrix}
    0 & 0 \\ 
    0 & 1
\end{pmatrix}$ (with eigenvalues $\omega=0,1$), then $P(0)= |\alpha|^2$ and $P(1) = |\beta|^2$. The state at the end is equal to
\begin{equation}
    |\Psi^\text{after}| = \frac{\hat{\Pi}_0\ket\Psi}{\sqrt{P(0)}} \text{ or }\frac{\hat{\Pi}_1\ket\Psi}{\sqrt{P(1)}} 
\end{equation}
where $\hat{\Pi}_0 = \begin{pmatrix}
    1 & 0 \\ 0 & 0 \end{pmatrix}$ and $\hat{\Pi}_1 = \begin{pmatrix} 0 & 0 \\ 0 & 1
\end{pmatrix}$ are rank-1 projectors, i.e. $\hat{\Pi}_0^2 = \hat{\Pi}.$
\newpage
\section{Two Qubit State}
Recall that a two qubit state is written as 
\begin{equation}
    \ket\Psi = \alpha\ket{00} + \beta\ket{01} + \gamma\ket{10} + \delta\ket{11}.
\end{equation}
An \emf{independent} or \emf{separable} state can be written as a tensor product 
\begin{equation}
    \ket{\Psi_\text{sep}} = \left(a\ket 0 + b\ket 1\right)_A \otimes \left(c\ket 0 + d\ket 1\right)_B = ac\ket{00} + ad\ket{01} + bc\ket{10} + bd\ket{11}.
\end{equation}
Note that $\alpha \delta - \beta\gamma = acbd-adbc = 0.$ We can immediately determine if a system can be separated by computing the \emf{concurrence}
\begin{equation}
    C = 2|\alpha \delta - \beta\gamma|.
\end{equation}
If $C\neq 0$, then the system is not separable and is known as \emf{entangled.}
\subsection{Schmidt Decomposition Theorem}
\begin{theorem}
    Any two-qubit pure state can be written as
    \begin{equation}
        \ket\Psi = \hat{U}_A \otimes \hat{U}_B \left(\lambda_0 \ket{00} + \lambda_1{11}\right),
    \end{equation}
    where $\lambda_0,\lambda_1$ are real, positive constants known as $\emf{singular values}$ and they satisfy $\lambda_0^2 + \lambda_1^2 = 1$. The operators $\hat{U}_A,\hat{U}_B$ are unitaries applied separately to each qubit.
\end{theorem}
Consider the unitary operators $\hat{U}_A = \begin{pmatrix}
    a & b \\ -b^* & a^*
\end{pmatrix}$ and $\hat{U}_B = \begin{pmatrix}
    c &d \\ -d^* & c^*
\end{pmatrix}.$ Therefore,
\begin{align}
    \ket\Psi &= \lambda_0\left(a\ket 0 + b\ket 1\right)(c\ket 0 + d\ket 1) + \lambda_1(-b^*\ket 0 + a^* \ket 1)(-d^*\ket 0 + c^*\ket 1) \\ 
    &= (\lambda_0 ac + \lambda_1 b^* d^*) \ket{00} + (\lambda_0 ad - \lambda_1 b^*c^*) \ket{01} + (\lambda_0 bc-\lambda_1a^*d^*)\ket{10} + (\lambda_0bd+\lambda_1a^*c^*)\ket{11}.
\end{align}
This looks very messy, but we can compute the concurrence (and after a length but straightforward computations), we get
\begin{equation}
    C = 2\lambda_0\lambda_1.
\end{equation}
Using $\lambda_0^2+\lambda_1^2=1$, we can obtain the quadratic equation
\begin{equation}
    \lambda^4 - \lambda^2 + (C/2)^2 = 0,
\end{equation}
so $\lambda_0,\lambda_1$ are determined by $C$. The maximum value of $C$ is $C_\text{max} = 1$, which occurs at $\lambda_\text{crit} = \frac{1}{\sqrt 2}.$ At $C=1,$ it is known as a \emf{maximally entangled state.}

This isn't justified yet, but $C$ is the measure of entanglement for 2-qubit states.
\begin{proof}
    Let us rewrite
    \begin{equation}
        \ket\Psi = \sum_{i,j=0}^{1}\chi_{ij}\ket i \ket j
    \end{equation}
    where $\chi_{ij}$ are elements of a $2\times 2$ matrix $\chi = \begin{pmatrix}
        \alpha & \beta \\ \gamma & \delta
    \end{pmatrix}.$ Note that $\chi$ is not hermitian, but both $\hat{\chi}\hat{\chi}^\dagger$ and $\hat{\chi}^\dagger\hat{\chi}$ are hermitian and their eigenvalues are positive.

    We can show they are hermitian by a direct computation. To show their eigenvalues are positive, note that $\bra{\phi}\ket{\phi} \ge 0$ for any state $\phi$ and we can write:
    \begin{equation}
        \bra{\phi}\hat{\chi}\hat{\chi}^\dagger\ket{\phi} = \bra{\phi'}\ket{\phi'} \ge 0.
    \end{equation} 
    Note that $\ket{\phi'}$ is an eigenvector of $\hat{\chi}\hat{\chi}^\dagger.$ Then all the eigenvalues are positive.

    Consider an aribtrary matrix $\begin{pmatrix}
        A & B \\ C & D
    \end{pmatrix}.$ The determinant can be determined by $\lambda^2 - (\text{Tr})\lambda + (\text{Det}) = 0.$ The trace of $\hat{\chi}\hat{\chi}^\dagger$ is $1$ and the determinant is $C^2/4.$ This allows us to calculate $\lambda_0,\lambda_1$. Define
    \begin{equation}
        \Lambda = \begin{pmatrix}
            \lambda_0 & 0 \\ 
            0 & \lambda_1
        \end{pmatrix}.
    \end{equation}
    This allows us to write
    \begin{align*}
        \hat{\chi}\hat{\chi}^\dagger &= \hat{U}\Lambda^2\hat{U}^\dagger \\ 
        \hat{\chi}^\dagger\hat{\chi} &= \hat{V}\Lambda^2\hat{V}^\dagger.
    \end{align*}
    Combining the two together, we end up with the \emf{singular value decomposition}
    \begin{equation}
        \hat{\chi} = \hat{U}\hat{\Lambda}\hat{V}^\dagger. 
    \end{equation}
    We can write an expression for each entry:
    \begin{equation}
        \chi_{ij} = \sum_{p=0}^{1}U_{ip}\lambda_p V_{jp}^*,
    \end{equation}
    which directly leads to the desired relationship.
\end{proof}
\subsection{Operations on Two Qubits}
There are various ways to perform operations. Here are a few ways:
\begin{enumerate}
    \item \emf{Local Unitaries} apply to only one qubit. Namely,
    \begin{equation}
        \ket{\Psi'} = (\hat{U}\otimes \hat{I})\ket{\Psi}.
    \end{equation}
    If $\hat{U} = \begin{pmatrix}
        a & b  \\ -b^* & a^*
    \end{pmatrix}$, then this operation can be represented by 
    \begin{equation}
        \begin{pmatrix}
            \alpha' \\ \beta' \\ \gamma' \\ \delta'
        \end{pmatrix} = \begin{pmatrix}
            a & 0 & b & 0 \\
            0 & a & 0 & b \\ 
            -b^* & 0 & a^* & 0 \\ 
            0 & -b^* & 0 & a^*
        \end{pmatrix}\begin{pmatrix}
            \alpha \\ \beta \\ \gamma \\ \delta
        \end{pmatrix} = \begin{pmatrix}
            a\hat{I} & b\hat{I} \\ 
            -b^*\hat{I} & a^*\hat{I}
        \end{pmatrix}\begin{pmatrix}
            \alpha \\ \beta \\ \gamma \\ \delta
        \end{pmatrix} = (\hat{U} \otimes \hat{I})\begin{pmatrix}
            \alpha \\ \beta \\ \gamma \\ \delta
        \end{pmatrix}.
    \end{equation}
    A similar relationship can be found for operations in the form $\hat{I} \otimes \hat{V}.$
\end{enumerate}
It is important to recognize that local operations can never increase entanglement. So how can we increase entanglement? We start with two qubits in $\ket 0 \ket 0$, and apply a unitary $\hat{U}_1 = \lambda_0 \hat{I} -i\lambda_1 \hat{Y}$ to qubit 1,
\begin{equation}
    \ket 0 \rightarrow \lambda_0\ket 0 + \lambda_1\ket 1.
\end{equation}
such that 
\begin{equation}
    \ket{\Psi_1} = \lambda_0\ket{00} + \lambda_1\ket{11}.
\end{equation}
We then apply a \emf{CNOT} gate by applying a bit flip to qubit 2 if qubit 1 is in $\ket 1$ and do nothing if qubit 1 is in $\ket 0$. However, we have to do this unitarily and reversibly. We can write:
\begin{equation}
    \text{CNOT} = \hat{\Pi}_0 \otimes \hat{I} + \hat{\Pi}_1 \otimes \hat{X}.
\end{equation}
so 
\begin{equation}
    \ket{\Psi_2} = \text{CNOT}(\Psi_1) = \lambda_0\ket{00} + \lambda_1\ket{11}. 
\end{equation}
We then apply local unitaries $\hat{U}_a$ and $\hat{U}_b,$ so 
\begin{equation}
    \ket\Psi_3 = (\hat{U}_a \otimes \hat{U}_b) (\lambda_0\ket{00} + \lambda_1\ket{11}).
\end{equation}

\newpage
\section{Universal Two-Qubit Gates}
A \emf{universal 2-qubit gate} (such as the CNOT gate), along with local unitary operators, can be used to create any two-qubit system. A CNOT gate can be represented as 
\begin{center}
    \begin{quantikz}
        \lstick{$\ket{a}$} & \ghost{X} & \qw & \ctrl{1} & \qw & \qw & \rstick{$\ket{a}$} \\
        \lstick{$\ket{b}$} & \ghost{X} & \qw & \targ{} & \qw &  \qw & \rstick{$\ket{a}\otimes \ket{b}$}
    \end{quantikz}
\end{center}
To test if other gates are universal, we can see if it can be transformed into a CNOT gate. For example, the control-Z gate,
\begin{center}
    \begin{quantikz}
        & \ghost{X} & \qw & \ctrl{1} & \qw & \qw \\
        & \ghost{X} & \qw & \gate{Z} & \qw & \qw
    \end{quantikz}
\end{center}
is also universal. This is equivalent since $HZH=X.$ This is represented by the \emf{control-Z} matrix, given by 
\begin{equation}
    \hat{U}_{CZ} = \begin{pmatrix}
        1 & 0 & 0 & 0 \\ 
        0 & 1 & 0 & 0 \\ 
        0 & 0 & 1 & 0 \\ 
        0 & 0 & 0 & -1
    \end{pmatrix}.
\end{equation}
The \emf{SWAP} gate is given by 
\begin{equation}
    SWAP = \begin{pmatrix}
        1 & 0 & 0 & 0 \\ 
        0 & 0 & 1 & 0 \\ 
        0 & 1 & 0 & 0 \\ 
        0 & 0 & 0 & 0
    \end{pmatrix}
\end{equation}
and reverses the roles of the two qubits, which is equivalent to the circuit
\begin{center}
    \begin{quantikz}
        \qw & \targ{} & \ctrl{1} & \targ{} & \qw \\
        \qw & \ctrl{-1} & \targ{} & \ctrl{-1} & \qw
    \end{quantikz}.
\end{center}
Note that the SWAP gate is not universal. However, the \emf{ROOT-SWAP} gate is universal and is given by:
\begin{equation}
    \sqrt{SWAP} = \begin{pmatrix}
        1 & 0 & 0 & 0 \\
        0 & (1+i)/2 & (1-i)/2 & 0 \\ 
        0 & (i-1)/2 & (1+i)/2 & 0 \\ 
        0 & 0 & 0 & 1
    \end{pmatrix}.
\end{equation}
We can use the SWAP gate along with local unitaries to create control-Z via the following:
\begin{center}
    \begin{quantikz}
        & \qw & \gate[2]{\sqrt{\textsc{swap}}} & \gate{Z} &  \gate[2]{\sqrt{\textsc{swap}}} & \gate{V} & \qw \\
        & \gate{Z} & & \qw & & \gate{V} & \qw
    \end{quantikz}        
\end{center}
where $V=\sqrt{Z} = (\hat{I}-i\hat{Z})/\sqrt{2}$, and can be checked by matrix multiplication.
\subsection{Maximally Entangled States}
Recall that a state is maximally entangled if and only if $C=2|\alpha\delta-\beta\gamma|=1.$ Let us see how we can construct such a state. Consider the circuit:
\begin{center}
    \begin{quantikz}
        \lstick{$\ket{0}$} & \gate{H} & \ctrl{1} & \qw \\ 
        \lstick{$\ket{0}$} & \qw & \gate{X} & \qw \\
    \end{quantikz}
\end{center}
So, the qubits gets transformed to:
\begin{equation}
    \ket{00} \to \frac{1}{\sqrt{2}}(\ket{00}+\ket{10}) \to \frac{1}{\sqrt2}(\ket{00} + \ket{11}) = \ket{\Phi_+},
\end{equation}
so the concurrence is $1$. It turns out we can construct more maximally entangled states. Namely, 
\begin{equation}
    \ket{\beta_k} = i(\hat{I}\otimes \hat{\sigma}_k)\ket{\beta}.
\end{equation}
\end{document}