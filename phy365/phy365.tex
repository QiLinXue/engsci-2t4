\documentclass{article}
\usepackage{qilin}
\tikzstyle{process} = [rectangle, rounded corners, minimum width=1.5cm, minimum height=0.5cm,align=center, draw=black, fill=gray!30, auto]
\title{PHY365: Quantum Information}
\author{QiLin Xue}
\date{Fall 2021}
\usepackage{mathrsfs}
\usetikzlibrary{arrows}
\usepackage{stmaryrd}
\usepackage{accents}
\newcommand{\ubar}[1]{\underaccent{\bar}{#1}}
\usepackage{pgfplots}
\numberwithin{equation}{section}

\begin{document}

\maketitle
\tableofcontents
\newpage
\section{Quantum Coins}
Consider a quantum coin that can be in a superposition of heads and tails. We can write its state as a vector:
\begin{equation}
    |\Psi\rangle = \alpha|H\rangle + \beta|T\rangle  
\end{equation}
which lives in the \emf{Hilbert Space.} Inner products of these vectors can be written as 
\begin{equation}
    \langle \Psi_1 | \Psi_2 \rangle.
\end{equation}
\emf{Born's Rule} tells us we can compute the probability of tails to be $|\beta|^2$ and the probability of heads is $|\alpha|^2$. When there are two quantum coins, there can be four combinations of heads and tails, written as:
\begin{equation}
    |\Psi\rangle = \alpha|HH\rangle + \beta|HT\rangle + \gamma TH\rangle + \delta |TT\rangle.
\end{equation}
In quantum mechanics, we can construct the following state:
\begin{equation}
    |\Psi\rangle = \frac{1}{\sqrt{2}}|HH\rangle + \frac{1}{\sqrt{2}}|TT\rangle,
\end{equation}
which represents \emf{entanglement.} If we measure the first coin, we can instantly know the outcome of the second coin, even if they are lightyears apart.
\end{document}