\documentclass{article}
\usepackage{qilin}
\tikzstyle{process} = [rectangle, rounded corners, minimum width=1.5cm, minimum height=0.5cm,align=center, draw=black, fill=gray!30, auto]
\title{PHY365: Quantum Information}
\author{QiLin Xue}
\date{Fall 2021}
\usepackage{mathrsfs}
\usetikzlibrary{arrows}
\usepackage{stmaryrd}
\usepackage{accents}
\newcommand{\ubar}[1]{\underaccent{\bar}{#1}}
\usepackage{pgfplots}
\numberwithin{equation}{section}

\begin{document}

\maketitle
\tableofcontents
\newpage
\section{Quantum Coins}
Consider a quantum coin that can be in a superposition of heads and tails. We can write its state as a vector:
\begin{equation}
    |\Psi\rangle = \alpha|H\rangle + \beta|T\rangle  
\end{equation}
which lives in the \emf{Hilbert Space.} Inner products of these vectors can be written as 
\begin{equation}
    \langle \Psi_1 | \Psi_2 \rangle.
\end{equation}
\emf{Born's Rule} tells us we can compute the probability of tails to be $|\beta|^2$ and the probability of heads is $|\alpha|^2$. When there are two quantum coins, there can be four combinations of heads and tails, written as:
\begin{equation}
    |\Psi\rangle = \alpha|HH\rangle + \beta|HT\rangle + \gamma TH\rangle + \delta |TT\rangle.
\end{equation}
In quantum mechanics, we can construct the following state:
\begin{equation}
    |\Psi\rangle = \frac{1}{\sqrt{2}}|HH\rangle + \frac{1}{\sqrt{2}}|TT\rangle,
\end{equation}
which represents \emf{entanglement.} If we measure the first coin, we can instantly know the outcome of the second coin, even if they are lightyears apart.
\subsection{Building a Better Computer}
How might we use quantum coins to help us build a ``better'' computer? Before we begin to understand and answer this question, let us understand some key concepts.

First, we can measure \emf{information} as the number of bits (binary digits) that are needed to specify a message. Each bit in a computer requires a physical system that has two possible configurations.
\begin{itemize}
    \item In semiconductor circuits, we use voltage.
    \item Magnetization is sometimes also used (i.e. in hard drives).
    \item Pits in optical storage.
    \item Paper tape with holes in it
\end{itemize}
Now let's extend the idea to quantum bits, i.e. \emf{qubits}. Let us use $|0\rangle$ and $|1\rangle$ to represent the two possible states of a quantum coin, and we can write a qubit as 
\begin{equation}
    |\Psi_1\rangle = \alpha|0\rangle + \beta|1\rangle,
\end{equation}
which isn't necessarily interesting.  If we have two qubits, we can write the state as
\begin{equation}
    |\Psi_2\rangle = \alpha|00\rangle + \beta|01\rangle + \gamma|10\rangle + \delta|11\rangle,
\end{equation}
where the following notation are equivalent:
\begin{equation}
    |00\rangle = |0\rangle|0\rangle = |0\rangle \otimes |0\rangle
\end{equation}
where $\otimes$ is the \emf{tensor product} of two vectors. To make it easier to write, we can also write it as:
\begin{equation}
    | \Psi_2 \rangle = \alpha |0_2\rangle + \beta|1_2\rangle + \gamma |2_2\rangle + \delta|3_2\rangle.
\end{equation}
For three qubits, we have
\begin{equation}
    |\Psi_3\rangle = \alpha |000\rangle + \beta|001\rangle + \gamma|010\rangle + \delta|011\rangle + \epsilon|100\rangle + \zeta|101\rangle + \eta|110\rangle + \theta|111\rangle.
\end{equation}
Therefore, $N$ qubits will have $2^N$ possible states. This suggests that quantum memory can get big, fast.
\subsubsection{Quantum Parallelism}
However, this is not the only difference. Each qubit operation, i.e. $|0\rangle \longleftrightarrow |1\rangle$ affect \textit{all} the probability amplitudes. This also suggests that quantum computers can be extremely efficient.

However, when we make measurements, $N$ qubits only leads to $N$ bits of information. Therefore, even though it is very efficient and quick, there is only a small amount of output.
\begin{example}
    Consider $f:\mathbb{Z}^+\rightarrow \mathbb{R}$ a periodic function that maps $x\in [0,2^L-1]$ (i.e. takes in an $L$ bit integer). There is some $X$ such that $f(x+X)=f(x)$ and we wish to find $X$.
    \vspace{2mm}

    In a classical computer, we would evaluate $f(x)$ for multiple values of $x$. In general, we would expect around $2^{L-1}$ calls in the routine.
    \vspace{2mm}

    However, in a quantum computer, we need $L$ qubits to store values of $x$ (i.e. in the. argument register) and $L$ qubits to store the result of $f(x)$ in the function register. Through a series of bit flips, we can create the state 
    \begin{equation}
        |x\rangle|0\cdots 0\rangle
    \end{equation}
    where the first braket is the input and the second braket is the function register. Then suppose we have a \emf{quantum operation} $\hat{U}_f$ defined such that
    \begin{equation}
        \hat{U}_f|x\rangle |0\rangle = |x\rangle |f(x)\rangle.
    \end{equation}
    But if we prepare the initial state of the register not in $x$, but in a superposition (achieved via a \emf{Hadamard gate}), then we can write:
    \begin{equation}
        \hat{U}_f\frac{1}{N}\left(\sum_{x=0}^{2^k-1} |x\rangle\right)|0\rangle = \frac{1}{N} \underbrace{\sum_{x=0}^{2^k-1} |x\rangle |f(x)\rangle}_{\text{massively entangled state}}.
    \end{equation}
    The difference is that all values of $f(x)$ are generated by a single call on $\hat{U}_f$. If we now apply something called the \emf{Quantum Fourier Transform}
    \begin{equation}
        \hat{U}_{QFT}\sum_x |x\rangle |f(x)\rangle = \frac{1}{N} \sum_x |x\rangle |\tilde{f}(x)\rangle,
    \end{equation}
    where $\tilde{f}$ is the \emf{fourier transform,} which you will get a discrete graph of vertical bars separated a distance by $\frac{n}{X}.$ If we do this a few times, we can extract what $X$ is.
\end{example}
Quantum computers allow us in principle to evaluate periods very efficient. This is a very important problem in \emf{number theory} since period finding helps a great deal in factoring.

Consider coprime $n,a$ and define 
\begin{equation}
    f(x) = a^x \text{ mod } n.
\end{equation} 
This is a periodic function with period $r$. If we can figure out what $r$ is, then
\begin{equation}
    \gcd(a^{r/2} \pm 1, n)
\end{equation}
is a factor of $n$. This is known as \emf{Shor's Algorithm.}
\section{Quantum Mechanics of Quantum Computers}
Suppose there are three qubits. Recall that there are $2^3=8$ possible configurations. These form a basis for a $8$-dimensional vector space. These basis states are known as a \emf{computational basis}.

For a single basis $|\Psi \rangle = \alpha |0\rangle + \beta 1\rangle$, where $\alpha,\beta$ are complex probability amplitudes, then we have
\begin{equation}
    |\alpha|^2+|\beta|^2 = 1 \iff \begin{pmatrix}
        \alpha^*, \beta^*
    \end{pmatrix}\begin{pmatrix}
        \alpha \\ \beta 
    \end{pmatrix} = 1.
\end{equation}
Now suppose we apply a transformation (i.e. operators and gates): 
\begin{align*}
    |\Psi \rangle &\mapsto |\Psi'\rangle \\ 
    \alpha \mapsto \alpha' \\
    \beta \mapsto \beta'.
\end{align*}
We can assume linearity (which has been experimentally validated), and therefore
\begin{align*}
    \alpha' = u_{00}\alpha + u_{01}\beta \\ 
    \beta' = u_{10}\alpha + u_{11}\beta
\end{align*}
which can be written as a matrix
\begin{equation}
    \begin{pmatrix}
        \alpha' \\ \beta'
    \end{pmatrix} = \begin{pmatrix}
        u_{00} & u_{01} \\ u_{10} & u_{11}\end{pmatrix} \begin{pmatrix}
        \alpha \\ \beta
    \end{pmatrix} \iff |\Psi'\rangle = \hat{U} |\Psi\rangle.
\end{equation}
And the complex conjugates are
\begin{equation}
    (\alpha'^*, \beta'^*) = (\alpha^*,\beta^*)\begin{pmatrix}
        u_{00}^* & u_{10}^* \\ u_{01}^* & u_{11}^*
    \end{pmatrix} \iff \langle \Psi'| = \langle \Psi | \hat{U}^\dagger.
\end{equation}
Here are some properties of the complex conjugate:
\begin{itemize}
    \item $(\hat{A}\hat{B})^\dagger = \hat{B}^\dagger\hat{A}^\dagger$
    \item $\langle \psi'|\psi'\rangle = \langle \psi | \hat{U}^\dagger\hat{U} | \Psi\rangle = 1 \iff \hat{U} \text{ is unitary,}$ which is true for all valid quantum operations on a closed system.
\end{itemize}
Let's look at some example gates:
\begin{itemize}
    \item Bit-flip gate:
    \begin{equation}
        \hat{X} = \begin{pmatrix}
            0 & 1 \\ 
            1 & 0.
        \end{pmatrix}
    \end{equation}
    along with the rest of the Pauli matrices:
    \begin{align}
        \hat{Y} &= \begin{pmatrix}
            0 & -i \\ 
            i & 0.
        \end{pmatrix} \\ 
        \hat{Z} &= \begin{pmatrix}
            1 & 0 \\ 
            0 & -1.
        \end{pmatrix} \\
        \hat{I} &= \begin{pmatrix}
            1 & 0 \\ 
            0 & 1.
        \end{pmatrix}.
    \end{align}
    \item Phase-flip gate: $\hat{Z}.$ Note that the overall \emf{phase}, or ``global'' phase is irrelevant, since the norm of the probabilities stay the same.
\end{itemize}
\end{document}