\section{Information Representation}
\begin{itemize}
    \item Everything is encoded using binary, using $0$ and $1$. On a physical level, this could be represented as $0\text{ V}$ and $5\text{ V}$.
    \item With two digits, we can represent $4$ numbers: $00,01,10,11$ which correspond with $0,1,2,3$.
    \item We can represent $2^n$ values using $n$ binary digits.
    \item \textbf{Notation:}
    \begin{itemize}
        \item To represent that we are expressing a number in binary, we prepend $0b$ to the front.
        \item To represent hexadecimal, we prepend $0x$.
        \item Another notation is to write a number with a subscript. For example, $(123)_4$ is $27$ written in base $4$.
    \end{itemize}
    \begin{idea}
        An intuitive way of converting numbers between bases is to understand why we write numbers the way they are in base $10$. For example, suppose we have the number $42069$. This can be written as
        \begin{equation}
            42069 = 4 \times 10^4 + 2 \times 10^3 + 0 \times 10^2 + 6 \times 10^1 + 9 \times 10^0
        \end{equation}
        This is however, not unique. We could have instead just written $4206 \times 10^1 + 9\times 10^0$. The \textit{trick} to finding this \textit{unique} representation is to ``fit'' the greatest amount of a power of $10$, before moving on to the next (smaller) power of $10$. If there were $42069$ candies, you could represent it by trying to create as many groups of $10^4$ as possible, then form as many groups of $10^3$ as possible with the remainder, and so forth.
        \vspace{2mm}

        The same is true for other bases, except replace $10$ with the other base.
    \end{idea}
    \item An algorithm to convert from base 10 to base $b$ is as follows:
    \begin{enumerate}
        \item Divide the number by $b$. Let the remainder be $b_0$
        \item Take the whole part of the quotient, and divide it by $b$. Let the remainder be $b_1$.
        \item Repeat the previous step, except let the remainder be $b_2,b_3,\dots$.
        \item Stop when the quotient is $0$.
        \item The number in base $b$ is thus $(b_k \dots b_3b_2b_1b_0)_b$.
    \end{enumerate}
    \item To convert a number from base $b$ to decimal, we can write them in expanded form. For example:
    \begin{equation}
        (b_3b_2b_1b_0)_b = b_3 \times b^3 + b_2 \times b^2 + b_1 \times b^1 + b_0 \times b^0
    \end{equation}
    \item To convert a number from base $b_1$ to $b_2$, you can first convert the number to decimal, then convert it to $b_2$\footnote{There is another trick explained by bprp \href{https://www.youtube.com/watch?v=nm6wmLD5F9s}{here}}.
    \item We can create a logic gate that takes in two inputs $a$ and $b$ and gives two outputs, $s$ (for the sum) and $c$ (for the carry). This gives the following truth table
    \begin{center}
        \begin{tabular}{|c|c|c|c|}
            \hline
            a & b & s & c \\ 
            \hline 
            0 & 0 & 0 & 0 \\ 
            0 & 1 & 0 & 1 \\ 
            1 & 0 & 0 & 1 \\ 
            1 & 1 & 1 & 0 \\ 
            \hline 
        \end{tabular}
    \end{center}
    \item we can say that $s=1$ when $b=0$ $\verb#AND#$ $a=1$ $\verb#OR#$ $b=1$ $\verb#AND#$ $a=0$.
    \item We can also say that $c=1$ if $b=1$ $\verb#AND#$ $a=1$.
    \begin{idea}
        We can write this in Verilog (preview)
        \begin{verbatim}
            module add(a,b,c,s);
                input a, b;
                output c, s;
                assign c = a & b; // & = AND;
                assign S = (~b & a) | (b & ~a);
        \end{verbatim}
    \end{idea}
\end{itemize}