\documentclass{article}
\usepackage{qilin}
\tikzstyle{process} = [rectangle, rounded corners, minimum width=1.5cm, minimum height=0.5cm,align=center, draw=black, fill=gray!30, auto]
\title{ECE253: Computer and Digital Systems \\ \textbf{Summary}}
\author{QiLin Xue}
\date{Fall 2021}
\usepackage{mathrsfs}
\usetikzlibrary{arrows}
\begin{document}

\maketitle
\tableofcontents
\newpage
\section{Boolean Algebra}
We write multiplication for AND gates and addition for OR gates. This forms an algebraic structure with the following properties. The obvious ones:
\begin{itemize}
    \item $xy=yx$
    \item $x+y=y+x$
    \item $x(yz)=xy(z)$
    \item $x+(y+z)=(x+y)+z$
    \item $x(y+z)=xy+xz$
\end{itemize}
and the less obvious ones: 
\begin{itemize}
    \item $x + yz = (x+y)(x+z)$
    \item $x+xy = x$ (Absorption)
    \item $xy+x\bar{y}=x$ (Combining)
    \item $(x+y)(x+\bar{y})=x$
    \item $\overline{xy}=\bar{x}+\bar{y}$ (De Morgan's Theorem)
    \item $\overline{x+y}=\bar{x}\bar{y}$ 
    \item $x+\bar{x}y=x+y$
    \item $x(\bar{x}+y)=xy$
    \item $xy+tz+\bar{x}z=xy+\bar{x}z$
    \item $(x+y)(y+z)(\bar{x}+z)=(x+y)(\bar{x}+z)$
\end{itemize}
which can be proven using \textbf{perfect induction} (i.e. look at all cases) or algebraic manipulation.
\subsection{Sum of PRoducts and Product of Sums}
Given a truth table, the \textbf{minterm} that corresponds to each row is given by something like 
\begin{equation}
    m_3 = \bar{x}_1x_2x_3
\end{equation}
such that when $3=0b011$ is substituted in, $m_3=1$, and $m_3=0$ otherwise.

The \textbf{maxterm} corresponds to a sum, and $M_i=0$ if and only if the input is $i$. For example, 
\begin{equation}
    M_3 = x_1 + \bar{x_2} + \bar{x_3}
\end{equation}
\end{document}