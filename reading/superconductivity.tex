\documentclass{article}
\usepackage{qilin}
\tikzstyle{process} = [rectangle, rounded corners, minimum width=1.5cm, minimum height=0.5cm,align=center, draw=black, fill=gray!30, auto]
\title{Reading Project: Superconductivity}
\author{QiLin Xue}  
\date{Fall 2022}
\usepackage{mathrsfs}
\usetikzlibrary{arrows}
\usepackage{stmaryrd}
\usepackage{accents}
\newcommand{\ubar}[1]{\underaccent{\bar}{#1}}
\usepackage{pgfplots}
\numberwithin{equation}{section}
\usetikzlibrary{quantikz}
\usepackage[american]{circuitikz}
\newcommand{\equals}{=}

\begin{document}

\maketitle
\tableofcontents
\newpage
\section{Problem Set Two}
\subsection{Ginzburg-Landau Theory of Superconductivity}
Consider a potential 
\begin{equation}
    V(\Delta) = t|\Delta|^2 + u|\Delta|^4,
\end{equation}
which will be derived in the future. Here, $\Delta$ is the energy required to break a cooper pair. When $t$ goes from positive to negative, it signifies a phase transition, where
\begin{equation*}
    t = \frac{T-T_c}{T_c}.
\end{equation*}
\begin{center}
    \begin{tikzpicture}
        \begin{axis}[
        legend pos=outer north east,
        title=Potential $V(|\Delta|)$,
        axis lines = middle,
        xlabel = $|\Delta|$,
        ylabel = $V$,
        variable = t,
        trig format plots = rad,
        ]
        \addplot [
            domain=-1.1:1.1,
            samples=70,
            color=blue,
            ]
            {x^2 + x^4};
        \addlegendentry{$t>0$}
        \addplot [
            domain=-1.1:1.1,
            samples=70,
            color=blue,
            ]
            {-x^2 + x^4};
        \addlegendentry{$t<0$}
        \end{axis}
        \end{tikzpicture}
\end{center}
For $t>0,$ then $\Delta_* =0.$ For $t<0,$ we have $|\Delta_*|^2 = \frac{-t}{2u}.$ Therefore,
\begin{equation*}
    V(\Delta_*) = \begin{cases}
        0 & t>0 \\ 
        -\frac{t^2}{4u} & t<0.
    \end{cases}
\end{equation*}
We can use this to determine what the specific heat capacity is (which is something that is measurable). The specific heat capacity is 
\begin{equation*}
    C = -\frac{\partial^2 F}{\partial t^2}.
\end{equation*}
If we assume that we only have potential energy, we have $F = V,$ which gives 
\begin{equation*}
    C = \begin{cases}
        0 & t>0 \\ 
        \frac{!}{2u} & t< 0,
    \end{cases}
\end{equation*}
so experimentally there will be a discontinuity. Furthermore, the power law relationship that 
\begin{equation*}
    |\Delta| \sim |t|^{1/2}
\end{equation*}
can be experimentally verified.
\subsection{Fermions vs Bosons and 2nd Quantization}
Subatomic particles are either fermions (which obey the Pauli exclusion principle), and bosons.
\begin{itemize}
    \item Electrons are fermions
    \item Photons are bosons
    \item Phonons are bosons
\end{itemize}
Note that no two fermions can be in the same quantum state. For example, consider the two states $\ket{\uparrow}$ and $\ket{\downarrow}.$ If we think from a chemistry perspective, no two fermions can be in the $s$ shell with the same spin. This is a \textit{consequence} of the antisymmetry of the wavefunction. Namely, for fermions 
\begin{equation*}
    \psi(x_1,x_2) = -\psi(x_2,x_1),
\end{equation*}
and for bosons,
\begin{equation*}
    \psi(x_1,x_2) = \psi(x_2,x_1).
\end{equation*}
For \emf{second quantization,} we want to build a Hilbert space for many identical quantum particles. We can define \emf{sectors.}
\begin{idea}
    In many particle quantum mechanics, the number of particles is not a conserved quantity. For example, we can have annhilationa dn creation of particles (converting it from energy to mass and back). In superconductors, we typically have a bath of electrons where we can freely take and put back electrons.
\end{idea}
Instead, we can define \emf{Sectors,} each with a well-defined number of particles. For fermions,
\begin{itemize}
    \item 0 particles: $\ket{0}$ is the \emf{vacuum vector}
    \item 1 particle: $\{\ket{i,\sigma}\}_{i=1,\dots,N_s,\sigma=\uparrow,\downarrow}$ where $N_s$ is the number of sites and $\sigma$ is the site. We are assuming that there is one occupiable orbital site. This gives $2N_s$ single particle states.
    \item 2 particles: $\ket{i\sigma}\ket{j\tau}:$ The naive guess is $(2N_s)^2$ states, but we have to account for the fact that they can't occupy the same state. Instead, there are 
    \begin{equation*}
        \binom{2N_s}{2} = \frac{(2N_s)(2N_s-1)}{2} 
    \end{equation*}
    states.
    \item $n$ particles: We have 
    \begin{equation*}
        \binom{2N_s}{n}
    \end{equation*}
    states, which gets very big, very fast. This is another reason why single-particle quantum mechanics can only get us so far.
\end{itemize}
We wish to define operators which bring us between different sectors. We can define the \emf{creation operator}
\begin{equation*}
    c^\dagger_{i\sigma}\ket{\sigma} = \ket{i\sigma}
\end{equation*}
and the inverse is the \emf{annihilation operator}, where
\begin{equation*}
    c_{i\sigma}c_{i\sigma} = \ket{0}.
\end{equation*}
Note these identities, 
\begin{align*}
    c_{i\sigma}\ket{0} &= 0 \\ 
    (c_{i\sigma})^2 &= 0 \\ 
    (c^\dagger_{i\sigma}) &= 0.
\end{align*}
Furthermore, we can deal with the antisymmetric nature of fermions by noting that 
\begin{equation*}
    c^\dagger_{i\sigma}c^\dagger_{j\sigma'} = -c^\dagger_{j\sigma'}c^\dagger_{i\sigma},
\end{equation*}
so
\begin{equation*}
    \ket{(i\sigma)(j\sigma')} = -\ket{(j\sigma')(i\sigma)}.
\end{equation*}
In general, if we define the anti-commutator $\{A,B\}=AB+BA,$ then 
\begin{equation}
    \{c_{i\sigma},c^\dagger_{j\sigma'}\} = \delta_{ij}\delta_{\sigma\sigma'}.
\end{equation}
Furthermore, annhilation and creation operators anti-commute with themselves, i.e. 
\begin{align*}
    \{c_{i\sigma}, c_{j\sigma'}\} &= 0 \\
    \{c^\dagger_{i\sigma}, c^\dagger_{j\sigma'}\} &= 0.
\end{align*}
For \textit{bosons,} we have 
\begin{align*}
    [b_{i\sigma},b_{j\sigma}^\dagger] &= \delta_{ij}\delta_{\sigma\sigma'} \\
    [b_{i\sigma},b_{j\sigma}] &= 0 \\
    [b_{i\sigma}^\dagger,b_{j\sigma}^\dagger] &= 0,
\end{align*}
where $[A,B]=AB-BA.$

Usually, we can think of an analogy with the quantum harmonic oscillator, where 
\begin{equation}
    \hat{H} = \hbar\omega\left(\underbrace{a^\dagger a}_{n} + \frac{1}{2}\right),
\end{equation}
where we can treat $a^\dagger a$ as the energy level. As an analogy, we can treat the system as only one energy level, and each new particle we add on a constant energy. This is what motivates the creation and annihilation operators.
\end{document}