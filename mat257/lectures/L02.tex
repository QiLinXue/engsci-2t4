\section{Review: Continuity, Distance, and Linear Algebra Review}
\textit{Lecture 2 and 3 (Sept 13 and 15)}
\begin{itemize}
    \item The properties in the previous section will be important when giving a formal definition of continuity.
    \begin{definition}
        If $x,y\in \mathbb{R}^n$, then:
        \begin{equation}
            d(x,y) = \text{``Distance between $x$ and $y$''} = |x-y|
        \end{equation}
    \end{definition}
    \begin{theorem}
        \begin{enumerate}
            \item $d$ is \textbf{symmetric}, i.e. $d(x,y)=d(y,x)$.
            \item $d$ is \textbf{positive definite}, i.e. $d(x,y) \ge 0$ and $d(x,y)=0 \iff x=y$.
            \item Triangle Inequality: $d(x,z) \le d(x,y) + d(y,z).$
        \end{enumerate}
    \end{theorem}
    \begin{proof}
        We prove each separately:
        \begin{enumerate}
            \item $d(x,y) = |x-y| = |-(y-x)| = |-1| \cdot |y-x| = |y-x| = d(y,x)$
            \item $d(x,y) = 0 \iff |x-y| = 0 \iff x-y = 0 \iff x=y$
            \item We need to check that
            \begin{align}
                d(x,z) &\stackrel{?}{\le} d(x,y) + d(y,z) \\ 
                |x-z| &\stackrel{?}{\le} |x-y| + |y-z| \\ 
                |x-z| &\stackrel{?}{\le} |x-z|
            \end{align}
            where the third line comes from the previous triangle inequality. The last statement is true, and the steps are reversible, so we are done.
        \end{enumerate}
    \end{proof}
    \item This theorem is significant as these are the only properties that we need to know about distances to formally define continuity.

    \textit{Note:} In a future section, we will use these properties to \textit{define} a distance function (formally a metric), which is anything that satisfies these properties. This will allow us to generalize continuity to more abstract spaces.
    \item A note on notation. Our definition of a norm is known as the $L^2$ or \textit{Euclidean norm,} i.e.
    \begin{equation}
        |x|_{L^2} = \sqrt{\sum x_i^2}
    \end{equation}
    The $L^1$ norm can be defined as
    \begin{equation}
        |x|_{L^1} = \sum |x_i|
    \end{equation}
    and the infinity norm: 
    \begin{equation}
        |x|_\infty = \max|x_i|.
    \end{equation}
    \item Similarly, distances for $L^1$ and the infinity norms can be defined as $d_1(x,y)=|x-y|_1$ and $d_\infty(x,y)=|x-y|_\infty$.
    
    \textbf{Exercise:} Show that $d_1$ and $d_\infty$ also satisfy the properties of a distance.

    \item There is a bijection from the linear map $\mathbb{R}^n \to \mathbb{R}^m$ and a $m\times n$ matrix:
    \begin{equation}
        \left\{T:\mathbb{R}^n \to \mathbb{R}^m\right\} \longleftrightarrow M_{m\times n}(\mathbb{R})
    \end{equation}
    which is also a homomorphism. We can associate a matrix with any linear transformation, and any linear transformation is associated with a matrix. Here, the standard basis is used.
    \item Specifically, we have the map:
    \begin{equation}
        A \in M_{m\times n} \mapsto     L_A(x)= Ax
    \end{equation}
    where $x\in \mathbb{R}^n$, and 
    \begin{equation}
        T \mapsto M_T = \begin{pmatrix}
            Te_1 & Te_2 & \cdots & Te_n
        \end{pmatrix}
    \end{equation}
    \item We also need to show that this map is bijective, i.e both 
    \begin{equation}
        L_{M_T}=T,\quad M_{L_A}=A
    \end{equation}
    are both satisfied.
    \item Furthermore, we can also show that this map is a homomorphism. Note that the set of linear transformations is itself a vector space. Both $A\mapsto L_A$ and $T\mapsto M_T$ is linear, so 
    \begin{align}
        L_{aA+bB} &= aL_A + bL_b \\ 
        M_{aT+bS} &= aM_T+bM_s.
    \end{align}
    \item Furthermore, suppose we have two maps $T:\mathbb{R}^n \rightarrow \mathbb{R}^m$ and $S:\mathbb{R}^m \rightarrow \mathbb{R}^p$. To go from $\mathbb{R}^n$ to $\mathbb{R}^p$, we can take the composition 
    \begin{equation}
        S \circ T
    \end{equation}
    or
    \begin{equation}
        S \sslash T
    \end{equation}
    The bijection is a homomorphism, so we have 
    \begin{equation}
        M_S M_T = M_{S \circ T}
    \end{equation}
\end{itemize}