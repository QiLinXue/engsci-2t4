\section{Continuity}
\begin{itemize}
    \item Review: Recall that if $C \subseteq \mathbb{R}^n$, then we can map $C$ to its codomain via
          \begin{equation}
              F(C) = \{F(\gamma):\gamma \in C\}.
          \end{equation}
          Similarly, we can take the inverse image. For $D\in \mathbb{R}^m$:
          \begin{equation}
              F^{-1}(D) = \{\gamma \in \mathbb{R}^n: F(\gamma)\in D\}
          \end{equation}
          \begin{idea}
              It turns out that the inverse mapping is better behaved. For example,
              \begin{align}
                  F^{-1}(D_1 \cup D_2) & = F^{-1}(D_1) \cup F^{-1}(D_2) \\
                  F^{-1}(D_1 \cap D_2) & = F^{-1}(D) \cap F^{-1}(D_2)   \\
                  F^{-1}(D^C)          & = F^{-1}(D)^C
              \end{align}
              However, only the following is true for the regular mapping:
              \begin{align}
                  F(C_1 \cup C_2) = F(C_1) \cup F(C_2)
              \end{align}
          \end{idea}
    \item Consider the map $F: \mathbb{R}^n \rightarrow \mathbb{R}^m$, we have the following map
          \begin{equation}
              \begin{pmatrix}
                  x_1 \\ \vdots \\ x_n
              \end{pmatrix}\rightarrow \begin{pmatrix}
                  y_1 \\ \vdots \\ y_m
              \end{pmatrix} = \begin{pmatrix}
                  F_1(x_1\,\cdots\,x_n) \\
                  \vdots                \\
                  F_m(x_1\,\cdots\,x_n)
              \end{pmatrix}
          \end{equation}
          so we always get $F_i:\mathbb{R}^n\rightarrow \mathbb{R}$ for $i=1,\dots,m$, where $F_i$ is known as a coordinate function of $f.$ We can similarly write:
          \begin{equation}
              F_i = \pi_i \circ F = F \sslash \pi_i
          \end{equation}
          where $\pi_i$ is the projection onto the $i^\text{th}$ coordinate and $\circ$ denotes composition.
    \item Recall in one-dimensional calculus, the graph of a function $F:\mathbb{R}\rightarrow \mathbb{R}$ is given by $\Gamma_F = \{(x,f(x)):x\in \mathbb{R}\} \subset \mathbb{R}^2.$ We can generalize this to say that the graph of a function $F:\mathbb{R}^n\rightarrow \mathbb{R}^m$ lives in $\mathbb{R}^{n+m}.$
    \item We can also generalize limits.
          \begin{theorem}
              Suppose we have a function $F:\mathbb{R}^n \rightarrow \mathbb{R}^m$ and let $a\in A \subset \mathbb{R}^n.$ Then
              \begin{equation}
                  \lim_{x\to a}f(x) = b
              \end{equation}
              means that $\forall \epsilon>0, \exists \delta > 0$ s.t. $x\neq a \in B_\delta(a)\cap A \implies F(x) \in B_\epsilon(b).$ IF the limit exists, then it is unique.
          \end{theorem}
    \item Note that we don't necessarily require $a\in A$. However in the case, then the limit may not be unique: in fact, the limit can be anything.
    \item As a result, it makes more sense to talk about the limit as $x$ approaches $a\in\bar{A}$.
          \begin{definition}
              The function $F:A\subset \mathbb{R}^n \rightarrow \mathbb{R}^m$ is continuous at $a\in A$ if $\lim_{x\to a}f(x)=f(a).$
              \vspace{2mm}

              $F$ is continuous on $A$ if and only if it is continuous at every $a\in A$. This is equivalent to saying that for all $a$ and $\epsilon >0$, ther exists a $\delta > 0$ for all $x \in A$ such that $|x-a|<\delta \implies |f(x)-f(a)|<\epsilon.$
          \end{definition}
    \item There is another definition of continuity, which is equivalent to the above, but easier to work with.
          \begin{theorem}
              $F:\mathbb{R}^n\rightarrow \mathbb{R}^m$ is continuous if and only if for every open $V\subset \mathbb{R}^m$, the pre-image $F^{-1}(V)$ is also open.
          \end{theorem}
          \begin{theorem}
              $F:A\subset \mathbb{R}^n \rightarrow \mathbb{R}^m$ is continuous if and only if for every open $V \subset \mathbb{R}^m$, there is an open $U \subset \mathbb{R}^n$ such that
              \begin{equation}
                  F^{-1}(U) = U \cap A
              \end{equation}
          \end{theorem}
    \item Aside: We can define that $B\subset A$ is \textbf{open in $A$} if there exists a $U$ (that is open in $\mathbb{R}^n$) such that $B=U\cap A$. Note that being open in $A$ has the same nice properties as open sets in general.
          \begin{theorem}
              A function $F:A\subset \mathbb{R}^n \rightarrow \mathbb{R}^m$ is continuous if and only if whenever $U\subset \mathbb{R}^m$ is open, there exists an open $V\in \mathbb{R}^n$ such that $F^{-1}(U) = V \cap A$.
          \end{theorem}
          \begin{proof}
              We will only prove this theorem where $A=\mathbb{R}^n$, however the proof can easily extend to arbitrary $A$.

              $\implies$ Assume $U\subset \mathbb{R}^m$ is open, we need to show that $F^{-1}(U)$ is open. Pick $a\in F^{-1}(U)$, then $F(a) \in U$. We pick $\epsilon >0$ such that $B_\epsilon(F(a)) \subset U$, by continuity, we can find $\delta > 0$ such that $F(B_\delta(a)) \subset B_\epsilon(F(a)) \subset U,$ which is equivalent to saying $a\in B_\delta(a) \subset F^{-1}(U).$

              $\impliedby$ Given $a\in \mathbb{R}^n$ and $\epsilon >0$, consider $B_\epsilon(F(a))$ (which is open), so $F^{-1}(B_\epsilon(F(a))) \ni a$ is open. Since this is open, there exists $\delta > 0$ such that $a\in B_\delta(a) \subset F^{-1}(B_\epsilon(F(a))),$ so $F$ is continuous.
          \end{proof}
          \begin{theorem}
              Suppose we have a function $F:\mathbb{R}^n\rightarrow \mathbb{R}^m$ and $G: \mathbb{R}^m \rightarrow \mathbb{R}^p.$ Given that $F,G$ are continuous, then $G \circ F$ is continuous.
              \vspace{2mm}

              \begin{proof}
                  Given $U\subset \mathbb{R}^p$ is open, then
                  \begin{equation}
                      (G\circ F)^{-1}(U) = F^{-1}(G^{-1}(U)) = F^{-1}(\text{open}) = \text{open.}
                  \end{equation}
              \end{proof}
          \end{theorem}
          \begin{theorem}
              If $F:\mathbb{R}^n\rightarrow \mathbb{R}^m$ is continuous and $C\subset \mathbb{R}^n$ is compact, then $F(C)$ is compact.
              \vspace{2mm}

              Alternatively, a continuous image of a compact set is compact.
          \end{theorem}
          \begin{proof}
              Given an open cover $\{U_\alpha\}$ of $f(C)$, then $\{F^{-1}(U_\alpha)\}$ is an open cover of $C$, hence it has a finite subcover. This finite subcover in $\mathbb{R}^n$ corresponds to a finite subcover in $\mathbb{R}^m$.
          \end{proof}
          \begin{corollary}
              A continuous function on a compact set is bounded.
          \end{corollary}
        \subsection{Tutorial Notes}
        \begin{theorem}
            \textbf{Lebesque's Number Lemma:} If $K$ is compact, then given an open cover $\mathcal{O}$, there exists $r>0$ such that all balls of radius with less than $r$ are contained in some $U \in \mathcal{U}.$
        \end{theorem}
\end{itemize}