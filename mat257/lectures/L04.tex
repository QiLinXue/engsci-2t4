\section{Compactness}
\begin{itemize}
    \item We start with an open cover
          \begin{definition}
              An open cover of a set $A$ is a collection $\{U_\alpha\}$ of open sets in $\mathbb{R}^n$ such that
              \begin{equation}
                  \bigcup_\alpha U_\alpha \supset A
              \end{equation}
              A subcover of $\{U_\alpha\}_{\alpha \in I}$ is a collection where $\alpha$ runs over $I' \subset I$ such that
              \begin{equation}
                  \bigcup_{\alpha \subset I'} U_\alpha \supset A
              \end{equation}
          \end{definition}
          \begin{definition}
              $A$ is called compact if every open cover of $A$ has a finite sub-cover.
          \end{definition}
          \begin{example}
              If $F \subset \mathbb{R}^n$ is finite, then it is compact.
          \end{example}
          \begin{example}
              $\mathbb{R}$ is not compact.
              \begin{proof}
                  We just need to find a counterexample. Let $\mathbb{R} = \bigcup_{n\in \mathbb{Z}} (n-1,n+1).$
                  \vspace{2mm}

                  Alternatively, we can write $\mathbb{R} = \bigcup_{n\in \mathbb{Z}} (-n,n).$
              \end{proof}
          \end{example}
    \item We want to eventually classify all compact subsets of $\mathbb{R}^n$.
          \begin{theorem}
              The \textbf{Heine-Borel Theorem} tells us that $[0,1]$ is compact.
          \end{theorem}
          \begin{proof}
              Let $\{U_\alpha\}_{\alpha \in J}$ be an open cover of $I=[0,1].$ To find a finite subcover, we will look for a point $g$ such that we can find a finite subcover that covers $[0,g],$ and we want to push $g$ as far to the right as possible.
              \vspace{2mm}

              Let us define $G=\{g\in [0,1]:\exists J' \subset J \text{ s.t. } \bigcup_{\alpha\in J'} \supset [0,g]\}$ where $J'$ is finite. We want to show that $1\in G.$
              \vspace{2mm}

              Let $\gamma$ be the furthest we can ``push'' $g$. Specifically, let $\gamma = \text{sup} (G)$. This is well defined since $G$ is clearly bounded $G \subset [0,1]$ and is nonempty since $0\in G.$
              \vspace{2mm}

              We claim that $\gamma = 1$. Suppose this is not true, i.e. $\gamma < 1$. If this is true, then there exists some open set $U_\beta$ with $\beta \in J$ such that $\gamma \in U_\beta.$ Since $U_\beta$ is open, there exists $g'$ and $g''$ such that $\gamma \in [g',g''] \subset U_\beta.$ Therefore, $[0,g''] = [0,g']\bigcup [g',g''].$ Since $[0,g']$ has a finite cover, and $[g',g'']$ is covered by $U_\beta$, then this has a finite cover, so $g'' \in G$. However, $\text{sup}(G) \ge g'' > \gamma$, which leads to a contradiction.

          \end{proof}
          \begin{theorem}
              If $A\subset \mathbb{R}^n$ is compact and $B\subset \mathbb{R}^n$ is compact. Then:
              \begin{equation}
                  A \times B \subset \mathbb{R}^{n+m}
              \end{equation}
              is compact.
          \end{theorem}
          \begin{proof}
              Suppose $\mathcal{U}=\{U_\alpha\}$ is an open cover of $A\times B$.

              WLOG, each $U_\alpha$ is itself an open rectangle. The idea behind is that if we draw $A\times B$ as a rectangle, then every vertical line is a copy of $B$, which can be covered with finitely many covers. We then state and prove the following lemma.
              \begin{lemma}
                  For every $x\in A$, we can find an open set $N_x \ni x$ such that $N_x\times B$ can be covered with finitely many of the $U_\alpha$'s.
                  \begin{proof}
                      Write $U_\alpha = V_\alpha \times W_\alpha$ where $V_\alpha$ and $W_\alpha$ are open rectangles in $\mathbb{R}^n$ and $\mathbb{R}^m$, respectively.
                      \vspace{2mm}

                      Consider $\{W_\alpha: x \in V_\alpha\}$ which covers $B$, which is compact, so there is a subcover $\{W_{\alpha_1},\dots,W_{\alpha_p}\}$ that covers $B$. Therefore:
                      \begin{equation}
                          U_{\alpha_1},\dots,U_{\alpha_p}
                      \end{equation}
                      cover $\{x\}\times B$.
                      \vspace{2mm}

                      Let $N_x = \bigcap_{i=1}^n V_{\alpha_i} \subset V_{\alpha_i}$. We also have that for all $\alpha$, $V_\alpha \ni x$. Now
                      \begin{align}
                          N_x \times B & \subset \bigcup_{i=1}^p N_x \times W_{\alpha_i}         \\
                                       & \subset \bigcup_{i=1}^p V_{\alpha_i}\times W_{\alpha_i} \\
                                       & = \bigcup_{i=1}^p U_{\alpha_i},
                      \end{align}
                      which is a finite open cover of $N_x\times B$.
                  \end{proof}
              \end{lemma}
              We know that $\{N_x\}_{x\in A}$ is an open cover of $A$. By compactness, we can find $x_1,\dots,x_q$ such that
              \begin{equation}
                  \bigcup_{j=1}^q N_{x_j} \supset A,
              \end{equation}
              i.e.
              \begin{equation}
                  \bigcup_{j=1}^q N_{x_j}\times B \supset A\times B
              \end{equation}
              For each $j=1,\dots,q$, we can find $U_{ji}$ where $i=1,\dots,p(j)$ such that
              \begin{equation}
                  \bigcup_{i=1}^{p(j)} U_{ji}\supset N_{x_j}\times B.
              \end{equation}
              Now,
              \begin{equation}
                  \bigcup_{j=1}^p \bigcup_{i=1}^{p(j)}U_{ji} \supset A\times B.
              \end{equation}
          \end{proof}

          \begin{corollary}
              Any closed rectangles, i.e. regions in the form
              \begin{equation}
                  R = \prod_{i=1}^n [a_i,b_i]
              \end{equation}
              are compact.
          \end{corollary}
          \begin{proposition}
              A closed subset of a compact set is compact.
          \end{proposition}
          \begin{proof}
              Suppose $C$ is compact and $B \subset C$ is closed.

              This means that $B^C$ is open. Suppose $\{U_\alpha\}$ is an open cover of $B$. Then
              \begin{equation}
                  \{U_\alpha\} \cup \{B^C\}
              \end{equation}
              is an open cover of $C,$ so it has a finite subcover, which contains $U_{\alpha_1},\dots,U_{\alpha_p}$ and maybe $B^C$. If we just consider $U_{\alpha_1},\dots,U_{\alpha_p}$, and this is a finite cover of $B\subset C$.
          \end{proof}
          \begin{corollary}
              Every closed and bounded subset of $\mathbb{R}^n$ is compact.
          \end{corollary}
          \begin{definition}
              Bounded means that there exists $M\in \mathbb{R}$ such that for all $b\in B$, $|b|<M$. This is equivalent to saying that $B$ is contained in some closed rectangle.
          \end{definition}
          \subsection{Extra Notes from Tutorial}
    \item We can make the above corollary stronger (in $\mathbb{R}^n$):
          \begin{theorem}
              $S$ is compact if and only if $S$ is closed and bounded.
          \end{theorem}
          \begin{proof}
              We have already shown that if $S$ is closed and bounded, then $S$ is compact. We now show the other direction.
              \begin{itemize}
                  \item To show that $S$ is bounded, note that by compactness there exists a finite subcover of $S$. Since these subcovers are finite, we can draw a ball or rectangle around this subcover, so it's bounded.
                  \item To show that $S$ is closed, we can show $S^C$ is open. Let $x\in S^C$ and let $y\in S$ be the closest point to $x$ contained in $S$. We can create an open ball centered at $x$ with radius $\frac{1}{3}|x-y|.$ This doesn't intersect $S$, so $S^C$ is open and $S$ is closed.
              \end{itemize}
          \end{proof}
    \item This allows us to create three equivalent definitions of compactness. A set $S$ is compact if and only if:
          \begin{itemize}
              \item Every open cover has a finite subcover.
              \item It is closed and bounded (only true for $\mathbb{R}^n$)
              \item Every sequence of points in $S$ has a convergent subsequence converging to a point $S$.
          \end{itemize}
    \item Similarly, there are also three definitions of continuity.
          \begin{itemize}
              \item We can define continuity using $\epsilon-\delta.$
              \item A function is continuous if and only if pre-image of all open sets are open.
              \item A function is continuous at a point $p$ if whenever $\{a_n\}$ is a real sequence converging to $p$, the sequence $\{f(a_n)\}$ converges to $f(p).$
          \end{itemize}
    \item Finally, we can examine other ways of looking at the interior, boundary, exterior, and closure:
          \begin{itemize}
              \item $\text{Int}(S)$ is the largest open set in $S$
              \item The closure, $\bar{S}=\text{cl}(S)$ is the smallest closed set containing $S$.
          \end{itemize}
          \begin{example}
              \begin{enumerate}
                  \item If $C$ is closed:
                        \begin{enumerate}
                            \item Is $C \subseteq \cl(\Int C)$? {\color{red} False, consider the empty set.}
                            \item Is $C \supseteq \cl(\Int C)$? {\color{red} True, the closure of a subset of $S$ cannot contain points outside of $S$.}
                        \end{enumerate}
                  \item if $U$ is open:
                        \begin{enumerate}
                            \item Is $U\subseteq \Int\cl U$? {\color{red} True, since $\cl U \subseteq \Int \cl U$ and $U\subseteq \cl U$. }
                            \item Is $U\supseteq \Int\cl U$? {\color{red} False, consider $(1,2)\cup (2,3)$. }
                        \end{enumerate}
                  \item \begin{enumerate}
                            \item Is $\Bd S \subseteq \Bd \cl S$? {\color{red} False, consider $(1,2)\cup (2,3)$. }
                            \item Is $\Bd S \supseteq \Bd \cl S$? {\color{red} True, $\Bd S = \bar{S} \setminus \Int S$ and $\Bd \bar{S} = \bar{S} \setminus \Int \bar{S}$. We have $\Int U \subseteq \Int\bar{U}$, so this is true.}
                        \end{enumerate}
              \end{enumerate}
          \end{example}
          \newpage
    \item Given that $f$ is a continuous function from $\mathbb{R}^n$ to $\mathbb{R}^m$, the following statements are true:
          \begin{enumerate}
              \item $f(\text{open})=\text{open}$ {\color{red} Consider $f(x)=c$}
              \item $f(\text{closed})=\text{closed}$ {\color{red} Consider $f(x)=\arctan(x)$}
              \item $f(\text{bounded})=\text{bounded}$  {\color{red} Consider $f(x)=\tan(x)$}
              \item $f(\Int S)=\Int(f(S))$  {\color{red} Consider $f(x)=c$}
              \item $f(\Bd S)=\Bd f(S)$  {\color{red} Consider $f(x)=c$}
          \end{enumerate}
          but these two are true:
          \begin{enumerate}
              \item $f(\cl S)=\cl f(S)$
              \item $F(\text{compact})=\text{compact}$
          \end{enumerate}
\end{itemize}