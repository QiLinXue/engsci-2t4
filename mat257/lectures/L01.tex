\section{Introduction}
\begin{itemize}
    \item Everything in the course will lead to Stokes' Theorem:
    \begin{equation}
        \int\limits_C \dd{W} = \int\limits_{\partial C}W
    \end{equation}
    This generalizes a well-known theorem in one-dimensional calculus, known as the Fundamental Theorem of Calculus:
    \begin{equation}
        \int\limits_{[a,b]} F'(t) = F(b) - F(a) = \int\limits_{\partial[a,b]}F
    \end{equation}
    where $\partial[a,b] = \{\underbrace{b}_{+},\underbrace{a}_{-}\}$.
    \item \textbf{Continuity in} $\mathbb{R}^n$: Recall that continuity in $\mathbb{R}$ is formally defined via $\delta-\epsilon.$ However intuitively it means that if you wiggle the input by a tiny bit, you wiggle the output by a tiny bit.
    
    A similar way can be used to view continuity in $\mathbb{R}^n.$
    \begin{definition}
        For $x,y\in \mathbb{R}^n$, \textbf{the standard (Euclidean) inner product} of $x$ and $y$ denoted 
        \begin{equation}
            \langle x,y\rangle = \sum_{i=1}^n x_iy_i
        \end{equation}
        The \textbf{norm squared} is defined as
        \begin{equation}
            |x|^2 = \langle x,x\rangle
        \end{equation}
        and the \textbf{norm} of $x$ is defined:
        \begin{equation}
            |x| = \sqrt{|x^2|} = \sqrt{\sum_{i=1}^n x_i^2}
        \end{equation}
    \end{definition}
    \begin{idea}
        There are multiple ways of defining $\mathbb{R}^n$. Some people will define it as the set of all column vectors while others define it as the set of all row vectors. In linear algebra, the distinction is important but in real analysis, this distinction is not too important.
    \end{idea}
    \item A \textbf{bilinear} function $f(x,y)$ means that the function is linear in each of the two variables. This means that
    \begin{equation}
        f(ax+by, z) = af(x,z)+bf(y,z)
    \end{equation}
    and similarly the same thing for the other parameter.
    \item A \textbf{semi-linear} function $f(x)$ is one such that
    \begin{equation}
        f(ax)=|a|f(x)
    \end{equation}
    \begin{proposition}
        If $x,y,z \in \mathbb{R}^n$ and $a,b \in \mathbb{R}$, then:
        \begin{enumerate}
            \setcounter{enumi}{-1}
            \item $\langle \cdot, \cdot \rangle$ is bilinear and $|\cdot |$ is semi-linear. Also note that $\langle x,y\rangle = \langle y,x\rangle.$

            \item $|x| \ge 0$ and $|x|=0 \iff x=0$.
            \item Cauchy–Schwarz Inequality: $|\langle x,y\rangle| \le |x||y|$ and equality holds if and only if $x,y$ are dependent.
            \item Triangle Inequality: $|x+y| \le |x|+|y|$
            \item Polarization Identity: $\langle x,y\rangle = \frac{|x+y|^2-|x-y|^2}{4}$
        \end{enumerate}
    \end{proposition}
    \begin{proof}
        We prove each part separately, and skip $0$:
        \begin{enumerate}
            \item We have $|x|=\sqrt{\sum x_i^2} \ge 0$ since every $x_i^2$ is non-negative. Then $|x|=0$ if and only if $x_i=0$, i.e. $x=0$.
            \item Consider and note that $\left||y|^2x - \langle x,y\rangle y\right|^2 \ge 0.$ This is equal to $0$ if and only if the first term (a multiple of $x$) equals the second term (a multiple of $y$), which is equivalent to $x,y$ being dependent.
            
            Next, note that
            \begin{align}
                |s+t|^2 &= \langle s+t, s+t\rangle \\ 
                &= \langle s,s\rangle +\langle s,t\rangle + \langle t,s\rangle + \langle t,t\rangle \\ 
                &= |s|^2 + 2\langle s,t\rangle + |t|^2
            \end{align}
            Using this result, we can simplify the earlier expression to get that
            \begin{align}
                \left||y|^2x - \langle x,y\rangle y\right|^2 &= |y|^4 |x|^2 + \langle x,y\rangle^2|y|^2 - 2|y|^2 \langle x,y\rangle^2 \\ 
                &= |y|^2 (|y|^2|x|^2 - \langle x,y\rangle^2)
            \end{align}
            Since this quantity is non-negative, it follows that $|y|^2|x|^2 \ge \langle x,y\rangle^2$, which is what we wanted to show.
        \end{enumerate}
    \end{proof}
\end{itemize}