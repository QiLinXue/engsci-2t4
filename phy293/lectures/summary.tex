\documentclass{article}
\usepackage{qilin}
\tikzstyle{process} = [rectangle, rounded corners, minimum width=1.5cm, minimum height=0.5cm,align=center, draw=black, fill=gray!30, auto]
\title{PHY293: Waves and Modern Physics \\ \textbf{Summary}}
\author{QiLin Xue}
\date{Fall 2021}
\usepackage{mathrsfs}
\usetikzlibrary{arrows}
\begin{document}

\maketitle
\textit{Disclaimer: A large portion of this document is and will be stolen from David Morin's \href{https://scholar.harvard.edu/david-morin/waves}{Waves} book, which is currently still a draft.}
\section{Dampened Harmonic Motion}
\subsection{Introduction}
\textbf{The Setup:} An object undergoing dampened harmonic motion experiences a restoring force $-kx$ and a resistive force $-b \frac{dx}{dt}$. The differential equation is: 
\begin{equation}
    \frac{d^2x}{dt^2} + \gamma \frac{dx}{dt} + \omega_0^2 x
\end{equation}
where $\gamma = \frac{b}{m}$ and $\omega_0^2 = \frac{k}{m}.$
\begin{warning}
    Most authors prefer to write the differential equation as 
    \begin{equation*}
        \frac{d^2x}{dt^2} + \mathbf{2}\gamma \frac{dx}{dt} + \omega_0^2 x
    \end{equation*}
    as it makes the solution less complicated (i.e. less fractions). Therefore, be very careful when trying to find equations online as we may not all be defining variables the same way.
\end{warning}
\textbf{Motivation for Solution:} In general, a solution to a second order linear differential equation is a sum of exponentials, i.e. it is in the form of 
\begin{equation*}
    x(t) = Ae^{\alpha_1 t} + Be^{\alpha_2 t}
\end{equation*}
where $\alpha_1,\alpha_2$ are solutions to a particular quadratic equation (see Appendix for details), where there are three options: 
\begin{itemize}
    \item Quadratic equation has $2$ solutions $\implies$ Then $x(t)$ is a sum of 2 exponential decays.
    \item Quadratic equation has $1$ solution $\implies$ Then $x(t)$ is a single exponential decay.
    \item Quadratic equation has $0$ solutions $\implies$ Then the roots are complex. Recall from ESC194 that complex exponents lead to sinusoidal functions, so $x(t)$ will have a sinusoidal component. 
\end{itemize}
\subsection{Underdamping ($\gamma < 2\omega_0$)}
We can define
\begin{equation}
    \omega^2 = \omega_0^2 - \frac{\gamma^2}{4},
\end{equation}
which will be the new angular frequency. \textit{Damping reduces the frequency.\footnote{However, this is mostly irrelevant, because if $\gamma$ is large enough to make $\omega$ differ appreciably from $\omega_0$, then the motion becomes negligible after a few cycles anyways. For example, if $\omega$ differs from $\omega_0$ be even $20\%$, then after just $2$ cycles, the amplitude would have decrease to $0.01\%$ of the initial.}} The equation of motion is given by
\begin{equation}
    x_\text{underdamped}(t) = Ae^{-\gamma t/2}\cos(\omega t + \phi)
\end{equation}
where $A$ and $\phi$ are determined by initial conditions.
\subsection{Overdamping ($\gamma > 2\omega_0$)}
If $\gamma > 2\omega_0$, then the equation of motion is given by
\begin{equation}
    x_\text{overdamped}(t) = C_1e^{-\mu_1 t} + C_2 e^{-\mu_2 t}
\end{equation}
where we have\footnote{This shows why most physicists choose to use the $2\gamma$ factor, as it reduces a lot of fractions.} 
\begin{align}
    \mu_1 &= \frac{\gamma}{2} + \sqrt{\frac{\gamma^2}{4} - \omega_0^2} \\ 
    \mu_2 &= \frac{\gamma}{2} - \sqrt{\frac{\gamma^2}{4} - \omega_0^2}
\end{align}
and $C_1,C_2$ are determined by initial conditions.
\subsection{Critical Damping ($\gamma=2\omega_0$)}
Critical damping occurs at $\gamma=2\omega_0$, then the equation of motion is given by 
\begin{equation}
    x_\text{critical}(t) = (A+Bt)e^{-\omega_0t}
\end{equation}
where $A$ and $B$ are determined by initial conditions.

\textbf{Importance of Critical Damping:} Critically damped motion has the property that it \textit{converges} to the origin in the quickest manner, that is, quicker than both the overdamped and underdamped motions.

\newpage
\section{Appendix}
\subsection{Derivations}
Guessing a solution of the form $x(t)=Ce^{\alpha t}$, and substituting it into the ODE gives the characteristic equation 
\begin{equation}
    \alpha^2 + \gamma \alpha + \omega_0^2 = 0
\end{equation}
which has the solution 
\begin{equation}
    \alpha = \frac{-\gamma + \sqrt{\gamma^2 - 4\omega_0^2}}{2}
\end{equation}
The three cases for what the discriminant $\gamma^2-4\omega_0^2$ can be gives us the three cases of motion.

\textbf{Underdamping:} In this case, $\sqrt{\gamma^2-4\omega_0^2}$ is an imaginary number, so let us write: $\alpha = -\gamma/2 + i\sqrt{4\omega_0^2-\gamma^2}$. We can define $\omega$ such that 
\begin{equation}
    \alpha = -\gamma/2 + i\omega.
\end{equation}
Substituting this back into our original guess (and using a linear combination), we get: 
\begin{align}
    x_\text{underdamped}(t) &= C_1e^{(-\gamma/2 + i\omega)t}+C_2e^{(-\gamma/2 - i\omega)t} \\ 
    &= e^{-\gamma t/2}\left(C_1e^{i\omega t} + C_2e^{-i\omega t}\right)
\end{align}
Since $x(t)$ has to be real, the part inside the parentheses has to be real, which means the two terms are complex conjugates of each other. This means that if $C_1=Ce^{i\phi}$, then we must have $C_2 =Ce^{-i\phi}$. Making this substitution leads to 
\begin{equation}
    x_\text{underdamped}(t) = 2Ce^{-\gamma t/2}\cos(\omega t+\phi)
\end{equation}
\textbf{Overdamping:} Note that $\mu_1$ and $\mu_2$ are simply the two solutions to the characteristic equation, so we are left with a simple sum of exponentials: 
\begin{equation}
    C_1e^{-\mu_1 t} + C_2e^{-\mu_2 t},
\end{equation}
obtained by straightforward substitution.

\textbf{Critical Damping:} There is only one root, so a naive guess may be that $x(t) = Ce^{-\gamma/2 t},$ but this cannot be the case as there is only one parameter, $C$, which cannot satisfy two freely chosen initial conditions (i.e. initial position and velocity). It turns out (covered in ESC194) that another solution is $te^{-\gamma/2 t},$ so the full solution is the linear combination 
\begin{align}
    x_\text{critically damped}(t) &= C_1e^{-\gamma/2 t} + C_2te^{-\gamma/2 t} \\ 
    &= (C_1+C_2t)e^{-\gamma/2 t}
\end{align} 
\end{document}