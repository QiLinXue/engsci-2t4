\documentclass{article}
\usepackage{qilin}
\tikzstyle{process} = [rectangle, rounded corners, minimum width=1.5cm, minimum height=0.5cm,align=center, draw=black, fill=gray!30, auto]
\title{PHY293: Waves and Modern Physics \\ \textbf{Summary}}
\author{QiLin Xue}
\date{Fall 2021}
\usepackage{mathrsfs}
\usetikzlibrary{arrows}
\begin{document}

\maketitle
\textit{Disclaimer: A large portion of this document is and will be stolen from David Morin's \href{https://scholar.harvard.edu/david-morin/waves}{Waves} book, which is currently still a draft.}

\tableofcontents
\newpage

\section{Damped Harmonic Motion}
\subsection{Introduction}
\textbf{The Setup:} An object undergoing Damped harmonic motion experiences a restoring force $-kx$ and a resistive force $-b \frac{dx}{dt}$. The differential equation is: 
\begin{equation}
    \frac{d^2x}{dt^2} + \gamma \frac{dx}{dt} + \omega_0^2 x
\end{equation}
where $\gamma = \frac{b}{m}$ and $\omega_0^2 = \frac{k}{m}.$
\begin{warning}
    Most authors prefer to write the differential equation as 
    \begin{equation*}
        \frac{d^2x}{dt^2} + \mathbf{2}\gamma \frac{dx}{dt} + \omega_0^2 x
    \end{equation*}
    as it makes the solution less complicated (i.e. less fractions). Therefore, be very careful when trying to find equations online as we may not all be defining variables the same way.
\end{warning}
\textbf{Motivation for Solution:} In general, a solution to a second order linear differential equation is a sum of exponentials, i.e. it is in the form of 
\begin{equation*}
    x(t) = Ae^{\alpha_1 t} + Be^{\alpha_2 t}
\end{equation*}
where $\alpha_1,\alpha_2$ are solutions to a particular quadratic equation (see Appendix for details), where there are three options: 
\begin{itemize}
    \item Quadratic equation has $2$ solutions $\implies$ Then $x(t)$ is a sum of 2 exponential decays.
    \item Quadratic equation has $1$ solution $\implies$ Then $x(t)$ is a single exponential decay.
    \item Quadratic equation has $0$ solutions $\implies$ Then the roots are complex. Recall from ESC194 that complex exponents lead to sinusoidal functions, so $x(t)$ will have a sinusoidal component. 
\end{itemize}
\subsection{Underdamping ($\gamma < 2\omega_0$)}
We can define
\begin{equation}
    \omega^2 = \omega_0^2 - \frac{\gamma^2}{4},
\end{equation}
which will be the new angular frequency. \textit{Damping reduces the frequency.\footnote{However, this is mostly irrelevant, because if $\gamma$ is large enough to make $\omega$ differ appreciably from $\omega_0$, then the motion becomes negligible after a few cycles anyways. For example, if $\omega$ differs from $\omega_0$ be even $20\%$, then after just $2$ cycles, the amplitude would have decrease to $0.01\%$ of the initial.}} The equation of motion is given by
\begin{equation}
    x_\text{underdamped}(t) = Ae^{-\gamma t/2}\cos(\omega t + \phi)
\end{equation}
where $A$ and $\phi$ are determined by initial conditions.
\subsection{Overdamping ($\gamma > 2\omega_0$)}
If $\gamma > 2\omega_0$, then the equation of motion is given by
\begin{equation}
    x_\text{overdamped}(t) = C_1e^{-\mu_1 t} + C_2 e^{-\mu_2 t}
\end{equation}
where we have\footnote{This shows why most physicists choose to use the $2\gamma$ factor, as it reduces a lot of fractions.} 
\begin{align}
    \mu_1 &= \frac{\gamma}{2} + \sqrt{\frac{\gamma^2}{4} - \omega_0^2} \\ 
    \mu_2 &= \frac{\gamma}{2} - \sqrt{\frac{\gamma^2}{4} - \omega_0^2}
\end{align}
and $C_1,C_2$ are determined by initial conditions.
\subsection{Critical Damping ($\gamma=2\omega_0$)}
Critical damping occurs at $\gamma=2\omega_0$, then the equation of motion is given by 
\begin{equation}
    x_\text{critical}(t) = (A+Bt)e^{-\omega_0t}
\end{equation}
where $A$ and $B$ are determined by initial conditions.

\textbf{Importance of Critical Damping:} Critically damped motion has the property that it \textit{converges} to the origin in the quickest manner, that is, quicker than both the overdamped and underdamped motions.

\subsection{Energy of Underdamped Oscillations}
\textit{Note: We will only focus on very underdamped oscillations.}

\textbf{Underdamped:} For simplicity, let us assume $\phi=0.$ The energy of a damped harmonic oscillator is: 
\begin{equation}
    E = \frac{1}{2}m\dot{x}^2 + \frac{1}{2}kx^2.
\end{equation}
Substituting in $x(t)$ gives 
\begin{equation}
    E = \frac{1}{2}mA_0^2\exp\left(-\gamma t\right)(m\omega_0^2\sin^2(\omega_0t)+k\cos^2(\omega_0t)+\omega_0^2)
\end{equation}
This is very messy, so we want to make approximations.

\textbf{Very Underdamped:} If we look at the case where $\gamma \ll \omega_0$, we can reduce this to
\begin{equation}
    \boxed{E = \frac{1}{2}mA_0^2\omega_0^2\exp\left(-\gamma t\right) = E_0\exp(-\gamma t)}
\end{equation}
as the $\gamma,\gamma^2$ terms approach zero. We can double check that when $\gamma = 0$, this reduces to $E = \frac{1}{2}kA_0^2.$

We can define the lifetime to be $\tau = \frac{1}{\gamma}$.
\subsubsection{Rate of Energy Loss}
The rate of energy loss in a \textbf{very underdamped} system is given by 
\begin{equation}
    \frac{dE}{dt} = -\gamma E
\end{equation}
Note that $E$ here represents the average energy over a period $T$.
\subsubsection{$Q$-Factor}
We can define the $Q$-factor to be 
\begin{equation}
    \boxed{Q = \frac{\omega_0}{\gamma}}.
\end{equation}
If we consider a very underdamped oscillator (where $\gamma \ll \omega_0$), then 
\begin{equation}
    \frac{E(t_1)-E(t_1+T)}{E(t_1)} \approx \gamma T \approx \frac{2\pi \gamma}{\omega} = \frac{2\pi}{Q}.
\end{equation}
Therefore, we have $Q = \frac{E(t_0)}{(E(t_1)-E(t_1+T))/2\pi}$, which is equivalent to the ratio of the initial energy divided by the energy loss per radian: 
\begin{equation}
    \boxed{Q = \frac{\text{initial energy stored}}{\text{energy loss per radian}}}
\end{equation}
We can also use the $Q$-factor to write the differential equation as 
\begin{equation}
    \frac{d^2x}{dt^2}+\frac{\omega_0}{Q}\frac{dx}{dt}+\omega_0^2 x = 0
\end{equation}
and 
\begin{equation}
    \omega = \omega_0\left(1-\frac{1}{4Q^2}\right)^{1/2}
\end{equation}
\newpage
\section{Driven Harmonic Motion}x`
Suppose there is a driving force of the form $F=F_0\cos(\omega t).$ Our differential equation becomes 
\begin{equation}
    F_0\cos(\omega t) = \frac{d^2x}{dt^2}+\gamma \frac{dx}{dt} + \omega_0^2 
\end{equation}
\subsection{Undamped Forced Oscillations}
The solution is in the form of 
\begin{equation}
    x(t) = A(\omega )\cos(\omega t - \delta)
\end{equation}
where 
\begin{equation}
    \tan\delta = 0
\end{equation}
so $\delta = 0$ (if $\omega < \omega_0$) or $\delta = \pi$ (if $\omega > \omega_0$). We have
\begin{equation}
    A(\omega) = \left|\frac{a}{1-\omega^2/\omega_0^2}\right|
\end{equation}
where $a \equiv \frac{F_0}{k}.$
\subsection{Damped Forced Oscillations}
Similarly, the equation is in the same form, except 
\begin{equation}
    \tan\delta = \frac{\omega y}{\omega_0^2-\omega^2}
\end{equation}
and 
\begin{equation}
    A(\omega) = \frac{\omega_0^2 a}{\sqrt{(\omega_0^2-\omega^2)^2 + (\omega \gamma)^2}}
\end{equation}
There are three important regimes: 
\begin{itemize}
    \item $\omega \to 0$ gives $A(\omega) \rightarrow a = \frac{F_0}{k}$
    \item $\omega \to \omega_0$ gives $A(\omega) \to \frac{a\omega_0}{\gamma}$
    \item $\omega\to\infty$ gives $A(\omega)\to 0$
\end{itemize}
The phase shift is separated into three regimes as well: 
\begin{itemize}
    \item $\omega \to 0$ gives $\delta \to 0$
    \item $\omega \to \omega_0$ gives $\delta \to \frac{\pi}{2}$
    \item $\omega\to\infty$ gives $\delta \to \pi$.
\end{itemize}

\subsection{Power}
The power of the damping force is
\begin{equation}
    \bar{P}_\text{damping}=-\frac{1}{2}b(\omega A)^2
\end{equation}
and 
\begin{equation}
    \bar{P}_\text{driving}=\frac{1}{2}F_0\omega A\sin\delta = \frac{1}{2}b(\omega A)^2 
\end{equation}
after making the substitution $\sin\delta = \frac{\gamma \omega m A}{F_0}.$ We can also write $\bar{P}_\text{driving}$ in terms of the frequency: 
\begin{equation}
    \bar{P}_\text{driving} = \frac{F_0^2}{2\gamma m}\cdot \frac{\gamma^2\omega^2}{(\omega_0^2-\omega^2)^2+\gamma^2\omega^2}
\end{equation}
\subsubsection{The $\bar{P}(\omega)$ Curve}
If the driving frequency is close to the natural frequency, we can write 
\begin{equation}
    \omega^2-\omega_0^2 \approx -2\omega_0\Delta \omega,
\end{equation}
such that 
\begin{equation}
    \bar{P}(\omega) = \frac{F_0^2}{2m\gamma} \frac{\gamma^2}{4(\Delta \omega)^2+\gamma^2}
\end{equation}
\subsection{Transient Phenomena}
Recall that there are actually two solutions to the differential equation. One is the steady state forced oscillations, however there is another solution known as the transient response: 
\begin{equation}
    x_1(t)+x_2(t) = A(\omega)\cos(\omega_d t- \delta) + B\exp\left(-\frac{\gamma t}{2}\right)\cos\left(\omega_\text{damped} t + \phi\right)
\end{equation}

\newpage
\section{Coupled Harmonic Oscillators}
Suppose we h

\newpage
\section{Appendix}
\subsection{Derivations}
Guessing a solution of the form $x(t)=Ce^{\alpha t}$, and substituting it into the ODE gives the characteristic equation 
\begin{equation}
    \alpha^2 + \gamma \alpha + \omega_0^2 = 0
\end{equation}
which has the solution 
\begin{equation}
    \alpha = \frac{-\gamma + \sqrt{\gamma^2 - 4\omega_0^2}}{2}
\end{equation}
The three cases for what the discriminant $\gamma^2-4\omega_0^2$ can be gives us the three cases of motion.

\textbf{Underdamping:} In this case, $\sqrt{\gamma^2-4\omega_0^2}$ is an imaginary number, so let us write: $\alpha = -\gamma/2 + i\sqrt{4\omega_0^2-\gamma^2}$. We can define $\omega$ such that 
\begin{equation}
    \alpha = -\gamma/2 + i\omega.
\end{equation}
Substituting this back into our original guess (and using a linear combination), we get: 
\begin{align}
    x_\text{underdamped}(t) &= C_1e^{(-\gamma/2 + i\omega)t}+C_2e^{(-\gamma/2 - i\omega)t} \\ 
    &= e^{-\gamma t/2}\left(C_1e^{i\omega t} + C_2e^{-i\omega t}\right)
\end{align}
Since $x(t)$ has to be real, the part inside the parentheses has to be real, which means the two terms are complex conjugates of each other. This means that if $C_1=Ce^{i\phi}$, then we must have $C_2 =Ce^{-i\phi}$. Making this substitution leads to 
\begin{equation}
    x_\text{underdamped}(t) = 2Ce^{-\gamma t/2}\cos(\omega t+\phi)
\end{equation}
\textbf{Overdamping:} Note that $\mu_1$ and $\mu_2$ are simply the two solutions to the characteristic equation, so we are left with a simple sum of exponentials: 
\begin{equation}
    C_1e^{-\mu_1 t} + C_2e^{-\mu_2 t},
\end{equation}
obtained by straightforward substitution.

\textbf{Critical Damping:} There is only one root, so a naive guess may be that $x(t) = Ce^{-\gamma/2 t},$ but this cannot be the case as there is only one parameter, $C$, which cannot satisfy two freely chosen initial conditions (i.e. initial position and velocity). It turns out (covered in ESC194) that another solution is $te^{-\gamma/2 t},$ so the full solution is the linear combination 
\begin{align}
    x_\text{critically damped}(t) &= C_1e^{-\gamma/2 t} + C_2te^{-\gamma/2 t} \\ 
    &= (C_1+C_2t)e^{-\gamma/2 t}
\end{align}

\end{document}