\documentclass{article}
\usepackage{qilin}
\tikzstyle{process} = [rectangle, rounded corners, minimum width=1.5cm, minimum height=0.5cm,align=center, draw=black, fill=gray!30, auto]
\title{PHY293: Waves and Modern Physics \\ \textbf{Special Relativity}}
\author{QiLin Xue}
\date{Fall 2021}
\usepackage{mathrsfs}
\usetikzlibrary{arrows}
\begin{document}

\maketitle
\section{Basics}
Special relativity is based on two postulates:
\begin{itemize}
    \item The laws of physics are the same in all inertial reference frames.
    \item The speed of light will travel at the same speed measured in any reference frame.
\end{itemize}
While the first postulate is ``obvious,'' the second may be counterintuitive: If we throw a ball on a train at $1\text{ m/s}$, someone standing on a platform will see the ball travel faster than $1\text{ m/s}$. This is \textit{not} true for light.

With these two postulates, we can derive time dilation and length contraction. A person $A$ moving at a speed $v$ relative to person $B$ will observe the other person to experience time
\begin{equation}
    t' = t\gamma
\end{equation}
and person $B$'s rocket ship will be contracted a distance
\begin{equation}
    \ell' = \ell/\gamma
\end{equation}
where $\gamma = \frac{1}{\sqrt{1-(v/c)^2}}$.
\end{document}