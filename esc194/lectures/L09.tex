\section{Lecture 9}
\begin{itemize}
    \item We want to define a new number $f'(a)$ where it is defined as:
    \begin{equation}
        f'(a) \equiv \lim_{h\to 0} \frac{f(a+h)-f(a)}{h}
        \label{eq:}
    \end{equation}
    if it exists where $a \in$ domain of $f(x)$. Note this is known as the derivative of $f(x)$ evaluated at $x=a$.
    \begin{example}
        Determine the derivative of $f(x)=x^3$ at $a-1$. We can evaluate:
        \begin{align}
            f'(1)&=\lim_{h\to 0} \frac{f(1+h)-f(1)}{h} \\ 
            &=\lim_{h\to 0} \frac{(1+h)^3-1}{h} \\
            &= \lim_{h\to 0}\frac{1+3h+3h^2+h^3-1}{h} \\ 
            &= \lim_{h\to 0}\frac{3h+3h^2+h^3}{h} \\ 
            &= \lim_{h\to 0} (3+3h+h^2) & \text{provided that } h\neq 0 \\ 
            &= 3 &\text{polynomial limit theorem}
            \label{eq:}
        \end{align}
    \end{example}
    \item $h$ disappears when evaluating the limit, as a result, it is known as a \textbf{dummy variable}.
    \item The tangent line to the curve $f(x)$ at $x=a$ is given by the equation:
    \begin{equation}
        y_\text{tangent line}(x)\equiv f(a)+f'(a)(x-a)
        \label{eq:}
    \end{equation}
    \item We can define average speed between $t_0$ and $t_0+h$ to be:
    \begin{equation}
        v_\text{avg} =\frac{s(t_0+h)-s(t_0)}{h}
        \label{eq:}
    \end{equation}
    where $s(t)$ is the displacement. We can use our rigorous definition of the derivative to make a rigorous definition of ``speed at an instant $t_0$'' to be:
    \begin{equation}
        v(t_0) \equiv \lim_{h\to 0} \frac{s(t_0+h)-s(t_0)}{h}
        \label{eq:}
    \end{equation}
    \item We can extend further and define a new \textit{function} $f'(x)$ such that:
    \begin{equation}
        f'(x) \equiv \lim_{h\to 0} \frac{f(x+h)-f(x)}{h}
        \label{eq:}
    \end{equation}
    There are two variables here, but it is still rigorously defined since $h$ is still a dummy variable and it will still disappear after we evaluate the limit.
    \begin{definition}
        If $f'(a)$ exists we say $f(x)$ is differentiable at $a$.
    \end{definition}
    \begin{definition}
        If $f'(a)$ is differentiable at all $x\in $ domain of $f(x)$, we can say that $f(x)$ is a differentiable function.
    \end{definition}
    \begin{example}
        Let $f(x)=x^2$. Find $f'(x)$:
        \begin{align}
            f'(x) &= \lim_{h\to 0} \frac{(x+h)^2-x^2}{h} \\ 
            &= \lim_{h\to 0} \frac{(x^2_2hx_h^2-x^2}{h} \\ 
            &= \lim_{h\to 0} \frac{2hx+h^2}{h} \\ 
            &= \lim_{h\to 0} (2x+h)
        \end{align}
        where we canceled as $h\neq 0$. This is a polynomial function of $h$ as far as $\lim_{h\to 0}$ is concerned! Therefore, by the polynomial limit theorem, we get:
        \begin{equation}
            f'(x) = 2x
            \label{eq:}
        \end{equation}
    \end{example}
    \begin{example}
        Let $f(x)=x^n$. Prove that $f'(x)=nx^{n-1}$.
        \begin{align}
            f'(x) &= \lim_{h\to 0}\frac{(x+n)^n-x^n}{h} \\ 
            &= \lim_{h\to 0} \frac{x^n+nx^{n-1}h+\binom{n}{2}x^{n-2}h^2+\cdots + \cdots + h^n-x^n}{h} \\ 
            &=  \lim_{h\to 0} \frac{nx^{n-1}h+\binom{n}{2}x^{n-2}h^2+\cdots + \cdots + h^n}{h} \\ 
            &= \lim_{h\to 0} nx^{n-1}+\binom{n}{2}x^{n-2}h+\cdots + \cdots + h^{n-1} \\ 
            &= nx^{n-1}
        \end{align}
    \end{example}
    Later we will show that this is true for any real number, and not just for positive integers.
    \item Note that there are different notations. For example, the Leibniz notation gives us:
    \begin{equation}
        f'(x) \equiv \frac{df}{dx}
        \label{eq:}
    \end{equation}
    
\end{itemize}