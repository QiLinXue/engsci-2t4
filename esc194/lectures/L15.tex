\section{Defining Horizontal Asymptotes}
\begin{itemize}
    \item We can define horizontal asymptotes:
    \begin{definition}
        A horizontal asymptote occurs when $\lim_{x\to\infty}f(x)=L$. We can say that 
        $f(x)$ goes to $L$ as $x$ goes to infinity if for any $\epsilon>0$, a number $A$ can be found s.t. for all $x>A$, $|f(x)-L|<\epsilon$.
        \vspace{2mm}

        The idea behind this revolves around finding $f$ values as close to $L$ as might be wanted by going to large enough $x$ values.
    \end{definition}
    \item Geometrically, we can say that if $\displaystyle \lim_{x\to\infty}f(x) = L$, then the line $y=L$ is the horizontal asymptote of $f(x)$ at $x=\infty$.
    \begin{theorem}
        The reciprocal horizontal asymptote limit:
        \begin{equation}
            \lim_{x\to \pm \infty} \frac{1}{x^r} = 0
            \label{eq:}
        \end{equation}
    \end{theorem}
    \item Slant asymptotes are a thing too. For example, suppose we have the function:
    \begin{equation}
        f(x)=\frac{x+2}{1+\frac{1}{x^2}}
        \label{eq:}
    \end{equation}
    We might say that intuitively $f(x)\to x+2$ as $x\to \infty$.
    \begin{definition}
        If $\lim_{x\to \infty} \left[f(x)-(mx+b)\right]=0$, then $y=mx+b$ is a slant asymptote to $f(x)$ at $+\infty$.
    \end{definition}
    \item Curve Sketching Check-list:
    \begin{itemize}
        \item Find domain/range/limits at infinity, end points if they exist, vert/horz/slant asymptote
        \item Intercepts: Find x/y intercepts.
        \item Establish if $f(x)$ is symmetrical/even/odd/periodic
        \item Find $f'(x)$ then find all critical points and $f(c_\text{crit})$. Find when $f(x)$ is increasing/decreasing. Use 1st derivative test (QT2). Find vertical tangents/cusps if they exist.
        \item Find $f''(x)$, find where $f(x)$ is concave up/down. Find points of inflection if they exist. (Optional: use 2nd derivative test QT4 to confirm local max/min)
        \item Choose largest and smallest values of $f$ from the above as abs. max, min, if they exist.
    \end{itemize}
\end{itemize}