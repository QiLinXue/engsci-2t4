\section{Exponential Growth and Decay}
\begin{itemize}
    \item There are many instances in physics and nature where the growth of a function is related to the function at that point, such as:
    \begin{equation}
        \frac{df}{dt} = kf(t) \implies k =\frac{1}{f}\frac{df}{dt} = \frac{d}{dt}(\ln f)
        \label{eq:}
    \end{equation}
    Separating, we get:
    \begin{equation}
        \ln(f) + kt + C \implies f = Ce^{kt}
        \label{eq:}
    \end{equation}
    where $C$ is based off initial conditions.
    \item The \textbf{doubling time} refers to the time for a function to double:
    \begin{equation}
        2P_0 = P_0 e^{kt_2} \implies t_2 = \frac{\ln 2}{k}
        \label{eq:}
    \end{equation}
    \item In many areas (such as radioactive decay), the half life gives the time necessary for the function to half. This occurs in functions where the DE looks like:
    \begin{equation}
        \frac{df}{dt} = -kN
        \label{eq:}
    \end{equation}
    where $k>0$. Similarly, the half life is given by:
    \begin{equation}
        t_{1/2} = \frac{\ln 2}{k}
        \label{eq:}
    \end{equation}
    \item For compound interest, the annual interest is given by:
    \begin{equation}
        V(t) = V_0(1+i)^t
        \label{eq:}
    \end{equation}
    If we compound the interest more and more often, we get:
    \begin{equation}
        V(t) = V_0\left(1+\frac{i}{n}\right)^{nt}
        \label{eq:}
    \end{equation}
    Taking the limit as $n\to \infty$, we get:
    \begin{align}
        \lim_{n\to\infty} \left(1+\frac{i}{n}\right)^{nt} &= V_0\lim_{m\to \infty} \left(\left(1+\frac{1}{m}\right)^m\right)^{it} \\ 
        &= V_0e^{it}
    \end{align}
    where we made the substitution $m=n/i$.
    \item The \textbf{logistic model} is a more realistic model for population growth:
    \begin{equation}
        \frac{dP}{dt} = kP\left(1-\frac{P}{M}\right)
        \label{eq:}
    \end{equation}
    where $M$ is the carry capacity or max population:
    \begin{align}
        \int \frac{\dd{P}}{P(1-P/M)} &= k \int \dd{t} \\ 
        \int \left(\frac{1}{P}+\frac{1}{M-P}\right) \dd{P} &= k\int \dd{t} \\ 
        \ln|P|-\ln|M-P| &= kt + C \\ 
        \ln\left|\frac{P}{M-P}\right| &= kt + C \\ 
        \frac{P}{M-P} &= \pm e^{kt+C} \\
        P(t) &= \frac{M}{1+Ae^{-kt}}  
    \end{align}
    where $A \equiv \frac{M-P_0}{P_0}$ and $P_0 = P(t=0)$.
\end{itemize}