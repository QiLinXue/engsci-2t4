\section{The Natural Exponential Function}
\begin{itemize}
    \item The exponential function can be written as:
    \begin{equation}
        \ln e^{p/q} = \frac{p}{q}
        \label{eq:}
    \end{equation}
    where $p$ and $q$ are integers. Thus, there must also be some number $q$ such that:
    \begin{equation}
        \ln q = \pi \implies q = e^{\pi}
        \label{eq:}
    \end{equation}
    \begin{definition}
        If $z$ is irrational, then $e^z$ is the unique number such that:
        \begin{equation}
            \ln(e^z) = z
            \label{eq:}
        \end{equation}
    \end{definition}  
    \begin{definition}
        The exponential function is written as:
        \begin{equation}
            \exp\left(x\right)=e^x
            \label{eq:}
        \end{equation}
        for all real $x$.
    \end{definition}
    \item Here are the properties of $e^x$:
    \begin{enumerate}
        \item $\ln(e^x)=x$ where $x\in\mathbb{R}$
        \item The graph looks like the following
        \begin{center}
            \begin{tikzpicture}
                \begin{tikzpicture}
                \begin{axis}[
                legend pos=outer north east,
                title=Plot of $e^x$ along with $\ln x$,
                axis lines = middle,
                xlabel = $x$,
                ylabel = $y$,
                variable = t,
                trig format plots = rad,
                ymin=-2,
                ymax=2
                ]
                \addplot [
                    domain=-2:2,
                    samples=70,
                    color=blue,
                    ]
                    {exp(x)};
                \addlegendentry{$e^x$}
                \addplot [
                    domain=0:2,
                    samples=400,
                    color=blue,
                    ]
                    {ln(x)};
                \addlegendentry{$\ln(x)$}
                \addplot [
                    domain=-2:2,
                    samples=70,
                    color=black,
                    dotted
                    ]
                    {x};                
                \end{axis}
                \end{tikzpicture}
            \end{tikzpicture}
        \end{center}
        \item $e^x>0$
        
        The exponential function is always greater than zero. Comes from the fact that it is the inverse of the natural logarithm function.
        \item $e^0=1$
        \item $\lim_{x\to-\infty} e^x = 0$
        \item $e^{\ln x} = x$
        \item $e^{a+b}=e^ae^b$ for all real $a$ and $b$
        \item $e^{-b}=\frac{1}{e^b}$
        \item $\frac{d}{dx}e^x=e^x$
        \begin{prooof}
            $$\ln(e^x)= x \implies \frac{d}{dx} \ln(e^x) = \frac{1}{e^x} \cdot \frac{d}{dx}(e^x) = 1$$
        \end{prooof}
        \item $\frac{d}{dx}e^{u(x)}=e^u \frac{du}{dx}.$
        
        For example, $\frac{d}{dx}e^{kx} = ke^{kx}$ which is a relationship appears many times as this represents a relationship in which the rate of growth/decay is proportional to the actual amount.
        \item $\int e^x \dd{x} = e^x + C$
        \item $\int e^{g(x)}g'(x) \dd{x} = e^{g(x)}+C$
    \end{enumerate}
    \begin{example}
        Show $e^x>1+x$ for $x>0$. We can show this via integration:
        \begin{equation}
            e^x = 1+ \int_0^x e^{t} \dd{t} = 1+e^x-e^0 = e^x
            \label{eq:}
        \end{equation}
        Note that $\frac{d}{dx}e^x > 0$ is always increasing. Therefore, we can claim that $e^x > 1$ for $x>0$. Therefore:
        \begin{align}
            1 + \int_0^x e^t \dd{t} &> 1 + \int_0^x \dd{t} = 1+x \\ 
            e^x &> 1 + x
        \end{align}
        If we continue this, we will get:
        \begin{equation}
            e^x > 1+x+\frac{x^2}{2!}+\frac{x^3}{3!}+\cdots+\frac{x^n}{n!}
            \label{eq:}
        \end{equation}
        Try this yourself pls.
    \end{example}
    \begin{example}
        Sketch the curve $f(x) = x^4e^{-x}$. We go through our checklist:
        \begin{itemize}
            \item The domain of $f$ is $\mathbb{R}$ and the range is $f \ge 0$.
            \item The derivative is
            \begin{equation}
                f' = 4x^3e^{-x} + x^4(-1)e^{-x} = x^3e^{-x}(4-x)
            \end{equation}
            and is equal to $f'=0$ when $x=0,4$. We have $f' < 0$ for $x>4$ or $x<0$ and $f'>0$ for $0<x<4.$
            \item The local minimum is at $f(0)=0$ and the local maximum is at $f(4)=256e^{-4} \approx 4.2.$
            \item The second derivative is
            \begin{align}
                f''(x) &= 12x^2e^{-x} - 4x^3e^{-x} - 4x^3e^{-x} + x^4e^{-x} \\ 
                &= x^2e^{-x}(x^2-8x+12) \\ 
                &= x^2e^{-x}(x-6)(x-2)
            \end{align}
            We have:
            \begin{itemize}
                \item $f'' = 0$ at $x=0,2,6$.
                \item $f'' > 0$ for $x>6$ (concave up)
                \item $f'' < 0$ for $2<x<6$ (concave down)
                \item $f'' > 0$ for $x<2$ (concave up)
            \end{itemize}
            The points of inflection are at $x=2$ and $x=6$.
        \end{itemize}
        We can finally draw the picture:
        \begin{center}
            \begin{tikzpicture}
                \begin{tikzpicture}
                \begin{axis}[
                legend pos=outer north east,
                title=Example Graph,
                axis lines = middle,
                xlabel = $x$,
                ylabel = $y$,
                variable = t,
                trig format plots = rad,
                ]
                \addplot [
                    domain=-1.5:10,
                    samples=200,
                    color=blue,
                    ]
                    {x^4*exp(-x)};
                \addlegendentry{$x^4e^{-x}$}
                
                \end{axis}
                \end{tikzpicture}
            \end{tikzpicture}
        \end{center}
        Interestingly, this is part of a family of functions that have applications in medicine (among other fields).
    \end{example}
    \begin{example}
        Suppose we wish to integrate 
        \begin{equation}
            I = \int \frac{xe^{ax^2}}{e^{ax^2}+1} \dd{x}
        \end{equation}
        We motivate our solution by noting that the derivative of the denominator is contained in the numerator. Let $u=e^{ax^2}+1$, $\dd{u} = 2axe^{ax^2}\dd{x}$, such that
        \begin{equation}
            I = \frac{1}{2a} \int \frac{\dd{u}}{u} = \frac{1}{2a}\ln(e^{ax^2}+1) + C
        \end{equation}
        where we have removed the absolute value signs since the argument of the natural logarithm is always positive.
    \end{example}
    \begin{example}
        Let us calculate $\int_0^{\sqrt{2\ln 3}} xe^{-x^2/2} \dd{x}$. This type of integral will show up often in ECE286. Let $u=-\frac{1}{2}x^2$ and $\dd{u} = - x \dd{x}$. We can change the bounds to $u=0$ (at $x=0$) and $u=-\ln 3$ (at $x=\sqrt{2\ln 3}$). This gives 
        \begin{align}
            \int_0^{\sqrt{2\ln 3}} x e^{-x^2/2} \dd{x} &= - \int_0^{-\ln 3} e^u \dd{u} \\ 
            &= -[e^u]^{-\ln 3}_0 \\ 
            &= \frac{2}{3}
        \end{align}
    \end{example}
\end{itemize}