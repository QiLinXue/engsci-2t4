\section{Continuity}
\begin{itemize}
    \item A ``continuous function'' is intuitively clear, but how do we define it rigorously?
    \begin{definition}
        $f(x)$ is ``continuous at $c$'' if
        \begin{equation}
            \lim_{x\to c}f(x)=f(c)
            \label{eq:}
        \end{equation}
    \end{definition}
    \begin{definition}
        A function $f(x)$ is discontinuous at $c$ if it is not continuous.
    \end{definition}
    \item There are various types of discontinuity:
    \begin{itemize}
        \item \textbf{Jump Discontinuity-} For example:
        \begin{equation}
            f(x)=\frac{|x|}{x}
            \label{eq:}
        \end{equation}
        \item \textbf{Removable Discontinuity-} For example:
        \item Discontinuity because either $f(c)$ DNE or $\displaystyle \lim_{x\to c} f(x)$ DNE, or both.
    \end{itemize}
    \item There are also continuity theorems.
    \begin{theorem}
        If $f(x)$ is continuous at every $x\in[a,b]$, then $f(x)$ is integrable on $[a,b]$ i.e. $\int_a^b f(x) \dd{x}$ exists.
    \end{theorem}
    \begin{theorem}
        Given $f$, $g$ is continuous at $a$, then $f(x)+g(x)$ is continuous at $a$.
    \end{theorem}
    \begin{prooof}
        Apply the additivity L.T lmao.
    \end{prooof}
    \item There is also such a thing as a one-sided continuity.
    \begin{definition}
        $f(x)$ is continuous on the right at $c$ if $\displaystyle\lim_{x\to c^+}f(x)=f(c)$.
    \end{definition}
    \item amd for an interval.
    \begin{definition}
        $f(x)$ is continuous on $(a,b)$ iff $f(x)$ is continuous at all $x \in (a,b)$.
    \end{definition}
    \begin{definition}
        $f(x)$ is continuous on $[a,b]$ iff $f(x)$ is continuous on $(a,b)$ and $f(x)$ is continuous from the right of $a$ and from the left of $b$.
    \end{definition}
    \begin{theorem}
        If $g(x)$ is continuous at $a$ and $f(x)$ is continuous at $g(a)$, then $f(g(x))$ is continuous at $a$.
    \end{theorem}
    \item We can now introduce the \textbf{Intermediate Value Theorem}. Recap: We simply want there to be a number $q$ such that $q\cdot q=2$, i.e. $\sqrt{2}$ exists. We proved such a number does not exist among the rationals. So we imposed a new axiom CORA and then defined:
    \begin{equation}
        \sqrt{2}=\text{lub}\{x:x^2<2\}
        \label{eq:}
    \end{equation}
    which CORA guarantees exists as a real number. This however doesn't tell us that $\sqrt{2}\cdot\sqrt{2}=2$, but we want to use this result. How can we rigorously prove that:
    \begin{equation}
        [\text{lub}\{x:x^2<2\}]\cdot [\text{lub}\{x:x^2<2\}] = 2
        \label{eq:}
    \end{equation}
    \begin{theorem}
        \begin{enumerate}
            \item Given that $f(x)$ is continuous on $[a,b]$
            \item $C$ is some number such that $f(a)<G(a)<f(b)$.
            \item There exists some $C$ in $[a,b]$ such that $f(C)=G$.
        \end{enumerate}
    \end{theorem}
    \item The point of the IVT is that continuous functions don't skip over any $y$ values. Note that this is a property that intuitively we want continuous functions to have.
    \begin{example}
        Prove that there is a number $c$ such that $c\cdot c=2$ using IVT rather than CORA directly.
        \vspace{2mm}

        Consider $f(x)=x^2$ on $[1,2]$. It is easy to show that $f(x)$ is continuous on $[1,2]$ per the polynomial continuous theorem. Here $f(1)=1$ and $f(2)=4$. Note that $1<2<4$ so $f(1)<2<f(2)$.
        \vspace{2mm}
        
        The IVT shows that there must be some number $c$ where $1<c<2$ in this interval such that $f(c)=c\cdot c=2$. 
    \end{example}
    Note that if reals consisted of rationals only, then the IVT would not be true!
    \item Logically we could have started with the IVA (Intermediate Value Axiom) and used it to prove CORT (Completeness of Reals Theorem). But this would be messy: we would need to define functions before we had finished defining numbers.
    \begin{example}
        Prove there's at least one root of $x^{17}+1=3x$ in $[0,1]$.
        \vspace{2mm}

        Define $f(x)=x^{17}+1-3x$. $f(x)$ is continuous by polynomial continuity theorem. $f(0)=1$ and $f(1)=-1$, so $f(0)>0>f(1)$. By IVT there is some $c$ in $(0,1)$ such that $f(c)=0$. Therefore $c$ is a root of this equation.
    \end{example}
\end{itemize}