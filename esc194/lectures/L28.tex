\section{General Loagrithmic and Exponential Functions}
\begin{itemize}
    \item The general exponential function can be defined below:
    \begin{definition}
        \begin{equation}
            x^{z} = e^{z\ln x}
            \label{eq:}
        \end{equation}
        for $x>0$
    \end{definition}
    \item As a result, we have the following properties:
    \begin{align}
        x^{r+s} &= x^rx^s \\ 
        x^{r-s} &= \frac{x^r}{x^s} \\ 
        (x^r)^s &= x^{rs} \\ 
        \label{eq:}
    \end{align}
    For $r,s \in \mathbb{R}$ and $x>0$.
    \item We also have:
    \begin{align}
        \frac{d}{dx}x^p &= px^{p-1} \\ 
        \frac{d}{dx}x^r = rx^{r-1}
    \end{align}
    \begin{example}
        Evaluate
        \begin{equation}
            I = \int \frac{x \dd{x}}{(x^2+1)^{\sqrt{2}}}
            \label{eq:}
        \end{equation}
        Set $u=x^2+1$ such that $\dd{u}=2x\dd{x}$ such that:
        \begin{align}
            I &= \frac{1}{2} \int \frac{\dd{u}}{u^{\sqrt{2}}} \\ 
            &= \frac{1}{2}\left(\frac{1}{1-\sqrt{2}}\right)(x^2+1)^{1-\sqrt{2}}
        \end{align}
    \end{example}
    \begin{example}
        If $f(x)=x^x$, evaluate $f'(x)$.
        \vspace{2mm}

        We let:
        \begin{align}
            x^x &= e^{x\ln(x)} \\
            (x^x)' &= e^{x\ln(x)}\left(x\cdot \frac{1}{x}+\ln(x)\right) \\ 
            &= x^x(1-\ln(x)) 
        \end{align}
    \end{example}
    \item The derivative of $f(x)=p^x$ is given as:
    \begin{equation}
        \frac{d}{dx}p^u = p^u \ln(p)\frac{du}{dx}
    \end{equation}
    Notice that the $\ln(p)$ term goes to one as $p\to e$.
    \item We can introduce logarithm functions to a base $p$:
    \begin{equation}
        f(x) = \frac{\ln(x)}{\ln(p)},\, g(x)=p^x
        \label{eq:}
    \end{equation}
    such that:
    \begin{equation}
        f(g(x)) = \frac{\ln(p^x)}{\ln(p)} = x
        \label{eq:}
    \end{equation}
    so they are inverse functions.
    \begin{definition}
        The logarithm is defined as
        \begin{equation}
            \log_{p}(x) = \frac{\ln x}{\ln p}
        \end{equation}
    \end{definition}
    \begin{example}
        \begin{align}
            \frac{d}{dx} \log_{7}(2x^3-2) = \frac{6x^2-1}{(2x^3-x)\ln(7)}
        \end{align}
    \end{example}
    \item It is possible to estimate the value of $e$ by bounding it. We have that:
    \begin{equation}
        \ln x = \int_1^x \frac{\dd{t}}{t}
        \label{eq:}
    \end{equation}
    such that:
    \begin{equation}
        \ln\left(1+\frac{1}{n}\right) = \int_1^{1+1/n} \frac{\dd{t}}{t} < \int_1^{1+1/n} 1 \dd{t}
        \label{eq:}
    \end{equation}
    Since $frac{1}{t} < \frac{1}{1}$ for $t>0$. The upper bound then becomes:
    \begin{equation}
        1+\frac{1}{n}-1 = \frac{1}{n} \implies \ln\left(1+\frac{1}{n}\right) < \frac{1}{n}
        \label{eq:}
    \end{equation}
    We can similarly repeat this process:
    \begin{equation}
        1+\frac{1}{n} < e^{1/n} \implies (1+\frac{1}{n})^n < e
        \label{eq:}
    \end{equation}
    Note that if we take the limit as $n\to\infty$, \textit{intuitively} we would expect the upper bound to become closer and closer to the true value. We shall explore this further, and we can write the lower bound as:
    \begin{equation}
        \ln\left(1+\frac{1}{n}\right) = \int_{1}^{1+1/n} \frac{\dd{t}}{t} > \int_1^{1+1/n} \frac{\dd{t}}{1+1/n}
    \end{equation}
    since $\frac{1}{t}>\frac{1}{1+1/n}$. We can write this in logarithm form to get:
    \begin{equation}
        \ln\left(1+\frac{1}{n}\right)>\left(\frac{1}{1+1/n}\right)\left(1+\frac{1}{n}-1\right) = \frac{1}{n+1} \implies \left(1+\frac{1}{n}\right)^{n+1} > e
        \label{eq:}
    \end{equation}
    Putting it altogether, we have the following statement:
    \begin{idea}
        $e$ can be estimated with its lower and upper bound with the following:
        \begin{equation}
            \left(1+\frac{1}{n}\right)^n < e < \left(1+\frac{1}{n}\right)^{n+1}
        \end{equation}
    \end{idea}
\end{itemize}