\section{Applications of Derivatives}
\begin{itemize}
    \item Applications of Derivatives:
    \begin{definition}
        $f(x)$ has ``an absolute maximum at $c$'' if $f(c) \ge f(x)$ for all $x\in$ domain of $f(x)$. \textbf{Note} that $f(c)$ must exist!
        \vspace{2mm}

        For example, if $f(x) = \frac{\sin x}{x}$, it does not have an absolute maximum at $x=0$!

        \begin{center}
            \begin{tikzpicture}
            \begin{axis}[
            legend pos=outer north east,
            title=Example,
            axis lines = box,
            xlabel = $x$,
            ylabel = $y$,
            variable = t,
            trig format plots = rad,
            ]
            \addplot [
                domain=-15:15,
                samples=70,
                color=blue,
                ]
                {sin(x)/x};
            \addlegendentry{$\sin(x)/x$}
            
            \end{axis}
            \end{tikzpicture}
        \end{center}
    \end{definition}
    \begin{definition}
        $f(x)$ has a ``absolute max on $[a,b]$ etc'' if $f(c) \ge f(x)$ for all $x\in[a,b]$
    \end{definition}
    \begin{definition}
        $f(x)$ has a ``local max at $c$'' if $f(c) \ge f(x)$ for \textit{some} open interval containing $c$.
    \end{definition}
    \begin{theorem}
        The \textbf{extreme value theorem} (EVT) says that given $f(x)$ is continuous on $[a,b]$, then $f(x)$ has an absolute maximum $f(c)$ and an absolute minimum $f(x)$ for some $c,d \in [a,b]$.
        \vspace{2mm}

        However, functions do not need to be continuous to have an absolute max.

        \begin{proof}
            The outline of the proof is as follows:
            \begin{enumerate}
                \item Prove all continuous functions on $[a,b]$ are \textbf{bounded}
                \item Then prove all continuous functions on $[a,b]$ have a max and a min.
            \end{enumerate}
            Note that this is not the same thing! Remember that $\displaystyle f(x)=\frac{\sin x}{x}$ on $[-1,1]$ is bounded, but does not have an absolute maximum! However, this doesn't violate it since it's not continuous.
            \vspace{2mm}

            We will take (1) to be proven and just prove (2): Consider the set $S=\{f(x): a\le x\le b\}$. Since $S$ is a set of f-values from (1), $S$ is bounded above. By CORA, $\text{lub}(S)$ exists as a real number ,call it $M$. Therefore: $f(x) \le M$ for all $x\in [a,b]$
            \vspace{2mm}

            We now need to prove that there is some $c\ni[a,b]$ such that $f(c)=M$, i.e. $f(x)$ takes on the value $M$. We can prove this via contradiction:
            \vspace{2mm}

            Suppose $f(x)$ never equals $M$. We can then define:
            \begin{equation}
                g(x)\equiv \frac{1}{M-f(x)}
                \label{eq:}
            \end{equation}
            Note that $g(x)>0$. (It cannot be negative since $M > f(x)$). Therefore, $g(x)$ is also continuous on $[a,b]$ by A.C.T, Q.C.T, and by the fact that $f(x)\neq M$.
            \vspace{2mm}

            Therefore $g(x)$ is also bounded above by part (1). There exists a number $K$ such that $0<g(x)\le K$ where $K>0$. Taking the inverse, we have:
            \begin{align}
                \frac{1}{K} &\le \frac{1}{g(x)} \\ 
                \frac{1}{K} &\le M-f(x) \\ 
                f(x) &\le M-\frac{1}{K}
                \label{eq:}
            \end{align}
            This makes $M-\frac{1}{K}$ an upper bound of $S$. However, if $M$ is the least upper bound of $S$, this gives a contradiction.
            \vspace{2mm}

            Since there is a contradiction, there exists at least one $c\in [a,b]$ such that $f(c)=M$. Therefore, a maximum exists.
        \end{proof}
    \end{theorem}
    \item Fermat's Theorem:
    \begin{definition}
        $c$ is a ``critical point'' of $f(x)$ if $f'(c)=0$ or $f'(c)$ DNE.
    \end{definition}
    \item We have to be careful however, suppose we look at $f(x)=x^2$ on $x\in [1,2]$. It is continuous by the polynomial C.T., and $f(2)=4$ is an absolute maximum of $f(x)$ in $[1,2]$. However, this doesn't violate Fermat's theorem since it's an absolute max, not a local max!
    \item Another example is $f(x)=x^3$. We have $f'(0)=0$ but $f(0)$ is not a local max or min. We \textbf{cannot} reverse Fermat's theorem!
    \begin{idea}
        The motivation behind Fermat's theorem is as follows:
        \begin{enumerate}
                \item We often need to find local max, min.
                \item But how can we?
                \item It's usually easy to calculate $f'(x)$ and then find out where $f'(c)=0$ or DNE.
                \item While these critical points are not necessarily local max, min, only local max, min points will be in this set.\footnote{Like using a net to catch fish. Not everything you catch will be the fish you want but the fish you want will be among it.}
        \end{enumerate}
    \end{idea}
    \item This leads to a test for the absolute max/min on $[a,b]$. Given that $f(x)$ is continuous on $x\in [a,b]$. By the EVT there is an absolute max,min on $[a,b]$ for sure. Then, we can:
    \begin{enumerate}
        \item Find all $c_\text{crit}$ and $f(c_\text{crit})$.
        \item Find $f(a)$, $f(b)$.
        \item The largest of these number is absolute max, and the smallest is the absolute min.
    \end{enumerate}
    \begin{example}
        Let $f(x)=(9-x^2)^{1/2}$ on $x\in[-1,2]$:
        \begin{center}
            \begin{tikzpicture}
            \begin{axis}[
            legend pos=outer north east,
            title=Example,
            axis lines = box,
            xlabel = $x$,
            ylabel = $y$,
            variable = t,
            trig format plots = rad,
            ]
            \addplot [
                domain=-1:2,
                samples=70,
                color=blue,
                ]
                {(9-x^2)^0.5};
            
            \end{axis}
            \end{tikzpicture}
        \end{center}
        We can find the derivative as:
        \begin{equation}
            f'(x)=\frac{1}{2}(9-x^2)^{-1/2}(-2x)
            \label{eq:}
        \end{equation}
        using the chain rule, polynomial D.T., power D.T. \begin{enumerate}
            \item We now look for when $f'(c)0=0$, which only happens when $f(0)=3$. When is it undefined? One might be tempted to say at $-3$ or $3$ but they aren't in the interval.
            \item $f(-1)=\sqrt{8}$ and $f(2)=\sqrt{5}$
            \item Therefore, the absolute maximum is $f(0)=3$ and the absolute minimum is $f(2)=\sqrt{5}$.
        \end{enumerate}
    \end{example}
\end{itemize}