\section{Feynman's Trick of Differentiation}
\begin{itemize}
    \item If we want to graph a function $\ln(f(x))$, the domain is given by $x$ values such that $f(x)>0$. As a result:
    \begin{equation}
        \int \ln(x) \dd{x} \neq \frac{1}{x}
        \label{eq:}
    \end{equation}
    since $x$ cannot be negative.
    \item Instead, we define:
    \begin{equation}
        \ln|x| = \frac{1}{x}
        \label{eq:}
    \end{equation}
    And similarly the following definite integral is defined as:
    \begin{equation}
        \int \frac{\dd{x}}{x} = \ln|x|+C
        \label{eq:}
    \end{equation}
    \begin{theorem}
        \textbf{Feynman's trick of Differentiation} (otherwise known as logarithmic differention): The following was popularized by Richard Feynman during the first of his Feynman Lectures. If we have a function:
        \begin{equation}
            g(x)=g_1(x)g_2(x)g_3(x)\cdots g_n(x)
            \label{eq:}
        \end{equation}
         Then taking the natural logarithm of both sides, applying the chain rule, and simplifying gives:
         \begin{equation}
             g'(x) = g(x)\left(\frac{g_1'}{g_1}+\frac{g_2'}{g_2}+\cdots+\frac{g_n'}{g_n}\right)
             \label{eq:}
         \end{equation}
    \end{theorem}
    \begin{example}
        Take the derivative of $g(x) = \frac{x^4(x-1)}{(x+2)(x^2+1)}.$ While we can calculate this directly, we can also use the Feynman's trick to calculate the derivative: We get
        \begin{equation}
            g'(x) = \frac{x^4(x-1)}{(x+2)(x^2+1)} \left[\frac{4x^3}{x^4}+\frac{1}{x-1}-\frac{1}{x+2}-\frac{2x}{x^2+1}\right]
        \end{equation}
    \end{example}
\end{itemize}