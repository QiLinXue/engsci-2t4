\section{Volumes by Cylindrical Shells}
\begin{itemize}
    \item Sometimes, the washer method is difficult to apply. Suppose we wish to rotate a curve $f(x)$ from $x=a$ to $x=b$ around the $y$ axis. Then we can look at a small rectangle with width $\Delta x$ such that the volume of the cylindrical shell once rotated is:
    \begin{equation}
        V_i = f(x_i^*)\Delta x_i \cdot 2\pi x_i^*
        \label{eq:}
    \end{equation}
     so the volume is:
     \begin{equation}
         V = \int_a^b 2\pi x f(x) \dd{x}
         \label{eq:}
     \end{equation}
    This is known as the \textbf{shell method about y-axis}
    \item Similarly we can apply this for a curve rotated about the $x$ axis:
    \begin{equation}
        V = \int_a^b 2\pi y f(y) \dd{y}
        \label{eq:}
    \end{equation}
    which is the \textbf{shell method about x-axis}.
    \begin{warning}
        Note that this is the opposite to the washer method. If we rotated across the $x$ axis in this case, we integrate with respect to $y$.
    \end{warning}
    \begin{example}
        Find the volume bounded by $y^2-x^2=1$ and $y=2$ when they are rotated about the $x$ axis.
        \vspace{2mm}

        It is essentially a hyperbola. Solving for $x$ gives:
        \begin{equation}
            x = \pm \sqrt{y^2-1}
            \label{eq:}
        \end{equation}
        If we integrate with respect to $y$, then the bound is between $y=1$ and $y=2$. We can use symmetry and look at only positive values of $x$, then double it to get:
        \begin{equation}
            V = 2\int_1^2 (2\pi y)\sqrt{y^2-1} \dd{y}
            \label{eq:}
        \end{equation}
        Using the u substitution of $u=y^2-1$, we get:
        \begin{equation}
            V = 4\sqrt{3}\pi
            \label{eq:}
        \end{equation}
    \end{example}
    \begin{example}
        Find the volume bounded by $y=x^2$ and $y=\sqrt{x}$ when rotated about the $y$ axis.
        \vspace{2mm}

        We get:
        \begin{equation}
            V = \int_0^1 2\pi x (\sqrt{x}-x^2) \dd{x} = \frac{3\pi}{10}
            \label{eq:}
        \end{equation}
        where I have omitted the intermediate steps, where we can just apply the power rules.
    \end{example}
\end{itemize}