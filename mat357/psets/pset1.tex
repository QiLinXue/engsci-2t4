

\documentclass{article}
\usepackage{qilin}
\tikzstyle{process} = [rectangle, rounded corners, minimum width=1.5cm, minimum height=0.5cm,align=center, draw=black, fill=gray!30, auto]
\title{\vspace{-2cm}MAT357: Real Analysis \\ Problem Set 1}
\author{QiLin Xue}
\date{2022-2023}
\usepackage{mathrsfs}
\usetikzlibrary{arrows}
\usepackage{stmaryrd}
\usepackage{accents}
\newcommand{\ubar}[1]{\underaccent{\bar}{#1}}
\usepackage{pgfplots}
\numberwithin{equation}{section}

\begin{document}

\maketitle
\begin{enumerate}
    \item \begin{enumerate}[label=(\alph*)]
        \item We show the implication both ways. Suppose there is a fixed point. Then there exists an $a \in A$ such that $f(a)=a.$ The graph of $f$ is given by $(x,f(x)),$ which at $x=a$ becomes $(a,a),$ which crosses the diagonal.
        
        Now suppose that the graph given by $(x,f(x))$ crosses the diagonal, i.e. there exists some $a\in A$ such that $(x,f(x))=(a,a).$ Component-wise, this implies that $x=a$ and $f(x)=f(a)=a,$ which is the definition of a fixed point.
        \item Consider the trivial case where $f(0)=0$ or $f(1)=1.$ If either of this is true, then by definition there exists a fixed point. We now assume that $f(0) > 0$ and $f(1) < 1.$ Note that this is the only other case since the range of $f$ is restricted to $[0,1].$
        
        The diagonal of $[0,1]\times [0,1]$ is given by the function $g:[0,1]\to [0,1], g(x)=x.$ We then have $f(0)>g(0)=0$ and $f(1)<g(1)=1.$ By the intermediate value theorem, there must be an intersection, and by part (a), this intersection corresponds to a fixed point.
        \item This is not true. Consider $f:(0,1)\to (0,1), x \mapsto 0.$ This function will never cross the diagonal in the domain $(0,1),$ so there are no fixed points.
        \item This is also not true. Consider $f:(0,1)\to (0,1)$ defined piecewise by 
        \begin{equation}
            f(x) = \begin{cases}
                1  & 0 \le x \le \frac{1}{2} \\ 
                0 & \frac{1}{2} < x \le 1.
            \end{cases}
        \end{equation}
        This clearly doesn't intersect the diagonal since in the region $[0,1/2],$ we have $g(x)=x < 1$ and in the region $(1/2,1],$ we have $g(x)=x > 0.$
    \end{enumerate}
    \newpage
    \item \begin{enumerate}
        \item First, we will show that a uniformly continuous function is continuous. Suppose $f$ is uniformly continuous. We know that for each $\epsilon > 0$ there exists a $\delta > 0$ such that for all $t,$ we have $|x-t|<\delta$ implies that $|f(x)-f(t)|<\epsilon.$
        
        Then to show continuity, we need to show that for each $\epsilon > 0,$ there exists a $\delta'(t) > 0$ such that $|x-t|<\delta'(t)$ implies that $|f(x)-f(t)|<\delta',$ where the only difference is that $\delta'$ is allowed to vary along the domain of the function. If we set $\delta'(t)=\delta$ (i.e. the same $\delta$ as what uniform continuity gives us), then by uniform continuity, 
        \begin{equation}
            |x-t|<\delta'(t)=\delta \implies |f(x)-f(t)| < \epsilon
        \end{equation}
        is satisfied. To prove the converse is not true, we turn to part (c) of this problem, where we show that while $x^2$ is a continuous function, it is not uniformly continuous.
        \item Let $f(x)=2x.$ For each $\epsilon > 0,$ pick $\delta = \frac{\epsilon}{3}.$ Then
        \begin{equation*}
            |x-t| < \delta = \frac{\epsilon}{3}.
        \end{equation*} 
        Using this, we can bound 
        \begin{align*}
            |f(x)-f(t)| &= |2x-2t| \\ 
            &= 2|x-t| \\ 
            &< 2\frac{\epsilon}{3} \\ 
            &< \epsilon.
        \end{align*}
        Therefore, we found a value of $\delta$ such that $|x-t|<\delta \implies |f(x)-f(t)|<\epsilon$ for all $t\in (-\infty,\infty).$
        \item Suppose that $f(x)=x^2$ is uniformly continuous. Then for each $\epsilon>0,$ there exists a $\delta$ such that $|x-t|<\delta \implies |x^2-t^2| < \epsilon$ for all $t\in \mathbb{R}.$ We will show that this leads to a contradiction.
        
        With the given choice of $\delta,\epsilon,$ consider $t = \frac{\epsilon}{\delta}$ and $x = \frac{\epsilon}{\delta} + 0.9\delta.$ Clearly 
        \begin{equation*}
            |x-t| = |0.9\delta| < \delta,
        \end{equation*}
        so this is satisfied. What remains is to verify whether $|x^2-t^2|<\epsilon$ is satisfied or not. We can compute,
        \begin{align}
            |x^2-t^2| &= \left|\left(\frac{\epsilon}{\delta}+0.9\delta\right)^2 - \frac{\epsilon^2}{\delta^2}\right| \\ 
            &= \left|1.8\epsilon + 0.81\delta^2\right| \\ 
            &> |1.8\epsilon + 0| \\ 
            &> \epsilon.
        \end{align}
        But our assumption was that this choice of $\delta,\epsilon$ ensured that $|x^2-t^2|<\epsilon,$ hence a contradiction. Intuitively, this corresponds to the fact that the derivative of $x^2$ is unbounded.
    \end{enumerate}
    \newpage
    \item \begin{enumerate}
        \item We will use two useful facts in this computation. Because $\langle \cdot ,\cdot \rangle$ is bilinear, we can compute
        \begin{align}
            \langle x \pm y, x \pm y\rangle &= \langle x,x\pm y\rangle \pm \langle y, x\pm y\rangle \\ 
            &= \langle x,x\rangle \pm \langle x,y\rangle \pm \langle y,x\rangle \pm \langle y,y\rangle \\ 
            &= \langle x,x\rangle \pm 2\langle x,y\rangle + \langle y,y\rangle.
        \end{align}
        Using this, we can simplify using the fact that the inner product induces a norm,
        \begin{align}
            |x+y|^2+|x-y|^2 &= \langle x+y,x+y\rangle + \langle x-y,x-y\rangle \\ 
            &= \left(\langle x,x\rangle + 2\langle x,y\rangle  + \langle y,y\rangle \right)+ \left(\langle x,x\rangle - 2\langle x,y\rangle + \langle y,y\rangle\right) \\ 
            &= 2\langle x,x\rangle + 2\langle y,y\rangle \\ 
            &= 2|x|^2+2|y|^2.
        \end{align}
        \item We will assume that the map $f:V\times V\to F$ defined by 
        \begin{equation*}
            f(x,y) = \left|\frac{x+y}{2}\right|^2 - \left|\frac{x-y}{2}\right|^2
        \end{equation*}
        is an inner-product (Piazza post says we don't have to show bilinearity). Note the change of notation, to make it clear later on which inner product we are talking about.
        
        Now suppose for the sake of contradiction that the norm $|\cdot |$ arises from two different inner products, $f(\cdot,\cdot)$ and $g(\cdot,\cdot).$ That is, there exists $a^*,b^*\in V$ such that 
        \begin{equation*}
            f(a^*,b^*) \neq g(a^*,b^*).
        \end{equation*}  
        Now consider 
        \begin{align}
            x = \frac{a^*+b^*}{2},\quad\quad\quad\quad y = \frac{a^*-b^*}{2},
        \end{align}
        such that $a^*=x+y$ and $b^*=x-y.$ Then, 
        \begin{align}
            g(a^*,b^*) &= g(x+y,x-y) \\ 
            &= g(x,x-y)+g(y,x-y) \\ 
            &= g(x,x) - g(x,y) + g(y,x) - g(y,y) \\ 
            &= g(x,x) - g(y,y) \\ 
            &= |x|^2 - |y|^2
        \end{align}
        We can also compute,
        \begin{align}
            f(a^*,b^*) &= \left|\frac{a^*+b^*}{2}\right|_f^2- \left|\frac{a^*-b^*}{2}\right|_f^2 \\ 
            &= |x|^2 - |y|^2.
        \end{align}
        % Here, the subscript $f,g$ refer to the norms induced by $f,g$ respectively. 
        Therefore, since the two give the same result, we have $g(a^*,b^*)= f(a^*,b^*),$ which is a contradiction. Therefore, the inner product which defines the norm must be unique.
        % Note that for both computations are reversible.

        % We have 
        % \begin{align*}
        %     |x|^2 - |y|^2 &= |x|^2+ |y|^2 - 2|y|^2 \\ 
        %     &= \frac{1}{2}\left(|x+y|^2+|x-y|^2\right) - 2|y|^2 \\ 
        %     &= |a|^2 -2|y|^2 + |b|^2 \\ 
        %     &= |a|^2 - \frac{1}{2}|a-b|^2 + |b|^2
        % \end{align*}
    \end{enumerate}
    \item \begin{enumerate}[label=(\alph*)]
        \item \textbf{FALSE:} Consider the function $f:[0,1) \cup (1,2] \to [0,3]$ defined by 
        \begin{equation*}
            f(x) = \begin{cases}
                x & 0 \le x < 1 \\ 
                x + 1 & 1 < x \le 2.
            \end{cases}
        \end{equation*}
        Clearly, this is not continuous at $x=1,$ so it suffices to show that this is open. Consider an open set $(a,b)$ where $0 \le a < b< 2.$ We have two cases:
        \begin{itemize}
            \item Case 1: $a,b \le 1:$ When restricted to this set, $f((a,b))=(a,b),$ so it is open.
            \item Case 2: $a,b > 1:$ When restricted to this set, $f((a,b))=(a+1,b+1),$ so it is open.
        \end{itemize}
        Note that any open set can be constructed by the union of open sets of the above types. If $A,B$ are disjoint open sets in $[0,1)\cup (1,2]$ then $f(A),f(B)$ are also disjoint, since $f(x)$ is injective. Any open set $U$ in $[0,1)\cup (1,2]$ can be written as $U = \cup V,$ where $V$ are disjoint open sets that satisfy either of the two cases above. Then:
        $f(U)=\cup f(V).$ Since $f(V)$ is open, a collection of open sets is open, and therefore $f$ is an open function.
        \item \textbf{TRUE:} If $f$ is a homeomorphism, then $f$ is bijective and $f^{-1}:N\to M$ is continuous. For any open set $U\subset M,$ this continuity condition tells us that 
        \begin{equation*}
            \left(f^{-1}\right)^{-1}(U) 
        \end{equation*}
        is open. But since $f$ is bijective, the inverse of $f^{-1}$ is just $f.$ The above statement is therefore equivalent to saying that for any open set $U\subset M,$ we get that $f(U)$ is open, which is the definition for open function.
        \item \textbf{TRUE:} The function $f$ is a homeomorphism if and only if $f$ is bijective and continuous, and $f^{-1}$ is continuous. We already satisfied the bijective and continuous condition.
        
        To show that $f^{-1}$ is continuous, we use the fact that $f$ is open to show that $f(U)$ is open for any open set $U\subset M.$ But since $f$ is bijective, we can rewrite $f = (f^{-1})^{-1},$ so $\left(f^{-1}\right)^{-1}(U)$ is open, which is what it means for $f^{-1}:N\to M$ to be continuous.
        \item \textbf{FALSE:} Consider the function $f:\mathbb{R}\to\mathbb{R}$ defined piecewise to be 
        \begin{equation*}
         f(x) = \begin{cases}
            -x-2\pi & x \le -2\pi \\ 
            \sin(x) & -2\pi < x < 2\pi \\ 
            -x+2\pi & x \ge 2\pi,
         \end{cases}   
        \end{equation*} 

         which when plotted out, looks like the following (visualization purposes only):
         \begin{center}
            \begin{tikzpicture}
            \begin{axis}[
            legend pos=outer north east,
            title=Counter Xxample,
            axis lines =middle,
            xlabel = $x$,
            ylabel = $y$,
            variable = t,
            trig format plots = rad,
            ]
            \addplot [
                domain=-10:-2*pi,
                samples=70,
                color=blue,
                ]
                {-x-2*pi};
                \addplot [
                    domain=2*pi:10,
                    samples=70,
                    color=blue,
                    ]
                    {-x+2*pi};
                    \addplot [
                        domain=-2*pi:2*pi,
                        samples=70,
                        color=blue,
                        ]
                        {sin(x)};
            \end{axis}
            \end{tikzpicture}
         \end{center}
         It is easy to verify that $f$ is continuous. We will show that it is not open. Consider the set $U=(-2\pi,2\pi) \subset \mathbb{R}.$ Then,
         \begin{align*}
            f(U) &= \left\{\sin(x): -2\pi < x < 2\pi \right\} \\ 
            &= [-1,1].
         \end{align*} 
         To see why this is the case, for any $y\in [-1,1],$ we can compute $\arcsin(y),$ which is bounded between $-\pi/2$ and $+\pi/2,$ within our domain. Since $f(U)$ is closed, this is false.
         \item \textbf{TRUE:} First recognize that if $f$ is also an injection, then $f$ must be a homeomorphism, since the bijection and continuous condition is satisfied, and the continuity of $f^{-1}$ is satisfied as $f$ is open (which we have proved). Therefore, we just need to show that $f$ is injective.
         
         Let $x,y\in \mathbb{R}$ such that $f(x)=f(y).$ We want to show that $x=y.$ Suppose for the sake of contradiction, this was not true, and that $x\neq y.$ WLOG, let $x<y.$ By the extreme value theorem (which is valid since $f$ is continuous), there exists a $z\in \mathbb{R}$ such that $x<z<y$ and that $f(z)$ is the extremum point (either maximum or minimum). Again, without loss of generality, assume that $f(z)$ is the maximum, i.e. 
         \begin{equation*}
            f(z) \ge f(x) \forall x \in (a,b).
         \end{equation*}
         Then $f((x,y))$ is NOT open. To see why, consider $f(z) \in f((x,y)),$ and consider an open neighborhood $(f(z)-\delta,f(z)+\delta)$ around $f(z)$ with $\delta > 0.$ For any choice of $\delta > 0,$ this neighborhood cannot be continued in $f((a,b))$ since $f(z) + \frac{\delta}{2} > f(z).$ But since $f(z)$ is the maximum of $f((a,b))$ there cannot be any values in $f((a,b))$ that are greater than $f(z).$
         \item \textbf{FALSE:} Consider the function $f:S^1 \to S^1$ defined by 
         \begin{equation*}
            (\cos(\theta),\sin(\theta)) \mapsto (\cos(2\theta),\sin(2\theta)),
         \end{equation*}
         where $\theta \in \mathbb{R}.$ Clearly, this is a surjection since $(\cos \phi,\sin\phi)$ is mapped to by $(\cos \phi/2,\sin \phi/2).$ This map is open because consider the open set 
         \begin{equation*}
            U = \{(\cos\theta,\sin\theta):a < \theta < b\},
         \end{equation*}
         where $0 \le a,b < 2\pi.$ Then:
         \begin{align*}
            f(U) &= \{(\cos 2\theta,\sin 2\theta):a < \theta < b\} \\ 
            &= \{(\cos \theta,\sin\theta): 2a < \theta < 2b\}
         \end{align*}
         is open. We can also show that it is continuous. Consider the same $U$ as before. We have,
         \begin{align*}
            f^{-1}(U) &= f^{-1}\{(\cos \theta,\sin\theta): a < \theta < b\} \\ 
            &= \{(\cos \theta/2,\sin\theta/2):a < \theta < b\}  \cup \{(\cos(\theta/2 + \pi), \sin(\theta/2+\pi):a < \theta < b)\} \\ 
            &= \{(\cos\theta,\sin\theta):a/2<\theta<b/2\} \cup \{(\cos(\theta+\pi),\sin(\theta+\pi)):a/2<\theta<b/2\},
         \end{align*}
         which is open as it is the union of two open sets. Note that we have two sets in the first place because for each point $(\cos\theta,\sin\theta) \in S^1$ other than $(1,0),$ there are two points on the circle that gets mapped to it by $f.$ However, it is not homeomorphic because it is not injective. We have 
         \begin{equation*}
            f(1,0) = (1,0),\quad\quad\quad f(-1,0) = (1,0).
         \end{equation*} 
         Because $f$ is not injective, it cannot be bijective.
    \end{enumerate}
    \newpage
    \item Note that $\text{id}:f\mapsto f$ is a linear bijection by definition, as it is the identity map and the domain and codomain contain the same functions (but with a different metric).
    
    We will first show that $id$ is continuous.


    Consider a sequence $(f_n)$ that converges to the limit $g$ in $C_{max}.$ This means that for each $\epsilon > 0,$ there exists $N\in \mathbb{N}$ such that 
    \begin{equation*}
        d_{max}(f_n,g)<\epsilon
    \end{equation*}  
    for all $n \ge N.$ Rewriting this with the metric, we have 
    \begin{align*}
        \text{max}|f(x)-g(x)| < \epsilon \implies |f(x)-g(x)|<\epsilon.
    \end{align*}
    Then the map $id$ is continuous if the sequence $(\text{id}(f_n))=(f_n)$ converges to $(\text{id}(g))=(g)$ in $C_{int}.$ This converges if for every $\epsilon' > 0,$ there exists an $N' \in \mathbb{N}$ such that for all $n\ge N'$ we have 
    \begin{equation*}
        d_{int}(f_n,g) = \int_0^1 |f_n-g| \dd{x}.
    \end{equation*}
    We can choose $N'$ the same way we chose $N$ for the converging sequence in $C_{max},$ i.e. such that for all $n\ge N'$ we have $|f(x)-g(x)|<\epsilon'.$ We can compute,
    \begin{align*}
        \int_0^1 |f(x)-g(x)| \dd{x} < \int_0^1 \epsilon' \dd{x} = \epsilon,.
    \end{align*}
    which is equivalent to showing that $d_{int}(f_n,g)<\epsilon'.$ Therefore, $\text{id}$ is continuous.

    To show that $\text{id}^{-1}$ is not continuous, we will find a counterexample. Consider the sequence $(f_n)$ in $C_{int}$ that is defined by 
    \begin{equation*}
        f(x) = \begin{cases}
            nx & 0 \le x \le \frac{1}{n} \\ 
            -nx + 2 & \frac{1}{n} < x \le \frac{2}{n} \\ 
            0 & \text{else},
        \end{cases}
    \end{equation*}
    where $n>1.$ We claim that this converges to $g$ where $g(x)=0.$ We have,
    \begin{align*}
        d_{int}(f_n,g) &= \int_0^1 |f_n(x)| \dd{x} \\ 
        &= \int_0^{1/n} nx \dd{x} + \int_{1/n}^{2/n} (-nx + 2)\dd{x} \\ 
        &= \frac{1}{n}.
    \end{align*}
It's clear that this converges. For any $\epsilon > 0,$ we can choose $N=\text{max}\{2,\lceil 1/ \epsilon\rceil\}$ such that for any $n > N,$ we have 
    \begin{equation*}
        d_{int}(f_n,g) = \frac{1}{n} < \frac{1}{N}.
    \end{equation*}
    If $\epsilon > 1,$ then pick $N=2$ such that $d_{int}(f_n,g_n) < 1/2 < 1.$ If $\epsilon \le 1,$ then
    \begin{equation*}
        d_{int}(f_n,g_n) < \frac{1}{N} < \frac{1}{(\epsilon^{-1})}  = \epsilon,
    \end{equation*}  
    so it does converge. However, $\text{id}^{-1}$ is not continuous because $(f_n)$ does not converge to $0$ in $C_{max}.$ This is because 
    \begin{equation*}
        d_{max}(f_n,g) = \text{max}|f_n(x)-0| = \text{max}(f_n(x)) = 1. 
    \end{equation*}
    Therefore, for $\epsilon = \frac{1}{2},$ there does not exist any $N \in \mathbb{N}$ such that for $n \ge N$ implies that $d_{max}(f_n,g) < \epsilon = \frac{1}{2},$ since $d_{max}(f_n,g)=1$ always.

    Intuitively, the trick for this problem is to realize you can construct a skinny function such that the area disappears, but the height remains the same (sort of like the opposite of a $\delta$ function!).
\end{enumerate}

\end{document}