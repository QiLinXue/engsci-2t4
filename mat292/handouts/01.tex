\documentclass{article}
\usepackage{qilin}
\tikzstyle{process} = [rectangle, rounded corners, minimum width=1.5cm, minimum height=0.5cm,align=center, draw=black, fill=gray!30, auto]
\title{MAT292 \\ Lecture 1 Exercise}
\author{QiLin Xue}
\date{Fall 2021}
\usepackage{mathrsfs}
\usetikzlibrary{arrows}
\begin{document}
% \setlength\extrarowheight{20pt}
\newcommand{\tabitem}{~~\llap{\textbullet}~~}

\maketitle
\textbf{Exercise:} Let's consider some suggestions for an ODE describing the temperature of a coffee cup in a room. Each of the following suggested ODEs contradicts our intuition in some way. How?
\vspace{2mm}

% \begin{tabularx}{\textwidth}{cc}
%     \textbf{Equation} & \textbf{Which ``law(s)'' does it contradict?} \\ \bottomrule
%     $y'=y^2$ &  

%         \tabitem The temperature can never decrease.
%         \tabitem The temperature doesn't depend on $T$ 
%         \tabitem The rate of increase of temperature will always increase.
    
%     \\ \bottomrule
%     $y'=y''+2y$ &  \\ \bottomrule
%     $y'=\frac{T}{y}$ &  \\ \bottomrule 
%     $y'=y[e^{y-T}+y^3]$ &  \\ \bottomrule 
%     $y'=y-T$ &  \\ \bottomrule 
%     $y'=T-y$ &  \\ \bottomrule
% \end{tabularx}
\end{document}