\documentclass{article}
\usepackage{qilin}
\tikzstyle{process} = [rectangle, rounded corners, minimum width=1.5cm, minimum height=0.5cm,align=center, draw=black, fill=gray!30, auto]
\title{MAT292 \\ Tutorial 7 Solution}
\author{QiLin Xue}
\date{Fall 2021}
\usepackage{mathrsfs}
\usetikzlibrary{arrows}
\begin{document}
% \setlength\extrarowheight{20pt}
\newcommand{\tabitem}{~~\llap{\textbullet}~~}

\maketitle
\begin{enumerate}
    \item \begin{enumerate}[label=(\alph*)]
        \item The net force is
        \begin{equation}
            F = mg - \frac{c_d}{2}\rho Av^2 - \rho Vg
        \end{equation}
        \item 
        \begin{equation}
            \ddot{y} = \left(1-\frac{\rho_\text{fluid}}{\rho_\text{mass}}\right)g - \frac{c_d\rho A}{2m}\dot{y}^2 = \alpha - \beta 
        \end{equation}
        \item Second order nonlinear.
        \item We have 
        \begin{equation}
            \dot{v} = \alpha - \beta v^2
        \end{equation}
    \end{enumerate}
    \item \begin{enumerate}[label=(\alph*)]
        \item $v$ should reach a constant when $v^2 = \frac{\alpha}{\beta}.$
        \item Differentiating, we get
        \begin{equation}
            \frac{dv}{dt} = \frac{df}{dv}v
        \end{equation}
        \item Setting $\dot{v}$ equal, we get
        \begin{equation}
            \frac{df}{dy}v = \alpha - \beta v^2 \implies \frac{df}{dy} = \frac{\alpha}{v} - \beta v 
        \end{equation}
        \item Equilibrium occurs when 
        \begin{equation}
            v^2 = \frac{\alpha}{\beta},
        \end{equation}
        which is the same as earlier.
        \item If we plug in $f(0)=0$, we get $\alpha=0$ so $W=B.$ However physically this doesn't make any sense. 
        \item If $W=B$, the solution is just $y=0.$ If $W>B$, then we have
        \begin{equation}
            \frac{dv}{dy} = \frac{\alpha}{v} - \beta v
        \end{equation}
        so 
        \begin{equation}
            2\beta y + C = -\ln(\beta v^2 - \alpha)
        \end{equation}
        This show why plugging in $f(0)=0$ gives us such a strange relationship. Therefore, 
        \begin{equation}
            v = C\exp(-2\beta y) + \frac{\alpha}{\beta}
        \end{equation}
        \item and we see for a third time that as $y\to\infty$, we get $v=\frac{\alpha}{\beta}$. Note that we are ignoring certain cases, like what happens if $W=B$ or if $W<B$. IF $W<B$, then the argument using the $\ln$ would be negative, but we can take care of it by taking the absolute value.
    \end{enumerate}
\end{enumerate}

\end{document}