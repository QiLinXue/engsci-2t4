\documentclass{article}
\usepackage{qilin}
\tikzstyle{process} = [rectangle, rounded corners, minimum width=1.5cm, minimum height=0.5cm,align=center, draw=black, fill=gray!30, auto]
\title{MAT292 \\ Tutorial 4 Solution}
\author{QiLin Xue}
\date{Fall 2021}
\usepackage{mathrsfs}
\usetikzlibrary{arrows}
\begin{document}
% \setlength\extrarowheight{20pt}
\newcommand{\tabitem}{~~\llap{\textbullet}~~}

\maketitle
\begin{enumerate}
    \item
          \begin{enumerate}
              \item
                    $b,c$ should have units of $\frac{1}{\text{year}}$
              \item I expect it will diverge, unless both budgets are initially zero.
              \item We have
                    \begin{equation}
                        \begin{bmatrix}
                            x' \\ y'
                        \end{bmatrix} = \begin{bmatrix}
                            0 & b \\
                            c & 0
                        \end{bmatrix}\begin{bmatrix}
                            x \\ y
                        \end{bmatrix}
                    \end{equation}
                    We verify that the RHS has units of $\frac{\text{dollars}}{\text{year}}$.
              \item The equilibrium is $(0,0)$ and is unstable.
              \item We can compute the eigenvalues
                    \begin{align}
                        \lambda^2 = bc
                    \end{align}
                    so we have $\lambda=\pm \sqrt{bc}$.
              \item We substitute this in to get
                    \begin{equation}
                        \begin{bmatrix}
                            by \\ cx
                        \end{bmatrix} = \begin{bmatrix}
                            \sqrt{bc}x \\ \sqrt{bc}y
                        \end{bmatrix}
                    \end{equation}
                    plugging in $x=1$ gives the eigenvector
                    \begin{equation}
                        \begin{bmatrix}
                            1 \\ \sqrt{c/b}
                        \end{bmatrix}
                    \end{equation}
                    and the second eigenvector is $\begin{bmatrix}
                            1 \\ -\sqrt{c/b}
                        \end{bmatrix}.$
              \item We have
                    \begin{equation}
                        \begin{bmatrix}
                            x \\ y
                        \end{bmatrix} = Ae^{\sqrt{bc}t}\begin{bmatrix}
                            1 \\ \sqrt{c/b}
                        \end{bmatrix} + Be^{-\sqrt{bc}t}\begin{bmatrix}
                            1 \\ -\sqrt{c/b}
                        \end{bmatrix}
                    \end{equation}
          \end{enumerate}
          \newpage
    \item \begin{enumerate}
              \item We have $\sqrt{c/b}=3$. The lines are
                    \begin{center}
                        \begin{tikzpicture}
                            \begin{axis}[
                                    legend pos=outer north east,
                                    title=Eigenvectors,
                                    axis lines = middle,
                                    xlabel = $x$,
                                    ylabel = $y$,
                                    variable = t,
                                    trig format plots = rad,
                                ]
                                \addplot [
                                    domain=-3:3,
                                    samples=70,
                                    color=blue,
                                ]
                                {3*x};
                                \addplot [
                                    domain=-3:3,
                                    samples=70,
                                    color=blue,
                                ]
                                {-3*x};
                            \end{axis}
                        \end{tikzpicture}
                    \end{center}
              \item The phase portrait looks like
                    \begin{center}
                        \def\length{sqrt(y^2+(9*x)^2)}

                        \begin{tikzpicture}

                            \begin{axis}[
                                    grid=both,
                                    grid style={dashed,red!20},
                                    xmin = 0, xmax = 20,
                                    ymin = 0, ymax = 50,
                                    width = 0.8\textwidth,
                                    height = 0.5\textwidth,
                                    xlabel = {$x$},
                                    ylabel = {$y$},
                                    title={Phase Portrait},
                                    view = {0}{90},
                                ]

                                % Vector Field
                                \addplot3[
                                    quiver = {
                                            u = {(y)/\length},
                                            v = {(9*x)/\length},
                                            scale arrows = 1,
                                        },
                                    -stealth,
                                    domain = 0:20,
                                    domain y = 0:50,
                                    red
                                ]
                                {0};
                            \end{axis}
                        \end{tikzpicture}
                    \end{center}
              \item No.
              \item They will diverge.
          \end{enumerate}
    \item \begin{enumerate}
              \item I understand.
              \item We have
                    \begin{equation}
                        \begin{bmatrix}
                            x' \\ y'
                        \end{bmatrix} = \begin{bmatrix}
                            -2 & 1  \\
                            1  & -2
                        \end{bmatrix}\begin{bmatrix}
                            x \\ y
                        \end{bmatrix} + \begin{bmatrix}
                            C \\ C
                        \end{bmatrix}
                    \end{equation}
                    \item Did via Wolfram Alpha. Equilibrium occurs at $\begin{bmatrix}
                            x \\ y
                        \end{bmatrix} = \begin{bmatrix}
                            C \\ C
                        \end{bmatrix}.$ We then have
                    \begin{equation}
                        \begin{bmatrix}
                            u' \\ v'
                        \end{bmatrix} = \begin{bmatrix}
                            -2 & 1  \\
                            1  & -2
                        \end{bmatrix}\begin{bmatrix}
                            u \\ v
                        \end{bmatrix}
                    \end{equation}
                    and get
                    \begin{equation}
                        \begin{bmatrix}
                            x \\ y
                        \end{bmatrix} = Ae^{-3t}\begin{bmatrix}
                            -1 \\ 1
                        \end{bmatrix} + Be^{-t}\begin{bmatrix}
                            1 \\ 1
                        \end{bmatrix}
                    \end{equation}
              \item The solution to the nonhomogenous equation is
                    \begin{equation}
                        \begin{bmatrix}
                            x \\ t
                        \end{bmatrix} = Ae^{-3t}\begin{bmatrix}
                            -1 \\ 1
                        \end{bmatrix} + Be^{-t}\begin{bmatrix}
                            1 \\ 1
                        \end{bmatrix} + \begin{bmatrix}
                            C \\ C
                        \end{bmatrix}
                    \end{equation}
                    \item False according to Parveer.
          \end{enumerate}
\end{enumerate}

\end{document}