\documentclass{article}
\usepackage{qilin}
\tikzstyle{process} = [rectangle, rounded corners, minimum width=1.5cm, minimum height=0.5cm,align=center, draw=black, fill=gray!30, auto]
\title{MAT292 \\ Tutorial 1 Solution}
\author{QiLin Xue}
\date{Fall 2021}
\usepackage{mathrsfs}
\usetikzlibrary{arrows}
\begin{document}
% \setlength\extrarowheight{20pt}
\newcommand{\tabitem}{~~\llap{\textbullet}~~}

\maketitle
\begin{enumerate}
    \item \begin{enumerate}
        \item $\frac{dI}{dx} = -AI$ and $I(0)=I_0.$
        \item Linear, Separable, Autonomous
        \item $I(x)=I_0e^{-Ax}$
        \item Either 1/cm or 1/m. Looking at dimensions, note that $\frac{dI}{I}=-A\dd{x}$. Since the left hand side is dimensionless, the right hand side is dimensionless, so $A$ has the inverse units of $x$.
    \end{enumerate}
    \item \begin{enumerate}
        \item At $x=1/A$, we have $I(x) = I_0/e.$
        \item We have
        \begin{align}
            I_1 &= I_0e^{-Ax_1} \\ 
            I_2 &= I_0e^{-Ax_2}
        \end{align}
        Taking the log, we have 
        \begin{align}
            \ln I_1 &= \ln I_0 - Ax_1 \\ 
            \ln I_2 &= \ln I_0 - Ax_2.
        \end{align}
        Their difference gives
        \begin{equation*}
            \ln I_1 - \ln I_2 = Ax_2 - Ax_1
        \end{equation*}
        so 
        \begin{equation}
            A = \frac{1}{x_2-x_1}\ln\left(\frac{I_1}{I_2}\right)
        \end{equation}
        \item We have 
        \begin{equation}
            I = I_0e^{-A_1x_1-A_2x_2}
        \end{equation}
    \end{enumerate}
    \item \begin{enumerate}
        \item We have
        \begin{equation}
            \ln(I/I_0)=-A_1x_1-A_2x_2.
        \end{equation}
        Solving for $x_1$ gives 
        \begin{equation}
            A_1x_1 = \ln(I_0/I)-A_2x_2
        \end{equation}
        and similarly:
        \begin{equation}
            A_2x_2 = \ln(I_0/I)-A_1x_1
        \end{equation}
        \item We measure the intensity at various points. The peak (or dip) will be where the center is, and where it levels off is where the sphere ends. The difference between these two points is the radius.
        \item Looking at the radiation intensity through the healthy cells allows us to measure $A_\text{healthy}.$
        
        Then shining the radiation through the center of the sphere, we can use part (a) to find $A_\text{cancer},$ since we know $x_\text{cancer}=2r_\text{cancer}$ from part (b).
    \end{enumerate}
    
\end{enumerate}
\end{document}