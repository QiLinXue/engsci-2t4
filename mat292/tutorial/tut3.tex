\documentclass{article}
\usepackage{qilin}
\tikzstyle{process} = [rectangle, rounded corners, minimum width=1.5cm, minimum height=0.5cm,align=center, draw=black, fill=gray!30, auto]
\title{MAT292 \\ Tutorial 3 Solution}
\author{QiLin Xue}
\date{Fall 2021}
\usepackage{mathrsfs}
\usetikzlibrary{arrows}
\begin{document}
% \setlength\extrarowheight{20pt}
\newcommand{\tabitem}{~~\llap{\textbullet}~~}

\maketitle
\begin{enumerate}
    \item \begin{enumerate}
              \item We have
                    \begin{equation}
                        \frac{dy}{dt}=r\left(1-\frac{y}{K}\right)y
                    \end{equation}
              \item Equilibrium occurs at $y=K.$
              \item We have
                    \begin{equation}
                        \frac{dy}{dt} = r\left(1-\frac{y}{K}\right)y - H(y,t)
                    \end{equation}
          \end{enumerate}
    \item \begin{enumerate}
              \item The new ODE is
                    \begin{equation}
                        \frac{dy}{dt} = r\left(1-\frac{y}{K}\right)y - Ey
                    \end{equation}
              \item The dimensions of $H$ is fish/time and the units of $E$ is $[T]^{-1}.$
              \item Equilibrium occurs when $\frac{dy}{dt}=0$, which occurs at
                    \begin{equation}
                        r - \frac{ry}{K} -E = 0 \implies y = \frac{r-E}{r/K}
                    \end{equation}
                    and $y=0.$
                \item We can rewrite the derivative as $\frac{dy}{dt} = r-(r/K+E)y,$ such that the phase plot is a line. If $y$ decreases, then $dy/dt$ is positive which causes $y$ to increase. If $y$ increases, then $dy/dt$ is negative which causes $y$ to decrease. Therefore, it is stable.
                
                On the other hand, $y=0$ is unstable.
                \item When the fish population stays constant, we want $y'=0$, so rearranging, we have 
                \begin{equation}
                    E = r-\frac{ry}{K}
                \end{equation}
                
                % and $y''=0,$ so 
                % \begin{equation}
                %     \frac{d^2y}{dt^2} = r - \frac{2y}{K} - E = 0 
                % \end{equation}
                % so 
                % \begin{equation}
                %     E = r - \frac{2y}{K}
                % \end{equation}
                % \item Substituting in $y=\frac{r-E}{r/K}$, we get 
                % \begin{equation}
                %     E = r - 2\left(r-E\right) \implies E = r-2r+2E
                % \end{equation}
                % so 
                % \begin{equation}
                %     E = r.
                % \end{equation}
                \item We have $H(y,t)=Ey = E\left(\frac{r-E}{r/K}\right)$
                \item We have $E=\frac{r}{2}$. This is a quadratic. 
            \end{enumerate}
            \item \begin{enumerate}
                \item We have 
                \begin{equation}
                    \frac{dy}{dt} = r\left(1-\frac{y}{K}\right)y - h
                \end{equation}
                \item We use the quadratic equation 
                \begin{equation}
                    y = \frac{-r \pm \sqrt{r^2-4\left(-\frac{r}{K}\right)(-h)}}{-2r/K} = K\cdot\frac{r \pm \sqrt{r^2-4rh/K}}{2r}
                \end{equation}
                since $h<rK/4$, real solutions exist.
                \item The quadratic is concave down so the first equilibrium is unstable and the second equilibrium is stable.
                \item Same thing as above.
            \end{enumerate}
            \item \begin{enumerate}
                \item If $h>rK/4$, then there are no equilibrium. The fish population will monotonically decrease.
                \item At $h=rK/4$, there is only one rate of fishing such that the fish population will remain constant. There is no way for it to increase.
                \item If we let $H(E) = \frac{rK}{4}$, then 
                \begin{equation}
                    E\left(\frac{r-E}{r/K}\right) = \frac{4K}{4}.
                \end{equation}
                We can solve for $E$ in terms of $r$ and $K$ to figure out the effort that represents the transition from sustainable to unsustainable fishing.
                \item Trivially obvious.
            \end{enumerate}
\end{enumerate}

\end{document}