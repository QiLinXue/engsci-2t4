\documentclass{article}
\usepackage{qilin}
\tikzstyle{process} = [rectangle, rounded corners, minimum width=1.5cm, minimum height=0.5cm,align=center, draw=black, fill=gray!30, auto]
\title{MAT292 \\ Tutorial 8 Solution}
\author{QiLin Xue}
\date{Fall 2021}
\usepackage{mathrsfs}
\usetikzlibrary{arrows}
\begin{document}
% \setlength\extrarowheight{20pt}
\newcommand{\tabitem}{~~\llap{\textbullet}~~}

\maketitle
\begin{enumerate}
    \item \begin{enumerate}
        \item Note that
        \begin{equation}
            \mathcal{L}\{f\}(s) = 3 \cdot \frac{1}{s^2+1} 
        \end{equation}
        so 
        \begin{equation}
            f(t) = 3 \sin(t)
        \end{equation}
        We can check this works via the following:
        \item We have 
        \begin{align}
            \mathcal{L}\{af(t)+bg(t)+ch(t)\}(s) &= \int_0^\infty (af(t)+bg(t)+ch(t))e^{-st} \dd{t} \\ 
            &= \int_0^\infty af(t)e^{-st} \dd{t} + \int_0^\infty bg(t)e^{-st} \dd{t} + \int_0^\infty ch(t)e^{-st} \dd{t} \\
            &= a\int_0^\infty f(t)e^{-st}\dd{t} + b\int_0^\infty g(t)e^{-st}\dd{t} + c\int_0^\infty h(t)e^{-st}\dd{t} \\ 
            &= aF(s) + bG(s) + cH(s)
        \end{align}
        Note that we can break up the integrals since the individual integrals converge.
        \item We have 
        \begin{enumerate}
            \item We have
            \begin{equation}
                \mathcal{L}\{h\}(s) = \frac{1}{s-2}+\frac{2}{s}+\frac{2}{s+2}
            \end{equation}
            where $s \in (-2,2) \setminus 0$.
            \item We have 
            \begin{equation}
                \mathcal{L}\{f\}(s) = \frac{s+a^2}{s^2+a^2}
            \end{equation} 
            where $s\in \mathbb{R}$.
            \item Applying linearity, we have
            \begin{align}
                \mathcal{L}\{f\}(s) &= \sum_{k=0}^n \mathcal{L}\left\{\frac{t^k}
                {k!}\right\}(s) \\ 
                &= \sum_{k=0}^n \frac{1}{k!} \cdot \frac{k!}{s^{k+1}} \\ 
                &= \sum_{k=0}^n \frac{1}{s^{k+1}} \\ 
                &= \frac{1}{s}\left(\frac{1-s^{-{n+1}}}{1-s^{-1}}\right)
            \end{align}
            where $s>0$.

            As we take the limit $n\to \infty$, we need $|1/s|<1$ and $s>0$ so we need $s>1.$ Linearity still holds because it is countable.
        \end{enumerate}
        \item easy
    \end{enumerate}
    \item \begin{enumerate}
        \item Using my integration talents (i.e. deifnitely not integral calculator), I manually computed  $\frac{s^2+2}{s^3+4s}$
        \item $\frac{2}{s^3+4s}$
        \item $\sqrt{\frac{\pi}{s}}$
        \item $e^x$ and $e^x$ defined piecewise such that it's equal to $0$ if and only if $x=1$.
    \end{enumerate}
    \item \begin{enumerate}
        \item $f(x)=\sin(x^{x^{x^{x^{x^x}}}})$
        \item $e^{-x^2}$ and $e^{x^2}.$
    \end{enumerate}
    
\end{enumerate}

\end{document}