\section{The Initial Value Problem (IVP)}
\begin{itemize}
    \item How many initial conditions do we need, such that we only have one solution?
    \begin{theorem}
        Consider the IVP for the most general ODE $y'+p(t)y=g(t)$ with initial value $y(t_0)=y_0$ and an interval $I=(\alpha,\beta)$.
        \vspace{2mm}

        If: 
        \begin{itemize}
            \item $t_0 \in I$
            \item $p(t)$ continuous on $I$
            \item $g(t)$ continuous on $I$,
        \end{itemize}
        then this IVP has a solution and this solution is unique, and this solution exists for all $t\in I.$
        \vspace{2mm}

        Also, the ODE has a general solution that depends on only one constant $C$.
    \end{theorem}
    \begin{example}
        Suppose we have the IVP 
        \begin{equation}
            ty'+2y'=4t^2 \iff y'+2\frac{y}{t}+4t
        \end{equation}
        with $y(1)=2$ and let $t\neq 0.$ By the above theorem, this has the \textbf{unique} solution 
        \begin{equation}
            y = t^2 + \frac{1}{t^2}
        \end{equation}
    \end{example}
    \begin{theorem}
        Consider the IVP $y'=f(t,y)$ and $y(t_0)=y_0.$ Consider a rectangle $\alpha < t <\beta$, $\gamma < y < \delta$. If: 
        \begin{itemize}
            \item the point $(t_0,y_0)$ is in the rectangle: 
            \begin{center}
                \begin{tikzpicture}
                    \draw[->] (0,0) -- (6,0) node[right] {$t$};
                    \draw[->] (0,0) -- (0,5) node[right] {$y$};
  
                    \draw[draw=black] (2,2) rectangle ++(3,2);
                    \draw[dotted] (2,2) -- (2,0) node[below] {$\alpha$};
                    \draw[dotted] (5,2) -- (5,0) node[below] {$\beta$};
                    \draw[dotted] (2,2) -- (0,2) node[left] {$\gamma$};
                    \draw[dotted] (2,4) -- (0,4) node[left] {$\delta$};
  
                \end{tikzpicture}
            \end{center}
            \item $f$ is continuous in the rectangle
            \item $f_y$ is continuous in the rectangle 
        \end{itemize}
        Then the IVP has a unique solution. The solution exists for $\alpha < t<\beta$ for some interval $t_0 - h < t < t_0 + h$ where $h\neq 0$.
    \end{theorem}
    \item Remarks: 
    \begin{enumerate}
        \item Non-linear ODEs don't necessarily have a general solution that depend on a single constant.
        \item The solution we get might be implicit, i.e. $\sqrt{y^2+\ln(y)}=5t.$
    \end{enumerate}
    \begin{example}
        Consider the ODE
        \begin{equation}
            (y+t^2y)y' = 2t.
        \end{equation}
        We can write 
        \begin{equation}
            y'=f(t,y) = \frac{2t}{y+t^2y}
        \end{equation}
        and
        \begin{equation*}
            f_y(t,y) = - \frac{2t}{y^2+y^2t^2}.
        \end{equation*}
        The IVP is given by $f(0)=1$. The rectangle for which $y'$ and $f_y$ is continuous is
        \begin{equation}
            R = (-\infty, \infty) \times (0,\infty).
        \end{equation}
        We can solve this by separation of variables and get the curve $y(t)=\sqrt{2\ln(t^2+1)+1}$. We get
        \begin{center}
            \begin{tikzpicture}
            \begin{axis}[
            legend pos=outer north east,
            title=Solution to IVP,
            axis lines = middle,
            xlabel = $t$,
            ylabel = $y$,
            variable = t,
            trig format plots = rad,
            ymin = -1,
            ]
            \addplot [
                domain=-10:10,
                samples=70,
                color=blue,
                ]
                {ln(x^2+1)*2+1};            
            \end{axis}
            \end{tikzpicture}
        \end{center}
        It turns out that the solution exists for all $t$, but we could not predict this!
    \end{example}
    \begin{example}
        Now consider the same ODE but with the initial value $y(-2)=1.$ The solution is $y(t) = \sqrt{2\ln\left(\frac{t^2+1}{5}\right)+1}$, then the solution is in the following interval:
        \begin{center}
            \begin{tikzpicture}
            \begin{axis}[
            legend pos=outer north east,
            title=Solution to IVP,
            axis lines = middle,
            xlabel = $t$,
            ylabel = $y$,
            variable = t,
            trig format plots = rad,
            ymin = -1,
            xmax = 3,
            ]
            \addplot [
                domain=-10:1,
                samples=200,
                color=blue,
                ]
                {sqrt(2*ln((x^2+1)/5)+1)};
            \end{axis}
            \end{tikzpicture}
        \end{center}
        If instead the initial condition was $y(0)=0$, note that we cannot surround the box such that $f$ and $f_Y$ is not continuous in that rectangle (i.e. at $y=0$).
    \end{example}
    \begin{warning}
        Note that the $E-U$ theorem is not an if and only if statement, i.e. 
        \begin{equation}
            \text{condition fulfilled} \implies \text{solution exists}
        \end{equation}
        but 
        \begin{equation}
            \text{solution exists} \centernot\implies \text{condition fulfilled}
        \end{equation}
    \end{warning}
    \item Some clarifications about the Picard–Lindelöf (E \& U) theorem:
    \begin{itemize}
        \item We need $f(t,y)$ continuous in the rectangle to guarantee existence.
        \item We need $f_y(t,y)$ continuous in the rectangle to get uniqueness.
    \end{itemize}
    \item There are no general solution for nonlinear ODEs, for example, take $y'y=^2$, then using separation of variables, we get 
    \begin{equation}
        y = -\frac{1}{t+C},
    \end{equation}
    there is no $C$ such that $y(0)=y_0=0.$
\end{itemize}