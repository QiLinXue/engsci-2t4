\section{Multiplying like Bunnies}
\begin{itemize}
    \item Exponential growth is as follows: 
    \begin{equation}
        y' = ky \implies y = Ce^{kt}
    \end{equation}
    \item \textbf{Logistic Growth} If uninhibited, we assume exponential growth. However, in the long run, population is limited to $k$. We generally have 
    \begin{equation}
        y' = rh(y)y
    \end{equation}
    where $h(y)$ is a limiting factor. If $h(y)=1$, we have exponential growth, if $h(y)=0$ we have no growth.
    \item We want $h(y)\approx 1$ if $y$ is small: 
    \begin{itemize}
        \item $0<h(y)<1$ if $y<k$
        \item $h(y) = 0$ if $y=k$
        \item $h(y) < 0$ if $y > k$
    \end{itemize}
    The simplest function that satisfies these is
    \begin{equation}
        y = 1 - \frac{y}{k}
    \end{equation}
    so we have the equation 
    \begin{equation}
        \frac{dy}{dt} = r\left(1-\frac{y}{k}\right)y,
    \end{equation}
    which is an autonomous ODE. This allows us to draw a phase plot:
    \begin{center}
        \begin{tikzpicture}
        \begin{axis}[
        legend pos=outer north east,
        title=Phase Plot for r\equals k\equals 1,
        axis lines = middle,
        xlabel = $y$,
        ylabel = $y'$,
        variable = t,
        trig format plots = rad,
        ymax = 2,
        ymin=-2,
        xmax = 2,
        xmin = -1,
        ]
        \addplot [
            domain=-1:2,
            samples=70,
            color=blue,
            ]
            {3*x*(1-x)};
        \end{axis}
        \end{tikzpicture}
    \end{center}
    which is unstable at $x=0$ and stable at $x=k$.
    \begin{example}
        Refer to the previous example. For what values of $y$ does a solution have an inflection point? We have 
        \begin{equation}
            y' = r(1-y/k)y
        \end{equation}
        and differentiating 
        \begin{align}
            y'' &= \frac{d}{dt}y' = \frac{d}{dt}(r(1-y/k)y) \\ 
                &= r\frac{d}{dt}\left(y-\frac{y^2}{k}\right) \\ 
                &= r\left(1-\frac{2y}{k}\right)\frac{dy}{dt}
        \end{align}
        and inflection points occur at $y=0,k,k/2.$
    \end{example}
    \begin{warning}
        The definition of inflection point we will use in this course is simply $y''=0$, which is a weaker version of what we learned in ESC194.
    \end{warning}
\end{itemize}