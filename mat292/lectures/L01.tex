\section{Introduction}
\textit{Covers 1.1: Mathematical Models and Solutions}
\begin{itemize}
    \item Big Idea: Differential equations model physical situations:
          \begin{itemize}
              \item Take a physical situation and ODE-ify it (How do we model a cooling coffee cup?)
              \item Understand an ODE without solving it (What can we deduce directly from $y'=y^2$?)
              \item Study, categories, typecast ODEs and solve them
                    \begin{example}
                        Suppose we have $y'=y/t+\ln t$ and $y'=y^2+t$. Which of these are harder to solve (without actually solving them)?
                        \vspace{2mm}

                        It turns out that the second one is harder as it is \textit{non-linear}.
                    \end{example}
              \item Handle ODEs numerically (What do we do when we cannot solve an ODE that models a real life phenomenon?)
              \item The art of problem solving (How do I work with no strings attached?)
          \end{itemize}
          \item What is a differential equation?
          \begin{definition}
              A differential equation relates a function and its derivatives.
          \end{definition}
          \item We can understand ODEs without solving it:
          \begin{example}
              Let's consider a cup of coffee in a room. We want to model its change in temperature over time. How do we do this?
              \vspace{2mm}

              There are a lot of variables, so we have to simplify our model. The things we care about
              \begin{itemize}
                  \item The temperature of the coffee cup $y(t)$.
                  \item $t$ is in minutes.
                  \item $y(t)$ is in Celsius.
                  \item The temperature in the room $T$ (in Celsius).
              \end{itemize}
              The things we ignore / simplify:
              \begin{itemize}
                  \item Temperature variation within the cup
                  \item Temperature variation in the room
              \end{itemize}
              \textbf{Exercise:} Let's consider some suggestions for an ODE describing the temperature of a coffee cup in a room. Each of the following suggested ODEs contradicts our intuition in some way. How?

              \begin{itemize}
                  \item $y'=y^2$
                        \begin{itemize}
                            \item $T$ isn't in there
                            \item Temperature would always increase except if $y=0$.
                            \item The hotter the coffee, the faster it heats up.
                        \end{itemize}
                  \item $y'=\frac{T}{y}$
                        \begin{itemize}
                            \item If $T>0, y>0$, then $y'>0$
                            \item The model doesn't work for coffee at $0^\circ \text{C}$.
                        \end{itemize}
                  \item $y'=y[e^{y-T}+y^3]$
                  \item $y'=y-T$
                  \item $y'=T-y$
                        \begin{itemize}
                            \item There should be a parameter that describes the physical properties (rate of heating/cooling will be different for different materials)
                        \end{itemize}
              \end{itemize}
          \end{example}
          \begin{idea}
            Without solving an ODE, you can already make many predictions about its solution
            \vspace{2mm}

            (and then, for example, judge your model)
          \end{idea}
\end{itemize}