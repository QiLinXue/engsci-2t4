\section{Thermodynamics Systems}
\subsection{Definitions}
\begin{itemize}
    \item \textbf{System:} any piece of matter or region of space that we identify for purposes of analysis.
    \item \textbf{Surroundings:} everything outside the system
    \item \textbf{System boundary:} surface that separates the system from the surroundings (denoted by dotted line)
    \item \textbf{Closed System:} a system where the mass is fixed. Also known as \textit{control mass}.
    \begin{itemize}
        \item Energy can enter or leave the system.
        \item The system boundary may move.
        \item The system boundary may be imaginary.
        \item The system boundary may change its shape and size.
    \end{itemize}
    \item \textbf{Open System:} A system where both energy and mass can cross boundaries. Also known as \textit{control volume}.
    \begin{itemize}
        \item Control volumes may move.
    \end{itemize}
    \item \textbf{Isolated System:} A system where no mass or energy can cross the system boundary. It therefore does not interact with surroundings.
\end{itemize}
\subsection{Thermodynamic Properties}
\begin{itemize}
    \item A property of a system is any attribute that can be measured without knowing the history of the system.
    \begin{itemize}
        \item i.e. mass, volume, temperature, energy
    \end{itemize}
    \item To represent infinitesimal amounts, we use $\dd$ to represent quantities that are properties (i.e. $\dd{x}$) and  $\delta$ to represent quantities that aren't properties (i.e. $\delta x$).
    \item \textbf{Point Function:} Value depends only on the state of the system. All properties are point functions.
    \item \textbf{Path Function:} Value depends on the start and end state and the path followed to get from one to the other.
    \item \textbf{Intensive Properties:} Properties can be specified at a point within the system and are independent of system mass.
    \item \textbf{Extensive Properties:} properties that depend on the mass of the system.
    \begin{idea}
        For any extensive property, we can define a corresponding intensive property.
    \end{idea}
\end{itemize}
\subsection{Steady State and Equilibrium}
\begin{itemize}
    \item A system is at \textbf{steady state} if its properties do not change with time, even though it is exchanging energy or mass with its surroundings.
    \begin{idea}
        All systems that are left isolated eventually reach a state of equilibrium where their properties do not change with time.
    \end{idea}
    \item Systems at equilibrium do not interact with surroundings so they cannot do work. A system does work only when it is not in equilibrium.
    \item \textbf{Mechanical Equilibrium:} Pressure within system is the same.
    \item \textbf{Thermal Equilibrium:} Temperature within system is the same.
    \item \textbf{Phase Equilibrium:} Pressure and temperature within system is the same. Amounts of the phases remain constant.
\end{itemize}
\subsection{State and Process}
\begin{itemize}
    \item The \textbf{state of a system} is described by a complete list of its properties. 
    \item \textbf{Process:} The change of a system from one state to another.
    \item We can show a thermodynamic process by selecting two properties and show how they vary on a 2D graph. The properties chosen depend on the process.
    \begin{center}
        \begin{tikzpicture}[
            > = latex,
            dot/.style = {draw,fill,circle,inner sep=1pt}
            ]
            \draw[<->] (0,4.3) node[above right] {$p$} |- (4.3,0) node[right] {$V$};

            \node[dot, ,label={above right:$1$}] (1) at (4,2) {};
            \node[dot, label={above right:$2$}] (2) at (2,4) {};
            \draw[->, red] (1) to[out = 180, in=-90] (2);
          \end{tikzpicture}
    \end{center}
    The line is known as the \textbf{process path}.
    \item \textbf{Isothermal Process:} The temperature of the system is constant.
    \begin{itemize}
        \item This can be accomplished by surrounding a piston with a constant temperature path.
    \end{itemize}
    \item \textbf{Adiabatic Process:} There is no heat transfer to or from the system.
    \begin{itemize}
        \item This is achieved by surrounding the piston with perfect insulation.
    \end{itemize}
    \item \textbf{Isobaric Process:} The pressure of the system is constant.
    \begin{itemize}
        \item This can be accomplished by heating up a piston/chamber with a heat source.
    \end{itemize}
    \begin{idea}
        Suppose a piston rapidly compresses a gas confined in a cylinder. can we measure the gas pressure $(P)$ as a function of volume $(V)$?
        \vspace{2mm}

        The answer is no, as the system will not be in equilibrium during the process. Therefore, properties are only well defined at \textit{equilibrium.}
    \end{idea}
    \item Therefore, when we define a process path, we assume the process is \textbf{quasi-equilibrium} (very slowly).
    \item \textbf{Thermal Reservoir:} A system whose temperature remains constant despite heat transfer to or from it (i.e. atmosphere, lake).
    \begin{example}
        Suppose we have three thermal reservoirs at $T_1, T_1+\Delta T, T_1+2\Delta T, \dots, T_2$. We can put a system in contact with thermal reservoir $T_1$. We can then move it to the next thermal reservoir, until we get to $T_2$. This represents a quasi-equilibrium heating process when we take the limit where $\Delta T \rightarrow 0$.
    \end{example}
    \begin{center}
        \begin{tikzpicture}[
            > = latex,
            dot/.style = {draw,fill,circle,inner sep=1pt}
            ]
            \draw[<->] (0,4.3) node[above right] {$p$} |- (4.3,0) node[right] {$V$};

            \node[dot, ,label={right:$1$}] (1) at (4,1) {};
            \node[dot, label={above:$2$}] (2) at (1,3) {};
            \node[dot, label={above:$3$}] (3) at (2,3) {};
            \node[dot, label={right:$4$}] (4) at (4,2) {};

            \draw[->, red] (1) to[out = 180, in=-90] (2);
            \draw[->, red] (2) to (3);
            \draw[->, red] (3) to[out = -90, in=180] (4);
            \draw[->, red] (4) to (1);
          \end{tikzpicture}
    \end{center}
    \item \textbf{Cycle:} Any process, or series of processes, that result in the system in the same state it starts from. For example, the above is an example of a cycle:
    \item Going from $2\rightarrow 3$ is an isocharic process (constant volume) and going from $4\rightarrow 1$ is an isobaric process.
\end{itemize}