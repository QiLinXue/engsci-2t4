\section{Specific Heat, Steady Flow, Entropy}
\begin{itemize}
    \item We use $c_v$ to calculate change in internal energy for any process.
    \item We use $c_p$ to calculate change in enthalpy for any process.
    \item The mass flow rate into the control volume is given by 
    \begin{equation}
        \dot{m} = \rho A \bm{V}
    \end{equation}
    where $\bm{V}$ is the flow velocity.
    \item At steady state, we have 
    \begin{equation}
        \dot{m}_\text{in} = \dot{m}_\text{out} = \dot{m}
    \end{equation}
    \item The flow work is $Pv$.
    \item The kinetic energy is $\frac{1}{2}\bm{V}^2$
    \item and the specific potential energy is $gz$. The total energy transported per unit mass of fluid is given by (Bernoulli's Equation):
    \begin{equation}
        e = h + \frac{1}{2}\bm{V}^2 + gz
    \end{equation}
    \item The rate of energy transfer into the control volume is 
    \begin{equation}
        \dot{m}\left(h + \frac{1}{2}\bm{V}^2 + gz\right)
    \end{equation}
    \item Using the fact that the rate of energy entering CV is the rate of energy leaving CV, we can write out
    \begin{equation}
        \dot{Q} + \dot{W} + \dot{m}(h_1+\frac{1}{2}v_1^2+gz_1)=\dot{m}(h_2+\frac{1}{2}v_2^2+gz_2)
    \end{equation}
    Dividing this by $\dot{m}$, we can rewrite it as 
    \begin{equation}
        q+w = \Delta h + \frac{v_2^2-v_1^2}{2}+g\Delta z
    \end{equation}
    where $q=\dot{Q}/\dot{m}$ and $w=\dot{W}/\dot{m}$.
    \begin{example}
        In a turbine, the flow enters with higher $P_1,T_1,h_1$ and exits with low $P_2,T_2,h_2$.  Let us neglect heat loss to surroundings and changes in kinetic and potential energy. We have 
        \begin{equation}
            \cancel{\dot{Q}} + \dot{W} + \dot{m}\left(h_1+\cancel{\frac{1}{2}V_1^2}+\cancel{gz_1}\right)=\dot{m}(h_2+\cancel{\frac{1}{2}v_2^2}+\cancel{gz_2})
        \end{equation}
        which gives 
        \begin{equation}
            \dot{W}_\text{shaft} = \dot{m}(h_2-h_1)
        \end{equation}
        since $h_2<h_1$, we have $\dot{W}_\text{shaft}$, so turbine does work on surroundings.
    \end{example}
    \item For a compressor, we have something similar to the above, but $h_2>h_1$, so the surroundings does work on the compressor.
    \begin{example}
        For a pump, we enter with $P_1,h_1$ and exit with $P_2,h_2$ after increasing a distance of $\Delta z$. Let us neglect heat loss to the surroundings, changes in kinetic energy, and the temperature remains constant.
        \vspace{2mm}

        We get 
        \begin{equation}
            \dot{W}_\text{shaft} = \dot{m}[(h_2-h_1)+g(z_2-z_1)]
        \end{equation}
        and for an incompressible liquid, we have 
        \begin{equation}
            h_2-h_1=\cancel{c_v\Delta T}+v\Delta P.
        \end{equation}
        Using this, we get 
        \begin{equation}
            \dot{W}_\text{shaft} = \dot{m}[v\Delta P + g\Delta z]
        \end{equation}
    \end{example}
    \item Let us define a new extensive property known as entropy, defined as 
    \begin{equation}
        \text{entropy change} = \frac{\text{heat transferred}}{\text{temperature}}.
    \end{equation}
    The lower the entropy change associated with any heat transfer, the more work can be done with it.
    \item Let entropy be denoted as $S$. Then 
    \begin{equation}
        \dd{S} = \frac{\delta Q}{T}
    \end{equation}
    where $T$ is the temperature of the system boundary where the heat crosses.
    \item The entropy of a thermal reservoir is 
    \begin{equation}
        \Delta S = \int_1^2 \frac{\delta Q}{T} = \frac{1}{T}\int_1^2 \delta Q = \frac{Q_{12}}{T}
    \end{equation}
    \item Suppose we have two thermal reservoirs. The one at $A$ has temperature $T+\Delta T$ and the one at $B$ has temperature $T$. Then: 
    \begin{equation}
        \Delta S_A = \frac{Q}{T+\Delta T}
    \end{equation}
    and 
    \begin{equation}
        \Delta S_b = \frac{Q}{T}.
    \end{equation}
    The entropy reaching $B$ is greater than that leaving $A$, so we have generated entropy. Specifically,
    \begin{equation}
        S_\text{gen} =\Delta S_b - \Delta S_a
    \end{equation}
    \item In the limit as $\Delta T\rightarrow 0$, we have $\Delta S_b=\Delta S_A$, so $S_\text{gen}=0$.
    \begin{idea}
        There is no entropy generated between two thermal reservoirs whose temperatures differ by an infinitesimal amount.
    \end{idea}
    \item Entropy can be created, but not destroyed.
    \begin{theorem}
        The entropy of an isolated system will increase until the system reaches a state of equilibrium. The entropy of an isolated system remains constant.
    \end{theorem}
    \item A reversible process is a process that can be reversed by an infinitesimal change in the surroundings so that both the system and surroundings are restored to their initial conditions.
    \item No entropy is generated during a reversible process.
    \item A reversible process gives the least amount of work required for a process.
    \item If entropy is generated, the surroundings and system cannot be restored to their original state.
    \item Suppose we want to increase a system's temperature from $T_1$ to $T_2$ in a reversible matter. We can do this by placing it in contact with thermal reservoirs of temperatures $T_1+\Delta T,T_1+2\Delta T,\dots$ where $\Delta T \rightarrow 0$.
    \item In general, for any system 
    \begin{equation}
        \dd{S} = \frac{\delta Q}{T} + dS_\text{gen}.
    \end{equation}
    When the process is reversible and adiabatic, we have $\dd{S}=0$, i.e. the entropy is constant. This is known as \textbf{isentropic}.
\end{itemize}