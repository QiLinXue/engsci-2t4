\section{Energy}
\begin{itemize}
    \item Energy can be stored in microscopic forms at the molecular level. These do not require a change in position or velocity of the system, known as \textbf{internal energy.}
    \item Changes in internal energy typically correspond to changes in temperature and pressure.
          \subsection{Ideal Gas Equation}
    \item The Ideal Gas Law tells us that
          \begin{equation}
              PV = nR_uT
          \end{equation}
          or
          \begin{equation}
              PM = \rho R_uT
          \end{equation}
          where
          \begin{equation}
              R_u = 8.\pi \frac{\text{kJ}}{\text{kmol K}}
          \end{equation}
    \item This makes the following assumptions:
          \begin{itemize}
              \item Gas molecules are in constant random motion.
              \item All molecules have equal mass.
              \item The number of molecules is very large.
              \item Molecules collide perfectly elastically.
              \item No forces acts on molecules except during collisions.
              \item Molecules are point masses with negligible volume.
          \end{itemize}
    \item The \textbf{gram-mole} is the molar mass in $g$ and contains $N_A$ molecules.
    \item The \textbf{kilogram-mole} is the molar mass in kilograms.
          \subsection{Internal Energy Mean Square Speed}
    \item The mean square speed is given by
          \begin{equation}
              c_\text{rms}^2 = \frac{3k_BT}{m}
          \end{equation}
          The average kinetic energy of a single monoatomic molecule is
          \begin{equation}
              \frac{1}{2}mc_\text{rms}^2 = \frac{3}{2}\frac{k_BT}{m}
          \end{equation}
          \begin{warning}
              The above formula is only an approximation, and only applies for monoatomic molecules!
          \end{warning}
    \item The internal energy of a gas is given by
          \begin{equation}
              U = \frac{3}{2}NR_uT
          \end{equation}
          and is a function of temperature alone. For ideal gases which are not monoatomic, we can write
          \begin{equation}
              \Delta U = c_V(T_2-T_1)
          \end{equation}
          where $c_V$ is the specific heat\footnote{To be more precise, the specific heat at constant volume.}

          \subsection{First Law of Thermodynamics}

    \item The first law of thermodynamics is
          \begin{equation}
              \Delta U = Q + W
          \end{equation}
          \textbf{Sign convention:} Energy transferred to a system is positive and energy transferred from a system is negative.
    \item $Q_\text{net}$ represents the net heat added, and $W_\text{net}$ represents the net work done.
    \item We define power as
          \begin{equation}
              P = \frac{\delta W}{\dd{t}}.
          \end{equation}
    \item The heat transfer per unit mass of the system is
          \begin{equation}
              q = \frac{Q}{m}.
          \end{equation}
    \item Boundary work refers to the work a force does on a system by acting on its boundary. It is given by
          \begin{equation}
              W_\text{boundary} = -\int_{V_1}^{V_2}P\dd{V}.
          \end{equation}
          \begin{idea}
              Freely moving pistons result in constant pressure.
          \end{idea}
    \item For an isobaric process, we have
          \begin{equation}
              W_{12} = -P(V_2-V_1)
          \end{equation}
    \item For an isothermal process, we have
          \begin{equation}
              W_{12}  =-\int_{V_1}^{V_2} P \dd{V}  = -\int_{V_1}^{V_2} \frac{mRT}{V}\dd{V} =  -mRT \ln\frac{V_2}{V_1}
          \end{equation}
    \item For a polytropic process, we can write $PV^N = \text{constant}$. For $n\neq 1$, the work done is
          \begin{equation}
              W_{12} = \frac{CV_{2}^{1-n}-CV_1^{1-n}}{n-1}
          \end{equation}
    \item Flow work refers to the force that pushes a fluid element into a control volume. The work done in pushing fluid element is
          \begin{equation}
              W_\text{flow} = FL = PV.
          \end{equation}
    \item The shaft work (i.e. windmills), or more importantly the power associated with it, is
          \begin{equation}
              \dot{W}_\text{shaft} = 2\pi\omega\tau
          \end{equation}
          \subsection{Enthalpy}
    \item We can define enthalpy to be
          \begin{equation}
              H = U + PV
          \end{equation}
          \begin{idea}
              A good analogy is to think about a wizard creating a system out of nowhere. He/she/it needs to generate the internal energy $U$ of the system, but also move space for it, which requires an energy $PV$.
              \vspace{2mm}

              Alternatively, it measures the capacity of a fluid to do work.
          \end{idea}
    \item Enthalpy is not a fundamental property because it is defined as a function of other properties. It is useful as it make keeping track of flow work easier when analyzing control volume systems.
          \begin{idea}
              Heat transfer to a constant pressure system equals change in enthalpy.
          \end{idea}
          \subsection{Specific Heats}
    \item The specific heat at constant volume is 
    \begin{equation}
        c_V(T) = \left(\frac{\partial u}{\partial T}\right)_{v}
    \end{equation}
    At constant pressure, we have 
    \begin{equation}
        c_p(T) = \left(\frac{\partial h}{\partial T}\right)_{p}
    \end{equation}
    \item The molar specific heat is: 
    \begin{equation}
        \bar{c_v} = c_vM
    \end{equation}
    and similarly at constant pressure: 
    \begin{equation}
        \bar{c_p} = c_pM
    \end{equation}
    \item Note that for an ideal gas, $h=u+RT$, so the enthalpy of an ideal gas depends only on temperature. Therefore, we can write 
    \begin{equation}
        c_p(T) = \frac{\dd{h}}{\dd{T}}
    \end{equation}
    \item If we assume $c_v$ and $c_p$ don't depend on temperature, we get 
    \begin{equation}
        c_p - c_v = R
    \end{equation}
    \item We can define a new property (known as the adiabatic constant), as the specific heat ratio: 
    \begin{equation}
        \gamma = \frac{c_p}{c_v}
    \end{equation}
    \item Liquids and solids are mainly incompressible, so we cannot do boundary work. The only way to change internal energy is thus by heating it.
    \item For liquids and solids, we have
    \begin{equation}
        c_p = c_v = c
    \end{equation}
\end{itemize}