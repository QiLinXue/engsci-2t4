\section{Isentropic Processes, Bernoulli's Equation, and Phase Change}
\begin{itemize}
    \item For an ideal gas with constant specific heats, 
    \begin{equation}
        \Delta s = s_2- s_1 = c_v\ln\frac{T_2}{T_1} + R\ln\frac{v_2}{v_1}
    \end{equation}
    \item For an isentropic process $\Delta s=0$, we get 
    \begin{equation}
        \frac{T_2}{T_1} = \left(\frac{v_1}{v_2}\right)^{R/c_v}
    \end{equation}
    Since $c_p-c_v=R$ and $\gamma = \frac{c_p}{c_v}$, so 
    \begin{equation}
        \boxed{\frac{T_2}{T_1} = \left(\frac{v_1}{v_2}\right)^{\gamma-1}}
    \end{equation}
    Similarly, we have 
    \begin{equation}
        \boxed{\frac{T_2}{T_1} = \left(\frac{P_2}{P_1}\right)^{\frac{\gamma-1}{\gamma}}}
    \end{equation}
    and finally 
    \begin{equation}
    \boxed{\frac{P_2}{P_1}=\left(\frac{v_1}{v_2}\right)^\gamma}
    \end{equation}
    and gives $Pv^\gamma = \text{constant}.$
    \item If specific heats are not constant, then 
    \begin{equation}
        \Delta s = s^0(T_2)-S^0(T_1)-R\ln\frac{P_2}{P_1}
    \end{equation}
    where 
    \begin{equation}
        s^0(T) = \int_{T_\text{ref}}^T c_p(T) \frac{\dd{T}}{T}.
    \end{equation}
    For an isentropic process $\Delta s=0$, we get 
    \begin{equation}
        s^0(T_2)=s^0(T_1)+R\ln\frac{P_2}{P_1}
    \end{equation}
    \item We can define relative pressure as
    \begin{equation}
        P_r(T) = \exp\left(\frac{s^0(T)}{R}\right)
    \end{equation}
    Similarly, we can also define 
    \begin{equation}
        v_r = \frac{RT}{P_r}
    \end{equation}
    \item For a reversible compression or expansion, we have 
    \begin{equation}
        \delta w = -P \dd{v}.
    \end{equation}
    For an internally reversible process, we then have 
    \begin{equation}
        \delta q = \dd{u} + P \dd{v}.
    \end{equation}
    From Gibb's Equation, we have 
    \begin{equation}
        \dd{s} = \frac{1}{T}\dd{u} + \frac{P}{T}\dd{v} = \frac{1}{T}(\dd{u}+P\dd{v}) = \left(\frac{\delta q}{T}\right)_\text{int rev}
    \end{equation}
    \item Multiplying by the system mass, we have 
    \begin{equation}
        \dd{S} = \left(\frac{\delta Q}{T}\right)_\text{int rev}
    \end{equation}
    so the change in entropy is 
    \begin{equation}
        \Delta S = \int_1^2 \frac{\delta Q}{T}.
    \end{equation}
    For a thermal reservoir with $T$ constant, this becomes $\Delta S = \frac{Q_{12}}{T}$ and the rate becomes 
    \begin{equation}
        \frac{dS}{dt} = \frac{\dot{Q}_\text{int rev}}{T}
    \end{equation}
    \item For an irreversible process, we can write 
    \begin{equation}
        \Delta S = \int_1^2 \frac{\delta Q}{T} + S_\text{gen}
    \end{equation}
    Entropy can only be generated, not destroyed, so $S_\text{gen} >0$.
    \begin{idea}
        The area under a T-S curve gives the heat added for a reversible process.
    \end{idea}
    \item \textbf{Entropy Balances:} Note that entropy can be transferred to or from a control mass. It can also be generated within the system due to irreversibilities. 
    \item The rate of entropy transfer is 
    \begin{equation}
        \frac{dS}{dt} = \dot{S}_\text{in} - \dot{S}_\text{out} + \dot{S}_\text{gen}
    \end{equation}
    At steady state, $\frac{dS}{dt}=0$, and for an internally reversible system, $\dot{S}_\text{gen}=0$.
    \item For a control mass, the only way to transfer entropy is with heat. Therefore 
    \begin{equation}
        \dot{S}_\text{heat} = \frac{\dot{Q}_j}{T_j}
    \end{equation}
    and 
    \begin{equation}
        \frac{dS}{dt} = \sum_j \frac{\dot{Q}_j}{T_j}+\dot{S}_\text{gen}
    \end{equation}
    \item The entropy entering is 
    \begin{equation}
        \dot{S}_\text{mass,in}=\dot{m}_is_i
    \end{equation}
    and the entropy exiting is 
    \begin{equation}
        \dot{S}_\text{mass,out}=\dot{m}_es_e.
    \end{equation} 
    This then becomes 
    \begin{equation}
        \frac{dS}{dt} = \sum_j\frac{\dot{Q}_j}{T_j}+\sum_i \dot{m}_is_i - \sum_i \dot{m}_es_e + \dot{S}_\text{gen}
    \end{equation}
    \begin{idea}
        We can simplify this with various assumptions
        \begin{itemize}
            \item At steady state, we have $\frac{dS}{dt}.$
            \item If we only have one inlet and outlet, we have $\dot{m}_i=\dot{m}_e=\dot{m}$.
            \item If adiabatic, we have $\dot{Q}_j=0$.
            \item If reversible, we have $s_i=s_e$.
        \end{itemize}
    \end{idea}
    \item \textbf{Analyzing Steady Flow Devices:} When asked for the \textit{maximum work,} we should assume 
    \begin{itemize}
        \item heat losses are low, so \textbf{adiabatic}
        \item all other losses, so \textbf{reversible}
    \end{itemize}
    Therefore, maximum work occurs when the process is isentropic.
    \item If we know $P,T$ of the input and output. We can then calculate $s_1(P_1,T_1)$, $s_2(P_2,T_2)$, and set $s_1=s_2$.
    \item \textbf{Isobars:} It is helpful to draw isobars on a T-S diagram, which are lines where pressure is constant. We get: 
    \begin{equation}
        \frac{T_2}{T_1} = \exp\left(\frac{\Delta s}{c_p}\right)
    \end{equation}
    Note that temperature increases exponentially with increase in entropy. Therefore, isobars are exponential lines.
    \item We can then draw a line on the $S-T$ diagram representing our process between two isobars. 
    \item The work done by an isentropic turbine is thus 
    \begin{equation}
        \dot{W}_\text{t,s} = \dot{m}(h_{2s}-h_1) = \dot{m}c_p(T_{2s}-T_1)
    \end{equation}
    Note that the $2s$ represents the final temperature given an isentropic process.
    \item If the process was irreversible, the final entropy would be higher since entropy is generated. 
    \item The work done by an irreversible turbine is
    \begin{equation}
        \dot{W}_t = \dot{m}c_p(T_2-T_1)
    \end{equation}
    where $\dot{W}_t$ is the actual work output, which will always be less.
    \item The turbine efficiency is
    \begin{equation}
        \eta = \frac{\text{actual work output}}{\text{ideal work output}} = \frac{\dot{W}_t}{\dot{W}_{t,s}}=\frac{h_2-h_1}{h_{2s}-h_1}
    \end{equation}
    and is typically around $70-90\%$.
    \begin{example}
        Suppose we have a nozzle. Assume that $v_2 \gg v_1$. For an isentropic nozzle, exit velocity is $v_{2s}$ and for a real nozzle, exit velocity is $v_s$.  The nozzle efficiency is written in terms of kinetic energy: 
        \begin{equation}
            \eta_n = \frac{v_2^2}{v_{2s}^2}
        \end{equation}
    \end{example}
    \item For a compressor, we have 
    \begin{equation}
        \eta_c = \frac{h_{2s}-h_1}{h_2-h_1}.
    \end{equation}
    Remember that an ideal compressor needs less work.
    \item Let's look at flow. Incompressible flow tells us that $\Delta s= c_\text{avg} \ln\frac{T_2}{T_1}$ and isentropic flow tells us that $\Delta s=0$.
    \begin{idea}
        Isentropic and incompressible flow is isothermal.
    \end{idea}
    \item For incompressible substances, we have $\Delta h= c(T_2-T_1)+v(P_2-P_1)$. For an isothermal process, we have 
    \begin{equation}
        h_2-h_1 = v(P_2-P_1) = \frac{P_2-P_1}{\rho}.
    \end{equation}
    We can then write an energy balance: 
    \begin{equation}
        \dot{Q} + \dot{W} + \dot{m}(h_1+\frac{1}{2}v_1^2+gz_1)=\dot{m}(h_2+\frac{1}{2}v_2^2+gz_2).
    \end{equation}
    Assuming no work or energy transfer (isentropic and incompressible), we get 
    \begin{equation}
        \frac{P_1}{\rho}+\frac{v_1^2}{2}+gz_2 = \frac{P_2}{\rho}+\frac{v_2^2}{2}+gz_2
    \end{equation}
    \item \textbf{Phase Change:} High energy molecules escape from liquid and become vapour. Low energy molecules are recaptured by the liquid surface. Phase equilibrium is reached when the two rates are equal.
    \item Molecules leaving the liquid have higher energy than average so the liquid is cooled when evaporation occurs.
    \item The energy carried away is known as latent heat.
    \begin{idea}
        We have energy balance 
        \begin{equation}
            \dd{U} = h\dd{m} + \delta Q + \delta W
        \end{equation}
        entropy balance 
        \begin{equation}
            \delta Q = T\dd{s} - Ts\dd{m}.
        \end{equation}
        Combining, we get
        \begin{equation}
            \dd{U} = T\dd{S}-P\dd{V} + (h-Ts)\dd{m}
        \end{equation}
    \end{idea}
    \begin{definition}
        Let us define the Gibbs Energy 
        \begin{equation}
            G = H - TS
        \end{equation}
        which is an extensive property.
    \end{definition}
    \item Energy balance then becomes 
    \begin{equation}
        \dd{U} = T\dd{S} - P\dd{V} + g\dd{m}
    \end{equation}
    which is equivalent to 
    \begin{equation}
        \dd{S} = \frac{\dd{U}}{T} + \frac{P}{T}\dd{V}-\frac{g}{T}\dd{m}
    \end{equation}
    \item Notation: 
    \begin{itemize}
        \item $f$: saturated liquid (stands for $flussigkeit$ which is liquid in German)
        \item $g$: saturated vapour (Gas)
    \end{itemize}
    \item We use the following properties and postulates
    \begin{itemize}
        \item State Postulate: $S_f=S_f(U_f,V_f,m_f)$ and $S_g=S_g(U_g,V_g,m_g)$
        \item Entropy is extensive: $S=S_f+S_g$
        \item Second Law: At equilibrium, $\dd{S}=0$
        \item Isolated System: Total mass, volume, energy are fixed.
    \end{itemize}
    Starting from the Gibbs Equation for liquid and vapour and combining gives 
    \begin{equation}
        \dd{S}_g = -\frac{\dd U_f}{T_g}-\frac{P_g}{T_g}\dd{V}_f + \frac{g_g}{T_g}\dd{m}_f
    \end{equation}
    \item We can use $\dd{S}+\dd{S}_f+\dd{S}_g=0$ to derive that at equilibrium 
    \begin{itemize}
        \item $T_f=T_g$
        \item $P_f=P_g$
        \item $g_f=g_g$
    \end{itemize}
    The first two are known, but the third one is new. It is also known as the \textbf{chemical potential}. This tells us there is no exchange of mass between two phases whose chemical potentials are the same.
    \item Assume system is not at equilibrium and uniform pressure and temperature is everywhere. Then we must have $g_f>g_g.$
    \begin{idea}
        Mass transfer takes place from the phase with higher chemical potential to the phase with lower chemical potential.
    \end{idea}
    \item We can write 
    \begin{equation}
        \dd{G} = \dd{U} + P\dd{V} + V\dd{P} - T\dd{S} - S\dd{T}
    \end{equation}
    substituting $\dd{S}$ using the Gibbs equation, we get 
    \begin{equation}
        \dd{G} = g\dd{m} + V\dd{P} - S\dd{T}
    \end{equation}
    Gibbs Energy can change because of changes in mass, pressure, or temperature of the system.
    \item The total Gibbs energy is $G=gm$. Taking the differential, we have 
    \begin{equation}
        \dd{G}=g\dd{m}+m\dd{g}
    \end{equation}
    Simplifying, we get the Gibbs-Duhem Equation, which tells us that 
    \begin{equation}
        \dd{g} = v\dd{P} - s\dd{T}
    \end{equation}
    The chemical potential varies with pressure and temperature.
\end{itemize}