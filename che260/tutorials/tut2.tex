\documentclass{article}
\usepackage{qilin}
\tikzstyle{process} = [rectangle, rounded corners, minimum width=1.5cm, minimum height=0.5cm,align=center, draw=black, fill=gray!30, auto]
\title{CHE260: Tutorial Problems \\ \textbf{Tutorial 2 Solutions}}
\author{QiLin Xue}
\date{Fall 2021}
\usepackage{mathrsfs}
\usetikzlibrary{arrows}
\begin{document}

\maketitle
\begin{enumerate}
    \item We have 
    \begin{align}
        W &= -\int_{V_1}^{V_2} A/V+B \dd{V} \\ 
        &= \left[A\ln(V_1)+BV_1\right] - \left[A\ln(V_2)+BV_2\right]  \\ 
        &= A\ln(V_1/V_2) + B(V_1-V_2) = -12.59\text{ kJ}
    \end{align}
    \item We first find the final pressure: 
    \begin{equation}
        P_2 = P_1\left(\frac{V_1}{V_2}\right)^{1.35} = 2.00\text{ bar.}
    \end{equation}
    \begin{equation}
        W = \frac{P_2V_2-P_1V_1}{n-1} = -1284\text{ kJ.}
    \end{equation}
    \item The initial volume is given by $V_1$ such that 
    \begin{equation}
        P_1V_1 = mRT \implies V_1 = \frac{mRT}{P_1} = 0.1259 \text{ m}^3. 
    \end{equation}
    so $PV^{1.2} = 49.91.$ 
    and the final gas pressure is given by 
    \begin{equation}
        P_2V_2 = mRT \implies P_2\left(\frac{85.87}{P_2}\right)^{0.833} = mRT
    \end{equation}
    or 
    \begin{equation}
        25.67P_2^{0.167}=mRT \implies P_2 = 1.10\text{ bar}
    \end{equation}
    where the error comes from lazy rounding. Using $1.03$ from now on, we have 
    \begin{equation}
        V_2 = \left(\frac{49.91}{P_2}\right)^{1/1.2}= 0.547\text{ m}^3
    \end{equation}
    \begin{equation}
        W = \frac{P_2V_2-P_1V_1}{n-1} = \frac{4}{3} = -96.0\text{ kJ.}
    \end{equation}
    From the first law,
    \begin{equation}
        \Delta U = Q + W
    \end{equation}
    and $\Delta U = mc\Delta T = -66.59 \text{ kJ}$ and plugging this in for $Q$ gives $Q = 29.7\text{ kJ}.$ 
    \item We have $P_1V_1 = P_2V_2.$ Solving for $V_2$ gives $V_2=4\text{ m}^3.$ We can determine the mass of air using the ideal gas law: 
    \begin{equation}
        m = \frac{PV}{RT} = 4.64\text{ kg}.
    \end{equation}
    We have
    \begin{equation}
        W = P_2V_2\ln(V_1/V_2) = -277\text{ kJ}
    \end{equation}
    and finally $\Delta U = 0$ since it's an isothermal process so $Q=-W.$ 
\end{enumerate}
\end{document}