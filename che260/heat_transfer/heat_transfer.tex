\documentclass{article}
\usepackage{qilin}
\tikzstyle{process} = [rectangle, rounded corners, minimum width=1.5cm, minimum height=0.5cm,align=center, draw=black, fill=gray!30, auto]
\title{CHE260: Heat Transfer}
\author{QiLin Xue}
\date{Fall 2021}
\usepackage{mathrsfs}
\usetikzlibrary{arrows}
\begin{document}

\maketitle
\section{Mechanisms of Heat Transfer}
\begin{itemize}
        \item Mechanisms involve:
        \begin{itemize}
            \item Conduction: Transfer of heat through a medium that is stationary.
            \item Convection: Transfer of heat from a solid surface and an adjacent fluid that is moving.
            \item Radiation: Energy emitted by matter in the form of electromagnetic waves.
        \end{itemize}
        \item Conduction follows \textbf{Fourier's Law:}
        \begin{equation}
            \dot{Q} = - k A \nabla T
        \end{equation}
        \item The rate of heat transfer from the surface of a blackbody is given by the Stefan-Boltzmann Law 
        \begin{equation}
            \dot{Q}_\text{emit,max} = \sigma A T_s^4
        \end{equation}
        where $\sigma$ is the Boltzmann constant $\sigma = 5.67 \times 10^{-8} \frac{W}{m^2 K^4}$
        \item For an ideal body, we have
        \begin{equation}
            \dot{Q}_\text{emit} = \epsilon \sigma AT_s^4
        \end{equation}
        where $0 \le \epsilon \le 1$ is the emissivity.
        \item When radiation is incident on a surface, some will be absorbed and some reflected. The \textbf{absorptivity} $\alpha$ is defined such that
        \begin{align}
            \dot{Q}_\text{absorbed} &= \alpha \dot{Q}_\text{incident} \\ 
            \dot{Q}_\text{reflected} &= (1-\alpha)\dot{Q}_\text{incident}
        \end{align}
        \item Kirchoff's Law says that
        \begin{equation}
            \alpha = \epsilon
        \end{equation}
        \item For a small surface completed surrounded by a much larger surface net radiation is 
        \begin{equation}
            \dot{Q}_\text{net} = \epsilon \sigma A (T_s^4 - T_\text{surrounding}^4)
        \end{equation}
        \item Natural convection tells us the heat transfer coefficient is
        \begin{equation}
            h = c(T_S-T_\infty)^{1/4}
        \end{equation}
        where $c=4.2 \text{W}{m^2 K^{5/4}}$ and 
        \begin{equation}
            q_\text{conv}  = hA(T_S-T_\infty)
        \end{equation}
        \item Forced convection gives a constant $h = 250 W/m^2 K$
        \item Let's look at the \textbf{one-dimensional case:} If we look at a segment of length $\Delta x$, the rate of the increase of enthalpy is
        \begin{align}
            \dot{H} &= mc_p \frac{\partial T}{\partial t} \\ 
            &= \rho c_p A \Delta x \frac{\partial T}{\partial t}
        \end{align}
        The energy balance in this small segment gives
        \begin{align}
            \rho c_p A\Delta x\frac{\partial T}{\partial t} &= \dot{Q}_x - \dot{Q}_{x+\Delta x} \\ 
            &= (\dot{q}A)_x - (\dot{q}A)_{x+\Delta x}
        \end{align}
        Dividing by $\Delta x$ and taking the limit gives 
        \begin{equation}
            \rho c_p \frac{\partial T}{\partial t} = -\frac{1}{A} \frac{\partial(\dot{q}A)}{\partial x}
        \end{equation}
        Note that $A$ depends on the coordinate system, $\dot{q}$ depends on Fourier's Law $\dot{q} = -k\frac{dT}{dx}$.
        \item This gives us the heat conduction equation 
        \begin{equation}
            \frac{\partial^2 T}{\partial x^2} = \frac{1}{\alpha}\frac{\partial T}{\partial t}  
        \end{equation}
        where $\alpha=\frac{k}{\rho c_p}$ is the thermal diffusivity.
        \item At constant state, $\frac{\partial T}{\partial t} = 0$, so 
        \begin{equation}
            \frac{d^2 T}{dx^2} = 0
        \end{equation}
        \item In cylindrical coordinates, we have
        \begin{equation}
            \frac{1}{\alpha}\frac{\partial T}{\partial t} = \frac{1}{r}\frac{\partial}{\partial r}\left(r\frac{\partial T}{\partial r}\right)
        \end{equation}
        \item In spherical coordinates 
        \begin{equation}
            \frac{1}{\alpha}\frac{\partial T}{\partial t} = \frac{1}{r^2}\frac{\partial}{\partial r}\left(r^2 \frac{\partial T}{\partial r}\right)
        \end{equation}
        \item In general, we have 
        \begin{equation}
            \frac{1}{r^n}\frac{\partial}{\partial r}\left(r^n \frac{\partial T}{\partial r}\right) = \frac{1}{\alpha}\frac{\partial T}{\partial t}
        \end{equation}
        where 
        \begin{itemize}
            \item $n=0$ for cartesian
            \item $n=1$ for cylindrical 
            \item $n=2$ for spherical
        \end{itemize}
        \begin{idea}
            These equations can alternatively derived by applying the divergence formula in different coordinate systems. In particular, it is done by considering the heat equation 
            \begin{equation}
                \frac{1}{\alpha}\frac{\partial T}{\partial t} = \nabla \cdot T
            \end{equation}
            and applying spherical / cylindrical symmetry.
        \end{idea}
        \item To solve problems, we also need boundary conditions, and we often make the steady state assumption.
    \end{itemize}
\end{document}