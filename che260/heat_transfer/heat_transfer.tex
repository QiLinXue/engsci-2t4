\documentclass{article}
\usepackage{qilin}
\tikzstyle{process} = [rectangle, rounded corners, minimum width=1.5cm, minimum height=0.5cm,align=center, draw=black, fill=gray!30, auto]
\title{CHE260: Heat Transfer}
\author{QiLin Xue and John Zhang}
\date{Fall 2021}
\usepackage{mathrsfs}
\usetikzlibrary{arrows}
\begin{document}

\maketitle
% \section{Mechanisms of Heat Transfer}
% \begin{itemize}
%         \item Mechanisms involve:
%         \begin{itemize}
%             \item Conduction: Transfer of heat through a medium that is stationary.
%             \item Convection: Transfer of heat from a solid surface and an adjacent fluid that is moving.
%             \item Radiation: Energy emitted by matter in the form of electromagnetic waves.
%         \end{itemize}
%         \item Conduction follows \textbf{Fourier's Law:}
%         \begin{equation}
%             \dot{Q} = - k A \nabla T
%         \end{equation}
%         \item The rate of heat transfer from the surface of a blackbody is given by the Stefan-Boltzmann Law 
%         \begin{equation}
%             \dot{Q}_\text{emit,max} = \sigma A T_s^4
%         \end{equation}
%         where $\sigma$ is the Boltzmann constant $\sigma = 5.67 \times 10^{-8} \frac{W}{m^2 K^4}$
%         \item For an ideal body, we have
%         \begin{equation}
%             \dot{Q}_\text{emit} = \epsilon \sigma AT_s^4
%         \end{equation}
%         where $0 \le \epsilon \le 1$ is the emissivity.
%         \item When radiation is incident on a surface, some will be absorbed and some reflected. The \textbf{absorptivity} $\alpha$ is defined such that
%         \begin{align}
%             \dot{Q}_\text{absorbed} &= \alpha \dot{Q}_\text{incident} \\ 
%             \dot{Q}_\text{reflected} &= (1-\alpha)\dot{Q}_\text{incident}
%         \end{align}
%         \item Kirchoff's Law says that
%         \begin{equation}
%             \alpha = \epsilon
%         \end{equation}
%         \item For a small surface completed surrounded by a much larger surface net radiation is 
%         \begin{equation}
%             \dot{Q}_\text{net} = \epsilon \sigma A (T_s^4 - T_\text{surrounding}^4)
%         \end{equation}
%         \item Natural convection tells us the heat transfer coefficient is
%         \begin{equation}
%             h = c(T_S-T_\infty)^{1/4}
%         \end{equation}
%         where $c=4.2 \text{W}{m^2 K^{5/4}}$ and 
%         \begin{equation}
%             q_\text{conv}  = hA(T_S-T_\infty)
%         \end{equation}
%         \item Forced convection gives a constant $h = 250 W/m^2 K$
%         \item Let's look at the \textbf{one-dimensional case:} If we look at a segment of length $\Delta x$, the rate of the increase of enthalpy is
%         \begin{align}
%             \dot{H} &= mc_p \frac{\partial T}{\partial t} \\ 
%             &= \rho c_p A \Delta x \frac{\partial T}{\partial t}
%         \end{align}
%         The energy balance in this small segment gives
%         \begin{align}
%             \rho c_p A\Delta x\frac{\partial T}{\partial t} &= \dot{Q}_x - \dot{Q}_{x+\Delta x} \\ 
%             &= (\dot{q}A)_x - (\dot{q}A)_{x+\Delta x}
%         \end{align}
%         Dividing by $\Delta x$ and taking the limit gives 
%         \begin{equation}
%             \rho c_p \frac{\partial T}{\partial t} = -\frac{1}{A} \frac{\partial(\dot{q}A)}{\partial x}
%         \end{equation}
%         Note that $A$ depends on the coordinate system, $\dot{q}$ depends on Fourier's Law $\dot{q} = -k\frac{dT}{dx}$.
%         \item This gives us the heat conduction equation 
%         \begin{equation}
%             \frac{\partial^2 T}{\partial x^2} = \frac{1}{\alpha}\frac{\partial T}{\partial t}  
%         \end{equation}
%         where $\alpha=\frac{k}{\rho c_p}$ is the thermal diffusivity.
%         \item At constant state, $\frac{\partial T}{\partial t} = 0$, so 
%         \begin{equation}
%             \frac{d^2 T}{dx^2} = 0
%         \end{equation}
%         \item In cylindrical coordinates, we have
%         \begin{equation}
%             \frac{1}{\alpha}\frac{\partial T}{\partial t} = \frac{1}{r}\frac{\partial}{\partial r}\left(r\frac{\partial T}{\partial r}\right)
%         \end{equation}
%         \item In spherical coordinates 
%         \begin{equation}
%             \frac{1}{\alpha}\frac{\partial T}{\partial t} = \frac{1}{r^2}\frac{\partial}{\partial r}\left(r^2 \frac{\partial T}{\partial r}\right)
%         \end{equation}
%         \item In general, we have 
%         \begin{equation}
%             \frac{1}{r^n}\frac{\partial}{\partial r}\left(r^n \frac{\partial T}{\partial r}\right) = \frac{1}{\alpha}\frac{\partial T}{\partial t}
%         \end{equation}
%         where 
%         \begin{itemize}
%             \item $n=0$ for cartesian
%             \item $n=1$ for cylindrical 
%             \item $n=2$ for spherical
%         \end{itemize}
%         \begin{idea}
%             These equations can alternatively derived by applying the divergence formula in different coordinate systems. In particular, it is done by considering the heat equation 
%             \begin{equation}
%                 \frac{1}{\alpha}\frac{\partial T}{\partial t} = \nabla \cdot T
%             \end{equation}
%             and applying spherical / cylindrical symmetry.
%         \end{idea}
%         \item To solve problems, we also need boundary conditions, and we often make the steady state assumption.
%     \end{itemize}


% \section{Thermal resistance Networks and Contact resistance}
% read Chapter 17.2, 17.3

% We can think of a wall with a temperature difference across the two sides as a \textbf{thermal resistor} with a thermal resistance of $r_1 = \frac{T_1 - T_2}{\dot{Q}}$.

% In this way, we can put thermal resistors in series and get the equivalent thermal resistance as:

% $r_{equivalent} = \sum_{i} {r_i}$

% What about the situation where there are three materials stacked up on top of each other bridging two different temperatures, each with their own cross-sectional area?

% If we call the amount of heat transfer in each of these $\dot{Q}_1$, $\dot{Q}_2$, $\dot{Q}_3$, then the total heat transfer is

% $\dot{Q} = \dot{Q}_1 + \dot{Q}_2 + \dot{Q}_3$

% $\dot{Q} = \frac{T_1 - T_2}{\dot{r}_1} + \frac{T_1 - T_2}{\dot{r}_2} + \frac{T_1 - T_2}{\dot{r}_3}$

% = $(T_1 - T_2)[\frac{1}{r_1} + \frac{1}{r_2} + \frac{1}{r_3}]$

% = $\frac{T_1 - T_2}{r_{total}}$

% So our electrical analogy fits thermal resistance completely. Now we can apply this analogy on networks of thermal resistance.

% % \makebox[\textwidth]{\includegraphics[width=\paperwidth]{images/Thermal - electrical network.png}}

% However, a surface in real life is never completely smooth so never has a perfect contact. Between two rough surfaces, air is trapped and affects the heat conduction.

% Heat flux $\dot{q}$ $[\frac{w}{m^2}]$

% Define a thermal contact resistance
% $r_c = \frac{\delta T}{\dot{q}}$ $[\frac{m^2 * C}{W}]$

% Which is the resistance per unit area.

% reciprocal of $r_c$ is the $h_c$, the "thermal contact conductance".

% $h = \frac{1}{r_c} = \frac{\dot{q}}{\delta T}$

% $\dot{q} = h_c \delta T$

% Typical values of $h_c$:

% \begin{center}
% \begin{tabular}{||c c||} 
%  \hline
%  Metal Pairs & $h_c$ ($\frac{w}{m^2 * C}$) \\ [0.5ex] 
%  \hline\hline
%  Steel/Steel & ~$10^3$ \\ 
%  \hline
%  Al/Al & ~$10^4$ \\
%  \hline
%  Cu/Cu & ~$10^5$ \\
%  \hline
% \end{tabular}
% \end{center}

% \section{Contact resistance in Cylinders and Spheres}


% % \makebox[\textwidth]{\includegraphics[width=\paperwidth]{images/Cylinder.png}}

% For a long pipe, $\frac{dT}{dZ} << \frac{dT}{dr}$.
% So we approximate by assuming 1-dimensional conduction in the Z direction.

% $\frac{\partial}{\partial r}(r\frac{\partial T}{\partial r}) = 0$

% Boundary conditions:\\
% $r = r_1, T = T_1$ \\
% $r = r_2, T = T_2$
% \\
% \\

% Integrate:

% $r \frac{dT}{dr} = c_1$\\

% $\rightarrow \frac{dT}{dr} = \frac{c_1}{r}$\\
% $T(r) = c_1 ln(r) + c_2$\\
% \\

% Applying boundary conditions:

% $ T_1 = c_1 ln(r_1) + c_2$\\
% $ T_2 = c_1 ln(r_2) + c_2$\\
% \\

% $\rightarrow T_1 - T_2 = c_1[ln(r_1) - ln(r_2)]$ \\
% $\rightarrow c_1 = \frac{T_1 - T_2}{ln(\frac{r_1}{r_2})}$ \\
% $T_2 = \frac{T_1 - T_2}{ln(\frac{r_1}{r_2})}ln(r_2) + c_2$ \\
% $c_2 = T_2 - \frac{(T_1 - T_2}{ln(\frac{r_1}{r_2})} ln(r_2)$ \\


% $T(r) = \frac{T_1 - T_2}{ln(\frac{r_1}{r_2})} ln(r) - \frac{T_1 - T_2}{ln(\frac{r_1}{r_2})} ln(r_2) + T_2$ \\

% $T(r) = \frac{T_1 - T_2}{ln(\frac{r_1}{r_2})} ln(\frac{r}{r_2}) + T_2$ \\

% $\dot{Q}_{conduction} = -kA_1 \frac{dT}{dr}|_{(r = r_2)}$ \\
% $A_1 = 2\pi r_1 L$

% $\frac{dT}{dr}|_{(r = r_2)} = \frac{T_1 - T_2}{ln(\frac{r_1}{r_2})} ln(\frac{r_2}{r} - \frac{1}{r_2})|_{r = r_1}$

% Finally:
% $\dot{Q}_{conduction} = 2\pi L k \frac{T_1 - T_2}{ln(\frac{r_1}{r_2})})$

% Define thermal resistance of a cylinder:
% $r_{cylinder} = \frac{T_1 - T_2}{\dot{Q}_{conduction}}$
% $r_{cylinder} = \frac{ln(r_2/r_1)}{2\pi Lk}$

% For a sphere:
% $r_{sphere} = \frac{r_2 - r_1}{4\pi r_1 r_2 k}$

% \section{Conduction in Cylinders and Spheres}
% Let $h_1$ be the thermal contact conductance between the outer surface of the cylinder and the outer surface of the sphere. 
\section{Lecture 1}
\begin{idea}
    The three basic principles of engineering are:
    \begin{itemize}
        \item $F=ma$
        \item You can't push on a rope.
        \item A necessary condition for solving any given engineering problem is to know the answer before starting.
    \end{itemize}
\end{idea}
\end{document}