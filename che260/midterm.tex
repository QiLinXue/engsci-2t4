\documentclass{article}
\usepackage{qilin}
\tikzstyle{process} = [rectangle, rounded corners, minimum width=1.5cm, minimum height=0.5cm,align=center, draw=black, fill=gray!30, auto]
\title{CHE260: Thermodynamics Midterm}
\author{QiLin Xue}
\date{Fall 2021}
\usepackage{mathrsfs}
\usetikzlibrary{arrows}
\begin{document}

\maketitle
\section{Basic Thermodynamics}
\begin{itemize}
    \item The ideal gas law 
    \begin{equation}
        PV = mRT
    \end{equation}
    \item The first law of thermodynamics applies to \textbf{closed systems only}:
    \begin{equation}
        U = Q + W
    \end{equation}
    \item For an ideal gas,
    \begin{align}
        c_p &= c_v + R \\ 
        \Delta u &= c_v\Delta T \\ 
        \Delta h &= c_p\Delta T 
    \end{align}
    \item For a liquid or solid 
    \begin{equation}
        \Delta h = c\Delta T + v\Delta P
    \end{equation}
    \item For a control volume 
    \begin{align}
        \dot{m} &= \rho A V \\ 
        \dot{Q} + \dot{W} &= \dot{m}
    \end{align}
    and we get the energy rate balance 
    \begin{align}
        \left(\Delta h + \frac{V_2^2-V_1^2}{2}+g\Delta z\right)
    \end{align}
    \item At steady state, we have
    \begin{equation}
        \dot{m}_\text{in} = \dot{m}_\text{out}
    \end{equation}
    \item The work done in a polytropic process $(PV)^n=\text{constant}$ is 
    \begin{equation}
        W_{12} = \begin{cases}
            P_1V_1\ln\frac{V_1}{V_2} & n=1 \\ \\
            \frac{P_2V_2-P_1V_1}{n-1} & \neq 1
        \end{cases}
    \end{equation}
\end{itemize}
\section{Entropy}
\begin{itemize}
    \item Entropy can be defined as 
    \begin{equation}
        \dd{S} = \frac{\dd{Q}}{T} + \dd{S}_\text{gen}
    \end{equation}
    \item Entropy can also be defined statistically to be 
    \begin{eqnarray}
        S = k_B \ln \Omega 
    \end{eqnarray}
    where $k_B$ is Boltzmann's constant and $\Omega$ is the number of microstates.
    \item The change in entropy of an ideal gas with constant specific is given by (all three are equivalent): 
    \begin{align}
        \Delta s &= c_v\ln\frac{T_2}{T_1} +R\ln \frac{v_1}{v_2} \\ 
        \Delta s &= c_v \ln\frac{P_2}{P_1} + c_p\ln\frac{v_2}{v_1} \\ 
        \Delta s &= c_p \ln\frac{T_2}{T_1} - R\ln\frac{P_2}{P_1} \\
        \Delta s &= s_2^\circ - s_1^\circ - R\ln\frac{P_2}{P_1}
    \end{align}
    where $s_2^\circ,s_1^\circ$ are standard specific enthalpy values that can be searched up using a table.
    \item For an adiabatic and isentropic process, the quantity $Pv^\gamma$ is constant.
    \item We can perform an entropy rate balance. We have, 
    \begin{align}
        \frac{dS}{dt} &= \dot{S}_\text{in} - \dot{S}_\text{out} = \dot{S}_\text{gen} \\ 
        \frac{dS}{dt} &= \sum \frac{\dot{Q}}{T} + \sum \dot{m}_\text{in}s_\text{in} - \sum \dot{m}_\text{out}s_\text{out} + \dot{S}_\text{gen}.
    \end{align}
    At steady state, $\frac{dS}{dt} =0 $.
    \item We can write Gibb's Equation in the following forms 
    \begin{align}
        T\dd{s}&=\dd{u} + P\dd{v} \\ 
        T\dd{s}&=\dd{h} - v\dd{P}
    \end{align}
    \item Turbine efficiency: 
    \begin{equation}
        \eta_\text{turbine} = \frac{h_2-h_1}{h_{2s}-h_1}
    \end{equation}
    and compressor efficiency:
    \begin{equation}
        \eta_\text{compressor} = \frac{h_{2s}-h_1}{h_{s}-h_1}
    \end{equation}
    Remember that turbines create energy so they want to maximize work. Compressors focus on compression using the least amount of work.
    \item For an isentropic and incompressible process (isothermal), we get Bernoulli's equation: 
    \begin{equation}
        \frac{P}{\rho}+\frac{v^2}{2}+gz = \text{constant}
    \end{equation}
\end{itemize}
\section{Phase Change}
\begin{itemize}
    \item The Clapeyron Equation gives 
    \begin{equation}
        \frac{dP}{dT} = \frac{h_{fg}}{T(v_g-v_f)}
    \end{equation}
    where $h_{fg} \equiv h_g-h_f$. Assuming an ideal gas, we get the Clausius-Clapeyron equation: 
    \begin{equation}
        \frac{dP}{dT} = \frac{h_{fg}P}{RT^2}
    \end{equation}
    \item For phase equilibrium, Gibb's equation becomes 
    \begin{equation}
        s_g-s_f=\frac{h_g-h_f}{T}
    \end{equation}
    since pressure and temperature are constant for a system in equilibrium.
    \item The saturation pressure can be written as 
    \begin{equation}
        P_\text{sat} = C\exp\left(-\frac{h_{fg}}{RT_\text{sat}}\right)
    \end{equation}
    \item The quality of a mixture is defined aS 
    \begin{equation}
        x = \frac{\text{mass of vapour}}{\text{mass of mixture}} = \frac{m_g}{m}
    \end{equation}
    \item Suppose $\xi$ is an intensive quantity. The intensive quantity $\xi$ of a mixture is 
    \begin{equation}
        \xi = \xi_f + x(\xi_g-\xi_f) 
    \end{equation}
\end{itemize}
\section{Heat Engines}
\begin{itemize}
    \item The thermal efficiency of a heat engine is 
    \begin{equation}
        \eta_\text{th} = \frac{W_\text{net}}{Q_\text{in}}
    \end{equation}
    \item The thermal efficiency of a carnot engine is 
    \begin{equation}
        \eta_\text{th,carnot} = 1-\frac{Q_C}{Q_H}-1-\frac{T_C}{T_H}
    \end{equation}
    \item The coefficient of performance of a refrigerator is 
    \begin{equation}
        COP_R = \frac{Q_C}{W_\text{net}} = \frac{1}{T_H/T_C-1}
    \end{equation}
    and the coefficient of performance of a heat pump is 
    \begin{equation}
        COP_{HP} = \frac{Q_H}{W_\text{net}} = \frac{1}{1-T_C/T_H}
    \end{equation}
    \item For cyclic cycles, $\Delta U=0,$ so the work that an engine does is $W=Q_h-Q_c$.
\end{itemize}
\end{document}