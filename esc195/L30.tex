\section{Directional Derivatives and the Gradient Vector}
\begin{itemize}
    \item We can attempt to define differentiability if the directional derivative exists:
    \begin{equation}
        \lim_{\vec{h}\to\vec{0}}  \frac{f(\vec{x}_0+\vec{h})-f(\vec{x}_0)}{\vec{h}} = \lim_{\vec{h}\to\vec{0}} \frac{f(x_0+h_1, y_0+h+2, z_0+h_3) - f(x_0, y_0, z_0)}{\vec{h}}
    \end{equation}
    However, we cannot simply divide two vectors. We also can't let the denominator be $\lVert \vec{h} \rVert$ because we then lose information about the direction. Therefore, we need to re-define the derivative:
    \begin{definition}
        If $\lim_{h\to 0} \frac{g(h)}{|h|} = 0$, then we can say that $g(h) = o(h)$ where $o(h)$ is ```little-o''
    \end{definition}
    \item We can start with the one-dimensional case, and write:
    \begin{align}
        f'(x) &= \lim_{h\to 0} \frac{f(x+h)-f(x)}{h} \\
        &\implies \lim_{h\to 0}\frac{f(x+h)-f(x)-f'(x)h}{h} = 0 
    \end{align}
    Therefore:
    \begin{equation}
        [f(x+h)-f(x)]-f'(x)h = o(h)
    \end{equation}
    We want to look for terms that approach zero faster than $h\to 0$, so we can let them be $o(h)$ and leave them out of the final definition.
    \begin{example}
        Let $f(x)=x^2$. Then: $f(x+h)-f(x)=(x+h)^2-x^2=2xh+h^2$. Note that:
        \begin{equation}
            \lim_{h\to 0} \frac{h^2}{h} = \lim_{h\to 0}h = 0
        \end{equation}
        so we can say that $h^2$ is $o(h)$, so from our re-definition, we look for the term:
        \begin{equation}
            \text{(term)} \cdot h
        \end{equation}
        where the term gives the derivative. Therefore, $f'(x)=2x$.
    \end{example}
    \begin{idea}
        The overall idea is to write the derivative using only the numerator:
        \begin{equation}
            f(x+h)-f(x)
        \end{equation}
        and looking for only the \textit{important} terms and leaving out the unimportant terms. We can define the unimportant terms to be any terms that are $o(h)$. Once we have the important terms, they should (at least in the $1d$ case), be in the form of:
        \begin{equation}
            \text{(derivative)} \cdot h
        \end{equation}
        and we can read out the derivative without having to divide.
    \end{idea}
\end{itemize}