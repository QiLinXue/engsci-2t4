\section{Directional Derivatives and the Gradient Vector}
\begin{itemize}
    \item We can attempt to define differentiability if the directional derivative exists:
    \begin{equation}
        \lim_{\vec{h}\to\vec{0}}  \frac{f(\vec{x}_0+\vec{h})-f(\vec{x}_0)}{\vec{h}} = \lim_{\vec{h}\to\vec{0}} \frac{f(x_0+h_1, y_0+h+2, z_0+h_3) - f(x_0, y_0, z_0)}{\vec{h}}
    \end{equation}
    However, we cannot simply divide two vectors. We also can't let the denominator be $\lVert \vec{h} \rVert$ because we then lose information about the direction. Therefore, we need to re-define the derivative:
    \begin{definition}
        If $\lim_{h\to 0} \frac{g(h)}{|h|} = 0$, then we can say that $g(h) = o(h)$ where $o(h)$ is ```little-o''
    \end{definition}
    \item We can start with the one-dimensional case, and write:
    \begin{align}
        f'(x) &= \lim_{h\to 0} \frac{f(x+h)-f(x)}{h} \\
        &\implies \lim_{h\to 0}\frac{f(x+h)-f(x)-f'(x)h}{h} = 0 
    \end{align}
    Therefore:
    \begin{equation}
        [f(x+h)-f(x)]-f'(x)h = o(h)
    \end{equation}
    We want to look for terms that approach zero faster than $h\to 0$, so we can let them be $o(h)$ and leave them out of the final definition.
    \begin{example}
        Let $f(x)=x^2$. Then: $f(x+h)-f(x)=(x+h)^2-x^2=2xh+h^2$. Note that:
        \begin{equation}
            \lim_{h\to 0} \frac{h^2}{h} = \lim_{h\to 0}h = 0
        \end{equation}
        so we can say that $h^2$ is $o(h)$, so from our re-definition, we look for the term:
        \begin{equation}
            \text{(term)} \cdot h
        \end{equation}
        where the term gives the derivative. Therefore, $f'(x)=2x$.
    \end{example}
    \begin{idea}
        The overall idea is to write the derivative using only the numerator:
        \begin{equation}
            f(x+h)-f(x)
        \end{equation}
        and looking for only the \textit{important} terms and leaving out the unimportant terms. We can define the unimportant terms to be any terms that are $o(h)$. Once we have the important terms, they should (at least in the $1d$ case), be in the form of:
        \begin{equation}
            \text{(derivative)} \cdot h
        \end{equation}
        and we can read out the derivative without having to divide.
    \end{idea}
    \begin{definition}
        We say that $f$ is differentiable at $\vec{x}$ if and only if there exists a vector $\vec{y}$ such that: $f(\vec{x}+\vec{h})-f(\vec{x})=\vec{y}\cdot\vec{h} + o(\vec{h})$ where:
        \begin{equation}
            \vec{y} = \nabla f(\vec{x})
        \end{equation}
        is known as the \textbf{gradient} of $f$.
    \end{definition}
    \begin{example}
        For example, let $f(x,y)=x+y^2$ and let $\vec{h}=(h_1,h_2)$. Then:
        \begin{align}
            f(\vec{x}+\vec{h})-f(\vec{x}) &= f(x+h, y+h)-f(x,y) \\ 
            &= x+h_1 + (y+h_2)^2 - x-y^2 \\ 
            &= h_1 + 2yh_2 + h_2^2 \\ 
            &= (1\hat{i} + 2y\hat{j})\cdot \vec{h}+h_2^2
        \end{align}
        To finish, we need to demonstrate the the excess term $h_2^2=o(h)$. To do this, let $g(\vec{h})=h_2^2=(h_2 \hat{j}) \cdot (h_1 \hat{i} + h_2\hat{j})$. We write it as a dot product because we wish to divide by the magnitude later. Therefore, we can write that $g(\vec{h}) = h_2\hat{j} \cdot \vec{h}$ and:
        \begin{align}
            \frac{|g(\vec{h})|}{\lVert \vec{h}\rVert} &= \frac{\lVert h_2\hat{j}\rVert \lVert \vec{h}\rVert \cos\theta}{\lVert \vec{h} \rVert} \\ 
            &\le \frac{\lVert h_2\hat{j}\rVert \lVert \vec{h}\rVert}{\lVert \vec{h} \rVert} \\ 
            &=|h_2|
        \end{align}
        We know that $h_2 \to 0$ as $\vec{h}\to\vec{0}$ so $g(\vec{h})=h_2^2$ is $o(\vec{h})$ and we can claim that:
        \begin{equation}
            \nabla f(\vec{x}) = 1\hat{i}+2y\hat{j}
        \end{equation}
    \end{example}
    \begin{example}
        Let $f(x,y,z)=xyz$ and let $\vec{h}=(h_1,h_2,h_3)$. Then:
        \begin{align}
            f(\vec{x}+\vec{h})-f(\vec{x}) &= (x+h_1)(y+h_2)(z+h_3)-xyz \\ 
            &= xyz + xyh_3+xh_2z+xh_2h_3+h_1yz+h_1yh_3+h_1h_2z+h_1h_2h_3-xyz \\ 
            &= (yz \hat{i} + xz\hat{j} + xy\hat{k}) \cdot \vec{h} + xh_2h_3 + yh_1h_3+zh_1h_2+h_1h_2h_3
        \end{align}
        Without loss of generality, we show that the first remainder term is $o(h)$. Consider $g(\vec{h})=xh_2h_3 = (xh_2 \hat{j})\cdot \vec{h}$. Therefore:
        \begin{equation}
            \frac{|g(\vec{h})|}{\lVert \vec{h}\rVert} = \frac{|x||h_2|\lVert \vec{h}\rVert \cos\theta}{\lVert \vec{h}\rVert} \le |xh_2|
        \end{equation}
        Now taking the limit:
        \begin{equation}
            \lim_{\vec{h}\to \vec{0}}|xh_2| = 0
        \end{equation}
        so $xh_2h_3$ is $o(\vec{h})$, and so are the other terms. Therefore, we have:
        \begin{equation}
            \nabla f(\vec{x}) = yz\hat{i}+xz\hat{j}+xy\hat{k}
        \end{equation}
        Note the symmetry.
    \end{example}
    \begin{theorem}
        In cartesian coordinates:
        \begin{equation}
            \nabla f(x,y) = \frac{\partial f}{\partial x}\hat{i} + \frac{\partial f}{\partial y}\hat{j}
        \end{equation}
        or equivalently written as:
        \begin{equation}
            \nabla f(\vec{x})=f_x\hat{i} + f_y\hat{j}+f_z\hat{i}
        \end{equation}
    \end{theorem}
    \item Note that $\vec{x}$ is a vector, $f(\vec{x})$ is not a vector, but $\nabla f(\vec{x})$ is a vector.
    \begin{example}
        Let $f(\vec{x})=xy^2z^3$. We have:
        \begin{align}
            \frac{\partial f}{\partial x} &= y^2z^3 \\
            \frac{\partial f}{\partial y} &= 2xyz^3 \\ 
            \frac{\partial f}{\partial z} &= 3xy^2z^2
        \end{align}
        so the gradient is:
        \begin{equation}
            \nabla f = (y^2z^3, 2xy^3, 3xy^2z^2)
        \end{equation}
    \end{example}
    \begin{example}
        Suppose we have a vector function $\vec{r}=x\hat{i} + y\hat{j} + z\hat{k}$. The magnitude is: $$r(x,y,z)=\sqrt{x^2+y^2+z^2}$$ and taking the gradient of this yields:
        \begin{equation}
            \nabla r = \frac{\vec{r}}{r}
        \end{equation}
    \end{example}
    \item For a directional derivative, we can define it using:
    \begin{align}
        \frac{\partial f}{\partial x} &= \lim_{h\to 0}\frac{f(\vec{x}_0+h\vec{i})-f(\vec{x}_0)}{h} \\ 
    \end{align}
    and similar for the other partial derivatives.
    \begin{definition}
        The directional derivative of $f$ at $\vec{}_0$ in the direction $\hat{u}$ is given by:
        \begin{equation}
            f_{\hat{u}}(\vec{x}_0) = \lim_{h\to 0} \frac{f(\vec{x}_0+h\hat{u})-f(\vec{x}_0)}{h}
        \end{equation}
    \end{definition}
    \item The directional derivative can also be expressed as:
    \begin{equation}
        f_{\hat{u}}(\vec{x}_0) = \nabla f(\vec{x}_0)\cdot \hat{u}
    \end{equation}
    \begin{proof}
        We can write:
        \begin{equation}
            f(\vec{x}+\vec{h})-f(\vec{x})-\nabla f(\vec{x})\cdot\vec{h} + o(\vec{h})
        \end{equation}
        here: $\vec{h}=h\hat{u}$. Using this, the above is also equal to:
        \begin{equation}
            \nabla f(\vec{x})\cdot h\hat{u}+o(\vec{h})
        \end{equation}
        and therefore:
        \begin{equation}
            \frac{f(\vec{x}+h\hat{u})-f(\vec{x})}{h} = \nabla f \cdot \hat{u} + \frac{o(\vec{h})}{h}
        \end{equation}
        and taking the limit as $h\to 0$, we have the desired result.
    \end{proof}
    \begin{example}
        Suppose we have some temperature gradient $T(x,y)$ where $\frac{\partial T}{\partial y}=4\si{\degree\celsius\per\meter}$ and $\frac{\partial T}{\partial x} = 3\si{\degree\celsius\per\meter}$. Suppose we move in the direction $\hat{u}=\cos\theta\hat{i} + \sin\theta\hat{j}$. Then:
        \begin{equation}
            T_{\hat{u}} = \left(\frac{\partial T}{\partial x}\hat{i} + \frac{\partial T}{\partial y}\hat{j}\right)(\cos\theta\hat{i}+\sin\theta\hat{j}) = 3\cos\theta + 4\sin\theta
        \end{equation}
    \end{example}
    \begin{example}
        Suppose we have a parabolic hill described by $z(x,y)=20-x^2-y^2$ and we move straight up, or $\hat{u}=(0,-1)$. Then:
        \begin{align}
            \frac{\partial f}{\partial x} &= -2x \\ 
            \frac{\partial f}{\partial y} &= -2y
        \end{align}
        so:
        \begin{equation}
            z_{\hat{u}}=(-2x,-2y)\cdot (0,-1) = 2y
        \end{equation}
    \end{example}
    \item Note that:
    \begin{align}
        |f_{\hat{k}}(\vec{x})| &= |\nabla f \cdot \hat{u} | \\ 
        &= \lVert \nabla f\rVert \lVert \hat{u} \rVert |\cos\theta| \\ 
        &\le \lVert \nabla f \rVert
    \end{align}
    \begin{example}
        Suppose that $z=f(x,y) = A + x + 2y - x^2 -3y^2$ and we wish to find the steepest path down starting from $(0,0,A)$. We know that:
        \begin{align}
            \frac{\partial f}{\partial x} &= 1-2x \\ 
            \frac{\partial f}{\partial y} &= 2- 6y
        \end{align}
        such that:
        \begin{equation}
            \nabla f = (1-2x)\hat{i} + (2-6y)\hat{j} \implies -\nabla f = (2x-1)\hat{i}+(6y-2)\hat{j}
        \end{equation}
        The curve is given by:
        \begin{equation}
            \vec{r}(t) = x(t) \hat{i} + y(t)\hat{j}
        \end{equation}
        where $x'(t) = 2x(t)-1$ and $y'(t) = 6y(t)-2$. This is in parametric form and we can convert to cartesian form by writing the derivative as:
        \begin{equation}
            \frac{dy}{dx}=\frac{6y-2}{2x-1}
        \end{equation}
        and solving this differential equation to get:
        \begin{equation}
            3y=(2x-1)^3+1
        \end{equation}
    \end{example}
\end{itemize}