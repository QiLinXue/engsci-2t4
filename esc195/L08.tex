\section{Applications to Physics and Engineering}
\begin{itemize}
    \item The \textbf{hydrostatic pressure} depends on the density $\rho$, gravitational constant $g$ and the depth $d$:
    \begin{equation}
        p = \rho g d
    \end{equation}
    and the force of pressure acting on the surface is:
    \begin{equation}
        F = \rho g d \cdot A = pA
    \end{equation}
    \begin{figure}[ht]
        \centering
        \incfig{L8a}
    \end{figure}
    \begin{example}
        Suppose we have a curved container. The force acting on the entire container can be broken up into segments, each with a force of:
        \begin{equation}
            F_i = \underbrace{w(x_i^*)\Delta x_i}_\text{area} \cdot \underbrace{\rho gx_i^*}_\text{pressure}
        \end{equation}
        where $w(x)$ is the width of the container as a function of height. The force exerted on the container is thus:
        \begin{equation}
            F = \int_a^b \rho g x w(x) \dd{x}
        \end{equation}
    \end{example}
    \begin{example}
        Suppose we have a pipe half with a radius of $1\si{\meter}$ filled with water and we wish to find the force it exerts on the end face of the pipe. We can do this via:
        \begin{align}
            F &= \int_0^1 \rho gx 2\sqrt{1-x^2}\dd{x} \\ 
            &= 2\rho g\left(-\frac{1}{3}(1-x^2)^{3/2}\right)\Biggr|^1_0 \\ 
            &= \frac{2}{3}\rho g = 6533 \si{\newton}
        \end{align}
    \end{example}

\end{itemize}