\section{Tangent Planes and Linear Approximations}
\begin{itemize}
    \item To set up our problem, suppose we have a three dimensional surface $z(x,y)=20-x^2-y^2$ and we want to find a \textit{level curve} such that the altitude does not change. If we start at $(1,2)$, then we can solve for the height $C$ to be:
    \begin{equation}
        C = 20-1^2-2^2 = 15
    \end{equation}
    and the equation of the level curve would be:
    \begin{equation}
        x^2 + y^2 = 5
    \end{equation}
    \item The radius vector is given by $\vec{r}=(1,2)$ and the tangent vector is $\vec{t}$. Since they are perpendicular:
    \begin{equation}
        \vec{t} \cdot \vec{r} = 0 \implies t_11+t_22 = 0
    \end{equation}
    If we choose $t_1=2$ and $t_2=-$, then $\vec{t} = (2,-1)$ is the tangent vector. Note that the gradient is actually perpendicular to the tangent vector:
    \begin{equation}
        \nabla f \cdot \vec{t} = (-2x_1, - 2y) \cdot (2,-1) = -4x+2y
    \end{equation}
    and at $(1,2)$, the dot product is zero.
    \begin{theorem}
        The gradient of a curve will be perpendicular to the tangent of the level curve at a specific point.
        \begin{proof}
            Let $\vec{r}(t)=x(t)\hat{i}+y(t)\hat{j}$. We know that $\vec{t}=\vec{r}'(t)$. We can write:
            \begin{equation}
                f(\vec{r}(t)) = C
            \end{equation}
            and applying chain rule, we have:
            \begin{align}
                \frac{d}{dt} f(\vec{r}(t)) &= \nabla f(\vec{r})\cdot \vec{r}' \\ 
                &= \frac{dC}{dt} = 0
            \end{align}
            Therefore, $\nabla f(\vec{r}) \cdot \vec{r} = 0$ so the two are perpendicular. Note that this means this property is true for any curve that can be expressed as $f(x,y)=C$.
        \end{proof}
    \end{theorem}
    \item Let $\vec{t} = \left(\frac{\partial f}{\partial y}, -\frac{\partial f}{\partial x}\right)$ which gives:
    \begin{equation}
        \nabla f \cdot \vec{t} = \frac{\partial f}{\partial x} \frac{\partial f}{\partial y} - \frac{\partial f}{\partial x}\frac{\partial f}{\partial y} = 0
    \end{equation}
    We can use this to determine the equations of tangent and normal lines.
    \item If $(x,y)$ is a point on the tangent line at a certain point $(x_0,y_0)$, then:
    \begin{align}
        (x-x_0, y-y_0) \cdot \nabla f &= 0 \\ 
        (x-x_0)\frac{\partial f}{\partial x}(x_0,y_0) + (y-y_0)\frac{\partial f}{\partial y}(x_0, y_0) &= 0
    \end{align}
    \item For the normal line, let $(x,y)$ be a point on the normal line. Then:
    \begin{equation}
        (x-x_0, y-y_0)\cdot \vec{t} = 0
    \end{equation}
    and applying the same equation, we get:
    \begin{equation}
        (x-x_0)\frac{\partial f}{\partial y} (x_0,y_0) - (y-y_0)\frac{\partial f}{\partial x}(x_0,y_0)=0
    \end{equation}
    \item We can also find the equation for the normal line to a curve $f(x,y)=C$ at $(x_0,y_0)$. We start with the two-dimensional case:
    \begin{example}
        Suppose we have a curve $x^2+y^2=9$. Then we have:
        \begin{align}
            \frac{\partial f}{\partial x} &= 2x \\ 
            \frac{\partial f}{\partial y} &= 2y
        \end{align}
        The tangent line is given by:
        \begin{equation}
            (x-x_0)2x_0 + (y-y_0)2y_0 = 0
        \end{equation}
        Suppose that $(x_0,y_0)=\left(\frac{3}{\sqrt{2}}, \frac{3}{\sqrt{2}}\right)$. Then the tangent line is given by:
        \begin{equation}
            \left(x-\frac{3}{\sqrt{2}}\right)2\frac{3}{\sqrt{2}}+\left(y-\frac{3}{\sqrt{2}}\right)2\frac{3}{\sqrt{2}} = 0
        \end{equation}
        this gives:
        \begin{equation}
            x+y=\frac{6}{\sqrt{2}}
        \end{equation}
        Similarly, the normal line is given by:
        \begin{equation}
            \left(x-\frac{3}{\sqrt{2}}\right)2\frac{3}{\sqrt{2}}-\left(y-\frac{3}{\sqrt{2}}\right)2\frac{3}{\sqrt{2}} = 0 \implies y=x
        \end{equation}
    \end{example}
    \item Extending it to three dimensions. A level surface is given by: $f(x,y,z)=C$. 
    \item We know that the gradient $\nabla f(x_0,y_0,z_0)$ is perpendicular to the level surface curve $f(x,y,z)=C$.
    \begin{proof}
        Let $F(t)=x(t)\hat{i}+y(t)\hat{j}+z(t)\hat{k}$ be any curve on the surface such that:
    \begin{equation}
        f(x(t), y(t), z(t)) = f(\vec{r}(t)) = C
    \end{equation}
    Again:
    \begin{equation}
        \frac{d}{dt} f(\vec{r}(t)) = \frac{dc}{dt}=0
    \end{equation}
    Applying the chain rule, we get:
    \begin{equation}
        \nabla f(\vec{r}(t)) \cdot \vec{r'}(t)=0
    \end{equation}
    \end{proof}
    \begin{example}
        Suppose we have a sphere $x^2+y^2+z^2=25$. The gradient is then:
        \begin{equation}
            \nabla f = (2x,2y,2z)=2(x,y,z)
        \end{equation}
        The equation of the tangent plane to surface is:
        \begin{equation}
            (\vec{x}-\vec{x}_0)\cdot \nabla f = 0
        \end{equation}
        At $(0,5,0)$, we have:
        \begin{equation}
            (x-x_0)2x_0 + (y-y_0)2y_0+(z-z_0)2z_0 = 0 \implies y-5\cdot 10=0 \implies y=5
        \end{equation}
        The normal line is given by:
        \begin{equation}
            \vec{r}(q)=\vec{x}_0+q\nabla f(\vec{x}_0)
        \end{equation}
        and the equations for the normal line are:
        \begin{align}
            x &= x_0 + qf_x \\ 
            y &= y_0 + qf_y \\ 
            z &= z_0 + qf_z
        \end{align}
    \end{example}
    \begin{example}
        Find the normal line at $xy^2+2z^2=12$ at $(1,2,2)$. We have:
        \begin{align}
            f_x &= y^2 = 4 \\ 
            f_y &= 2xy = 4 \\ 
            f_z &= 4z = 8
        \end{align}
        and so:
        \begin{align}
            x &= 1 + 4a \\ 
            y &= 2+4q \\ 
            z &= 2+8q
        \end{align}
    \end{example}
    \begin{example}
        Suppose we have an offset sphere given by:
        \begin{equation}
            f = x^2+y^2+z^2-8x-8y-6z+24=0
        \end{equation}
        and an ellipsoid centered at the origin:
        \begin{equation}
            g = x^2+3y^2+2z^2=9
        \end{equation}
        and specifically at the point $(2,1,1)$. Let us attempt to show that the sphere is tangent to the ellipsoid at this specific point. We do this via:
        \begin{equation}
            \nabla f = (2x-8, 2y-8, 2z-6) \implies \nabla f(2,1,1)=(-4,-6,-4)
        \end{equation}
        and:
        \begin{equation}
            \nabla g = (2x,6y,4z) \implies \nabla f(2,1,1)=-\nabla f
        \end{equation}
        The gradient is the same at that point and we can verify that they also touch at that point by showing $g(2,1,1)=f(2,1,1)$. 
    \end{example}
    \begin{example}
        Suppose we have another sphere and ellipsoid:
        \begin{align}
            f &= x^2+y^2+z^2-4y-2z+2 = 0 \\ 
            g &= 3x^2+2y^2-2z = 1
        \end{align}
        at $P(1,1,2)$. To show that they are perpendicular at this point, we take their gradient vector and show that they are perpendicular (i.e. dot product is zero)
        \begin{align}
            \nabla f(1,1,2) &= (2,-2,2) \\
            \nabla g(1,1,2) &= (6,4,-2)
        \end{align}
        and the dot product is $12-8-4=0$ so they are perpendicular.
    \end{example}
    \begin{example}
        Suppose we have a curve and an ellipsoid:
        \begin{align}
            \vec{r}(t) &= (\frac{3}{2}(t^2+1), t^4+1, t^3) \\ 
            x^2+2y^2+3z^2 &= 20
        \end{align}
        at $(3,2,1)$. Notice that they intersect at $(3,2,1)$. The gradient of the ellipsoid is:
        \begin{equation}
            \nabla f(3,2,1) = (6,8,6)
        \end{equation}
        and the tangent vector of the curve is: $\vec{r}(1)=(3,4,3) = \frac{1}{2}\nabla f$. The tangent to the curve is parallel to the gradient, so the normal vector must be perpendicular to the surface. 
    \end{example}
    \begin{example}
        Suppose someone's head is described by the ellipsoid $x^2+y^2+2z^2=7$, and there is a bee at $P(1,2,1)$. There is a bee-swatter at $2x+3y+z=49$. The bee moves with a speed of $4$ along the normal. When does it hit the bee swatter?
        \vspace{2mm}

        The gradient is:
        \begin{equation}
            \nabla f=(2x,2y,4z) \implies \nabla f(1,2,1) = (2,4,4)
        \end{equation}
        and the normal line is:
        \begin{equation}
            (x,y,z)= (1+2q,2+4q,1+4q)
        \end{equation}
        such that:
        \begin{equation}
            \vec{r}(q) = (1+2q,2+4q,1+4q)
        \end{equation}
        We want: $\lVert \vec{r}'(t) \rVert = 4$. We can pick $q=\frac{2}{3}t$ to get this. Therefore:
        \begin{equation}
            \vec{r}(t) = \left(1+\frac{4}{3}t, 2+\frac{8}{3}t,1+\frac{8}{3}t\right)
        \end{equation}
        To find when it intersects with the plane $2x+3y+z=49$, we set:
        \begin{equation}
            2(1+4t/3)+3(2+8t/3)+(1+8t/3)=49 \implies t = 3
        \end{equation}
        So the location of the bee once it hits the bee-swatter at $t=3$ is $(5,10,9)$.
    \end{example}
\end{itemize}