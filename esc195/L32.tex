\section{Tangent Planes and Linear Approximations}
\begin{itemize}
    \item To set up our problem, suppose we have a three dimensional surface $z(x,y)=20-x^2-y^2$ and we want to find a \textit{level curve} such that the altitude does not change. If we start at $(1,2)$, then we can solve for the height $C$ to be:
    \begin{equation}
        C = 20-1^2-2^2 = 15
    \end{equation}
    and the equation of the level curve would be:
    \begin{equation}
        x^2 + y^2 = 5
    \end{equation}
    \item The radius vector is given by $\vec{r}=(1,2)$ and the tangent vector is $\vec{t}$. Since they are perpendicular:
    \begin{equation}
        \vec{t} \cdot \vec{r} = 0 \implies t_11+t_22 = 0
    \end{equation}
    If we choose $t_1=2$ and $t_2=-$, then $\vec{t} = (2,-1)$ is the tangent vector. Note that the gradient is actually perpendicular to the tangent vector:
    \begin{equation}
        \nabla f \cdot \vec{t} = (-2x_1, - 2y) \cdot (2,-1) = -4x+2y
    \end{equation}
    and at $(1,2)$, the dot product is zero.
    \begin{theorem}
        The gradient of a curve will be perpendicular to the tangent of the level curve at a specific point.
        \begin{proof}
            Let $\vec{r}(t)=x(t)\hat{i}+y(t)\hat{j}$. We know that $\vec{t}=\vec{r}'(t)$. We can write:
            \begin{equation}
                f(\vec{r}(t)) = C
            \end{equation}
            and applying chain rule, we have:
            \begin{align}
                \frac{d}{dt} f(\vec{r}(t)) &= \nabla f(\vec{r})\cdot \vec{r}' \\ 
                &= \frac{dC}{dt} = 0
            \end{align}
            Therefore, $\nabla f(\vec{r}) \cdot \vec{r} = 0$ so the two are perpendicular. Note that this means this property is true for any curve that can be expressed as $f(x,y)=C$.
        \end{proof}
    \end{theorem}
    \item Let $\vec{t} = \left(\frac{\partial f}{\partial y}, -\frac{\partial f}{\partial x}\right)$ which gives:
    \begin{equation}
        \nabla f \cdot \vec{t} = \frac{\partial f}{\partial x} \frac{\partial f}{\partial y} - \frac{\partial f}{\partial x}\frac{\partial f}{\partial y} = 0
    \end{equation}
    We can use this to determine the equations of tangent and normal lines.
    \item If $(x,y)$ is a point on the tangent line at a certain point $(x_0,y_0)$, then:
    \begin{align}
        (x-x_0, y-y_0) \cdot \nabla f &= 0 \\ 
        (x-x_0)\frac{\partial f}{\partial x}(x_0,y_0) + (y-y_0)\frac{\partial f}{\partial y}(x_0, y_0) &= 0
    \end{align}
    \item For the normal line, let $(x,y)$ be a point on the normal line. Then:
    \begin{equation}
        (x-x_0, y-y_0)\cdot \vec{t} = 0
    \end{equation}
    and applying the same equation, we get:
    \begin{equation}
        (x-x_0)\frac{\partial f}{\partial y} (x_0,y_0) - (y-y_0)\frac{\partial f}{\partial x}(x_0,y_0)=0
    \end{equation}
\end{itemize}