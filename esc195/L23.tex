\section{Cylinders and Quadratic Cylinders}
\begin{itemize}
    \item The simplest \textit{non-planar} function is in the form of:
    \begin{equation*}
        Ax^2+By^2+Cz^2 + Dxy + Exz + Fyz + Gx + Hy + Iz + J =0
    \end{equation*}
    whicha re known as \textbf{quadratic surfaces.} There are nine distinct types of surfaces that can be represented. A list of shapes can be found in the Stewart textbook.
    \item There are several properties that are worth investigating to understand a quadratic surface:
    \begin{itemize}
        \item Domain/Range
        \item Intercepts with coordinate axes
        \item Traces (intercepts with coordinates planes)
        \item Sections (Intersections with other planes)
        \item Center
        \item Symmetry
        \item Bounded / Unbounded
    \end{itemize}
    \begin{example}
        Suppose we look at the hyperboloid of two sheets:
        \begin{equation}
            \frac{x^2}{a^2}+\frac{y^2}{b^2}-\frac{z^2}{c^2}=-1 \implies z = \pm c\sqrt{1+\frac{x^2}{a^2}+\frac{y^2}{b^2}}
        \end{equation}
        This means the domain is $(x,y) \in (-\infty, \infty)$ and the range is $z \ge c$ or $z \le c$. We now look at intercepts and we realize that at $x=0$ and $y=0$, we have $z = \pm c$. We then look at the traces:
        \begin{itemize}
            \item For the $xy$ plane, we have nothing since $z \neq 0$.
            \item For the $xz$ plane, we have $y=0$ and we get a hyperbola:
            \begin{equation}
                z = \pm c\sqrt{1+\frac{x^2}{a^2}}
            \end{equation}
            \item For the $yz$ plane $(x=0$, we have another hyperbola)
            \begin{equation}
                z = \pm c\sqrt{1+\frac{y^2}{b^2}}
            \end{equation}
        \end{itemize}
        We then look at sections. We can start with the plane defined by $z=z_0$ with $|z_0|>c$. This gives:
        \begin{equation}
            \frac{a^2}{a^2}+\frac{y^2}{b^2}=\frac{z_0^2}{c^2}-1
        \end{equation}
        since $z_0^2>c^2$, the RHS is greater than zero so the section is an ellipse. The center is at the origin. No offset terms. There is also a lot of symmetry:
        \begin{align}
            x &\to -x \\ 
            y &\to -y \\ 
            z &\to -z
        \end{align}
        so it is symmetric about each coordinate axis and it is unbounded in all directions.
    \end{example}
    \item Another important concept is the \textbf{projection}. If we have two three dimensional currves and they intersect, then we can define:
    \begin{definition}
        A curve of intersection: $C=(x,y,z)$ is defined such that $z=f(x,y)$ and $z=g(x,y)$. Then :$f(x,y)=g(x,y)$.
    \end{definition}
    If we set $z=0$, then the curve of intersection is given by $(x,y,z=0)$. This is known as the projection and can be seen as the shadow of the points of intersection.
    \begin{example}
        Suppose we have a cone: $x^2+y^2=2z^2$. We can rearrange this to:
        \begin{equation}
            z=\pm\sqrt{(x^2+y^2)/2}
        \end{equation}
        We can look at the plane $y+4z=5$. We then have:
        \begin{equation}
            \frac{5-y}{4}=\sqrt{\frac{x^2+y^2}{2}}
        \end{equation}
        After solving, we get:
        \begin{equation}
            \frac{x^2}{25/7} + \frac{(y+5/7)^2}{200/49} = 1
        \end{equation}
        gives the projection onto the plane $z=0$.
    \end{example}
    \item Recall that a vector can be written like $\vec{r}=3\hat{i}+2\hat{j}+5\hat{k}$. However, instead of having constants, we can have functions instead:
    \begin{equation}
        \vec{f}(t) = f_1(t)\hat{i} + f_2(t)\hat{j} + f_3(t)\hat{k} = (f_1(t), f_2(t), f_3(t))
    \end{equation}
    which is known as a vector-valued function or a \textbf{vector function}
    \item One simple example is that of a straight line, where we have:
    \begin{equation}
        \vec{f}(t) = (f_1,f_2,f_3) = (a_1+b_1t,a_2+b_2t,a_3+b_3t)
    \end{equation}
    \item Suppose we have a three dimensional function $y=g(x)$. We can let $f_1(t)=t=x$. This gives $f_2(t)=g(t)$, and $f_3(t)=0$. This gives:
    \begin{equation}
        \vec{f}(t) = (t, g(t), 0)
    \end{equation}
    and the same concept can be extended to three dimensions.
    \begin{definition}
        Let the vector function $\vec{f}$ be defined on some interval $I$ containing the point $t_0$, except possibly at $t_0$ itself, and let $\vec{L}$ be a vector. Then:
        \begin{equation}
            \lim_{t\to t_0} \vec{f}(t) = \vec{L}
        \end{equation}
        if 
        \begin{equation}
            \lim_{t\to t_0} \lVert\vec{f}(t)-\vec{L}\rVert = 0
        \end{equation}
    \end{definition}
    \item The simplest case is when $\lim_{t\to t_0} \vec{f}(t) = \vec{L}$. Then we can immediately see that:
    \begin{equation}
        \lim_{t\to t_0} \lVert \vec{f}(t) \rVert = \lVert \vec{L} \rVert
    \end{equation}
    \begin{warning}
        Note that this is not an if and only if statement. It is totally possible for two vectors to have the same magnitude, but are not equal to each other.
    \end{warning}
    \item Limit rules carry over from two dimensional calculus. Given $\vec{f}(t) \to \vec{L}$ and $\vec{g}(t) \to \vec{M}$ and $u(t) \to A$ as $t\to t_0$. Then:
    \begin{itemize}
        \item $\vec{f}+\vec{g} \to \vec{L}+\vec{M}$
        \item $\alpha \vec{f}(t) \to \alpha \vec{L}$
        \item $u(t)\vec{f}(t) \to A\vec{L}$
        \item $\vec{f}(t)\cdot \vec{g}(t) \to \vec{L}\cdot \vec{M}$
        \item $\vec{f}(t) \times \vec{g}(t) \to \vec{L} \times \vec{M}$
    \end{itemize}
    \item We have $\vec{f}(t) \to \vec{L}$ \textit{if and only if} $f_1(t) \to L_1$, $f_2(t) \to L_2$, and $f_3(t) \to L_3$.
    \begin{example}
        Suppose we have a vector function:
        \begin{equation}
            f(t) = \frac{\sin t}{t} \hat{i} + (2+t^2)\hat{j} + e^{t^2}\cos t\hat{k}
        \end{equation}
        then:
        \begin{equation}
            \lim_{t\to 0} \vec{f}(t) = (1,2,1)
        \end{equation}
    \end{example}
    \item A vector function $\vec{f}(t)$ is continuous at $t_0$ if $\lim_{t\to t_0} \vec{f}(t) =\vec{f}(t_0)$.
\end{itemize}