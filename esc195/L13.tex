\section{Infinite Sequences and Series}
% \subsection{Sequences}
\begin{itemize}
    \item We use curly brackets to indicate a sequence, such as:
    \begin{equation}
        f(n) = \frac{1}{n} = \left\{1, \frac{1}{2}, \frac{1}{3}, \frac{1}{4}, \dots\right\}
    \end{equation}
    Alternatively, we can use $a_n$ to represent a sequence.
    \begin{definition}
        A sequence $\{a_n\}$ is:
        \begin{itemize}
            \item increasing iff $a_n < a_{n+1}$
            \item non-decreasing iff $a_k \le a_{n+1}$
            \item decreasing iff $a_n > a_{n+1}$
            \item non-increasing iff $a_n \ge a_{n+1}$
        \end{itemize}
        A function that satisfies any of these are known as \textbf{monotonic} functions.
    \end{definition}
    \item Bounded functions have an upper or lower bound, while unbounded functions diverge to infinity or negative infinity.
    \begin{example}
        Suppose we wish to prove that $2^k$ is unbounded. We wish to find $k$ such that $a_k > M$ or $2^k > M$. Taking the natural logarithm of both sides, we have:
        \begin{equation}
            k > \frac{\ln M}{\ln 2}
        \end{equation}
        which is possible to do and we are done.
    \end{example}
    \begin{example}
        Suppose we wish to find if $a_n = \frac{n^2}{e^n}$ is bounded or unbounded. This can be approached by working with derivatives through the function $f(x) = \frac{x^2}{e^x}$, represented in the following plot:
        \begin{center}
            \begin{tikzpicture}
            \begin{axis}[
            legend pos=outer north east,
            title=Example,
            axis lines = box,
            xlabel = $x$,
            ylabel = $y$,
            variable = t,
            trig format plots = rad,
            ]
            \addplot [
                domain=0:5,
                samples=70,
                color=blue,
                ]
                {x^2/e^x};
            \end{axis}
            \end{tikzpicture}
        \end{center}
        Taking the derivative $f'(x)=xe^{-x}(2-x)$, we see that $f$ decreases for $x>2$ so this means that $a_n$ decreases for $n>2$
    \end{example}
    \begin{warning}
        Not everything in functions carries over to sequences. For example, $f(x)=\frac{1}{x-\sqrt{2}}$ is unbounded but $a_n = \frac{1}{n-\sqrt{2}}$ is bounded since $n \neq \sqrt{2}$ is impossible.
    \end{warning}
    \item We can only take the limit of a sequence as $n\to\infty$.
    \begin{definition}
        We can define $\lim_{n\to\infty} a_n = L$ iff for every $\epsilon>0$, there exists an integer $k>0$ such that if $n \ge k$, then $|a_n - L| < \epsilon$.
    \end{definition}
    \begin{example}
        Let us prove $\lim_{n\to\infty} \frac{n}{n+1}=1$. We  find $k$ such that $\left|\frac{n}{n+1}-1\right|<\epsilon$ for $n\ge k$. This can be rewritten as:
        \begin{equation}
            \left|\frac{1}{n+1}\right| < \epsilon 
        \end{equation}
        or $|n+1| > \frac{1}{\epsilon}$. Thus, if we choose $k = \frac{1}{\epsilon}$ such that if we choose $n>k=\frac{1}{\epsilon}$, then:
        \begin{equation}
            \left|\frac{n}{n+1}-1\right| = \left|\frac{1}{n+1}\right| < \left|\frac{1}{n}\right| < \frac{1}{k} = \epsilon
        \end{equation}
        Therefore, $\lim_{n\to\infty} \frac{n}{n+1}=1$.
    \end{example}
    \begin{theorem}
        \textbf{Uniqueness of a Limit}: If $\lim_{n\to\infty} = L$ and $\lim_{n\to\infty}a_n = M$, then $L=M$.
    \end{theorem}
    \begin{definition}
        If a sequence has a limit, it is said to be convergent. Otherwise, it is divergent.
    \end{definition}
    \item This leads to the following:
    \begin{enumerate}
        \item If a sequence is convergent, it is bounded.
        \item If a sequence is unbounded, it is divergent.
        \item A bounded sequence is not necessarily convergent.
    \end{enumerate}
    \item For example, $a_n = \cos \pi n$ is bounded but not convergent.
    \begin{theorem}
        \textbf{Monotonic Sequence Theorem}: A bounded nondecreasing sequence converges to its least upper bound. A bounded non increasing sequence converges to its greatest lower bound.
    \end{theorem}
    \item The limit has a few properties. Let $\lim_{n\to\infty} a_n = L$ and $\lim_{n\to\infty}b_n = M$. Then:
    \begin{enumerate}
        \item $\lim_{n\to\infty}(a_n+b_n) = L+M$
        \item $\lim_{n\to\infty} \alpha a_n = \alpha L$ for $\alpha \in \mathbb{R}$.
        \item $\lim_{n\to\infty} a_nb_n = L \cdot M$
        \item $\lim_{n\to\infty} \frac{1}{b_n} = \frac{1}{M}$ for $b_n \neq 0, M \neq 0$.
        \item $\lim_{n\to\infty} \frac{a_n}{b_n} = \frac{L}{M}$ for $b_n \neq 0, M\neq 0$.
    \end{enumerate}
    \begin{theorem}
        \textbf{Pinching Theorem for Sequences}: If for large $n$, $a_n \le b_n \le c_n$ and if $\lim_{n\to\infty}a_n = L$ and $\lim_{n\to\infty} = L$, then $\lim_{n\to\infty}b_n = L$.
    \end{theorem}
    \begin{example}
        Suppose we wish to find the limit $\lim_{n\to\infty} \frac{\sin (n\pi/6)}{n}$. We can let:
        \begin{equation}
            -\frac{1}{n} \le \frac{\sin(n\pi/6)}{n} \le \frac{1}{n}
        \end{equation}
        Since $\lim_{n\to\infty} -\frac{1}{n} = \lim_{n\to\infty}\frac{1}{n} = 0$, then the original limit must also be zero.
    \end{example}
    \begin{theorem}
        Supose we have a sequence: $c_n = g(f_n)$. Given $\lim_{n\to\infty}c_n = C$. If $f$ is continuous at $c$, in the traditional way, then: $\lim_{n\to\infty} f(c_n) = f(c)$.
    \end{theorem}
    \begin{example}
        Let us look at the function $\sin\left(\frac{1}{n^2+1}\right)$. We know that $\lim_{n\to\infty} \frac{1}{n^2+1}=0$, so:
        \begin{equation}
            \lim_{n\to\infty} \sin\left(\frac{1}{n^2+1}\right) = \sin(0) = 0
        \end{equation}
        where we have applied the previous theorem.
    \end{example}
\end{itemize}