\section{Applications of Taylor Polynomials}
\begin{itemize}
    \item Recall that the $n^\text{th}$ degree polynomial is:
    \begin{equation}
        T_n = \sum_{i=0}^n \frac{f^{(i)}(a)}{i!}(x-a)^i
    \end{equation}
    The first degree polynomial is a linear approximation and the second degree is a quadratic approximation (at least, near $x=a$).
    \item We can use Taylor series to estimate erros:
    \begin{enumerate}
        \item Alternating series: $|R_n(x)| < |a_{n+1}|$
        \item Taylor's formula: $|R_n| < \frac{M(x-a)^{n+1}}{(n+1)!}$
    \end{enumerate}
    \begin{example}
        Suppose we want to use the Taylor expansion of $f(x) = \sqrt{x}$ at $a=1$ and use it to evaluate $\sqrt{1.25}$. The first four derivatives are:
        \begin{align}
            f(x) = x^{1/2} && f(1) = 1\\ 
            f'(x)=\frac{1}{2}x^{-1/2} && f'(1)= \frac{1}{2} \\ 
            f''(x)=-\frac{1}{4}x^{-3/2} && f''(1) = -\frac{1}{4} \\ 
            f'''(x)=\frac{3}{8}x^{-5/2} && f'''(1) = \frac{3}{8} \\ 
            f''''(x)=-\frac{15}{16}x^{-7/2} && f''''(1) = -\frac{15}{16}
        \end{align}
        which gives:
        \begin{equation}
            \sqrt{x} \simeq T_3(x) = 1+ \frac{1}{2}(x-1) - \frac{1}{4} \frac{(x-1)^2}{2!} + \frac{3}{8}\frac{(x-1)^3}{3!}
        \end{equation}
        and the error is:
        \begin{equation}
            |R_3(x)| < |a_4| = \frac{15}{16}\frac{(x-1)^4}{4!}
        \end{equation}
        so:
        \begin{equation}
            \sqrt{1.125} \simeq 1 + \frac{0.25}{2} - \frac{0.25^2}{8} + \frac{0.25^3}{16} \pm \frac{5}{128}0.25^4 \approx 1.11816 \pm 0.00015
        \end{equation}
    \end{example}
    \begin{example}
        Consider the maclaurin series of $\cos x$ about $a=0$. We want to find the error for $-\frac{\pi}{4}<x<\frac{\pi}{4}$, we have:
        \begin{equation}
            \cos x \simeq T_3(x) = 1-\frac{x^2}{2!} + \frac{x^4}{4!} - \frac{x^6}{6!}
        \end{equation}
        and the error would be:
        \begin{equation}
            |R_3(x)| < \left|\frac{x^8}{8!}\right| < \frac{(\pi/4)^8}{8!} \approx 3.6 \times 10^{-6}
        \end{equation}
        Note that we can also use our alternating series result to get the same error.
    \end{example}
    \begin{example}
        Suppose we wish to find $\ln(1.4)$ to within $0.001$ using the series expansion for $\ln(1-x)$:
        \begin{equation}
            \ln(1-x) = - x - \frac{x^2}{2}-\frac{x^3}{3}-\frac{x^4}{4}-\cdots
        \end{equation}
        which converges for $|x|<1$. Here, we want $x=-0.4$ and we get an alternating series. The remainders are then"
        \begin{align}
            |R_1(x)| &< \left|\frac{x^2}{2}\right| = 0.8 \\ 
            |R_2(x)| &< \left|\frac{x^3}{3}\right| = 0.02 \\ 
            |R_3(x)| &< \left|\frac{x^4}{4}\right| = 0.006 \\ 
            |R_4(x)| &< \left|\frac{x^5}{5}\right| = 0.002 \\ 
            |R_5(x)| &< \left|\frac{x^6}{6}\right| = 0.0007 < 0.001
        \end{align}
        Therefore, we can take the first five terms of the expansion and get an answer within $0.0007$ of the actual answer.
    \end{example}
    \item The binomial theorem tells us:
    \begin{align}
        (a+b)^k &= a^k + ka^{k-1}b + \frac{k(k-1)}{2!}a^{k-2}b^2 + \cdots + \frac{k(k-1)(k-2)\cdots (k-n+1)}{k!}a^{k-n}b^{n} \\ 
        &= \sum_{n=0}^k \binom{k}{n}a^{k-n}b^n
    \end{align}
    \item If we let $a=1$ and $b=x$, we get:
    \begin{equation}
        (1+x)^k = \sum_{n=0}^k \binom{k}{n}x^n
    \end{equation}
    which represents a power series. We can also get the coefficients using the Taylor series:
    \begin{align}
        f(x)=(1+x)^k && f(0)=1 \\ 
        f'(x)=k(1+x)^{k-1} && f'(1) = k \\ 
        f''(x)=k(k-1)(1+x)^{k-1} && f''(0) = k(k-1)
    \end{align}
    such that:
    \begin{equation}
        f(x) = \sum_{n=0}^\infty \frac{f^{(n)}(0)}{n!}x^n
    \end{equation}
    which is equiavlent to before. We can compare the ratio test:
    \begin{equation}
        \left|\frac{a_{n+1}}{a_n}\right| = \left|\frac{k(k-1)(k-2)\cdots (k-x)(x^{n+1})}{(n+1)!}\cdot \frac{n!}{k(k-1)\cdots (k-n+1)x^n}\right| = \left|\frac{k-n}{n+1}x\right|
    \end{equation}
    which approaches $|x|<1$ as $n\to \infty$. The interval convergence is:
    \begin{align}
        k \le -1 && I = (-1,1) \\ 
        -1<k<0: && I= (-1,1] \\ 
        k \ge 0: && I = [-1,1]
    \end{align}
    \begin{example}
        Suppose we have $f(x)=\frac{1}{\sqrt{2+x}}$. We then have:
        \begin{align}
            (2+x)^{-0.5}{n} &= \frac{1}{\sqrt{2}}\left(1+\frac{x}{2}\right)^{-1/2} \\ 
            &= \frac{1}{\sqrt{2}}\sum_{n=0}^\infty \binom{-1/2}{n}\left(\frac{x}{2}{}\right)^n \\ 
            &= \frac{1}{\sqrt{2}}\left[1-\frac{1}{2}\frac{x}{2}+\frac{3}{4}\frac{(x/2)^2}{2!}-\cdots\right]
        \end{align}
        and the radius of convergence is $2$ since we want $|x/2|<1$.
    \end{example}
    \begin{example}
        This is an extremely important linear approximation:
        \begin{equation}
            (1+x)^k \approx 1+kx
        \end{equation}
    \end{example}
\end{itemize}