\section{Integration}
\subsection{Recap of Integration}
\begin{itemize}
    \item The definite integral has the geometric interpretation as the area under the curve $f(x)$ between $x=a$ and $x=b$ and the $x$ axis:
    \begin{equation}
        \int_a^b f(x) \dd{x}
    \end{equation}
    but can be rigorously defined using a Riemann sum:
    \begin{equation}
        \int_a^b f(x) \dd{x} \equiv \lim_{\Vert P \rVert} \sum_{i=1}^n f(x_i^*)\Delta x_i
        \label{eq:}
    \end{equation}
    Often, we have a uniform partition, such that $\Delta x_i = \frac{b-a}{n}$ where $n$ is the number of partitions. And if we choose to use the right hand endpoint, then:
    \begin{equation}
        f(x_i^*) = f(x_i) = f(x_i) = f\left(a+\frac{b-a}{n}i\right)
    \end{equation}
    \begin{example}
        To solve $\int_0^5 x^2 \dd{x}$, we can choose a uniform partition with:
        \begin{equation}
            \Delta x = \frac{5-0}{n} = \frac{5}{n}
        \end{equation}
        and:
        \begin{equation}
            x_i^* = x_i = i\Delta x \implies f(x_i^*)=(i\Delta x)^2 = \left(i\frac{5}{n}\right)^2
        \end{equation}
        The area approximation is:
        \begin{align}
            A \simeq \sum_{i=1}^n \Delta x_i f(x_i^*) &= \sum_{i=1}^n \left(\frac{5}{n}\right)\left(i\frac{5}{n}\right)^2 \\ 
            &= \frac{125}{n^2} \sum_{i=1}^n i^2 = \frac{125}{n^3} \frac{n(n+1)(2n+1)}{6}
        \end{align}
        Taking the limit as $n\to \infty$, we get:
        \begin{equation}
            \int_0^5 x^2 \dd{x} =  \lim_{n\to\infty} \frac{125}{6}\left(2+\frac{2}{n}+\frac{1}{n^2}\right) = \frac{5^3}{3}.
        \end{equation}
    \end{example}
    \begin{example}
        To evaluate $\int_1^2 x^{-2} \dd{x}$, we can choose
        \begin{equation}
            x_i^* = \sqrt{x_{i-1}x_i}
        \end{equation}
        and a uniform partition of:
        \begin{equation}
            \Delta x = \frac{2-1}{n} = \frac{1}{n}
        \end{equation}
        such that:
        \begin{equation}
            x_i = 1+i\Delta x = 1 + \frac{i}{n} = \frac{n+i}{n}
        \end{equation}
        and
        \begin{equation}
            x_{i-1} = \frac{n+i-1}{n}
        \end{equation}
        such that the area is:
        \begin{align*}
            A &\simeq \sum_{i=1}^n \Delta xf(x_i^*) \\ 
            &= \sum_{i=1}^n \frac{1}{n} \left(\frac{1}{x_i^*}\right)^2 \\ 
            &= \sum_{i=1}^n \frac{1}{n} \frac{1}{x_{i-1}x_i} \\ 
            &= \sum_{i=1}^n \frac{1}{n} \frac{n}{n+i-1}\cdot \frac{n}{n+i} \\
            &= \sum_{i=1}^n n \frac{1}{n+i-1} \cdot \frac{1}{n+i} \\ 
            &= \sum_{i=1}^n \left(\frac{1}{n+i-1}-\frac{1}{n+i}\right) \\ 
            &= n\left[\sum_{i=1}^n \frac{1}{n+i-1} - \sum_{i=1}^n \frac{1}{n+i}\right] \\ 
            &= n\left[\sum_{i=0}^n \frac{1}{n+i} - \sum_{i=1}^n \frac{1}{n+i}\right] \\ 
            &= n\left[\frac{1}{n}+\frac{1}{n+1}+\frac{1}{n+2}+\cdots+\frac{1}{2n-1}-\frac{1}{n+1}-\frac{1}{n+2}-\cdots-\frac{1}{2n}\right] \\ 
            &=n\left(\frac{1}{n}-\frac{1}{2n}\right) \\ 
            &= 1 - \frac{1}{2} = \frac{1}{2}
        \end{align*}
        The part where we cancel out everything is called a \textbf{telescoping series}. Notice how the area doesn't depend on $n$ so we get the exact area, even if we let $n=1$!.
    \end{example}
    \item We need a better way to do integration, so we can define:
    \begin{equation}
        F(x) \equiv \int_a^x f(t) \dd{t}
    \end{equation}
    such that $F'(x) = f(x)$. This is the definition of the antiderivative. This leads to the fundamental theorem of calculus:
    \begin{equation}
        \int_a^b f(t) \dd{t} = F(h) - F(a)
    \end{equation}
    and the indefinite integral can be written as:
    \begin{equation}
        \int f(x) \dd{x} = G(x) + C
    \end{equation}
    The main problem now becomes trying to \textit{find antiderivatives}, which is much easier than Riemann sums, though still more difficult than calculating derivatives.
\end{itemize}
\subsection{Techniques of Integration}
\begin{itemize}
    \item \textbf{Integration by Parts} attempts to reverse the product rule:
    \begin{equation}
        (fg)' = fg' + f'g
    \end{equation}
    Taking the integral of both sides gives:
    \begin{align}
        f(x)g(x) &= \int f(x)g'(x) \dd{x} + \int f'(x) g(x) \dd{x} \\ 
        \int f(x)g'(x) \dd{x} &= \int f(x) g'(x) \dd{x} = f(x)g(x) - \int f'(x)g(x) \dd{x}
    \end{align}
    If the second integral is easier than the first, then we have made substaintial progress.
    \begin{idea}
        Integration of parts tells us that:
        \begin{equation}
            \int u \dd{v} = uv - \int v \dd{u}
        \end{equation}
    \end{idea}
    \begin{example}
        To solve $\int xe^{2x}$, we can let:
        \begin{align}
            &u=x            &\dd{v} = e^{2x} \dd{x} \\ 
            &\dd{u}=\dd{x} &v=\frac{1}{2}e^{2x} 
        \end{align}
        which gives:
        \begin{align}
            & \frac{1}{2}xe^{2x} - \int \frac{1}{2}e^{2x} \dd{x} \\ 
            &= \frac{1}{2}xe^{2x} - \frac{1}{4}e^{2x} + C
        \end{align}
        We can check:
        \begin{align}
            \frac{d}{dx}\left(\frac{1}{2}xe^{2x}-\frac{1}{4}e^{2x}+C\right) \\ 
            &= xe^{2x} + \frac{1}{2}e^{2x}-\frac{2}{4}e^{2x} \\ 
            &= xe^{2x}
        \end{align}
    \end{example}
    \begin{example}
        To solve $\int x^2\sin(2x) \dd{x}$, we let:
        \begin{align}
            &u=x^2            &\dd{v} = \sin 2x\dd{x} \\ 
            &\dd{u}=2x\dd{x} &v=-\frac{1}{2}\cos(2x) 
        \end{align}
        which gives:
        \begin{equation}
            =-\frac{1}{2}x^2\cos 2x + \int x\cos(2x) \dd{x}
        \end{equation}
        and we can apply integration by parts a second time, if we let:
        \begin{align}
            &u=x            &\dd{v} = \cos 2x\dd{x} \\ 
            &\dd{u}=\dd{x} &v=\frac{1}{2}\sin(2x) 
        \end{align}
        which gives us:
        \begin{align}
            &=-\frac{1}{2}x^2\cos(2x) + \frac{1}{2}x\sin(2x) - \int \frac{1}{2}\sin(2x) \dd{x} \\ 
            &=-\frac{1}{2}x^2\cos(2x)+\frac{1}{2}x\sin(2x) +\frac{1}{4}\cos(2X)+C
        \end{align}
    \end{example}
    \begin{example}
        To solve $I=\int e^{x}\sin x\dd{x}$, we can let:
        \begin{align}
            &u=\sin x            &\dd{v} =e^x\dd{x} \\ 
            &\dd{u}=\cos x\dd{x} &v=e^x
        \end{align}
        to give us:
        \begin{equation}
            = e^x\sin x - \int e^{x}\cos x \dd{x}
        \end{equation}
        We apply integration by parts a second time:
        \begin{align}
            &u=\cos x            &\dd{v} =e^x\dd{x} \\ 
            &\dd{u}=-\sin x\dd{x} &v=e^x
        \end{align}
        to get:
        \begin{align}
            I&=e^x\sin x - e^x\cos x - \underbrace{\int e^x\sin x \dd{x}}_{I} \\ 
            2I &= e^x\left(\sin x - \cos x\right) + C' \\ 
            I &= \frac{1}{2}e^x\left(\sin x - \cos x\right) + C
        \end{align}
        and we are done.
    \end{example}
    \begin{example}
        We can also solve integrals that do not appear to have parts, such as $\int \ln x \dd{x}$. We choose:
        \begin{align}
            &u=\ln x            &\dd{v} =\dd{x} \\ 
            &\dd{u}=\frac{1}{x}\dd{x} &v=x
        \end{align}
        to give us:
        \begin{equation}
            \ln x - \int \dd{x} = x\ln x - x + C
        \end{equation}
    \end{example}
    \item For a definite integral, we can write IBP as:
    \begin{equation}
        f(x)g(x)\Big|^b_a-\int_a^b f'(x)g(x) \dd{x}
    \end{equation}
    \begin{example}
        It is \textit{possible} to apply integration of parts to find the integral of $\int \tan x \dd{x} = \int \frac{\sin x}{\cos x} \dd{x}$. We can let:
        \begin{align}
            &u=\frac{1}{\cos x}=\sec x            &\dd{v} =\sin x\dd{x} \\ 
            &\dd{u}=\sec x\tan x &v=-\cos x
        \end{align}
        this gives us:
        \begin{align}
            \int \tan x \dd{x} &= -\frac{\cos x}{\cos x}+ \int \tan x \dd{x}
        \end{align}
        Notice that we could try to subtract the original integral from both sides and get:
        \begin{equation}
            0 = -1
        \end{equation}
        which is clearly wrong! However, we forgot the constant of integration, so the correct statement would be:
        \begin{equation}
            0 + C' = -1 + C
        \end{equation}
        which does not tell us anything interesting. This brings We can see this concretely by repeating the same steps but trying to evaluate the definite integral $\int_a^b \tan x \dd{x}$ instead, which gives:
        \begin{equation}
            \int_a^b \tan \dd{x} = \left(-1\right)\Big|^{x=b}_{x=a} + \int_a^b \tan x \dd{x} \implies 0 = (-1) - (-1) \implies 0=0
        \end{equation}
        which confirms our suspcision that this isn't anything useful, but it's also not an incorrect statement.
    \end{example}
    \begin{warning}
        Sometimes it is possible to get more than one answer through various means that differ by a constant factor when solving indefinite integrals. When this happens, nothing is wrong: we simply need to consider the constant of integration.
    \end{warning}
    \begin{idea}
        But how do we know \textit{which} values of $u$ and $\dd{v}$ we should pick? A common strategy is to use \textbf{LIATE}:
        \begin{enumerate}
            \item L: Logarithms
            \item I: Inverse Trig
            \item A: Algebraic
            \item T: Trigonometric
            \item E: Exponential
        \end{enumerate}
        If a function consists of two terms, the term that is higher up (closer to L) usually gets differentiated and the term near the bottom (closer to E) usually gets integrated. See \href{https://math.stackexchange.com/questions/768332/liate-how-does-it-work}{this} for how it works, and this \href{https://www.youtube.com/watch?v=fnHxq8ZK0rE}{video} for a tutorial.
    \end{idea}
\end{itemize}
