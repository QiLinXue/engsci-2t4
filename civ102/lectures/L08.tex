\section{Lecture 8: Factors of Safety}
\begin{itemize}
    \item In lecture 4, we derived that the maximum tension is:
    \begin{equation}
        F_\text{max}=12900\si{\kilo\newton}
        \label{eq:}
    \end{equation}
    We wish to actually magnify this by a factor of safety. The failure stress is:
    \begin{equation}
        \sigma_\text{ultimate}=1860\si{\mega\pascal}
        \label{eq:}
    \end{equation}
    and the area of one wire ($\diameter=5\si{\milli\meter}$) s:
    \begin{equation}
        19.63 \si{\milli\meter\squared}
        \label{eq:}
    \end{equation}
    Thus each wire will have a maximum force of
    \begin{equation}
        F=36.52\si{\kilo\newton}
        \label{eq:}
    \end{equation}
    so we require that:
    \begin{equation}
        N > 12924/36.52 = 354\text{ wires}
        \label{eq:}
    \end{equation}
    \item For perfect materials and workmanship, Rankine recommended the safety factor for dead loads to be $2$ and for live loads to be $4$.
    \item For good ordinary materials and workmanship, in the case of metals, the factors of safety should be $3$ and $6$.
    \item For working with timber, the factors of safety should be $4 \to 5$ and $8\to 10$
    \item For masonry, the factors of safety should be $4$ and $8$.
\end{itemize}