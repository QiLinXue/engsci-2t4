\section{Lecture 9: Weight a Moment! What is I?}
\begin{itemize}
    \item For a system to be in equilibrium, the forces need to sum up to zero.
    \item The \textbf{moment of inertia} describes how much an object resists a change in its angular velocity.
    \item Recall the period of a mass on a spring under a linear force is:
    \begin{equation}
        T = 2\pi\sqrt{\frac{m}{k}} \implies T \propto \sqrt{m}
        \label{eq:}
    \end{equation}
    The rotational analog of mass is rotational inertia, so the period of an object under the influence of a linear restoring torque is propotoinal to:
    \begin{equation}
        T \propto \sqrt{I}
        \label{eq:}
    \end{equation}
    \item The strong axis refers to the orientation of the body which has greater rotational inertia and the \textbf{weak axis} refers to the orientation with the less rotational inertia.
    \item The moment of inertia of a point mass $m$ measured from a pivot a radius $r$ away is given by:
    \begin{equation}
        I = mr^2
        \label{eq:}
    \end{equation}
    If we have a mass on both ends of a rod, and apply a force of $F=\frac{M}{r}$ on a mass perpendicular to the rod, then the effective force on both masses is:
    \begin{equation}
        F_\text{effective} = \frac{M}{2r}
        \label{eq:}
    \end{equation}
    \item Some angular kinematics results:
    \begin{align}
        s = \theta r \\ 
        \frac{d^2s}{dt^2}=a \\ 
        \alpha = \frac{d^2\theta}{dt^2}=\frac{1}{r}\frac{d^2s}{dt^2}
    \end{align}
\end{itemize}