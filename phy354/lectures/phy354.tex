\documentclass{article}
\usepackage{qilin}
\tikzstyle{process} = [rectangle, rounded corners, minimum width=1.5cm, minimum height=0.5cm,align=center, draw=black, fill=gray!30, auto]
\title{PHY354: Advanced Classical Mechanics}
\author{QiLin Xue}
\date{Fall 2021}
\usepackage{mathrsfs}
\usetikzlibrary{arrows}
\usepackage{stmaryrd}
\usepackage{accents}
\newcommand{\ubar}[1]{\underaccent{\bar}{#1}}
\usepackage{pgfplots}
\numberwithin{equation}{section}

\begin{document}

\maketitle
\tableofcontents
\newpage
\section{Motivation}
In previous courses, we focused on $\vec{F}=m\vec{a}$ and its consequences. In this course, we will do it in a more elegant way, i.e. via Euler, Lagrange, Hamilton, Jacobi, Noether.  Some disadvantages of Newton's approach are:
\begin{itemize}
    \item Can be difficult to apply to complex situations and extended objects
    \item Difficult to see how chaos theory arises
    \item Obscures relationship between quantum and classical mechanics
\end{itemize}
\section{Lagrangian Mechanics}
We first start with a few examples showing Lagrangian Mechanics in action, and we will later discuss generalizations and intricacies:
\begin{example}
    Let us consider a simple system with a mass $m$ on a spring with spring constant $k$. Instead of solving this by writing $F=-kx$, let us write the following strange combination of kinetic and potential energy, known as the \textbf{Lagrangian}.:
    \begin{equation}
        L = T - V,
    \end{equation}
    where $T$ is the kinetic energy and $V$ is the potential energy. Using $T=\frac{1}{2}m\dot{x}^2$ and $V=\frac{1}{2}kx^2$, we get:
    \begin{equation}
        L = \frac{1}{2}m\dot{x}^2 - \frac{1}{2}kx^2.
    \end{equation}
    Consider the following equation, known as the Euler-Lagrange Equation:
    \begin{equation}
        \frac{d}{dt}\left(\frac{\partial L}{\partial \dot{x}}\right) = \frac{\partial L}{\partial x}
    \end{equation}
    which gives 
    \begin{equation}
        \frac{d}{dt}\left(m\dot{x}\right) = -kx \implies m\ddot{x} = -kx.
    \end{equation}
\end{example}
\begin{example}
    Suppose we have a general potential $V=V(x).$ Then the E-L equation gives
    \begin{equation}
        m\ddot{x} = -\frac{\partial V(x)}{\partial x}.
    \end{equation}
    Notice that $-\frac{\partial V(x)}{\partial x}$ is just the force!
\end{example}
If we have more than one dimension, i.e. $x,y,z$, then we'll have 3 E-L equations, and we'll have to solve them separately. The Lagrangian would be 
\begin{equation}
    L = \frac{1}{2}m\sum_i \dot{x}_i^2 - V(x_i)
\end{equation}
and the 3 different E-L equations would be
\begin{equation}
    \frac{d}{dt}\left(\frac{\partial L}{\partial \dot{x}_i}\right) = \frac{\partial L}{\partial x_i} \implies m\ddot{x}_i = -\frac{\partial V}{\partial x_i}
\end{equation}
for $i=1,2,3$. These equations are just the $3$ components of $m\ddot{\vec{r}} = - \vec{\nabla} V(\vec{r})$. The important idea here is that we can determine the equations of motion without knowing the forces! We just need to figure out the potential energy. We can try a more complex example:
\begin{example}
    Suppose we have a mass on a spring that acts as a pendulum. The spring has equilibrium length $\ell$ and at an angle $\theta$, the spring has length $r=\ell + x$. We will work in polar coordinates $x,\theta.$ The kinetic energy is
    \begin{equation}
        T = \frac{1}{2}m\vec{v}^2 = \frac{1}{2}m\left(\dot{x}^2 + (\ell+x)^2\dot{\theta}^2\right),
    \end{equation}
    and the potential energy is
    \begin{equation}
        V = -mg(\ell+x)\cos\theta + \frac{1}{2}kx^2.
    \end{equation}
    The $E-L$ equation for $x$ gives
    \begin{align}
        \frac{d}{dt}\left(\frac{\partial L}{\partial \dot{x}}\right) &= \frac{\partial L}{\partial x} \\
        m\ddot{x} &= m(\ell + x)\dot{\theta}^2 + mg\cos\theta - kx.
    \end{align}
    The term $m(\ell + x)\dot{\theta}^2$ is the centripetal force, $mg\cos\theta$ is the radial gravitational force, and $-kx$ is the spring force. The $\theta$ component is 
    \begin{equation}
        m(\ell + x)^2\dot{\theta} = mg(\ell + x)\sin\theta
    \end{equation}
    which corresponds to 
    \begin{equation}
        \vec{\tau} = \frac{d\vec{L}}{dt}
    \end{equation}
    where $\vec{\tau}$ is torque and $\vec{L}$ is the angular momentum.
\end{example}
\section{Principle of Least Action}
In principle, there are an infinite number of paths a particle can take from point A to point B. Traditionally, we will solve this by analyzing the forces at every single point to find the path it actually takes. We take an alternative approach. We define the action 
\begin{equation}
    S \equiv \int_{t_0}^{t_1} L(q(t),\dot{q}(t),t) \dd{t},
\end{equation}
where $q$ is a generalized coordinate.

We assign each path a particular action, and we claim that the particle takes the path with the least action. However, this may be tricky since it depends on $q(t_0),q(t_0+\epsilon),\dots,q(t_1)$, an infinite number of variables. The terminology here is that the action $S$ is a functional of $q(t)$, i.e. we write it as
\begin{equation}
    S[q(t)] = \lim_{\epsilon \to 0} S(q(t_0), q(t_0+\epsilon), \dots, q(t_1)).
\end{equation}
We want to find the extremum of this action. In one dimension, this occurs when
\begin{equation}
    f(x_0+\delta)=f(x_0)=o(\delta^2)
\end{equation}
where $o(\delta^2)$ is the order of second order. For functionals, we can consider a small perturbation $q_c(t) + \delta q(t)$. Then we claim that an extremum occurs when
\begin{equation}
    S[q_c(t)+\delta q(t)] - S[q_c(t)] = o(\delta q^2).
\end{equation}
We will prove this later.
\end{document}