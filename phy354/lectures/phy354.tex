\documentclass{article}
\usepackage{qilin}
\tikzstyle{process} = [rectangle, rounded corners, minimum width=1.5cm, minimum height=0.5cm,align=center, draw=black, fill=gray!30, auto]
\title{PHY354: Advanced Classical Mechanics}
\author{QiLin Xue}
\date{Fall 2021}
\usepackage{mathrsfs}
\usetikzlibrary{arrows}
\usepackage{stmaryrd}
\usepackage{accents}
\newcommand{\ubar}[1]{\underaccent{\bar}{#1}}
\usepackage{pgfplots}
\numberwithin{equation}{section}

\begin{document}

\maketitle
\tableofcontents
\newpage
\section{Motivation}
In previous courses, we focused on $\vec{F}=m\vec{a}$ and its consequences. In this course, we will do it in a more elegant way, i.e. via Euler, Lagrange, Hamilton, Jacobi, Noether.  Some disadvantages of Newton's approach are:
\begin{itemize}
    \item Can be difficult to apply to complex situations and extended objects
    \item Difficult to see how chaos theory arises
    \item Obscures relationship between quantum and classical mechanics
\end{itemize}
\section{Lagrangian Mechanics}
We first start with a few examples showing Lagrangian Mechanics in action, and we will later discuss generalizations and intricacies:
\begin{example}
    Let us consider a simple system with a mass $m$ on a spring with spring constant $k$. Instead of solving this by writing $F=-kx$, let us write the following strange combination of kinetic and potential energy, known as the \textbf{Lagrangian}.:
    \begin{equation}
        L = T - V,
    \end{equation}
    where $T$ is the kinetic energy and $V$ is the potential energy. Using $T=\frac{1}{2}m\dot{x}^2$ and $V=\frac{1}{2}kx^2$, we get:
    \begin{equation}
        L = \frac{1}{2}m\dot{x}^2 - \frac{1}{2}kx^2.
    \end{equation}
    Consider the following equation, known as the Euler-Lagrange Equation:
    \begin{equation}
        \frac{d}{dt}\left(\frac{\partial L}{\partial \dot{x}}\right) = \frac{\partial L}{\partial x}
    \end{equation}
    which gives 
    \begin{equation}
        \frac{d}{dt}\left(m\dot{x}\right) = -kx \implies m\ddot{x} = -kx.
    \end{equation}
\end{example}
\begin{example}
    Suppose we have a general potential $V=V(x).$ Then the E-L equation gives
    \begin{equation}
        m\ddot{x} = -\frac{\partial V(x)}{\partial x}.
    \end{equation}
    Notice that $-\frac{\partial V(x)}{\partial x}$ is just the force!
\end{example}
If we have more than one dimension, i.e. $x,y,z$, then we'll have 3 E-L equations, and we'll have to solve them separately. The Lagrangian would be 
\begin{equation}
    L = \frac{1}{2}m\sum_i \dot{x}_i^2 - V(x_i)
\end{equation}
and the 3 different E-L equations would be
\begin{equation}
    \frac{d}{dt}\left(\frac{\partial L}{\partial \dot{x}_i}\right) = \frac{\partial L}{\partial x_i} \implies m\ddot{x}_i = -\frac{\partial V}{\partial x_i}
\end{equation}
for $i=1,2,3$. These equations are just the $3$ components of $m\ddot{\vec{r}} = - \vec{\nabla} V(\vec{r})$. The important idea here is that we can determine the equations of motion without knowing the forces! We just need to figure out the potential energy. We can try a more complex example:
\begin{example}
    Suppose we have a mass on a spring that acts as a pendulum. The spring has equilibrium length $\ell$ and at an angle $\theta$, the spring has length $r=\ell + x$. We will work in polar coordinates $x,\theta.$ The kinetic energy is
    \begin{equation}
        T = \frac{1}{2}m\vec{v}^2 = \frac{1}{2}m\left(\dot{x}^2 + (\ell+x)^2\dot{\theta}^2\right),
    \end{equation}
    and the potential energy is
    \begin{equation}
        V = -mg(\ell+x)\cos\theta + \frac{1}{2}kx^2.
    \end{equation}
    The $E-L$ equation for $x$ gives
    \begin{align}
        \frac{d}{dt}\left(\frac{\partial L}{\partial \dot{x}}\right) &= \frac{\partial L}{\partial x} \\
        m\ddot{x} &= m(\ell + x)\dot{\theta}^2 + mg\cos\theta - kx.
    \end{align}
    The term $m(\ell + x)\dot{\theta}^2$ is the centripetal force, $mg\cos\theta$ is the radial gravitational force, and $-kx$ is the spring force. The $\theta$ component is 
    \begin{equation}
        \frac{d}{dt}m(\ell + x)^2\dot{\theta} = mg(\ell + x)\sin\theta
    \end{equation}
    which corresponds to 
    \begin{equation}
        \vec{\tau} = \frac{d\vec{L}}{dt}
    \end{equation}
    where $\vec{\tau}$ is torque and $\vec{L}$ is the angular momentum.
\end{example}
\section{Principle of Stationary Action}
In principle, there are an infinite number of paths a particle can take from point A to point B. Traditionally, we will solve this by analyzing the forces at every single point to find the path it actually takes. We take an alternative approach. We define the action 
\begin{equation}
    S \equiv \int_{t_0}^{t_1} L(q(t),\dot{q}(t),t) \dd{t},
\end{equation}
where $q$ is a generalized coordinate.

We assign each path a particular action, and we claim that the \textit{particle takes the path with the least action}. However, this may be tricky since it depends on $q(t_0),q(t_0+\epsilon),\dots,q(t_1)$, an infinite number of variables. The terminology here is that the action $S$ is a functional of $q(t)$, i.e. we write it as
\begin{equation}
    S[q(t)] = \lim_{\epsilon \to 0} S(q(t_0), q(t_0+\epsilon), \dots, q(t_1)).
\end{equation}
We want to find the extremum of this action. For functionals, we can consider a small perturbation $q_a(t)=q_c(t) + a \beta(t)$. Then we claim that an extremum occurs when
\begin{equation}
    \frac{\partial}{\partial a}S[\beta(t)] = 0,
\end{equation}
for any function $\beta(t).$ We will now offer a proof for the E-L equations.
\begin{proof}
    We can write 
    \begin{align*}
        \frac{\partial}{\partial a} S[q_a(t)] &= \frac{\partial}{\partial a}\int_{t_0}^{t_1} L(q_a(t), \dot{q}_a(t), t) \dd{t} \\
        &= \int_{t_0}^{t_1} \frac{\partial L}{\partial a}(q_a(t), \dot{q}_a(t), t) \dd{t} \\
        &= \int_{t_0}^{t_1} \left[\frac{\partial L}{\partial q_a}\frac{\partial q_a}{\partial a} + \frac{\partial L}{\partial \dot{q}_a}\frac{\partial \dot{q}_a}{\partial a}\right]\dd{t} \\
        &= \int_{t_0}^{t_1} \left[\frac{\partial L}{\partial q_a}\beta(t) + \frac{\partial L}{\partial \dot{q}_a}\dot{\beta}(t)\right] \dd{t}.
    \end{align*}
    Note that the second term can be simplified using integration by parts 
    \begin{align*}
        \int_{t_0}^{t_1}\frac{\partial L}{\partial \dot{q}_a}\dot{\beta}(t)\dd{t} &= \cancel{\frac{\partial L}{\partial \dot{q}_a}\beta(t)\Bigg|^{t_1}_{t_0}} - \int_{t_0}^{t_1}\frac{d}{dt}\left(\frac{\partial L}{\partial \dot{q}_a}\right)\beta(t) \dd{t}.
    \end{align*}
    The first term goes to zero since the end and start points do not change. Our original derivative is then
    \begin{equation}
        \frac{\partial}{\partial a}S[q_a(t)] = \int_{t_0}^{t_1}\left[\frac{\partial L}{\partial q_a} - \frac{d}{dt}\left(\frac{\partial L}{\partial \dot{q}_a}\right)\right]\beta(t) \dd{t}.
    \end{equation}
    Since this integral is zero, and $\beta$ can be arbitrary, then it must be true that
    \begin{equation}
        \frac{\partial L}{\partial q_a} - \frac{d}{dt}\left(\frac{\partial L}{\partial \dot{q}_a}\right) = 0,
    \end{equation}
    which is the E-L equation. Note that this easily generalizes to multiple dimensions, and hold for any set of generalized coordinates.
\end{proof}
\textbf{Remarks:} The stationary point always refers to a local minimum or a saddle point, but \textit{never} a maximum.

Furthermore, note that $L$ is not unique. Here are some acceptable modifications:
\begin{itemize}
    \item We can multiply $L$ by a constant.
    \item We can add a total time derivative to the Lagrangian: $L \to L + \frac{df}{dt}.$
    \begin{proof}
        The action will pick up a $\int_{t_0}^{t_1} \frac{df}{dt} \dd{t}$ term, which by the FTC, it adds a constant to the action, and thus the extremums will remain the same.
    \end{proof}
\end{itemize}
\subsection{Generalizations}
In general, if we have some integral dependent on a path 
\begin{equation}
    I \equiv \int_{x_1}^{x_2} F(y(x),y'(x)) \dd{x},
\end{equation}
then $I$ is an extremum when
\begin{equation}
    \frac{d}{dx}\frac{\partial f}{\partial y'} = \frac{\partial f}{\partial y}
\end{equation}
\begin{example}
    Let's try a trivial problem: What is the shortest distance between $2$ points on a plane? Consider a path $\ell$ that connects $(x_1,y_1)$ and $(x_2,y_2)$. The path length is then
    \begin{equation}
        \ell = \int\limits_{P}\sqrt{1+y'^2}\dd{x},
    \end{equation}
    which is an extremum when
    \begin{align}
        \frac{d}{dx}\frac{\partial \sqrt{1+y'^2}}{\partial y'} &= \frac{\partial \sqrt{1+y'^2}}{\partial y} \\ 
        \frac{d}{dx} \frac{y'}{\sqrt{1+y'^2}} &= 0,
    \end{align}
    which is true when $y'(x)=\text{constant},$ which corresponds to a straight line.
\end{example}
\subsection{Brachistochrome Problem}
Consider a bead sliding down a frictionless wire, whose endpoints are $(x_1,y_1)$ and $(x_2,y_2)$. What shape of wire will minimize the time it takes to slide down?

We can formulate in terms of a variational calculus problem, i.e. by optimizing a ``time functional'' over all paths $y(x)$. The time it takes is 
\begin{align*}
    T &= \int_1^2 \dd{t} \\ 
    &= \int_1^2 \frac{\dd{s}}{v}
    &= \int_{x_1}^{x_2} \frac{\sqrt{1+(y')^2}}{v}\dd{x},
\end{align*}
or alternatively we can write it as 
\begin{equation}
    T = \int_{y_1}^{y_2} \frac{\sqrt{1+(x')^2}}{v} \dd{y},
\end{equation}
which turns out to be easier since $v=\sqrt{2gy}$. Then:
\begin{equation}
    T = \frac{1}{\sqrt{2g}} \int_{y_1}^{y_2} \underbrace{\frac{\sqrt{1+(x')^2}}{\sqrt{y}}}_{F(y,x')} \dd{y}.
\end{equation}
Using the E-L equation, we have
\begin{align*}
    0 &= -\frac{d}{dy} \frac{\partial F}{\partial x'} + \frac{\partial F}{\partial x} \\ 
    0 &= -\frac{d}{dy} \frac{x'}{\sqrt{y}\sqrt{1+(x')^2}}.
\end{align*}
Since the derivative is $0$, that means $\frac{\partial F}{\partial x'}=a$ must be a constant. Rearranging, we get
\begin{equation}
    x' = \sqrt{\frac{y}{1/a^2-y}} = \sqrt{y/(C-y)}
\end{equation}
where $C\equiv 1/a^2.$ Now solving for $x(y)$, and using the substitution $y=c\sin^2\theta$, we end up as:
\begin{align}
    y &= \frac{c}{2}(1-\cos 2\theta) \\ 
    x &= \frac{c}{2}(2\theta-\sin(2\theta)) + \beta.
\end{align}
There are two boundary conditions and two unknowns. If we instead chose to do the problem by integrating the functional with $\dd{x}$, then we'll end up with a very messy second order product rule expression in the E-L equations. However, we can simplify things using a simple theorem.
\begin{theorem}
    For $S[y] = \int \mathcal{L}(x,y,y') \dd{x}$, if $\mathcal{L}(x,y,y')=\mathcal{L}(y,y')$, that is $\mathcal{L}$ is independent of $x$, then 
    \begin{equation}
        \frac{\partial L}{\partial y'}y' - \mathcal{L} = \text{constant}.
    \end{equation}
\end{theorem}
\begin{proof}
    Differentiate the quantity
    \begin{align}
        \frac{d}{dx}\left(\frac{\partial \mathcal{L}}{\partial y'}y'- \mathcal{L}\right) &= \frac{d}{dx}\frac{\partial L}{\partial y'}y' + \frac{\partial \mathcal{L}}{\partial y'}{y''} - \frac{\partial \mathcal{L}}{\partial y}y' - \frac{\partial \mathcal{L}}{\partial y'}y'' \\ 
        &= \left(\frac{d}{dx} \frac{\partial \mathcal{L}}{\partial y'}-\frac{\partial \mathcal{L}}{\partial y}\right)y' = 0 
    \end{align}
\end{proof}
Applying this theorem, and using,
\begin{align*}
    F(y,y') = \sqrt{\frac{1+(y')^2}{y}},
\end{align*}
we get 
\begin{align*}
    a &= \frac{\partial F}{\partial y'} - F \\ 
    0 &= \frac{y'}{\sqrt{y}\sqrt{1+(y')^2}} - \sqrt{\frac{1+(y')^2}{y}}.
\end{align*}
After rearranging, we get a first order differential equation that is equivalent to the 
\subsection{Constrained Systems}
Sometimes, we have a constraint. For example in a pendulum when working with $x$ and $y$ coordinates, we can introduce a \emf{holonomic constraint:}
\begin{equation}
    x^2+y^2-l^2=0.
\end{equation}
In general, holonomic constraints are in the form of $f(q_1,\dots,q_n)=0.$
\end{document}