\section{Lecture Seven: Systems of equation}
\begin{itemize}
    \item Suppose we wish to solve the system of equation:
    \begin{align}
        x-2y&=1 \\ 
        3x+2y &= 11
    \end{align}
    We can do this numerous ways. If we look at the row picture:
    \begin{center}
        \begin{tikzpicture}
        \begin{axis}[
        legend pos=outer north east,
        title=Row Picture,
        axis lines = box,
        xlabel = $x$,
        ylabel = $y$,
        variable = t,
        trig format plots = rad,
        ]
        \addplot [
            domain=0:5,
            samples=70,
            color=blue,
            ]
            {0.5*x-0.5};
        \addlegendentry{$ x-2y=1$}
        \addplot [
            domain=0:5,
            samples=70,
            color=red,
            ]
            {-1.5*x+5.5};
        \addlegendentry{$3x+2y=11$}
        \end{axis}
        \end{tikzpicture}
    \end{center}
    where the intersection point is at $(3,1)$.
    \item Now let's look at the column picture. Instead of seeing two equations, we are going to express these two equations as a single vector equation:
    \begin{equation}
        x\begin{bmatrix}
            1\\3
        \end{bmatrix}
        +y\begin{bmatrix}
        -2\\2
        \end{bmatrix}=\begin{bmatrix}
            1\\11
        \end{bmatrix}
        \label{eq:}
    \end{equation}
    The solution requires us to find linear combinations of the vectors $\begin{bmatrix}
        1\\3
    \end{bmatrix}$ and $\begin{bmatrix}
        -2\\2
    \end{bmatrix}$ that equal $\begin{bmatrix}
        1\\11
    \end{bmatrix}$
    \begin{center}
        \begin{tikzpicture}[scale=0.5]
            \draw[->] (-2,0) -- (10,0);
            \draw[->] (0,-2) -- (0,10);

            \draw[thick,->] (0,0) -- (1,3) node[right] {$\begin{bmatrix}
                1\\3
            \end{bmatrix}$};
            
            \draw[thick,->] (0,0) -- (-2,2) node[left] {$\begin{bmatrix}
                -2\\2
            \end{bmatrix}$};
            
            \draw[thick,->] (0,0) -- (1,11) node[right] {$\begin{bmatrix}
                1\\11
            \end{bmatrix}$};
        \end{tikzpicture}
    \end{center}
    and the linear combination that gives the solution is:
    \begin{equation}
        3\begin{bmatrix}
            1\\3
        \end{bmatrix}
        +1\begin{bmatrix}
        -2\\2
        \end{bmatrix}=\begin{bmatrix}
            1\\11
        \end{bmatrix}
        \label{eq:}
    \end{equation}
    \item A little later in the course, we will use matrices to represent these systems, for example:
    \begin{equation}
        \underbrace{\begin{bmatrix}
            1&-2\\3&2
        \end{bmatrix}}_\text{Matrix A}\underbrace{\begin{bmatrix}
            x\\y
        \end{bmatrix}}_{\text{vector } \vec{x}}=\underbrace{\begin{bmatrix}
            1\\11
        \end{bmatrix}}_{\text{vector }\vec{b}}
    \end{equation}
    where the matrix-vector product $A\vec{x}$ on the LHS is defined to be the equivalent of the column picture, e.g.:
    \begin{equation}
        A\vec{x} = x\begin{bmatrix}
            1\\3
        \end{bmatrix}
        +y\begin{bmatrix}
        -2\\2
        \end{bmatrix}
        \label{eq:}
    \end{equation}
     \item This leads to the dot product rule of calculating $A\vec{x}$:
     \begin{equation}
         \underbrace{\begin{bmatrix}
            &&&&& \\ 
            &&&&& \\ 
            a_i&b_i&c_i&d_i&e_i&f_i \\ 
            &&&&& \\ 
            &&&&& \\ 
            &&&&&
        \end{bmatrix}}_{A}
         \underbrace{\begin{bmatrix}
            a_j \\ b_j \\ c_j \\ d_j \\ e_j \\ f_j
        \end{bmatrix}}_{\vec{x}} =
         \underbrace{\begin{bmatrix}
            \\ \\ a_ia_j + b_ib_j +c_ic_j+d_id_j+e_ie_j+f_if_j \\ \\ \\ \\
         \end{bmatrix}}_{A\vec{x}}
     \end{equation}
     \begin{idea}
         What is a matrix? In the simplest of terms, it is a rectangular array of numbers, such as:
         \begin{equation}
             A = \begin{bmatrix}
                 4&8&3\\2&1&-9
             \end{bmatrix}
             \label{eq:}
         \end{equation}
        which has two rows and three columns. $\therefore$ this isa $2\times 3$ matrix. The general way to denote a matrix is via:
        \begin{equation}
            A = \begin{bmatrix}
                a_{11} & a_{12} & a_{13} \\ a_{21} & a_{22} & a_{23}
            \end{bmatrix}
            \label{eq:}
        \end{equation}
        where $a_{ij}$ is the entry in row $i$ and column $j$. Addition and scalar multiplication works in the same way.
     \end{idea}
     \item We can also multiply two matrices $A$ and $B$ if and only if $A$ has $n$ columns and $B$ has $n$ rows.
\end{itemize}