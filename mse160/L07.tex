\section{Technical Mechanical Behaviour}
\begin{itemize}
    \item The ultimate tensile strength, denoted as $UTS$ or $\sigma_\text{UTS}$ is the highest stress that a material can occur in a material.
    \item The \textit{proportional limit} describes the point on the tensile stress-strain curve where elastic deformations no longer occur.
    \item Along with the yield strength, it is difficult to define this objectively. Instead, we use the following convention:
    \begin{definition}
        The yield strength is the stress value at which if the material is unloaded at that point, the strain will be $\epsilon = 0.002 = 0.2\%$.
    \end{definition}
    \item Uniform deformation occurs when the thickness of the material is uniform. Non-uniform deformation occurs when an increase in strain results in a decrease in stress. This is known as \textbf{necking}.
    \item Crystalline imperfections are called \textbf{dislocations}. The dislocation density can be reduced by heating (annealing), and plastic deformation can increase the density.
    \item Dislocations are known as \textbf{linear imperfection} as they can form a ``tunnel'' like bonding environment where there is a dislocation line.
    \item Plastic deformation is the result of the step by step movement of linaer imperfections:
    \begin{enumerate}
        \item As a force is applied, the bonds to the right of the dislocation stretch and eventually break
        \item A new set of bonds re-forms
        \item This breaking and re-forming happens again
        \item And again, and so forth.
    \end{enumerate}
    \begin{warning}
        Remember that this mechanism is \textit{not} the breaking of all bonds at the same time on a plane of atoms.
    \end{warning}
\end{itemize}