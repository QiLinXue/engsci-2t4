\section{Technical Mechanical Behaviour}
\begin{itemize}
    \item The ultimate tensile strength, denoted as $UTS$ or $\sigma_\text{UTS}$ is the highest stress that a material can occur in a material.
    \item The \textit{proportional limit} describes the point on the tensile stress-strain curve where elastic deformations no longer occur.
    \item Along with the yield strength, it is difficult to define this objectively. Instead, we use the following convention:
    \begin{definition}
        The yield strength is the stress value at which if the material is unloaded at that point, the strain will be $\epsilon = 0.002 = 0.2\%$.
    \end{definition}
    \item Uniform deformation occurs when the thickness of the material is uniform. Non-uniform deformation occurs when an increase in strain results in a decrease in stress. This is known as \textbf{necking}.
\end{itemize}