\section{Crystallographic Planes and Directions}
\begin{itemize}
    \item We develop a set of notation to describe directions. It comes with a set of rules:
    \begin{itemize}
        \item Translate vector, if it simplifies things.
        \item Determine projection onto $x$, $y$, $z$.
        \item Reduce to lowest integers.
        \item Enclose in square brackets (negative signs are moved above, no commas)
    \end{itemize}
    \begin{example}
        We can find the crystallographic directions of both the blue and the red vectors in the figure below.
        \begin{center}
            \incfig{crystallographic_intro}
        \end{center}
        For the blue vector, we first let the tail by the origin. The vector travels $1$ in the negative $x$ direction, $0$ in the $y$ direction, and $-1$ in the negative $z$ direction, which gives $[\bar{1}\,0\,\bar{1}]$.
        \vspace{2mm}

        For the red vector, it travels in the negative $y$ direction for $0.5$ and in the negative $z$ direction for $1$ so we get, after getting rid of fractions: $[0\,\bar{1}\,\bar{2}]$.
    \end{example}
    \item To denote a family of directions, we can use braket notation. All face diagonals can be written as:
    \begin{equation}
        <0\,1\,1>
    \end{equation}
    which includes:
    \begin{equation}
        [0\,1\,1],\, [0\,\bar{1}\,1],\, [0\,1\,\bar{1}],\, [1\,0\,1], \dots
    \end{equation}
    \item To determine crystallographic planes, we use \textbf{crystallographic planes}. It uses the following set of rules:
    \begin{itemize}
        \item Translate plane so origin is not on the plane (defining new origin).
        \item Determine distance to intercept plane by travelling along each axis, from the origin.
        \item Take the reciprocol of the distance.
        \item Enclose in parentheses (use $h$, $k$, $\ell$). Negatives go above, no commas.
    \end{itemize}
    \begin{example}
        We can represent the following plane using Miller indices:
        \begin{center}
            \incfig{miller-1}
        \end{center}
        We can start from the origin and travel a length of $1$ in both the positive $x$, $y$, and $z$ direction. Taking the reciprocal, we still have $1,1,1$, so the final miller index is $(1\,1\,1)$.
    \end{example}
    \begin{example}
        We can represent the following plane using Miller indices:
        \begin{center}
            \incfig{miller-2}
        \end{center}
        We can start from the origin and travel a length of $1$ in the $x$ direction, $1/2$ in the $y$ direction, and $\infty$ in the $z$ direction. Taking the reciprocal we get $(1\,2\,0)$.
    \end{example}
    \item We can represent \textbf{families of planes} using curly braces. For example, faces of a cube can be represented by:
    \begin{equation}
        \{0\,0\,1\}
    \end{equation}
    which represent the following planes:
    \begin{equation}
        (1\,0\,0),\, (0\,1\,0),\, (0\,0\,1)
    \end{equation}
    corresponding to the below blue, yellow, and red plane respectively:
    \begin{center}
        \incfig{miller-3}
    \end{center}
\end{itemize}