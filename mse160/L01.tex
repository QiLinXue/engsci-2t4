\section{Introduction}
\subsection{Types of Material}
\begin{itemize}
    \item There are three classes of material (though not all materials fall under these categories):
    \begin{itemize}
        \item Metals
        \item Ceramics 
        \item Polymers
    \end{itemize}
    \item \textbf{Metals} (e.g. Fe, Cr, Cu, Zn, Al) are held together with \emph{mellatic} bonds and is described by \textbf{bond theory}.
    \item \textbf{Ceramics} (e.g. poreclain, concrete) are held together with \emph{ionic} bonds and are \emph{brittle}. A lot of them are metal oxides.
    \item \textbf{Polymer} (Teflon\textregistered, Gore-tex\textregistered, polyethylene) \emph{tend} to be from \emph{covalent bonds}
    \begin{warning}
        The word plastic actually describes a material property, and not a material type. There are plastics that are not polymers.
    \end{warning}
    \item Examples of materials that do not fall under this classification scheme include wood, skin, superconductors, and more.
\end{itemize}
\subsection{Elastic Behaviour}
\begin{itemize}
    \item Hooke's law tells us that $F=-k\Delta x$, where $\Delta x$ is the displacement from equilibrium.
    \item \textbf{Engineering stress} is defined as $\sigma = \frac{F}{A_0}$ where $A_0$ is the \textit{initial} (unloaded) cross-sectional area.
    \begin{warning}
        Due to material properties, the cross sectional area of a spring can change as it elongates or compresses, so the engineering stress only refers to the initial cross sectyional area. The \textit{true stress} refers to the force divided by the real area.
    \end{warning}
    \item \textbf{Engineering strain} is defined as $\varepsilon = \frac{\Delta \ell}{\ell_0}$ and the two are related via the \textbf{Young's Modulus}:
    \begin{equation}
        \sigma = E\varepsilon
    \end{equation}
    \item There are two possible definitions for elastic deformation. When viewing it from a macroscopic perspective:
    \begin{definition}
        During elastic deformation, the sample dimensions return to their original dimensions upon unloading.
    \end{definition}
    but it is also possible to view it from a microscopic perspective:
    \begin{definition}
        During elastic deformation, atoms return to their original positions upon unloading.
    \end{definition}
\end{itemize}
\subsection{Simple Model for Bonding in solids}
\begin{itemize}
    \item A crudge (but quite accurate) model is to assume nearby atoms in a solid are connected by springs. (This is actually Einstein's model of solid, except he modeled the interactions as quantum harmonic oscillators)
    \begin{idea}
        A more realistic model would be using the Lennard-Jones potential, which gives the force between two atoms as:
        \begin{equation}
            V = -\frac{a_1}{r^{13}}+\frac{a_2}{r^7}
        \end{equation}
        and is graphically represented below (here, $a_1=5$ and $a_2=3$ for illustration purposes only)
        \begin{center}
            \begin{tikzpicture}
            \begin{axis}[
            legend pos=outer north east,
            title=Lennard Jones Force,
            axis lines = box,
            xlabel = $r$,
            ylabel = $F$,
            variable = t,
            trig format plots = rad,
            ]
            \addplot [
                domain=1:2,
                samples=70,
                color=blue,
                ]
                {-5/x^13+3/x^7};            
            \end{axis}
            \end{tikzpicture}
        \end{center}
    \end{idea}
    When the two atoms are close to each other, the force scales roughly linearly with displacement, which is exactly the description of Hooke's Law.
    \item Specifically, the Young's Modulus can be recovered by defining it as:
    \begin{equation}
        E \propto \frac{dF}{dr} \Big|_{r=r_0}
    \end{equation}
    where $r_0$ is the equilibrium distance and is only dependent on the material. Permanently deforming a metal will not change its Young's Modulus.
\end{itemize}
\subsection{Getting a stress-strain curve}
\begin{itemize}
    \item The tensile specimen is in a \textbf{dogbone} shape as illustrated below:
    \begin{figure}[ht]
        \centering
        \incfig{dogbone}
    \end{figure}
\end{itemize}
\subsection{Poisson's Ratio and Shear}
\begin{itemize}
    \item When a material deforms, it does not deform in only one direction. The \textbf{poisson's ratio} $\nu$ relates the strain in all three directions:
    \begin{equation}
        \nu = -\frac{\varepsilon_R}{\varepsilon_Z}=-\frac{\varepsilon_x}{\varepsilon_z}=-\frac{\varepsilon_y}{\varepsilon_z}
    \end{equation}
    for a cylindrically symmetrical material.
    \item Shear stress is defined as
    \begin{equation}
        \tau = \frac{F}{A_0}
    \end{equation}
    and shear strain is defined as:
    \begin{equation}
        \gamma = \frac{\Delta \ell}{\ell_0}
    \end{equation}
    \item Similarly, shear stress and strain is related via the shear modulus $G$:
    \begin{equation}
        \tau = G\gamma
    \end{equation}
    \item The Young's modulus and the shear modulus is related via the poisson ratio:
    \begin{equation}
        E = 2G(1+\nu)
    \end{equation}
\end{itemize}