\documentclass{article}
\usepackage{qilin}
\tikzstyle{process} = [rectangle, rounded corners, minimum width=1.5cm, minimum height=0.5cm,align=center, draw=black, fill=gray!30, auto]
\title{Phy450: Relativistic Electrodynamics \\ Quiz 6}
\author{QiLin Xue}
\date{Spring 2023}
\usepackage{mathrsfs}
\usetikzlibrary{arrows}
\usepackage{stmaryrd}
\usepackage{accents}
\newcommand{\ubar}[1]{\underaccent{\bar}{#1}}
\usepackage{pgfplots}
\usepackage{siunitx}
\usepackage{esint}
\begin{document}

\maketitle
We can use Leibniz's rule,
\begin{align}
    \partial_\mu(x_\nu A^{\mu\nu} S) &= (\partial_\mu x_\nu) A^{\mu\nu}S + x_\nu (\partial_{\mu}A^{\mu\nu}) \partial_\mu S + x_\nu A^{\mu\nu}(\partial_\mu S) \\
    &= (\partial_\mu x^\lambda g_{\lambda\nu}) A^{\mu\nu}S  + x_\nu A^{\mu\nu}(\partial_\mu S) \\ 
    &= \delta^{\lambda}_{\mu} g_{\lambda\nu} A^{\mu\nu}S+ x_\nu A^{\mu\nu}(\partial_\mu S) \\ 
    &= g_{\mu\nu} A^{\mu\nu}S+ x_\nu A^{\mu\nu}(\partial_\mu S), \label{eq:final}
\end{align}
where $\partial_\mu A^{\mu\nu}=0$ since $A$ is a constant tensor. We claim that this first term is zero. Let $\xi = g_{\mu\nu} A^{\mu\nu}.$ Then:
\begin{align}
    \xi &= \frac{1}{2}(\xi+\xi) \\ 
       &= \frac{1}{2}\left(g_{\mu\nu} A^{\mu\nu} + g_{\nu\mu} A^{\mu\nu}\right) \\ 
       &= \frac{1}{2}\left(g_{\mu\nu} A^{\mu\nu} - g_{\nu\mu} A^{\nu\mu}\right) \\ 
       &= \frac{1}{2}\left(g_{\mu\nu} A^{\mu\nu} - g_{\mu\nu} A^{\mu\nu}\right) \\ 
       &= \frac{1}{2}(\xi - \xi) \\ 
       &= 0.
\end{align}
The second line is due to the symmetry of $g,$ third line is due to anti-symmetry of $A.$ In general, contracting a symmetric tensor with an antisymmetric one will give us zero. Therefore, only the second term of equation \ref{eq:final} survives, and we are left with 
\begin{equation}
    \partial_\mu(x_\nu A^{\mu\nu} S)  = x_\nu A^{\mu\nu}(\partial_\mu S).
\end{equation}
In this problem, we only assumed the anti-symmetry of $A^{\mu\nu},$ the symmetry of $g^{\mu\nu},$ and the fact that $A$ is constant. Therefore, this derivation will also hold for general curved spacetimes.
\end{document}