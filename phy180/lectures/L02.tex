\section{Lecture 2: Calculus of Motion}
\begin{itemize}
    \item The position, velocity, and acceleration are related by derivatives and integrals.
    \begin{equation}
        x(t) \xrightarrow{\frac{d}{dt}} v(t) \xrightarrow{\frac{d}{dt}} a(t)
        \label{eq:}
    \end{equation}
    \begin{equation}
        x(t) \xleftarrow{\int \dd{t}} v(t) \xleftarrow{\int \dd{t}} a(t)
        \label{eq:}
    \end{equation}

    \item Each quantity is consisted of certain dimensions that are not dependent on whether the quantity is a scalar or vector.
    \begin{itemize}
        \item Dimensions of length are represented by $L$.
        \item Dimensions of time are represented by $T$.
        \item Dimensions of mass are represented by $M$.
    \end{itemize}
    \begin{example}
        Suppose $x(t)=2 \text{ m} + (3 \text{ m/s}^3)t^3$. What is $v(t)$?
        \vspace{2mm}

        We can take derivatives, and dropping the units, we get:
        \begin{align}
            v &= \frac{dx}{dt} \\ 
            &= \frac{d}{dt}(2+3t^3) \\
            &= 6t^2
        \end{align}
    \end{example}
    \begin{example}
        Let $v=3\text{ m/s}$. What is $x(t)$? suppose at $t=0$, we have $x=2 \text{ m}$.
        \vspace{2mm}

        To solve, we need to integrate with respect to time:
        \begin{align}
            x(t) &= \int 3 \dd{t} \\ 
            &= 3t+C
        \end{align}
        We can determine the integration constant by plugging in the relationship $x(t=0)=3(0)+C$ which gives $C=2 \text{ m}$. Therefore:
        \begin{equation}
            x(t) = (3 \text{ m/s})t+2 \text{ m}
            \label{eq:}
        \end{equation}
    \end{example}
    \item A definite integral represents the area under a graph between two certain points. We can compare this to the indefinite integral 
    \begin{equation}
        F(t)=\int f(t) dt
        \label{eq:}
    \end{equation}
    such that
    \begin{equation}
        \int_a^b \dd{t} = F(b)-F(a)
        \label{eq:}
    \end{equation}
    known as the fundamental theorem of calculus
    \begin{example}
        Suppose $f = 5t^2$. What is $\int_{t=1}^{t=2} f(t) \dd{t}$?
        \vspace{2mm}

        We first integrate $f(t)$:
        \begin{equation}
            \int 5t^2 dt = \frac{5}{3}t^3 + C \equiv F(t)
            \label{eq:}
        \end{equation}
        We then use the fundamental theorem of calculus:
        \begin{equation}
            F(2)-F(1)=\left(\frac{5}{3}2^3+\cancel{C}\right)-\left(\frac{5}{3}1^3+\cancel{C}\right)=\frac{35}{3}
            \label{eq:}
        \end{equation}
    \end{example}
    \item We can apply this to position, velocity, and acceleration. If we want to determine:
    \begin{equation}
        \int_{t_i}^{t_f} v(t) \dd{t} = F(t_f)-F(t_i) = x_f-x_i = \Delta x
        \label{eq:}
    \end{equation}
    Similarly, the definite integral of acceleration gives the change in velocity:
    \begin{equation}
        \int_{t_i}^{t_f} a(t) \dd{t} = \Delta v(t)
        \label{eq:}
    \end{equation}
    \item Similarly, there are several position functions $x(t)$ that lead to the same $v(t)$. Information is lost.
    \item We can interpret these results graphically.
    \begin{example}
        Suppose we have the following $x(t)$ curve, we can draw the corresponding $v(t)$ curve:

        \begin{center}
            \begin{tikzpicture}
            \begin{axis}[
            legend pos=outer north east,
            title=Position as a function of time,
            axis lines = box,
            xlabel = $x$,
            ylabel = $y$,
            variable = t,
            trig format plots = rad,
            ]
            \addplot [
                domain=0:1.5,
                samples=70,
                color=blue,
                ]
                {-4*x*(x-1)^3+1};            
            \end{axis}
            \end{tikzpicture}
        \end{center}
        \begin{center}
            \begin{tikzpicture}
            \begin{axis}[
            legend pos=outer north east,
            title=Position as a function of time,
            axis lines = box,
            xlabel = $x$,
            ylabel = $y$,
            variable = t,
            trig format plots = rad,
            ]
            \addplot [
                domain=0:1.5,
                samples=70,
                color=blue,
                ]
                {-1*(x-1)^3-3*(x-1)^2*x};  
            
            \draw[dotted,thick] (-0.15,0) -- (1.65,0);
            \end{axis}
            \end{tikzpicture}
        \end{center}
        Pay especially close attention to how the points line up when $v=0$.
    \end{example}
    \item We can go in the opposite direction as well by looking at how the area is changing via time graphically.
    \item For the special case where $v(t)=v_0$, we can differentiate to get:
    \begin{equation}
        a(t) = \frac{dv}{dt}=\frac{d}{dt}v_0=0
        \label{eq:}
    \end{equation}
    And integrate it to get:
    \begin{equation}
        \Delta x = \int_{t_i}^{t_f} v_0 \dd{t} = v_0 \Delta t \implies x(t)=v_0t+x_0
    \end{equation}
    \item Similarly we can take a constant acceleration $a(t)=a_0$ so that we can determine that:
    \begin{equation}
        v(t) = \int a_0 \dd{t} = a_0t+v_0
        \label{eq:}
    \end{equation}
    and the position is given by:
    \begin{equation}
        x(t) = \int (a_0t+v_0) \dd{t} = \frac{1}{2}a_0t^2+v_0t+x_0
        \label{eq:}
    \end{equation}
    
    
\end{itemize}
% -1\left(x-1\right)^{3}-3\left(x-1\right)^{2}x