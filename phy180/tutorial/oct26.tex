\documentclass{article}
\usepackage{phy}

\title{PHY180: Classical Mechanics \\ Tutorial Questions}
\author{QiLin Xue}
\date{\today}
\usepackage{siunitx}
\usepackage{mathrsfs}
\usetikzlibrary{arrows}
\begin{document}

\maketitle
\subsection*{Problem 1}
Assume that the elevator undergoes unciform acceleration, then the scale reading as it moves up with acceleration $a$ should be:
\begin{equation}
    W'=m(g+a)
    \label{eq:}
\end{equation}
Letting $m=W/g$ be the mass of the student with $W$ being the weight before the elevator moves, then:
\begin{equation}
    W'=\frac{W}{g}(g+a) \implies a = g\left(\frac{W'}{W}-1\right)
    \label{eq:}
\end{equation}
So the top speed would be:
\begin{equation}
    v_\text{max} = a\Delta t = g\Delta t\left(\frac{W'}{W}-1\right) \approx 2.1\si{\meter\per\second}
    \label{eq:}
\end{equation}
\subsection*{Problem 2}
At steady state, each mass has the same acceleration of:
\begin{equation}
    a = \frac{F}{2m}
    \label{eq:}
\end{equation}
For mass $B$, the only force acting on it is the spring force, so this must mean from Newton's second law:
\begin{equation}
    m\left(\frac{F}{2m}\right)=k\Delta x \implies \Delta x = \frac{F}{2k}
    \label{eq:}
\end{equation}
so the distance between the two carts is $\boxed{L+\frac{F}{2k}}$. At the instant we let go, box $B$ would still have the same acceleration and box $A$ would have the same acceleration as box $B$ but in the opposite direction.

\noindent After I stop pulling, I expect the carts to undergo SHM around their center of mass, which moves at a constant velocity. When cart $B$ is moving at its fastest, the spring is not stretched.

\subsection*{Problem 3}
\textbf{(a)} The quantity $F_0$ represents the initial force at $t=0$.

\noindent \textbf{(b)} The time integral of the force gives us the change in momentum:
\begin{equation}
    p_f - 0 = \int_0^\tau F_0 e^{-t/\tau} \dd{t} = -F_0\tau e^{-t/\tau}\Big|^{t=\tau}_{t=0} = F_0\tau \left(1-e^{-1}\right)
    \label{eq:}
\end{equation}
\noindent \textbf{(c)} We do the same thing to get:
\begin{equation}
    p_f = \int_0^{5\tau} F_0e^{-t/\tau}\dd{t} = F_0\tau\left(1-e^{-1/5}\right)
    \label{eq:}
\end{equation}
\textbf{(d)} As $t\to\infty$, we get:
\begin{equation}
    p_f = F_0\tau
    \label{eq:}
\end{equation}
\textbf{(e)} Suppose this time occurs at $t=N\tau$, then the momentum after this time has elapsed is:
\begin{equation}
    fF_0\tau = F_0\tau\left(1-e^{-1/N}\right)
    \label{eq:}
\end{equation}
where $f=95\%$. Solving this, we get:
\begin{equation}
    e^{-1/N} = 1-f \implies -\frac{1}{N} = \ln(1-f) \implies N = \frac{1}{\ln\left(\frac{1}{1-f}\right)}=0.334
    \label{eq:}
\end{equation}
for a total time of:
\begin{equation}
    t=N\tau = 0.167\si{\milli\second}.
    \label{eq:}
\end{equation}


\end{document}
