\documentclass{article}
\usepackage{qilin}
\tikzstyle{process} = [rectangle, rounded corners, minimum width=1.5cm, minimum height=0.5cm,align=center, draw=black, fill=gray!30, auto]
\title{MAT347: Groups, Rings and Fields \\ Midterm 2 Review}
\author{QiLin Xue}
\date{Fall 2021}
\usepackage{mathrsfs}
\usetikzlibrary{arrows}
\usepackage{stmaryrd}
\usepackage{accents}
\newcommand{\ubar}[1]{\underaccent{\bar}{#1}}
\usepackage{pgfplots}
\numberwithin{equation}{section}

\begin{document}

\maketitle
\tableofcontents
\newpage
\section{Basic Definitions}
Of sets:
\begin{itemize}
    \item \emf{Ring:} Set $(R,+,\times)$ where $(R,+)$ is abelian group, $\times$ is associative and distributive.
    \item \emf{Integral domain:} No zero divisors
    \item \emf{Field:} Finite integral domain
    \item \emf{Subring:} Subgroup of $R$ that is closed under multiplication.
    \item \emf{Ideal:} subring that is closed under left/right multiplication, i.e. $rI \subseteq I.$
    \item \emf{Quotient Ring:} $R/I$ where $I$ is ideal
\end{itemize}
Of elements:
\begin{itemize}
    \item \emf{Zero divisor:} an element that divides $0$
    \item \emf{Unit:} an element that has an inverse
\end{itemize}
Other:
\begin{itemize}
    \item \emf{Ring homomorphism:} $\phi(a+b)=\phi(a)+\phi(b),\phi(ab)=\phi(a)\phi(b)$
\end{itemize}
\section{Basic Theorems}
\begin{itemize}
    \item \emf{First Isomorphism Theorem:} $\varphi:R\to S,$ then $\ker \varphi$ is an ideal and 
    \begin{equation}
        R/\ker\varphi \cong \varphi(R)
    \end{equation}
    \emf{Natural projection} is $R\to R/I$ defined by $r\mapsto r + I.$ Every ideal is the kernel of a ring homomorphism.
    \item \emf{}
\end{itemize}
\end{document}