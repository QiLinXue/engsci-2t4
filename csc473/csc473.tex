\documentclass{article}
\usepackage{qilin}
\tikzstyle{process} = [rectangle, rounded corners, minimum width=1.5cm, minimum height=0.5cm,align=center, draw=black, fill=gray!30, auto]
\title{CSC473: Algorithms}
\author{QiLin Xue}  
\date{Spring 2022}
\usepackage{mathrsfs}
\usetikzlibrary{arrows}
\usepackage{stmaryrd}
\usepackage{accents}
\newcommand{\ubar}[1]{\underaccent{\bar}{#1}}
\usepackage{pgfplots}
\numberwithin{equation}{section}
\usetikzlibrary{quantikz}
\usepackage[american]{circuitikz}
\newcommand{\equals}{=}
\usepackage{algpseudocode}

\begin{document}

\maketitle
\tableofcontents
\newpage
\section{Global Min Cut}
\begin{itemize}
    \item \textbf{Input:} Undirected, unweighted connected graph $G=(V,E).$
    \item \textbf{Output:} Smallest set of edges that disconnects $G.$
\end{itemize}
\begin{center}
    \begin{tikzpicture}[x=2cm, y=2cm]
        \tikzstyle{vertex}=[circle,fill=black!25,minimum size=20pt,inner sep=2pt]

        \draw[dashed] (-1,-1) rectangle (1.3,1) node[above] {$S$};
        \draw[dashed] (1.7,-1) rectangle (4,1) node[above] {$V\setminus S$};

        \node[vertex] (G_1) at (0,0) {};
        \node[vertex] (G_3) at (1,0) {a};
        \node[vertex] (G_2) at (0.5,0.5) {};
        \node[vertex] (G_4) at (0.5,-0.5) {};
        \node[vertex] (G_5) at (2,0) {b};
        \node[vertex] (G_7) at (3,0) {};
        \node[vertex] (G_6) at (2.5,0.5) {};
        \node[vertex] (G_8) at (2.5,-0.5) {};

        \draw (G_1) -- (G_2) -- (G_3) -- (G_4) -- (G_1) -- cycle;
        \draw (G_5) -- (G_6) -- (G_7) -- (G_8) --(G_5) -- cycle;

        % % \draw (G_5) -- (G_2) -- (G_4) -- cycle;
        \draw (G_3) -- (G_5);
    \end{tikzpicture}

    \begin{tikzpicture}[x=2cm, y=2cm]
        \tikzstyle{vertex}=[circle,fill=black!25,minimum size=20pt,inner sep=2pt]
        \node[vertex] (G_1) at (0,0) {};
        \node[vertex] (G_3) at (1,0) {};
        \node[vertex] (G_2) at (0.5,0.5) {c};
        \node[vertex] (G_4) at (0.5,-0.5) {};

        \draw (G_1) -- (G_2) -- (G_3) -- (G_4) -- (G_1) -- cycle;

        % % \draw (G_5) -- (G_2) -- (G_4) -- cycle;
        \draw (G_1) -- (G_3);
    \end{tikzpicture}
\end{center}
For the first graph, disconnect the edge separating $a$ and $b$ and for the second graph, disconnect the edges around node $c.$ We wish to instead return $S,T \subseteq V$ with $S\cap T = \emptyset$ such that the desired edges are simply 
\begin{equation*}
    E(S,T) = \{(u,v) \in E : u\in S, v\in T\}.
\end{equation*} 
The global min-cut is therefore the output $S\subseteq V$ such that $S \neq \emptyset,V$ where $|E(S,V\setminus S)|$ is minimized.

Notice that this is very similar to the max flow problem. Consider the source-sink min-cut problem, where the input is $G=(V,E)$ as before, and two nodes $s,t\in V.$ The desired output is the same $S$ with the same minimization, except with one extra constraint: that is, $s\in S$ and $t\notin S.$ Any max flow algorithm can determine this.

To solve the global min cut problem, we can fix $t$ and choose $s$ to be any of the other $V-1$ nodes, then run the max flow algorithm on each case. The best max flow algorithm (Which came out in a recent paper) does max flow in $O(m^{1+O(1)}) \approx O(n^2),$ so this algorithm would give a time complexity of $O(n^3).$ However, we can improve this to $O(n^2\log^2(n)).$
\end{document}