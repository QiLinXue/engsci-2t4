\documentclass{article}
\usepackage{qilin}
\tikzstyle{process} = [rectangle, rounded corners, minimum width=1.5cm, minimum height=0.5cm,align=center, draw=black, fill=gray!30, auto]
\title{MAT448: Algebraic Geometry}
\author{QiLin Xue}
\date{Spring 2022}
\usepackage{mathrsfs}
\usetikzlibrary{arrows}
\usepackage{stmaryrd}
\usepackage{accents}
\newcommand{\ubar}[1]{\underaccent{\bar}{#1}}
\usepackage{pgfplots}
\numberwithin{equation}{section}

\begin{document}

\maketitle
\tableofcontents
\newpage
\section{Introduction}
Algebraic geometry is the study of geometric objects defined by polynomials. Consider $x^2+y^2=1$ and $x^2+y^2=-1$ plotted in $\mathbb{R}^2.$ These two algebraic varieties are different as real varieties but the same (namely, isomorphic) as complex varieties. For example, consider the coordinate change $(x,y) \leftrightarrow (ix,iy).$

Let $k$ be an algebraically closed field, i.e. $k=\mathbb{C}.$ Consider an element of the affine space 
\begin{equation}
    a \in \mathbb{A}^n = \{a=(a_1,\dots,a_n)|a\in k\} = \text{affine space}/k
\end{equation}
and polynomial
\begin{align}
    f \in R = k[x_1,\dots,x_n]
\end{align}
acting on $\mathbb{A}^n.$ Then, $f(a) = f(a_1,\dots,a_n) \in k$ and we wish to examine 
\begin{equation}
    V(f) = \{a\in \mathbb{A}^n | f(a)=0\},
\end{equation}
or perhaps a collection of polynomials, known as an \emf{affine algebraic set}
\begin{equation}
    V(f_1,\dots,f_r) = \{a\in \mathbb{A}^n | f_i(a) =0\forall i\} \subseteq \mathbb{A}^n.
\end{equation}
We could also consider a subset $S\subset R_n,$ so we can take the ideal generated by $S$ in $R_n$ as $I(S).$ We claim that 
\begin{equation}
    V(S) = V(I(S)).
\end{equation}
Note that if $S=\{0\},$ then $V(S)=\mathbb{A}^n.$ Also, $V((1))=V(R_n)=\emptyset.$ Here, $(1)$ refers to the ideal generated by the identity $1.$
\section{Commutative Algebra}
All our rings will be commutative with identity $1\neq 0.$
\begin{proposition}
    $R$ is Noetherian iff equivalently:
    \begin{enumerate}[label=(\alph*)]
        \item every ideal $I \subset R$ is finitely generated,
        \item every ascending chain (AAC) of ideals terminates,
        \item every non-empty set of ideals contains maximal elements.
    \end{enumerate}
\end{proposition}
Recall that an ascending chain of ideals is the following $I_1 \subseteq I_2 \subseteq I_3 \subseteq \cdots.$
\begin{proof}
    (a)$\implies$(b): Suppose we have an ascending chain. Let $I= \bigcup_n I_n$ be an ideal in $R.$ We know that $I$ is finitely generated, i.e. $I=(r_1,\dots, r_k).$ There exists an $N$ such that $r_i \in I_N$ for all $i,$ so that we must have $I_N=I.$
    \vspace{2mm}

    (b)$\implies$(c): Let $\Sigma$ be a non-empty set of ideals. Suppose for the sake of contradiction there are no maximal elements. Take any $I_0\in \Sigma.$ Since it is not maximal, we can find an $I_1$ such that $I_0 \subseteq I_1,$ and this doesn't terminate. But by assumption, we know that the ACC terminates, so contradiction.
    \vspace{2mm}

    (c)$\implies$(a): Given $f\in I \subset R$ and $\Sigma = \{J=\text{ideal in R s.t. }J\subset I,J=f.g\}.$ Take $J\in \Sigma$ to be a maximum element. We know that $J\subset I.$  Take $f\in I \setminus J.$ Then $J \subset (J,f) \subset I,$ which is a contradiction. The only way to resolve this is if $J=I$ (i.e. not a proper subset).
\end{proof}
Also,
\begin{proposition}
    \begin{enumerate}[label=(\roman*)]
        \item If $R$ is Noetherian, and $I\subset R$ is an ideal, then $R/I$ is Noetherian.
        \item If $R$ is a Noetherian integral domain (i.e. we can make a fraction field $R\subset Frac(R)$) with $S\subset R$ and $0\neq S.$ Then $S^{-1}R$ is Noetherian. 
    \end{enumerate}
\end{proposition}
\begin{theorem}
    Hilbert's Theorem: If $R$ is Noetherian, then $R[x]$ is Noetherian.
\end{theorem}
\begin{corollary}
    Suppose $A$ is a finitely generated noetherian $R$-algebra, then $A$ is noetherian. Here, we mean that $A=R(a_1,\dots,a_n).$
\end{corollary}
\begin{proof}
    Consider $J\subset R[x]$ is an ideal. Our goal is to show that this is finitely generated. Consider 
    \begin{equation}
        C_n = \{a\in R |\exists f\in J,f(x)=ax^n+lower\}
    \end{equation}
    Clearly, $C_n$ is an ideal in $R.$ We also have an ACC 
    \begin{equation*}
        C_n \subset C_{n+1} \subset C_{n+2} \subset \cdots,
    \end{equation*}
    because we can take $f(x)$ and multiply it by $x.$ Since $R$ is Noetherian, this terminates at some $C_N=C_{N+1}$. For $n\le N,$ take $\{a_{n_j}\}$ given by $C$ and $f_{n_j}=a_{n_j}x^n+\text{lower}\in J.$

    If $f(x)=cx^n + \cdots + \text{lower}$ in $J,$ then $c\in C_n$ if $n \le N.$ Let $c=\sum_j b_{n_j}a_{n_j}$ with $b_{n_j}\in R,$ so 
    \begin{equation*}
        f(x)-\sum_j b_{n_j}f_{n_j}(x) \in C_{n-1},
    \end{equation*}
    and we can use induction to finish. If $n>N,$ we have 
    \begin{equation}
        f(x) - \sum_j b_{n_j}f_{N_j}(x) x^{n-N}
    \end{equation}
    which has degree smaller or equal to $n-1.$
\end{proof}
What this says is that given any set of polynomial equations, there exists a finite set of polynomial equations such that their simultaneous solutions are the same.
\subsection{Relations among algebraic affine sets}
\begin{enumerate}[label=(\alph*)]
    \item $I_1 \subset I_2 \implies V(I_2) \subseteq V(I_1)$
    \item $V(I_1 \cap I_2) = V(I_1) \cup V(I_2)$
    \item $V\left(\sum_{\lambda}I_{\lambda}\right) = \bigcap_{\lambda} V(I_{\lambda})$
\end{enumerate}
\begin{definition}
    The Zariski topology on $\mathbb{A}^n$ has as closed sets the affine algebraic sets $V(I).$
\end{definition}
This is the topology we will be using for algebraic geometry, but it is also very lousy and course. For example, consider $\mathbb{A}^1.$ Two open sets will always have a nonzero intersection.

We can go in the reverse direction. LEt $X\subset \mathbb{A}^n.$ Let $I(x) = \{f\in R_n|f(P)=a\forall P\in X\}.$ Note:
\begin{enumerate}[label=(\roman*)]
    \item $I(x)$ is an ideal in $R_n$
    \item $I(X)$ is a radical ideal.
\end{enumerate}
\begin{proposition}
    \begin{enumerate}[label=(\alph*)]
        \item $X \subseteq Y \implies I(Y)\subseteq I(X).$
        \item $X\subset \mathbb{A}^n$ for any $X\subset V(I(X)).$
        \item If $X=V(I)$ then $X=V(I(X)).$
    \end{enumerate}
\end{proposition}
\end{document}