\documentclass{article}
\usepackage{qilin}
\tikzstyle{process} = [rectangle, rounded corners, minimum width=1.5cm, minimum height=0.5cm,align=center, draw=black, fill=gray!30, auto]
\title{AER372: Control Systems}
\author{QiLin Xue}
\date{Spring 2022}
\usepackage{mathrsfs}
\usetikzlibrary{arrows}
\usepackage{stmaryrd}
\usepackage{accents}
\newcommand{\ubar}[1]{\underaccent{\bar}{#1}}
\usepackage{pgfplots}
\numberwithin{equation}{section}

\begin{document}

\maketitle
\tableofcontents
\newpage
\section{System Models}
Terminology:
\begin{itemize}
    \item \emf{System:} A collection of components of interest, demarcated by a boundary, interacting through certain physical principles (device / process / plant)
    \item \emf{System Parameters (C)}: Properties that define the components of the system 
    \item \emf{State Variables (X)}: A minimal set of variables that completely identify the state of the system at each moment. 
    \item \emf{Static System:} The output vector $Y(t)$ depends only on the input vector $U(t)$ at time $t.$ For any given input, state variables do not change, i.e. 
    \begin{equation}
        Y(t) = H(U(t),C)
    \end{equation}
    \item \emf{Dynamic System:} The current value of the output depends on the past history as well as the present values of the input. For any given input, state variables change in time.
\end{itemize}
\end{document}