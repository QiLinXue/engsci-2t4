\documentclass{article}
\usepackage{qilin}
\tikzstyle{process} = [rectangle, rounded corners, minimum width=1.5cm, minimum height=0.5cm,align=center, draw=black, fill=gray!30, auto]
\title{PHY294: Review Sheet}
\author{QiLin Xue}
\date{Winter 2022}
\usepackage{mathrsfs}
\usetikzlibrary{arrows}
\usepackage{siunitx}
\begin{document}

\maketitle
\section{The Three-Dimensional Schrödinger Equation.}
In general, the Schrödinger Equation can be written as
\begin{equation}
    \nabla^2 \psi = \frac{2M}{\hbar^2}[U-E]\psi
\end{equation}
where $\nabla^2$ is the Laplacian, which is a fancy way of writing:
\begin{align*}
    \nabla^2\psi &= \frac{\partial^2 \psi}{\partial x^2} + \frac{\partial^2 \psi}{\partial y^2} + \frac{\partial^2 \psi}{\partial z^2} & \text{(Cartesian)} \\
    &= \frac{\partial^2 \psi}{\partial r^2} + \frac{1}{r}\frac{\partial \psi}{\partial r} + \frac{1}{r^2}\frac{\partial^2 \psi}{\partial \phi^2} + \frac{\partial^2 \psi}{\partial z^2} & \text{(Cylindrical)} \\ 
    &= \frac{1}{r}\frac{\partial^2}{\partial r^2}(r\psi) + \frac{1}{r^2\sin\theta}\frac{\partial}{\partial\theta}\left(\sin\theta\frac{\partial\psi}{\partial\theta}\right) + \frac{1}{r^2\sin^2\theta}\frac{\partial^2\psi}{\partial \phi^2}
\end{align*}
\subsection{Particle in a Box}
For a particle in an infinite square well, the allowed energies are
\begin{equation}
    E = \frac{\hbar^2 \pi^2}{2M}\left(\frac{n_x^2}{L_x^2}+\frac{n_y^2}{L_y^2}+\frac{n_z^2}{L_z^2}\right)
\end{equation}
Here, $n_x,n_y,n_z$ are known as \emf{quantum numbers}. When the same energy occurs at different quantum numbers, then we call it a \emf{degeneracy.}
\subsection{Central Force Problem}
\subsubsection*{2-Dimensions}
For the two-dimensional case, our solution is in the form of $\psi(r,\theta) = R(r)\Theta(\theta).$ Solving for $\Theta(\theta)$ gives:
\begin{align*}
    \Phi(\phi) = A\sin(m\phi) + B\cos(m\phi),
\end{align*}
which must satisfy the property $\Phi(\phi)=\Phi(\phi+2\pi),$ forcing $m$ to be an integer. The radial part $R(r)$ cannot be solved since it depends on the specific potential.
\subsubsection*{3-Dimensions}
The solution is in the form of $\phi(r,\theta,\phi)=R(r)\Theta(\theta)\Phi(\phi).$ 
\begin{itemize}
    \item Solving for $\Phi(\phi)$ gives the same equation as the $\Theta(\theta)$ case in 2-D, leading to the same quantum numbers. 
    \item Solving for $\Theta(\theta)$ gives a parameter $k=l(l+1)$ where $l \ge |m|$ and is an integer.
\end{itemize} 
\subsection{Angular Momentum}
The magnitude of the angular momentum is quantized and given by
\begin{equation}
    L = \sqrt{l(l+1)}\hbar.
\end{equation} 
The angular momentum acts in a certain direction away from the $z$-axis, and
\begin{equation}
    L_z = m\hbar,
\end{equation}
where $m$ has the usual restrictions. As a result, we can often draw these as vectors on a diagram. For a given $l$, there are $2l+1$ possible vectors.
\subsection{Hydrogen Atom}
Consider the simplest (yet useful) potential $U(r) = \frac{-ke^2}{r}$, for a hydrogen atom. The $R(r)$ part is given by
\begin{equation}
    \frac{d^2}{dr^2}(rR)  = \frac{2m_e}{\hbar^2}\left[-\frac{ke^2}{r}+\frac{l(l+1)\hbar^2}{2m_er^2}-E\right](rR),
\end{equation}
which only has acceptable solutions if
\begin{equation}
    E = -\frac{m_e(ke^2)^2}{2\hbar^2}\frac{1}{n^2} =  - \frac{E_R}{n^2},
\end{equation}
where $n$ has the restriction $n>l.$
\subsection{Hydrogenic Wave Functions}
The solution for $R(r)$ is in the form of
\begin{equation}
    R(r) = Ae^{-r/(na_B)}
\end{equation}
where $a_B = \frac{h^2}{m_eke^2}$ is the Bohr radius, and $A$ is some function. The radial probability density is
\begin{equation}
    P(r) = 4\pi r^2 |R(r)|^2
\end{equation}
\subsection{Hydrogen-like Ions}
We can extend this discussion to any ion with one electron bounded to a nucleus of charge $Ze$. Note that $\Phi,\Theta$ are left unchanged and everything else can be derived by replacing $ke^2$ with $Zke^2$ and as a direct consequence, replacing $a_B$ with $a_B/Z.$
\end{document}